\begin{example}[twoRounds]    
    We show in Fig.~\ref{fig:twoRounds}(c) the estimated dependency graph that our static analysis algorithm returns for the program $\kw{twoRounds(k)}$ in Fig.~\ref{fig:twoRounds}(a).
    Vertices and query annotations are the same as the ones in Fig.~\ref{fig:twoRounds}(b) and they are simply inferred by scanning the program.
    As we said before, the edges are estimated using control flow and data flow analysis.
    For the $\kw{twoRounds(k)}$ example, every edge in Fig.~\ref{fig:twoRounds}(b) is precisely inferred by our combined analysis, this is why Fig.~\ref{fig:twoRounds}(c) contains exactly the same edges.
    The weight of every vertex is computed using a reachability-bound estimation algorithm which output a symbolic expression over the input variables, in the example only $k$, representing an upper bound on the number of times each assignment is executed.
    % \wq{symbolic and provide a sound upper bound on its execution-based weight of the corresponding vertex in the execution-based dependency graph.
    % $w_k(\trace) \leq \trace(k)$. $\trace(k)$ means getting the value of variable $k$ in the trace $\trace$. The soundness of this step is proved in Theorem~\ref{thm:addweight_soundness}.}
    For example, consider the vertex $x^{3}$, its weight is $k$ and this provides an upper bound on the values returned by the weight function $\lambda \trace. \rho(\trace)k$ associated with vertex $x^{3}$ in Fig.~\ref{fig:twoRounds}(b) for any initial state. 
    % Indeed, 
    % for any initial trace $\trace_0$, when $w_{x^{3}}(\trace_0)$ executes the program and counts the
    % execution times of command $3$,
    % we expect that this counts is at most the the loop iterations, i.e. $k$'s initial value from $\trace_0$.
    
    The algorithm searching for the longest walk first finds a path $l^6:{}^1_1 \to a^5: {}^k_1 \to x^3: {}^k_1$, and then constructs a walk based on this path. Every vertex on this walk is visited once, and the number of vertices with query annotation $1$ in this walk is $2$, which is the upper bound we expect.
    {It is worth to note here that $x^3$ and $a^5$ can only be visited once because there isn't an edge to go back to them, even though they both have the weight $k$}.
    In this sense, instead of simply computing the weighted length of this path ($2k+1$) as adaptivity $\pathsearch$ computes the upper bound $2$. Note that $2$ is not always tight, for example when $k = 0$.
    
{\small
\begin{figure}
\centering
\begin{subfigure}{1.0\textwidth}
\begin{centering}
$
    \begin{array}{l}
    \kw{towRounds(k)} \triangleq \\
           \clabel{ \assign{a}{0}}^{0} ;
            \clabel{\assign{j}{k} }^{1} ;\\
            \ewhile ~ \clabel{j > 0}^{2} ~ \edo ~ 
            \Big(
             \clabel{\assign{x}{\query(\chi[j] \cdot \chi[k])} }^{3}  ; 
             \clabel{\assign{j}{j-1}}^{4} ;
            \clabel{\assign{a}{x + a}}^{5}       \Big);\\
            \clabel{\assign{l}{\query(\chi[k]*a)} }^{6}
        \end{array}
$
\caption{}
\end{centering}
\end{subfigure}
\begin{subfigure}{.5\textwidth}
%}
\qquad
\begin{centering}
\begin{tikzpicture}[scale=\textwidth/15cm,samples=200]
\draw[] (0, 10) circle (0pt) node
{{ $a^0: {}^{\lambda \trace_0. 1}_{0}$}};
\draw[] (0, 7) circle (0pt) node
{\textbf{$x^3: {}^{\lambda \trace_0. \env(\trace_0) k}_{1}$}};
\draw[] (0, 4) circle (0pt) node {{ $a^5: {}^{\lambda \trace_0. \env(\trace_0) k}_{0}$}};
\draw[] (0, 1) circle (0pt) node
{{ $l^6: {}^{\lambda \trace_0. 1}_{1}$}};
% Counter Variables
\draw[] (8, 9) circle (0pt) node {\textbf{$j^1: {}^{\lambda \trace_0. 1}_{0}$}};
\draw[] (8, 6) circle (0pt) node {{ $j^4: {}^{\lambda \trace_0. \env(\trace_0) k}_{0}$}};
%
% Value Dependency Edges:
\draw[ ultra thick, -latex, densely dotted,] (0, 1.5)  -- (0, 3.5) ;
\draw[ ultra thick, -latex, densely dotted,] (0, 4.5)  -- (0, 6.5) ;
\draw[ -latex] (0, 4.5)  to  [out=-230,in=230]  (0, 9.5) ;
\draw[ -Straight Barb] (1.5, 3.8) arc (120:-200:1);
\draw[ -Straight Barb] (9, 6.5) arc (150:-150:1);
\draw[ -latex] (8, 6.5)  -- (8, 8.5) ;
\draw[ -latex] (0, 1.5)  to  [out=-230,in=230]  (0, 9.5) ;
% Control Dependency
\draw[ -latex] (2, 7)  -- (6, 9) ;
\draw[ -latex] (2, 7)  -- (5.5, 6) ;
\draw[ -latex] (2, 4.5)  -- (6, 9) ;
\draw[ -latex] (2, 4.5)  -- (5.5, 6) ;
% Edges Produced By Transitivity
\draw[ -latex] (2, 1)  -- (6, 9) ;
\draw[ -latex] (2, 1)  -- (5.5, 6) ;
\draw[ -latex] (0, 1.5)  to  [out=50,in=-50]  (0, 6.5) ;
\end{tikzpicture}
\caption{}
\end{centering}
\end{subfigure}
   \begin{subfigure}{.45\textwidth}
   \begin{centering}
   \begin{tikzpicture}[scale=\textwidth/15cm,samples=200]
\draw[] (0, 10) circle (0pt) node
{{ $a^0: {}^1_{0}$}};
\draw[] (0, 7) circle (0pt) node
{\textbf{$x^3: {}^{k}_{1}$}};
\draw[] (0, 4) circle (0pt) node
{{ $a^5: {}^{k}_{0}$}};
\draw[] (0, 1) circle (0pt) node
{{ $l^6: {}^{1}_{1}$}};
% Counter Variables
\draw[] (6, 9) circle (0pt) node {\textbf{$j^1: {}^{1}_{0}$}};
\draw[] (6, 6) circle (0pt) node {{ $j^4: {}^{k}_{0}$}};
%
% Value Dependency Edges:
\draw[ ultra thick, -latex, densely dotted,] (0, 1.5)  -- (0, 3.5) ;
\draw[ ultra thick, -latex, densely dotted,] (0, 4.5)  -- 
(0, 6.5) ;
\draw[  -latex] (0, 4.5)  to  [out=-230,in=230]  
(0, 9.5) ;
\draw[  -Straight Barb] (1.5, 3.5) arc (120:-200:1);
\draw[  -Straight Barb] (6.5, 6.5) arc (150:-150:1);
\draw[  -latex] (6, 6.5)  -- (6, 8.5) ;
% Control Dependency
\draw[ -latex] (1.5, 7)  -- (5, 9) ;
\draw[ -latex] (1.5, 4)  -- (5, 9) ;
\draw[ -latex] (1.5, 7)  -- (5, 6) ;
\draw[ -latex] (1.5, 4)  -- (5, 6) ;
\draw[  -latex] (0, 1.5)  to  [out=-230,in=230]  (0, 9.5) ;
% Edges Produced By Transitivity
\draw[ -latex] (2, 1)  -- (5, 9) ;
\draw[ -latex] (2, 1)  -- (5, 6) ;
\draw[ -latex] (0, 1.5)  to  [out=50,in=-50]  (0, 6.5) ;
\end{tikzpicture}
\caption{}
   \end{centering}
   \end{subfigure}
\vspace{-0.4cm}
 \caption{(a) The program $\kw{towRounds(k)}$, an example 
%  of a program 
with two rounds of adaptivity (b) The corresponding semantics-based dependency graph (c) The estimated dependency graph from $\THESYSTEM$.
}
\label{fig:twoRounds}
\end{figure}
}
\end{example}