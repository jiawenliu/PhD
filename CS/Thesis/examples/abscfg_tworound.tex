% The edge $(0 \xrightarrow{a' \leq 0} 1)$ on the top tells us the command 
% $\clabel{\assign{a}{0}}^0$ is executed with a continuation point $1$ such that the
% % where the 
% command $\clabel{\assign{j}{k}}^1$ will be executed next.
% The annotation $a' \leq 0$ is a difference constraint 
% computed for
% % by abstracting
% the expression $0$ in the assignment command $\assign{a}{0}$.
% %  from the function $\absexpr(0)$.
% It represents that the value of $a$ is less than or equal to $0$ after the
% execution of $\clabel{\assign{a}{0}}^0$ and before executing $\clabel{\assign{j}{k}}^1$.
% Another example edge $5 \xrightarrow{a' \leq a + x } 2$ describes the execution of
%  the command
% $\clabel{\assign{a}{x + a}}^{5}$.
% This edge has difference constraint $a' \leq a+x $.
% The $a'$ on the left side represents the value of $a$ after executing this assignment command. 
% $a' \leq a+x $ denotes the value of $a$ after executing $\assign{a}{x + a}$ is at most $a$'s value plus $x$'s value before this execution.
The edge $(1 \xrightarrow{j' \leq k} 2)$ on the top tells us the command 
$\clabel{\assign{j}{k}}^1$ is executed with a continuation point $2$ such that the
% where the 
guard $\clabel{j > 0}^2$ will be evaluated next.
The annotation $j' \leq k$ is a difference constraint 
computed for
% by abstracting
the expression $k$ from the assignment command $\assign{j}{k}$.
%  from the function $\absexpr(0)$.
It represents that the value of $j$ is less than or equal to value of input variable $k$ after the
execution of $\clabel{\assign{a}{0}}^0$ and before executing the loop.
% Another example edge $5 \xrightarrow{a' \leq a + x } 2$ describes the execution of
%  the command
% $\clabel{\assign{a}{x + a}}^{5}$.
The boolean constraint $j \leq 0 $ on the edge $2 \xrightarrow{j \leq 0} 6$
represents the negation of the testing guard $j > 0$
of the $\ewhile$ command with header at label $2$.
% The edge from $3$ to $4$ comes from the query request command $\clabel{\assign{x}{\query(\chi[j])} }^{3}$.
% The constraint over this edge, $x' < Q_m$ describes after executing $\assign{x}{\query(\chi[j])}$,
% % $\clabel{\assign{x}{\query(\chi[j])} }^{3}$, 
% the query request results stored in $x$ is bounded by $Q_m$. 

\begin{figure} 
    \centering
    \begin{subfigure}{1.0\textwidth}
      \begin{centering}
      $
          \begin{array}{l}
          \kw{towRounds(k)} \triangleq \\
                 \clabel{ \assign{a}{0}}^{0} ;
                  \clabel{\assign{j}{k} }^{1} ;\\
                  \ewhile ~ \clabel{j > 0}^{2} ~ \edo ~ 
                  \Big(
                   \clabel{\assign{x}{\query(\chi[j] \cdot \chi[k])} }^{3}  ; 
                   \clabel{\assign{j}{j-1}}^{4} ;
                  \clabel{\assign{a}{x + a}}^{5}       \Big);\\
                  \clabel{\assign{l}{\query(\chi[k]*a)} }^{6}
              \end{array}
      $
      \caption{}
      \end{centering}
      \end{subfigure}
    \begin{subfigure}{.47\textwidth}
        \begin{centering}
      \begin{tikzpicture}[scale=\textwidth/20cm,samples=200]
      \draw[] (-7, 10) circle (0pt) node{{ $0$}};
      \draw[] (0, 10) circle (0pt) node{{ $1$}};
      \draw[] (0, 7) circle (0pt) node{\textbf{$2$}};
      \draw[] (0, 4) circle (0pt) node{{ $3$}};
      \draw[] (0, 1) circle (0pt) node{{ $4$}};
      \draw[] (-7, 1) circle (0pt) node{{ $5$}};
      % Counter Variables
      \draw[] (6, 7) circle (0pt) node {\textbf{$6$}};
      \draw[] (6, 4) circle (0pt) node {{ $\lex$}};
      %
      % Control Flow Edges:
      \draw[  -latex] (-6, 10)  -- node [above] {$\top$}(-1.5, 10);
      \draw[ -latex] (0, 9.5)  -- node [left] {$j' \leq k$} (0, 7.5) ;
      \draw[ -latex] (0, 6.5)  -- node [right] {$j > 0 $}  (0, 4.5);
      \draw[ -latex] (0, 3.5)  -- node [right] {$\top $} (0, 1.5) ;
      \draw[ -latex] (-0.5, 1)  -- node [above] {$j' \leq j - 1$} (-6, 1) ;
      \draw[ -latex] (-6, 1.5)  -- node [left] {$\top$} (-0.5, 7)  ;
      \draw[ -latex] (0.5, 7)  -- node [above] {$ j \leq 0 $}  (5.5, 7);
      \draw[ -latex] (6, 6.5)  -- node [right] {$\top$} (6, 4.5) ;
      \end{tikzpicture}
      \caption{}
        \end{centering}
        \end{subfigure}
        \begin{subfigure}{.5\textwidth}
          \begin{centering}
        %   \todo{abstract-cfg for two round}
        \begin{tikzpicture}[scale=\textwidth/20cm,samples=200]
        \draw[] (-10, 10) circle (0pt) node{{ $0: 1$}};
        \draw[] (0, 10) circle (0pt) node{{ $1: 1$}};
        \draw[] (0, 7) circle (0pt) node{\textbf{$2: k$}};
        \draw[] (0, 4) circle (0pt) node{{ $3: k$}};
        \draw[] (0, 1) circle (0pt) node{{ $4: k$}};
        \draw[] (-10, 1) circle (0pt) node{{ $5: k$}};
        % Counter Variables
        \draw[] (6, 7) circle (0pt) node {\textbf{$6: 1$}};
        \draw[] (6, 4) circle (0pt) node {{ $\lex: 1$}};
        %
        % Control Flow Edges:
      \draw[  -latex] (-8, 10)  -- node [above] {$\top$}(-1.5, 10);
      \draw[ -latex] (0, 9.5)  -- node [left] {$j' \leq k$} (0, 7.5) ;
      \draw[ -latex] (0, 6.5)  -- node [right] {$j > 0 $}  (0, 4.5);
      \draw[ -latex] (0, 3.5)  -- node [right] {$\top $} (0, 1.5) ;
      \draw[ -latex] (-1.5, 1)  -- node [above] {$j' \leq j - 1$} (-8, 1) ;
      \draw[ -latex] (-8, 1.5)  -- node [left] {$\top$} (-1.5, 7)  ;
      \draw[ -latex] (1.5, 7)  -- node [above] {$j \leq 0 $}  (4.5, 7);
      \draw[ -latex] (6, 6.5)  -- node [right] {$\top$} (6, 4.5) ;
        \end{tikzpicture}
        \caption{}
          \end{centering}
          \end{subfigure}
      \caption{(a) The same $\kw{towRounds(k)}$ program as Figure~\ref{fig:twoRounds}
      (b) The abstract control flow graph for $\kw{towRounds(k)}$  (c) The abstract control flow graph with the reachability bound for $\kw{towRounds(k)}$.}
      \label{fig:abscfg_tworound}
    \end{figure}