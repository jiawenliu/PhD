\begin{example}[twoRounds]
    We illustrate the key technical components of our framework through a simple adaptive data analysis with two rounds of adaptivity.
    In this analysis, an analyst asks $k+1$ queries to a mechanism in two phases.
    In the first phase, the analyst asks $k$ queries and stores the answers that are provided by the mechanism. In the second phase, the analyst constructs a new query based on the results of the previous $k$ queries and sends this query to the mechanism. 
    The mechanism is abstract here and our goal is to use static analysis to provide an upper bound on adaptivity to help choose the mechanism.
    This data analysis assumes that the data domain $\univ$ 
    contains at least $k$ numeric attributes 
    (every query in the first phase focuses on one), which we index just by natural numbers.
    The implementation of this data analysis in the language of {\THESYSTEM} is presented in Fig.~\ref{fig:twoRounds}(a).

The \emph{semantics-based dependency graph} of the $\kw{twoRounds(k)}$ program
we gave in Fig.~\ref{fig:twoRounds}(a) is described in Fig.~\ref{fig:twoRounds}(b) (we use dashed arrows for two edges that will be highlighted in the next step, for the moment these can be considered similar to the other edges---i.e. solid arrows). We have all the variables that are assigned in the program with their labels, and edges representing dependency relations between them. 
For example, we have two edges $(l^6, a^5)$ and $(a^5, x^3)$ describing the dependency between the variables assigned by queries. The vertices $l^6$ and $x^3$ are the only ones with query annotation $1$ (the subscript), since they are the only two variables that are in assignments involving  queries. Notice that the graph contains cycles---in this example it contains two self-loops. These cycles capture the fact that the variables $a^5$ and $j^4$ are updated at every iteration of the loop using their previous value. Cycles are essential to capture mutual dependencies like the ones that are generated in loops. Adaptivity is a quantitative notion, so capturing this form of dependencies is not enough. This is why we also use weights. The weight of a vertex is a function that given an initial state returns a natural number representing 
the number of times the assignment corresponding to a vertex is visited during the program execution starting in this initial state.  
For example, the vertex $l^{6}$ has weight {$\lambda \trace.1$} since for every initial state {$\trace$} the corresponding assignment will be executed one time, the vertex $a^5$ on the other hand has weight $\lambda \trace. \env(\trace) k$ since the corresponding assignment will be executed a number of times that correspond to the value of $k$ in the initial state $\trace$, and $\env$ is the operator reading value of $k$ from $\trace$.

{\small
\begin{figure}
\centering
\begin{subfigure}{1.0\textwidth}
\begin{centering}
$
    \begin{array}{l}
    \kw{towRounds(k)} \triangleq \\
           \clabel{ \assign{a}{0}}^{0} ;
            \clabel{\assign{j}{k} }^{1} ;\\
            \ewhile ~ \clabel{j > 0}^{2} ~ \edo ~ 
            \Big(
             \clabel{\assign{x}{\query(\chi[j] \cdot \chi[k])} }^{3}  ; 
             \clabel{\assign{j}{j-1}}^{4} ;
            \clabel{\assign{a}{x + a}}^{5}       \Big);\\
            \clabel{\assign{l}{\query(\chi[k]*a)} }^{6}
        \end{array}
$
\caption{}
\end{centering}
\end{subfigure}
\begin{subfigure}{.5\textwidth}
%}
\qquad
\begin{centering}
\begin{tikzpicture}[scale=\textwidth/15cm,samples=200]
\draw[] (0, 10) circle (0pt) node
{{ $a^0: {}^{\lambda \trace_0. 1}_{0}$}};
\draw[] (0, 7) circle (0pt) node
{\textbf{$x^3: {}^{\lambda \trace_0. \env(\trace_0) k}_{1}$}};
\draw[] (0, 4) circle (0pt) node {{ $a^5: {}^{\lambda \trace_0. \env(\trace_0) k}_{0}$}};
\draw[] (0, 1) circle (0pt) node
{{ $l^6: {}^{\lambda \trace_0. 1}_{1}$}};
% Counter Variables
\draw[] (8, 9) circle (0pt) node {\textbf{$j^1: {}^{\lambda \trace_0. 1}_{0}$}};
\draw[] (8, 6) circle (0pt) node {{ $j^4: {}^{\lambda \trace_0. \env(\trace_0) k}_{0}$}};
%
% Value Dependency Edges:
\draw[ ultra thick, -latex, densely dotted,] (0, 1.5)  -- (0, 3.5) ;
\draw[ ultra thick, -latex, densely dotted,] (0, 4.5)  -- (0, 6.5) ;
\draw[ -latex] (0, 4.5)  to  [out=-230,in=230]  (0, 9.5) ;
\draw[ -Straight Barb] (1.5, 3.8) arc (120:-200:1);
\draw[ -Straight Barb] (9, 6.5) arc (150:-150:1);
\draw[ -latex] (8, 6.5)  -- (8, 8.5) ;
\draw[ -latex] (0, 1.5)  to  [out=-230,in=230]  (0, 9.5) ;
% Control Dependency
\draw[ -latex] (2, 7)  -- (6, 9) ;
\draw[ -latex] (2, 7)  -- (5.5, 6) ;
\draw[ -latex] (2, 4.5)  -- (6, 9) ;
\draw[ -latex] (2, 4.5)  -- (5.5, 6) ;
% Edges Produced By Transitivity
\draw[ -latex] (2, 1)  -- (6, 9) ;
\draw[ -latex] (2, 1)  -- (5.5, 6) ;
\draw[ -latex] (0, 1.5)  to  [out=50,in=-50]  (0, 6.5) ;
\end{tikzpicture}
\caption{}
\end{centering}
\end{subfigure}
   \begin{subfigure}{.45\textwidth}
   \begin{centering}
   \begin{tikzpicture}[scale=\textwidth/15cm,samples=200]
\draw[] (0, 10) circle (0pt) node
{{ $a^0: {}^1_{0}$}};
\draw[] (0, 7) circle (0pt) node
{\textbf{$x^3: {}^{k}_{1}$}};
\draw[] (0, 4) circle (0pt) node
{{ $a^5: {}^{k}_{0}$}};
\draw[] (0, 1) circle (0pt) node
{{ $l^6: {}^{1}_{1}$}};
% Counter Variables
\draw[] (6, 9) circle (0pt) node {\textbf{$j^1: {}^{1}_{0}$}};
\draw[] (6, 6) circle (0pt) node {{ $j^4: {}^{k}_{0}$}};
%
% Value Dependency Edges:
\draw[ ultra thick, -latex, densely dotted,] (0, 1.5)  -- (0, 3.5) ;
\draw[ ultra thick, -latex, densely dotted,] (0, 4.5)  -- 
(0, 6.5) ;
\draw[  -latex] (0, 4.5)  to  [out=-230,in=230]  
(0, 9.5) ;
\draw[  -Straight Barb] (1.5, 3.5) arc (120:-200:1);
\draw[  -Straight Barb] (6.5, 6.5) arc (150:-150:1);
\draw[  -latex] (6, 6.5)  -- (6, 8.5) ;
% Control Dependency
\draw[ -latex] (1.5, 7)  -- (5, 9) ;
\draw[ -latex] (1.5, 4)  -- (5, 9) ;
\draw[ -latex] (1.5, 7)  -- (5, 6) ;
\draw[ -latex] (1.5, 4)  -- (5, 6) ;
\draw[  -latex] (0, 1.5)  to  [out=-230,in=230]  (0, 9.5) ;
% Edges Produced By Transitivity
\draw[ -latex] (2, 1)  -- (5, 9) ;
\draw[ -latex] (2, 1)  -- (5, 6) ;
\draw[ -latex] (0, 1.5)  to  [out=50,in=-50]  (0, 6.5) ;
\end{tikzpicture}
\caption{}
   \end{centering}
   \end{subfigure}
% \vspace{-0.4cm}
 \caption{(a) The program $\kw{towRounds(k)}$, an example 
%  of a program 
with two rounds of adaptivity (b) The corresponding semantics-based dependency graph (c) The estimated dependency graph from $\THESYSTEM$.
}
\label{fig:twoRounds}
\end{figure}
}
\end{example}