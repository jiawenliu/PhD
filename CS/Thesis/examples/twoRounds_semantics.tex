\begin{example}[Evaluation Trace for A Two Rounds Data Analysis Example Program]
    In the following example program $\kw{towRounds(k)}$,    the analyst asks in total $k+1$ queries to the mechanism in two phases.
    %
    \[         \begin{array}{l}
      \kw{towRounds(k)} \triangleq \\
             \clabel{ \assign{a}{0}}^{0} ;
              \clabel{\assign{j}{k} }^{1} ;\\
              \ewhile ~ \clabel{j > 0}^{2} ~ \edo ~
              \Big(
               \clabel{\assign{x}{\query(\chi[j] \cdot \chi[k])} }^{3}  ;
               \clabel{\assign{j}{j-1}}^{4} ;
              \clabel{\assign{a}{x + a}}^{5}       \Big);\\
              \clabel{\assign{l}{\query(\chi[k]*a)} }^{6}
          \end{array}
          \]
    In the first phase, the analyst asks $k$ queries and stores the answers that are provided by the mechanism. 
    In the second phase, the analyst constructs a new query based on the results of the previous $k$ queries and sends this query to the mechanism. More specifically, I assume that, in this example, the domain $\dbdom$ 
    contains at least $k$ numeric attributes, which I index just by natural numbers. 
    The queries inside the while loop correspond to the first phase and compute an approximation of 
    the product of the empirical mean of the first $k$ attributes. 
    The query outside the loop corresponds to the second phase and computes an approximation of the empirical mean where each record is weighted by the sum of the empirical mean of the first $k$ attributes.
    %
    % Queries are of the form $q(e)$ where $e$ is an expression with a special variable $\chi$ representing a possible row. Mainly $e$ represents a function from $X$ to some domain $U$, for example $U$ could be $[-1,1]$ or $[0,1]$. This function characterizes the linear query I are interested in running. As an example, $x \leftarrow q(\chi[2])$ computes an approximation, according to the used mechanism, of the empirical mean of the second attribute, identified by $\chi[2]$. Notice that I don't materialize the mechanism but I assume that it is implicitly run when I execute the query. 
    % \jl{We use $\chi$ to abstract a possible row in the database and }
    % queries are of the form $\query(\qexpr)$, where $\qexpr$ is a special expression 
    With the initial trace
    $\trace_0 = [(k, in, 2, \bullet)]$,
    the trace generated from this example is 
   %   where $k$ is the 
   %  initial value of input variable $k$ given by user,
   %  we observe the execution trace as
   \\
    $
    \left[\begin{array}{l}
    % \trace_0 \tracecat
     (a, 0, 0, \bullet),
    (j, 1, 2, \bullet),
    (j>0, 2, \etrue, \bullet),
    (x, 3, v_1, \chi[2]*\chi[2]),
    (j, 4, 1, \bullet),
    (a, 5, v_1, \bullet),\\
    (j>0, 2, \etrue, \bullet),
    (x, 3, v_2, \chi[1]*\chi[2]),
    (j, 4, 0, \bullet),
    (a, 5, v_1 + v_2, \bullet),
    (j>0, 4, \efalse, \bullet),\\
    (l, 6, v_3, \chi[2]*( v_1 + v_2))
    \end{array} \right]
    $.
\end{example}