
\subsection{Reachability Bound Analysis}
\label{sec:intro-methodology-reachability}
\subsubsection{Execution-Based Program Analysis}
% \label{sec:intro-exe}

\subsubsection{Static Program Analysis}
% \label{sec:intro-exe}
\subsection{Adaptivity Analysis}
\label{sec:intro-methodology-adapt}
\subsubsection{Dependency Analysis Methodology via Analyzing Execution}
% \label{sec:intro-exe}
% \begin{enumerate}
%  \item
%  \textbf{Adaptive Data Analysis Formalization}
% The first challenge is \emph{how to define formally} a model for adaptive data analysis which is general enough to support the methods I discussed above and would permit to formulate the notion of adaptivity these methods use. 
% I take the approach of designing a programming framework for submitting queries to some \emph{mechanism} giving access to the data mediated by one of the techniques I mentioned before, e.g., adding Gaussian noise, randomly selecting a subset of the data, using the reusable holdout technique, etc. 
% In this approach, a program models an \emph{analyst} asking a sequence of queries to the mechanism. The mechanism runs the queries on the data applying one of the methods discussed above and returns the result to the program. The program can then use this result to decide which query to run next. 
% % Overall, I'm interested in controlling the generalization of the results of the queries which are returned by the mechanism, by means of adaptivity. 
% \item 
% This is because I focus on the generalization error of queries and not their post-processing. % 
\paragraph{Existing Methodology and Limitations}
There are many techniques for analyzing a program's semantics or execution, such are program simulation, 
semantics analysis, abstract interpretation, etc., which are widely used in defining
and formalizing the program's properties.
% dependency analysis. 
However, it is not straightforward how to apply them to analyzing and formalizing the \emph{adaptivity}
for adaptive data analysis program.
There are following three limitations among existing execution-based analysis techniques.
\begin{enumerate}
\item The state-of-art dependency relation analysis techniques don't
consider both control influence and value influence. However, both of these
two influences are included in the intuitive query dependency 
in adaptive data analysis programs.
% require
\item In the existing data dependency analysis area, there isn't any technique taking into account
the dependency depth. However, this dependency depth is the key quantity in formalizing the intuitive \emph{adaptivity}.
\item The intuitive \emph{adaptivity} doesn't accumulate linearly through the dependency relation and 
dependency depth. There is only very limited works related to this non-linear quantity property analysis.
This requires us to design new analysis techniques in order to formalize this quantity precisely.
% here isn't any research combining the two pieces of information. No need to mention the adaptivity analysis.
\end{enumerate}
%
\paragraph{New Methodology}
 To formalize this intuitive \emph{adaptivity} as a quantitative program property, 
% I develop an execution-based analysis in Chapter~\ref{ch:dynamic}.
I combine three program analysis methodologies.
%  This execution-based analysis is designed in three steps through different methodologies as follows,
\begin{enumerate}
   \item First one is to define the \emph{dependency relation} between every query, 
   through the methodology of semantic data dependency analysis.
   %
  % \\
   \item The second one is to define the \emph{dependency quantity} 
  %  analysis, 
  based on the \emph{dependency relation} above through the methodology of execution-based reachability bound analysis.
   \item The last one is the methodology of constructing the dependency graph, by combining the \emph{dependency relation} and \emph{dependency quantity} defined above.
  %  is developed through 
      \end{enumerate}
   % \item 
\subsubsection{Static Program Analysis Methodology}
\label{sec:intro-static}
There are many static program analysis techniques, which are widely used in dependency analysis. 
However, it is not straightforward when applied to adaptivity analysis.
\begin{enumerate}
\item To the best of my knowledge,
existing static dependency analysis techniques do not consider both control influence and value influence.
But in order to analyze and estimate the \emph{adaptivity} statically, as discussed above, we need to consider both of them.
\item The existing static analysis techniques in the complexity or resource cost estimation areas,
cannot give accurate reachability times on programs 
at every execution location.
\item Moreover, existing static analysis techniques and their research focus consider
either only the dependency analysis
or only the program quantitative property (such as complexity or resource cost),
but not both of them.
% No need to mention the adaptivity analysis.
\end{enumerate}

\paragraph{New Methodology}
I develop a static program adaptivity analysis framework, named {\THESYSTEM} in Chapter~\ref{ch:static}.
%  which 
% {\THESYSTEM} combines data flow and control flow analysis with reachability bound analysis~\cite{GulwaniZ10}. 
% This new program analysis gives tighter bounds on the adaptivity of a program than the ones one would achieve by directly using the data and control flow analyses or the ones that one would achieve by directly using reachability bound analysis techniques alone. Specifically as follows in the same 3 aspects as the execution-based analysis 
% while through static program analysis techniques, a sound estimated result will be given in each aspect as follows.
This new methodology combines data flow and control flow analysis with reachability bound analysis.
% ~\cite{GulwaniZ10}. 
This new program analysis gives tighter bounds on the adaptivity of a program than the ones one would achieve 
by directly using the data and control flow analyses or the ones that one would achieve 
by directly using reachability bound analysis techniques alone. 
% Specifically as follows in the same 
\begin{enumerate}
% \item The data dependency relation analysis through the static data flow analysis technique.
% \item The dependency quantity analysis through the static program reachability bound analysis techniques.
% \item The program adaptivity estimation, through newly designed algorithms based on the results estimated above, 
% computing the adaptivity upper bound soundly 
% and accurately.
\item The {\THESYSTEM} first adopts static data flow analysis technique for estimating the data \emph{dependency relation}
%  through the 
% in Section~\ref{sec:alg_weightedgegen}.
% This analysis corresponds to the first step in execution-based adaptivity analysis. 
% The estimated result produced from 
% this step is proved as a sound upper bound for the \emph{dependency relation} from execution-based analysis.
\item In the second part, {\THESYSTEM} adopts the static program reachability bound analysis techniques.
\item 
{\THESYSTEM} also uses the methodology of constructing dependence graph for approximating the execution-based dependency graph.
%  in Section~\ref{sec:dynamic-adapt}.
Then, based on this graph, {\THESYSTEM} 
% I design an algorithm
%  based on the results estimated above, 
% computing 
it adopts the path search methodology from the graph theory and estimates the longest path on this graph.
\end{enumerate}