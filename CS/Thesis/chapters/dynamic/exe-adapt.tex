
Based on the variable \emph{may-dependency} relation in Section~\ref{subsec:dynamic-datadep} and 
the dependency quantity analysis in Section~\ref{subsec:dynamic-reachability}.
% gives us the edges, 
I firstly define the execution-based dependency graph, then formalize the \emph{adaptivity} in this section.
% \wq{Just a few sentences here, some overview of this subsection. See 4.2 for instance.}
\paragraph{Execution Based Dependency Graph}
\label{para:execution-base-graph-def}
Based on the variable \emph{may-dependency} relation,
% gives us the edges, 
we define the execution-based dependency graph.
% \wq{Just a few sentences here, some overview of this subsection. See 4.2 for instance.}
\begin{defn}[Execution Based Dependency Graph]
\label{def:trace_graph}
Given a program ${c}$,
its \emph{execution-based dependency graph} 
$\traceG({c}) = (\traceV({c}), \traceE({c}), \traceW({c}), \traceF({c}))$ is defined as follows,
{
  \small
\[
\begin{array}{rlcl}
  \text{Vertices} &
  \traceV({c}) & := & \left\{ 
  x^l \in \mathcal{LV}
  ~ \middle\vert ~ x^l \in \lvar(c)
  \right\}
  \\
  \text{Directed Edges} &
  \traceE({c}) & := & 
  \left\{ 
  (x^i, y^j) 
%   \in \mathcal{LV} \times \mathcal{LV}
  ~ \middle\vert ~
  x^i, y^j \in \lvar(c) \land \vardep(x^i, y^j, c) 
  % \text{\mg{$\land$ instead of ,}}
  \right\}
  \\
  \text{Weights} &
  \traceW({c}) & := & 
%   \left
  \{ 
  (x^l, w) 
  % \in \mathcal{LV} \times \mathbb{N}
  ~ \vert ~ 
  w : \mathcal{T} \to \mathbb{N}
  \land
  x^l \in \lvar(c) 
  \\ & & &
  \land
  % n = \max \left\{ 
    % ~ \middle\vert~
  \forall \vtrace \in \mathcal{T}_0(c), \trace' \in \mathcal{T} \sthat \config{{c}, \trace} \to^{*} \config{\eskip, \trace\tracecat\vtrace'} 
  \implies w(\trace) = \vcounter(\vtrace', l) 
  %  \right\}
%   \right
\}
  \\
  % \text{Query Label} &
  \text{Query Annotation} &
  \traceF({c}) & := & 
\left\{(x^l, n)  
% \in  \mathcal{LV}\times \{0, 1\} 
~ \middle\vert ~
 x^l \in \lvar(c) \land
n = 1 \Leftrightarrow x^l \in \qvar(c) \land n = 0 \Leftrightarrow  x^l \notin \qvar(c)
\right\}
\end{array}.
\]
}
\end{defn}
%
There are four components of the execution-based dependency graph. 
The vertices $\traceV(c)$ is the set of program $c$'s labeled variables $\lvar(c)$,
which are statically collected.
The query annotation is 
a set of pairs $\traceF(c) \in \mathcal{P}(\mathcal{LV} \times \{0, 1\} )$ 
mapping each $x^l \in \traceV(c)$ to $0$ or $1$, 
indicating whether this labeled variable is in program $c$'s query variable set $\qvar(c)$.
{
The weights is a set of pairs, $(x^l, w) \in \mathcal{LV} \times (\mathcal{T} \to \mathbb{N})$,
with a labeled variable as first component and
its weight $w$ the second component.
Weight $w$ for
% a labeled variable 
$x^l$ is a function $w : \mathcal{T} \to \mathbb{N}$
mapping from a starting trace to a natural number.
When program executes under this starting trace $\trace$,
$\config{{c}, \trace} \to^{*} \config{\eskip, \trace\tracecat\vtrace'} $, it generates an execution trace $\trace'$.
This natural number is the evaluation times of the labeled command corresponding to the vertex, 
computed by the counter operator $w(\trace) = \vcounter(\vtrace', l)$.
We can see in the execution-based dependency graph of $\kw{twoRounds}$ in 
Figure~\ref{fig:twoRounds_example}(b), the weight of vertices in the while loop is  $\env(\trace) k$, which depends on the value of the user input $k$ specified in the starting trace $\tau$.
The directed edges $\traceE({c})$ is also a set of pairs with two labeled variables $ (x^i, y^j) \in \mathcal{LV} \times \mathcal{LV}$, from $x^i$ pointing to $y^j$ in the graph.
The edges are constructed directly from our variable may-dependency relation. 
For any two vertices $x^{i}$ and $y^{j}$ in $\traceV(c)$, if they satisfy the variable may-dependency relation $\vardep(x^i, y^j, c)$, there is a direct edge between the two vertices in our execution-based dependency graph for program $c$.
} 
In most data analysis programs $c$ we are interested, there are usually some user input variables, such as $k$ in $\kw{twoRounds}$. 
We denote $\mathcal{T}_0(c)$ as the set of initial traces in which all the input variables in $c$ are initialized, it is also reflected in $\traceW({c})$.    

\paragraph{Trace-based Adaptivity}

% \wq{
% Given 
% a program $c$'s execution-based dependency graph 
% % $G_{trace}(c)(\trace) = (\vertxs, \edges, \weights, \qflag)$,
% $\traceG({c}) = (\traceV({c}), \traceE({c}), \traceW({c}), \traceF({c}))$
% we define adaptivity 
% with respect to $\trace$ by the finite walk in the graph, which has the most query requests along the walk.
% }

Given 
a program $c$'s execution-based dependency graph 
% $G_{trace}(c)(\trace) = (\vertxs, \edges, \weights, \qflag)$,
$\traceG({c})$,
we define adaptivity 
with respect to an initial trace $\trace_0 \in \mathcal{T}_0(c)$ by the finite walk in the graph, which has the most query requests along the walk.
We show the definition of a finite walk as follows.
%
% The query length of a walk $k$ is the number of vertices which correspond to query variables in the vertices sequence of this walk. 
% Instead of counting all 
% the vertices in $k$'s vertices sequence, i

\begin{defn}[Finite Walk (k)].
  \label{def:finitewalk}
  \\
%   Given a program $c$'s execution-based dependency graph $\traceG({c})(\trace)$, 
%   a \emph{finite walk} $fw$ in $\traceG({c})(\trace)$ is a sequence of edges $(e_1 \ldots e_{n - 1})$ 
%   for which there is a sequence of vertices $(v_1, \ldots, v_{n})$ such that:
%   \begin{itemize}
%       \item $e_i = (v_{i},v_{i + 1})$ for every $1 \leq i < n$.
%       \item every vertex $v \in \traceV({c}) $ appears in $(v_1, \ldots, v_{n})$ at most 
%       \wq{$\traceW({c})(\trace)$} times.  
%   \end{itemize}
%   %
%   The length of $fw$ is the number of vertices in its vertex sequence, i.e., $\len(k) = n$.
  Given the execution-based dependency graph $\traceG({c}) = (\traceV({c}), \traceE({c}), \traceW({c}), \traceF({c}))$ of a program $c$,
  a \emph{finite walk} $k$ in $\traceG({c})$ is a 
  function $k: \mathcal{T} \to $ sequence of edges.
  For a initial trace $\trace_0 \in \mathcal{T}_0(c)$, 
  $k(\trace_0)$ is a sequence of edges $(e_1 \ldots e_{n - 1})$ 
  for which there is a sequence of vertices 
  $(v_1, \ldots, v_{n})$ such that:
  \begin{itemize}
      \item $e_i = (v_{i},v_{i + 1}) \in \traceE(c)$ for every $1 \leq i < n$.
      \item every $v_i \in \traceV(c)$
      and $(v_i, w_i) \in \traceW(c)$, 
       $v_i$ appears in $(v_1, \ldots, v_{n})$ at most 
    %   \wq{$\traceW({c})(\trace)$} 
    $w(\trace_0)$
      times.  
  \end{itemize}
  %
  The length of $k(\trace_0)$ is the number of vertices in its vertices sequence, i.e., $\len(k)(\trace_0) = n$.
 \end{defn}

We use $\walks(\traceG(c))$ to denote 
% \mg{``the set'', not ``a set''}a set containing all finite walks $k$ in $G$;
the set containing all finite walks $k$ in $\traceG(c)$;
and $k_{v_1 \to v_2} \in \walks(\traceG(c))$ with $v_1, v_2 \in \traceV(c)$ denotes the walk from vertex $v_1$ to $v_2$ . 
\\
We are interested in queries, so we need to recover the 
variables corresponding to queries from the walk. We define the query length of a walk, 
instead of counting all 
the vertices in $k$'s vertices sequence, we just count the number of vertices which correspond to query variables in this sequence.
%
% \mg{I don't understand this definition. Is wrt a single query?if yes, who is chosing the query? Or is it any query?}
% \jl{It is for any query, as long as the vertex is a query variable, in another worlds, this length just counting the number of query variables in the walk, instead of counting all 
% the vertices.}
% \todo{Make the definition clear}
\begin{defn}[Query Length of the Finite Walk($\qlen$)].
\label{def:qlen}
\\
% Given 
% % labelled weighted graph $G = (\vertxs, \edges, \weights, \qflag)$, 
% a program $c$'s execution-based dependency graph $\traceG(c)(\trace)$
%  and a \emph{finite walk} $k$ in $\traceG(c)(\trace)$ with its vertex sequence $(v_1, \ldots, v_{n})$, 
% %  the length of $k$ w.r.t query is defined as:
% The query length of $k$ is the number of vertices which correspond to query variables in $(v_1, \ldots, v_{n})$ as follows, 
% \[
%   \qlen(k) = \len\big( v \mid v \in (v_1, \ldots, v_{n}) \land \qflag(v) = 1 \big)
% \]
% , where $\big(v \mid v \in (v_1, \ldots, v_{n}) \land \qflag(v) = 1 \big)$ is a subsequence of $(v_1, \ldots, v_{n})$.
Given 
% labelled weighted graph $G = (\vertxs, \edges, \weights, \qflag)$, 
the execution-based dependency graph $\traceG({c}) = (\traceV({c}), \traceE({c}), \traceW({c}), \traceF({c}))$ of a program $c$,
 and a \emph{finite walk} 
%  $k$ in $\traceG(c)(\trace)$
 $k \in \walks(\traceG(c))$. 
%  with its vertex sequence $(v_1, \ldots, v_{n})$, 
%  the length of $k$ w.r.t query is defined as:
The query length of $k$ is a function $\qlen(k): \mathcal{T} \to \mathbb{N}$, such that with an initial trace  $\trace_0 \in \mathcal{T}_0(c)$, $\qlen(k)(\trace_0)$ is
the number of vertices which correspond to query variables in the vertices sequence of the walk $k(\trace_0)$
$(v_1, \ldots, v_{n})$ as follows, 
\[
  \qlen(k)(\trace_0) = |\big( v \mid v \in (v_1, \ldots, v_{n}) \land \qflag(v) = 1 \big)|.
\]
\end{defn}
The definition of adaptivity is then presented in Def~\ref{def:trace_adapt} below.

\begin{defn}
  [Adaptivity of a Program].
  \label{def:trace_adapt}
  \\
  Given a program ${c}$, 
  its adaptivity $A(c)$ is function 
  $A(c) : \mathcal{T} \to \mathbb{N}$ such that for an
  % with respect to a starting trace $\trace$ 
  initial trace $\trace_0 \in \mathcal{T}_0(c)$, 
  % is defined as follows:
  %
 $$
  A(c)(\trace_0) = \max \big 
  \{ \qlen(k)(\trace_0) \mid k \in \walks(\traceG(c)) \big \} $$
  \end{defn}