\chapter{The Program Dynamic Analysis for Adaptivity}
\label{ch:adapt-dynamic}
In this chapter, we formally introduce the language we will focus on for writing data analyses.  
This is a simple loop language with some primitives for calling queries. 
After defining the syntax of the language and showing an example, we will define its trace-based operational semantics. This is the main technical ingredient we will use to define the program's adaptivity.
% We will conclude this chapter by discussing the limitation of this language with respect to static analysis for adaptivity.

\section{Syntax of Query While Language}
\label{sec:dynamic-syntax}

\section{Trace-based Operational Semantics}
\label{sec:dynamic-os}
%
\section{Program Adaptivity}
\label{sec:dynamic-dynamic}
%
\section{Examples}
\label{sec:dynamic-examples}
%
\section{Implementation}
\label{sec:dynamic-implementation}
%
\section{Related Work}
\label{sec:dynamic-relatedwork}
In terms of techniques, our work relies on ideas from both static analysis and dynamic analysis. 
We discuss closely related work in both areas.

