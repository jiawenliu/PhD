In this section, I present my 
Improved full-spectrum adaptivity analysis 
following the same architecture as Figure~\ref{fig:structure}.
The extended language is presented first in Section~\ref{sec:refine-exe-language},
following with the first significant improvement
through the execution-based analysis in Section~\ref{sec:refine-exe}.
The other significant improvement w.r.t. the 
static adaptivity analysis is presented in Section~\ref{sec:refine-static}.
%
\subsection{Language Extension}
\label{sec:refine-exe-language}
In this chapter, we formally introduce the language we will focus on for writing data analyses.  
This is a simple loop language with some primitives for calling queries. 
After defining the syntax of the language and showing an example, we will define its trace-based operational semantics. 
This is the main technical ingredient we will use to define the program's adaptivity.

%
%
\paragraph{Labeled Language}
% \mg{It is ok to list all the operations in the appendix but for the main paper it is better to save space.}
\[
\begin{array}{llll}
\mbox{Arithmetic Operators} 
& \oplus_a & ::= & + ~|~ - ~|~ \times 
%
~|~ \div ~|~ \max ~|~ \min\\  
% ~|~ \div \\  
\mbox{Boolean Operators} 
& \oplus_b & ::= & \lor ~|~ \land
\\
%
\mbox{Relational Operators} 
& \sim & ::= & < ~|~ \leq ~|~ == 
\\  
%
\mbox{Arithmetic Expression} 
& \aexpr & ::= & 
n ~|~ {x} ~|~ \aexpr \oplus_a \aexpr  
 ~|~ \elog \aexpr  ~|~ \esign \aexpr
\\
%
\mbox{Boolean Expression} & \bexpr & ::= & 
%
\etrue ~|~ \efalse  ~|~ \neg \bexpr
 ~|~ \bexpr \oplus_b \bexpr
%
~|~ \aexpr \sim \aexpr 
\\
%
\mbox{Expression} & \expr & ::= & v ~|~ \aexpr ~|~ \bexpr ~|~ [\expr, \dots, \expr]
\\  
%
\mbox{Value} 
& v & ::= & { n ~|~ \etrue ~|~ \efalse ~|~ [] ~|~ [v, \dots, v]}  
\\ 
&&&
\highlight
{
~|~ (r, x_1, \ldots, x_n) := c
}
\\
%
\mbox{Query Expression} 
& {\qexpr} & ::= 
& { \qval ~|~ \aexpr ~|~ \qexpr \oplus_a \qexpr ~|~ \chi[\aexpr]} 
\\
%
\mbox{Query Value} & \qval & ::= 
& {n ~|~ \chi[n] ~|~ \qval \oplus_a  \qval ~|~ n \oplus_a  \chi[n]
    ~|~ \chi[n] \oplus_a  n}
\\
% \\%
\mbox{Label} 
& l & ::= & (n \in \mathbb{N} \cup \{\lin, \lex\}) ~|~ (l, n)
\\ 
%
\mbox{Labeled Command} 
& {c} & ::= &  
\clabel{\assign{x}{\expr}}^l 
~|~ \clabel{\assign{x}{\query(\qexpr)}}^l
~|~  \clabel{\eskip}^l
~|~ \ewhile \clabel{\bexpr}^{l} \edo {c}
~|~ \eif(\clabel{\bexpr}^{l} , {c}, {c}) 
\\ 
&&&
\highlight
{
~|~ \clabel{\efun}^l: x(r, x_1, \ldots, x_n) := c
~|~ \clabel{\assign{x}{\ecall(x, e_1, \ldots, e_n)}}^l
}
~|~ {c};{c}  
\\ 
% \\
\mbox{Event} 
& \event & ::= & 
    ({x}, l, v, \bullet) ~|~ ({x}, l, v, \qval)  ~~~~~~~~~~~ \mbox{Assignment Event} \\
&&& ~|~(\bexpr, l, v, \bullet)   ~~~~~~~~~~~~~~~~~~~~~~~~~~~~~~~~~~ \mbox{Testing Event}
\\
% &&& \text{\mg{I think it would be better to use quadruples for events, where the}}\\
% &&& \text{\mg{first element is either a variable or a boolean expression and }}\\
% &&& \text{\mg{the last is either a query value or some default value $\bullet$}}\\
%
% \mbox{Trace} & \trace
% & ::= & \cdot | \trace \cdot \event | \trace \tracecat \trace 
% \\
%
% \mbox{Trace} & \trace
% & ::= & [] ~|~ \event:: \trace ~|~ \trace \tracecat \trace  \\
\mbox{Trace} & \trace
& ::= & [] ~|~ \trace :: \event\\
% &&& \text{\mg{I don't understand why you need both :: and ++ as constructors.}}\\
% &&& \text{\jl{Because append is to the left but we are adding element to the left in the OS}}\\
% &&& \text{\jl{I was too sticky to the convention, it is a good idea to append to the left and just use $::$}}
% %
% \mbox{Event Signature} & \sig
% & ::= & (x, l, n) | (x, l, n, \query) | (b, l, n)
% \\
% %
\end{array}
\]
% \todo{change trace notation into list, and update corresponding operator nations}
% \\
% \wqside{"$\cdot$" has two meanings? empty, delimit. Trace is list of event?}
We use following notations to represent the set of corresponding terms:
\[
\begin{array}{lll}
\mathcal{VAR} & : & \mbox{Set of Variables}  
\\ 
%
\mathcal{VAL} & : & \mbox{Set of Values} 
\\ 
%
\mathcal{QVAL} & : & \mbox{Set of Query Values} 
\\ 
%
\cdom & : & \mbox{Set of Commands} 
\\ 
%
\eventset  & : & \mbox{Set of Events}  
\\
%
\eventset^{\asn}  & : & \mbox{Set of Assignment Events}  
\\
%
\eventset^{\test}  & : & \mbox{Set of Testing Events}  
\\
%
\ldom  & : & \mbox{Set of Labels}  
\\
%%
\mathcal{VAL}  & : & \mbox{Set of Labeled Variables}  
\\
%%
\dbdom  & : & \mbox{{Set of Databases}} 
\\
%
{\mathcal{T}} & : & \mbox{Set of Traces}
\\
%
\mathcal{T}_0(c) & : & \mbox{Set of Initial Traces, where all the input variables of the program $c$ are initialized.
}
\\
%
% \qdom = {[-1,1]} & : & \mbox{{Domain of Query Results}}\\
\qdom & : & \mbox{{Domain of Query Results}}\\
% &&\text{\mg{I don't think you need to hard code [-1,1] here}}\\
\end{array}
\]
%
%
%
Environment $ \env : {\mathcal{T}}  \to \mathcal{VAR} \to \mathcal{VAL} \cup \{\bot\}$
% \mgside{The following definition is missing one case, also it is better to say that $y\neq x$.}
% \[
% \begin{array}{lll}
% \env(\trace  \tracecat [(x, l, v, \cdot)]) x \triangleq v
% &
% \env(\trace \tracecat [(y, l, v, \cdot)]) x \triangleq \env(\trace) x, y \neq x
% &
% \env(\trace \tracecat [(b, l, v, \cdot)]) x \triangleq \env(\trace) x
% \\
% \env(\trace \tracecat [(x, l, v, \qval)]) x \triangleq v
% &
% \env(\trace \tracecat [(y, l, v, \qval)]) x \triangleq \env(\trace) x, y \neq x
% &
% \env({[]} ) x \triangleq \bot
% \end{array}
% \]
\[
\begin{array}{lll}
\env(\trace  \traceadd (x, l, v, \bullet)) x \triangleq v
&
\env(\trace \traceadd (y, l, v, \bullet)) x \triangleq \env(\trace) x, y \neq x
&
\env(\trace \traceadd (b, l, v, \bullet)) x \triangleq \env(\trace) x
\\
\env(\trace \traceadd (x, l, v, \qval)) x \triangleq v
&
\env(\trace \traceadd (y, l, v, \qval)) x \triangleq \env(\trace) x, y \neq x
&
\env({[]} ) x \triangleq \bot
\end{array}
\]
%
%
% \subsection{Trace-based Operational Semantics for Language \mg{What is ``Language''?}}
\paragraph{{Trace-based Operational Semantics for {\tt Query While} Language}}
{
\begin{mathpar}
\boxed{ \config{\trace,\aexpr} \aarrow v \, : \, \mbox{Trace  $\times$ Arithmetic Expr $\Rightarrow$ Arithmetic Value} }
\\
% \text{\mg{Missing. Without these rules it is difficult to understand why we need a trace to evaluate expressions.}}
% \\
\inferrule{ 
  \empty
}{
 \config{\trace,  n} 
 \aarrow n
}
\and
\inferrule{ 
  \env(\trace) x = v
}{
 \config{\trace,  x} 
 \aarrow v
}
\and
\inferrule{ 
  \config{\trace, \aexpr_1} \aarrow v_1
  \and 
  \config{\trace, \aexpr_2} \aarrow v_2
  \and 
   v_1 \oplus_a v_2 = v
}{
 \config{\trace,  \aexpr_1 \oplus_a \aexpr_2} 
 \aarrow v
}
\and
\inferrule{ 
  \config{\trace, \aexpr} \aarrow v'
  \and 
  \elog v' = v
}{
 \config{\trace,  \elog \aexpr} 
 \aarrow v
}
\and
\inferrule{ 
  \config{\trace, \aexpr} \aarrow v'
  \and 
  \esign v' = v
}{
 \config{\trace,  \esign \aexpr} 
 \aarrow v
}
\\
\boxed{ \config{\trace, \bexpr} \barrow v \, : \, \mbox{Trace $\times$ Boolean Expr $\Rightarrow$ Boolean Value} }
\\% \\
\inferrule{ 
  \empty
}{
 \config{\trace,  \efalse} 
 \barrow \efalse
}
\and 
\inferrule{ 
  \empty
}{
 \config{\trace,  \etrue} 
 \barrow \etrue
}
\and 
\inferrule{ 
  \config{\trace, \bexpr} \barrow v'
  \and 
  \neg v' = v
}{
 \config{\trace,  \neg \bexpr} 
 \barrow v
}
\and 
\inferrule{ 
  \config{\trace, \bexpr_1} \barrow v_1
  \and 
  \config{\trace, \bexpr_2} \barrow v_2
  \and 
   v_1 \oplus_b v_2 = v
}{
 \config{\trace,  \bexpr_1 \oplus_b \bexpr_2} 
 \barrow v
}
\and 
\inferrule{ 
  \config{\trace, \aexpr_1} \aarrow v_1
  \and 
  \config{\trace, \aexpr_2} \aarrow v_2
  \and 
   v_1 \sim v_2 = v
}{
 \config{\trace,  \aexpr_1 \sim \aexpr_2} 
 \barrow v
}
\\
\boxed{ \config{\trace, \expr} \earrow v \, : \, \mbox{Trace $\times$ Expression $\Rightarrow$ Value} }
\\
\inferrule{ 
  \config{\trace, \aexpr} \aarrow v
}{
 \config{\trace,  \aexpr} 
 \earrow v
}
\and
\inferrule{ 
  \config{\trace, \bexpr} \barrow v
}{
 \config{\trace,  \bexpr} 
 \earrow v
}
\and
\inferrule{ 
  \config{\trace, \expr_1} \earrow v_1
  \cdots
  \config{\trace, \expr_n} \earrow v_n
}{
 \config{\trace,  [\expr_1, \cdots, \expr_n]} 
 \earrow [v_1, \cdots, v_n]
}
\and
\inferrule{ 
  \empty
}{
 \config{\trace,  v} 
 \earrow v
}
\\
\boxed{ \config{\trace, \qexpr} \qarrow \qval \, : \, \mbox{Trace  $\times$ Query Expr $\Rightarrow$ Query Value} }
\\
\inferrule{ 
  \config{\trace, \aexpr} \aarrow n
}{
 \config{\trace,  \aexpr} 
 \qarrow n
}
\and
\inferrule{ 
  \config{\trace, \qexpr_1} \qarrow \qval_1
  \and
  \config{\trace, \qexpr_2} \qarrow \qval_2
}{
 \config{\trace,  \qexpr_1 \oplus_a \qexpr_2} 
 \qarrow \qval_1 \oplus_a \qval_2
}
\and
\inferrule{ 
  \config{\trace, \aexpr} \aarrow n
}{
 \config{\trace, \chi[\aexpr]} \qarrow \chi[n]
}
\and
\inferrule{ 
  \empty
}{
 \config{\trace,  \qval} 
 \qarrow \qval
}
 \end{mathpar}
%
The trace based operational semantics rules are defined in Figure \ref{fig:os}.
%
\begin{figure}
%   \text{\mg{Several skip are missing labels. Do we need fresh labels or we reuse l?}}
%   \\
%   \text{\jl{Both are good for OS, but generate fresh label will need extra arguments in soundness proof, so rescuing l is better}}
%   \\
% \text{\mg{Also, why we use ++, cannot we just define lists as adding elements on the right?}}  \\
% \text{\jl{I was too sticky to the convention, it is a good idea to append to the left and just use $::$ as construtor}}  \\
% \text{\mg{It is also unclear why we store the boolean expression in if and while, besides the boolean value.}}\\
% \text{\jl{When proving the soundness of dependency between trace-based and program-based,}}\\
% \text{\jl{The variable used in the boolean expression is useful in proving the inversion Lemmas.}}
{
\begin{mathpar}
\boxed{
\mbox{Command $\times$ Trace}
\xrightarrow{}
\mbox{Command $\times$ Trace}
}
\and
\boxed{\config{{c, \trace}}
\xrightarrow{} 
\config{{c',  \trace'}}
}
\\
\inferrule
{
\empty
}
{
\config{\clabel{\eskip}^l,  \trace } 
\xrightarrow{} 
\config{\clabel{\eskip}^l, \trace}
}
~\textbf{skip}
%
\and
%
\inferrule
{
\event = ({x}, l, v, \bullet)
}
{
\config{[\assign{{x}}{\aexpr}]^{l},  \trace } 
\xrightarrow{} 
\config{\clabel{\eskip}^l, \trace \traceadd \event}
}
~\textbf{assn}
%
\and
%
{
\inferrule
{
 \config{\trace, \qexpr }\qarrow \qval
 \and 
\query(\qval) = v
\and 
\event = ({x}, l, v, \qval)
}
{
\config{{[\assign{x}{\query(\qexpr)}]^l, \trace}}
\xrightarrow{} 
\config{{\clabel{\eskip}^l,  \trace \traceadd \event} }
}
~\textbf{query}
}
%
\and
%
\inferrule
{
  \config{\trace, b} \barrow \etrue
 \and 
 \event = (b, l, \etrue, \bullet)
}
{
\config{{\ewhile [b]^{l} \edo c, \trace}}
\xrightarrow{} 
\config{{
c; \ewhile [b]^{l} \edo c,
\trace \traceadd \event}}
}
~\textbf{while-t}
%
%
\and
%
\inferrule
{
  \config{\trace, b} \barrow \efalse
 \and 
 \event = (b, l, \efalse, \bullet)
}
{
\config{{\ewhile [b]^{l}, \edo c, \trace}}
\xrightarrow{} 
\config{{
  \clabel{\eskip}^l,
\trace \traceadd \event}}
}
~\textbf{while-f}
%
%
\and
%
%
\inferrule
{
\config{{c_1, \trace}}
\xrightarrow{}
\config{{c_1',  \trace'}}
}
{
\config{{c_1; c_2, \trace}} 
\xrightarrow{} 
\config{{c_1'; c_2, \trace'}}
}
~\textbf{seq1}
%
\and
%
\inferrule
{
  \config{{c_2, \trace}}
  \xrightarrow{}
  \config{{c_2',  \trace'}}
}
{
\config{{\clabel{\eskip}^l; c_2, \trace}} \xrightarrow{} \config{{ c_2', \trace'}}
}
~\textbf{seq2}
%
\and
%
%
\inferrule
{
  \config{\trace, b} \barrow \etrue
 \and 
 \event = (b, l, \etrue, \bullet)
}
{
 \config{{
\eif([b]^{l}, c_1, c_2), 
\trace}}
\xrightarrow{} 
\config{{c_1, \trace \traceadd \event}}
}
~\textbf{if-t}
%
\and
%
\inferrule
{
 \config{\trace, b} \barrow \efalse
 \and 
 \event = (b, l, \efalse, \bullet)
}
{
\config{{\eif([b]^{l}, c_1, c_2), \trace}}
\xrightarrow{} 
\config{{c_2, \trace \traceadd \event}}
}
~\textbf{if-f}
% %
\and
%
\highlight{
\inferrule
{
 c' = (c)^{+n}
 \and 
 \event = (f, l, (r, x_1, \ldots, x_n) := c', \bullet)
}
{
\config{{
  [\efun]^l: f(r, x_1, \ldots, x_n) := c, \trace}}
\xrightarrow{} 
\config{{\clabel{\eskip}^l, \trace \traceadd \event}}
}
~\textbf{fun-def}
%
}
\\
\highlight{
%
\inferrule
{
  \config{ \trace, f} \earrow (r, x_1, \ldots, x_n) := c
\and 
\config{{
  \clabel{\assign{x_1}{e_1}}^{(l, 1)}; \ldots;
  \clabel{\assign{x_n}{e_n}}^{(l, n)}, \trace}} 
  \xrightarrow{}^* 
  \config{{\clabel{\eskip}^{(l, n)}, \trace_1}}
  \\ 
  \config{{\clabel{c}^{(l)}, \trace_1}}
  \xrightarrow{}^* 
  \config{{\clabel{\eskip}^{l}, \trace'}}
  \and
  \config{\trace', r } \earrow v
  \and
 \event = (x, l, v, \bullet)
}
{
\config{{
  \clabel{\assign{x}{\ecall(f, e_1, \ldots, e_n)}}^l, \trace}}
\xrightarrow{} 
\config{{\clabel{\eskip}^l, \trace' :: \event}}
}
~\textbf{fun-call}
}
%
\end{mathpar}
}
% \end{subfigure}
    \caption{Trace-based Operational Semantics for Language.}
    \label{fig:os}
\end{figure}
%
\\
\highlight{
  \begin{defn}[Label Increase]
    \label{def:label_inc}  
    Label Increase $ + : {\ldom \to \mathbb{N} \to \ldom}$, increase a label $l$ by a natural number $n$:
\[
    n + n' \triangleq n'' ~ n, n' \in \mathbb{N} \land \config{[], n + n'} \aarrow n''
   \qquad (l, n) + n' \triangleq (l + n', n'') ~ n, n' \in \mathbb{N} \land \config{[], n + n'} \aarrow n''
   \]
\end{defn}
The case of $(l, n) + n'$ will never happen during evaluation.
By Operational semantics, the only place the label increase is in rule \textbf{fun-def},
$c' = (c)^{+n}$, where $c$ is the function body.
By the rule \textbf{fun-call}, and the label augment in Definition~\ref{def:comlabel_aug}, the function body $c$ will never be augmented.
%
\begin{defn}[Command Label Increase] 
  \label{def:comlabel_inc}
Command Label Increase $ {(\cdot)}{}^{+n} : {\cdom \to \cdom}$, increase the label in command by $n$.
\[
\begin{array}{ll}
  (\clabel{\assign{x}{\expr}}^l){}^{+n} & \triangleq \clabel{\assign{x}{\expr}}^{l + n}\\
(\clabel{\assign{x}{\query(\qexpr)}}^l)^{+n} & \triangleq \clabel{\assign{x}{\query(\qexpr)}}^{l + n}\\
(\clabel{\eskip}^l)^{+n} & \triangleq \clabel{\eskip}^{l + n}\\
(\ewhile \clabel{\bexpr}^{l} \edo {c'})^{+n} & \triangleq \ewhile \clabel{\bexpr}^{l+n} \edo {(c')^{+n}}\\
(\eif(\clabel{\bexpr}^{l} , {c_1}, {c_2}))^{+n}  & \triangleq \eif(\clabel{\bexpr}^{l+n} , {(c_1)^{+n}}, {(c_2)^{+n}})\\
% (\clabel{\efun}^l: x(r^l, x_1, \ldots, x_n) := c)^{+n} & \triangleq \clabel{\efun}^{l + n}: x(r^l, x_1, \ldots, x_n) := (c)^{+n} \\
(\clabel{\efun}^l: x(r^l, x_1, \ldots, x_n) := c)^{+n} & \triangleq \clabel{\efun}^{l + n}: x(r^l, x_1, \ldots, x_n) := c \\
(\clabel{\assign{x}{\ecall(x, e_1, \ldots, e_n)}}^l)^{+n} & \triangleq \clabel{\assign{x}{\ecall(x, e_1, \ldots, e_n)}}^{l + n}\\
({c_1};{c_2})^{+n} &  \triangleq {(c_1)}^{+n};{(c_2)}^{+n}
\end{array}
\]
\end{defn}
%
\begin{defn}[Command Label Augment] 
  \label{def:comlabel_aug}
  Command Label Augment $ \clabel{\cdot}^{l} : {\cdom \to \cdom}$, augment the label in command with a label $l$ 
in order to record the calling site.
\[
\begin{array}{ll}
  \clabel{\clabel{\assign{x}{\expr}}^{l'}}{}^{l} & \triangleq \clabel{\assign{x}{\expr}}^{(l, l')}\\
  \clabel{\clabel{\assign{x}{\query(\qexpr)}}^{l'}}^{l} & \triangleq \clabel{\assign{x}{\query(\qexpr)}}^{(l, l')}\\
  \clabel{\clabel{\eskip}^{l'}}^{l} & \triangleq \clabel{\eskip}^{(l, l')}\\
  \clabel{\ewhile \clabel{\bexpr}^{l'} \edo {c'}}^{l} & \triangleq \ewhile \clabel{\bexpr}^{(l, l')} \edo {(c')^{l}}\\
  \clabel{\eif(\clabel{\bexpr}^{l'} , {c_1}, {c_2})}^{l}  & \triangleq \eif(\clabel{\bexpr}^{(l, l')} , {(c_1)^{l}}, {(c_2)^{l}})\\
  \clabel{\clabel{\efun}^{l'}: x(r^l, x_1, \ldots, x_n) := c}^{l} & \triangleq \clabel{\efun}^{(l, l')}: x(r^l, x_1, \ldots, x_n) := c \\
  \clabel{\clabel{\assign{x}{\ecall(x, e_1, \ldots, e_n)}}^{l'}}^{l} & \triangleq \clabel{\assign{x}{\ecall(x, e_1, \ldots, e_n)}}^{(l, l')}\\
  \clabel{{c_1};{c_2}}^{l} &  \triangleq \clabel{c_1}^{l};\clabel{c_2}^{l}
\end{array}
\]
\end{defn}
}
% Each command is now labeled with a label $l$, either a natural number standing for the line of code where the command appears, or a symbol of $in$ or $ex$ used for annotating the input variables, and the exist point of the program. Notice that we associate the label $l$ to the conditional predicate $\bexpr$ in the if statement, and to the while guard counter $\bexpr$ in the $\ewhile$ statement.
% We abuse the same notation $c$ for labeled command in the rest of the paper.
% \\
% \todo{notation}
The labeled variables and assigned variables are set of variables annotated by a label. 
We use  
%$\mathcal{LVAR} = \mathcal{VAR} \times \mathcal{L} $ 
$\mathcal{LV}$ represents the universe of all the labeled variables and 
$\avar_c \in \mathcal{P}(\mathcal{VAR} \times \mathbb{N}) \subset \mathcal{LV}$ and 
$\lvar_c \in \mathcal{P}(\mathcal{VAR} \times \mathcal{L}) \subseteq \mathcal{LV}$,
represents the the set of assigned variables and labeled variables for a labeled command $c$,
defined in Definition~\ref{def:lvar} and \ref{def:avar}.
%
\\
$FV: \expr \to \mathcal{P}(\mathcal{VAR})$, computes the set of free variables in an expression. To be precise,
$FV(\aexpr)$, $FV(\bexpr)$ and $FV(\qexpr)$ represent the set of free variables in arithmetic
expression $\aexpr$, boolean expression $\bexpr$ and query expression $\qexpr$ respectively.
Labeled variables in $c$ is the set of assigned variables and all the free variables
showing up in $c$ with a default label $in$. 
The free variables
showing up in $c$, which aren't defined before be used, are actually the input variables of this program.
%
\begin{defn}[Assigned Variables (
% $\avar_{c} \subseteq \mathcal{VAR} \times \mathbb{N}$ or 
$\avar : \cdom \to \mathcal{P}(\mathcal{VAR} \times \mathbb{N})$)]
% labelled Variables 
% (
% % $\lvar_{c} \subseteq \mathcal{VAR} \times \mathbb{N}$ or 
% $\lvar : \cdom \to \mathcal{P}(\mathcal{VAR} \times \mathcal{L})$
\label{def:avar}
{\footnotesize
$$ \avar_{c} \triangleq
  \left\{
  \begin{array}{ll}
      \{{x}^l\}                   
      & {c} = [{\assign x e}]^{l} 
      \\
      \{{x}^l\}                   
      & {c} = [{\assign x \query(\qexpr)}]^{l} 
      \\
      \avar_{{c_1}} \cup \avar_{{c_2}}  
      & {c} = {c_1};{c_2}
      \\
      \avar_{{c}} \cup \avar_{{c_2}} 
      & {c} =\eif([\bexpr]^{l}, c_1, c_2) 
      \\
      \avar_{{c}'}
      & {c}   = \ewhile ([\bexpr]^{l}, {c}')
\end{array}
\right.
$$
}
\end{defn}
%

\begin{defn}[labelled Variables 
(
% $\lvar_{c} \subseteq \mathcal{VAR} \times \mathbb{N}$ or 
$\lvar : \cdom \to \mathcal{P}(\mathcal{LV})$]
\label{def:lvar}
{\footnotesize
$$
  \lvar_{c} \triangleq
  \left\{
  \begin{array}{ll}
      \{{x}^l\} \cup FV(\expr)^{in}                  
      & {c} = [{\assign x e}]^{l} 
      \\
      \{{x}^l\}   \cup FV(\qexpr)^{in}                
      & {c} = [{\assign x \query(\qexpr)}]^{l} 
      \\
      \lvar_{{c_1}} \cup \lvar_{{c_2}}  
      & {c} = {c_1};{c_2}
      \\
      \lvar_{{c}} \cup \lvar_{{c_2}} \cup FV(\bexpr)^{in}
      & {c} =\eif([\bexpr]^{l}, c_1, c_2) 
      \\
      \lvar_{{c}'} \cup FV(\bexpr)^{in}
      & {c}   = \ewhile ([\bexpr]^{l}, {c}')
\end{array}
\right.
$$
}
\end{defn}
%
%
%
% is a subset of the program's assigned variables, where every variable in this set is assigned by a query in the program.
% \mg{The set of query variables of a program is the set of variables set to the result of a query in the program.}\\
We also defined the set of query variables for a program $c$,
it is the set of variables set to the result of a query in the program formally in Definition~\ref{def:qvar}.
% \mg{In the next definition, why do you call it a vector? It seems that you define it as a set.}\\
% \jl{fixed}\\
%
% \begin{defn}[Query Variables ($\qvar_{c} \subseteq \mathcal{VAR} \times \mathbb{N}$)].
  % \\
\begin{defn}[Query Variables ($\qvar: \cdom \to \mathcal{P}(\mathcal{LV})$)] 
  \label{def:qvar}
Given a program $c$, its query variables 
% \mg{it seems you are missing the $_c$ subscript. Also, this is a minor point but I don't think it is a good idea to use a subscript, cannot you just use $\qvar(c)$.}
$\qvar(c)$ is the set of variables set to the result of a query in the program.
% \jl{fixed}
It is defined as follows:
{\footnotesize
$$
  % \qvar_{{c}} \triangleq
  \qvar(c) \triangleq
  \left\{
  \begin{array}{ll}
      \{\}                  
      & {c} = [{\assign x \expr}]^{l} 
      \\
      \{{x}^l\}                  
      & {c} = [{\assign x \query(\qexpr)}]^{l} 
      \\
      \qvar(c_1) \cup \qvar(c_2)  
      & {c} = {c_1};{c_2}
      \\
      \qvar(c_1) \cup \qvar(c_2) 
      & {c} =\eif([\bexpr]^{l}, c_1, c_2) 
      \\
      \qvar(c')
      & {c}   = \ewhile ([\bexpr]^{l}, {c}')
\end{array}
\right.
$$
}
\end{defn}
%
It is easy to see that a program $c$'s query variables is a subset of 
its labeled variables, $\qvar(c) \subseteq \lvar(c)$.
%
% \mg{In this definition as well as in others, I have the impression that you assume that the labelled variables are unique in the program. For example, it would not make sense to assign a query to the same labelled variable over and over. If this is the case, we need to make this very explicit in the paper.}
% \jl{TODO}
%
Every labeled variable in a program is unique, formally as follows with proof in Appendix~\ref{apdx:lemma_sec123}.
\begin{lem}[Uniqueness of the Labeled Variables]
  \label{lem:lvar_unique}
  For every program $c \in \cdom$ and every two labeled variables such that
  $x^i, y^j \in \lvar(c)$, then $x^i \neq y^j$.
  \[
    \forall c \in \cdom, x^i, y^j \in \mathcal{L} \sthat x^i, y^j \in \lvar(c)\implies x^i \neq y^j.
    \]
\end{lem}
%
%
%
\clearpage


\subsection{Accurate Execution-Based Adaptivity Analysis}
\label{sec:refine-exe}
%
% The program's adaptivity in the formal model through the execution-based analysis,
% % which we define over the program's execution-based dependency graph from the dynamic 
% % analysis 
% in Definition~\ref{def:trace_adapt}, isn't precise enough w.r.t. the intuitive adaptivity rounds.
% It comes across an over-approximation 
% % on the program's
% %  intuitive adaptivity rounds.
% % It is 
% resulted from difference between its Dependency Depth analysis and the \emph{variable may-dependency} definition.
% It occurs when the weight is computed over the traces different from the traces used in 
% witness the \emph{variable may-dependency} relation.
% As shown in the Example~\ref{ex:multipleRoundSingle_example}.
% \begin{example}[Over-Defined Adaptivtiy Example]
    \label{ex:overdefined_adapt}
    The program's adaptivity in our formal model,
    % which we define over the program's execution-based dependency graph from the dynamic 
    % analysis 
    in Definition~\ref{def:trace_adapt} also
     comes across an over-approximation on the program's
     intuitive adaptivity rounds.
    It is resulted from difference between its weight calculation and the \emph{variable may-dependency} definition.
    It occurs when the weight is computed over the traces different from the traces used in 
    witness the \emph{variable may-dependency} relation.
    % control flow can be decided in a particular way in front of conditional branches, while the static analysis fails to witness. 
    
    % We use one example to show the over-approximated definition, 
    As the program in Figure~\ref{fig:overdefn_example}(a),
    % This example is the variant of the multiple rounds strategy, 
    % we call it a multiple rounds odd iteration algorithm.
    % This example is still 
    which is a variant of the multiple rounds strategy, 
    % we call it a multiple rounds single iteration algorithm, 
    named $\kw{multipleRoundSingle(k)}$ with input $k$.
    % as the input variable.
    In this algorithm, 
    at line 7 of every iteration, 
    a query $\query(\chi[y] + p)$ based on previous query results stored in $p$ and $y$ is asked by the analyst like in the multiple rounds strategy. 
    The difference is that only the query answers from the one single iterations ($j = k - 2 $) are 
    % used to $b$. 
    used in this query $\query(\chi[y] + p)$.
    Because the execution trace updates 
    %   $b$ using the query answers at odd iterations, so the answers from even iterations do not affect the queries at odd iterations. From the query-based dependency graph in Figure~\ref{fig:overappr_example}(b), we can see that there is no edge from queries at odd iterations (such as $q_1,q_3,q_5$) to queries at even iteration(such as $q_2,q_4$). The longest path is dashed with a length $3$.  However, {\THESYSTEM} fails to realize that odd iteration will always execute then branch and even iteration means else branch, so its dependency graph considers both branches for every iteration. In this sense, the dependency graph by {\THESYSTEM} is similar to the one in the multiple rounds strategy. We show the estimated graph in Figure~\ref{fig:overappr_example}(c). The estimated upper bound is then, $5$, instead of $3$. 
    $p$ using the constant $0$ for all the iterations where ($j \neq k - 2$) at line $10$ after the 
    query request at line $7$.
    In this way, all the query answers stored in $p$ will not be accessed in next query request at line $7$ in the iterations 
    where  ($j \neq k - 2$).
    Only query answer at one single iteration where ($j = k - 2 $) will be used in next query request
    $\query(\chi[y] + p)$ at line $7$.
    So the adaptivity for this example is $2$. 
    % so the answers from odd iterations do not affect the queries at even iterations. 
    % However, from the execution-based dependency graph in Figure~\ref{fig:overappr_example}(b), 
    However, our adaptivity model fails to realize that there is only dependency relation 
    between $p^7$ and $p^7$ in one single iteration, 
    not the others. 
    % there is no edge from queries at odd iterations (such as $q_1,q_3,q_5$) to queries at even iteration(such as $q_2,q_4$). The longest path is dashed with a length $3$.  
    As shown in the execution-based dependency graph in Figure~\ref{fig:overdefn_example}(b), 
    there is an edge from $p^7$ to itself representing the existence of \emph{Variable May-Dependency} from $p^7$ on itself,
    and the visiting times of labeled variable $p^7$ is 
    $w_k(\trace_0)$ with a initial trace $\trace_0$. 
    % will always execute then branch and even iteration means else branch, so 
    % % its dependency 
    % it considers both branches for every iteration. 
    % In this sense, the weight estimated for $y^6$ and $w^6$ are both 
    % $k$.
    As a result, the walk with the longest query length 
    is
    $p^7  \to \cdots \to p^7 \to y^4  \to z^1 $ with the vertex $p^7$ visited $w_k(\trace_0)$,
    as the dotted arrows. 
    The adaptivity 
    % the Program-Based Dependency graph from {\THESYSTEM} by finding 
    based on
    this walk
    % walk with the longest query length 
    is $2 + w(\trace_0)$, instead of $2$. 
    % %
    % T% estimated from the Program-Based Dependency graph from by finding the walk with the longest query length 
    % is $1 + 2 * k$, instead of $1 + K$.
    Though the $\THESYSTEM$ is able to give us $2 + k$,  as an accurate bound w.r.t this definition.
    %  we show the estimated graph in Figure~\ref{fig:overappr_example}(c). 
    
        {\small
    \begin{figure}
     \centering
    %}
    \quad
    \begin{subfigure}{.35\textwidth}
    \begin{centering}
    $
        \begin{array}{l}
            \kw{multipleRoundsSingle(k)}\\
               \clabel{ \assign{j}{0}}^{0} ; 
                \clabel{\assign{z}{\query(0)} }^{1} ;             
                \clabel{\assign{p}{0} }^{2} ; \\
                \eif(\clabel{ k = 0}^{3}, 
                \clabel{ \assign{y}{\query(z)}}^{4}, \clabel{\eskip}^5);\\
                \ewhile ~ \clabel{j \neq k}^{6} ~ \edo ~ \\
                \Big(
                 \clabel{\assign{p}{\query(\chi[y]+p)} }^{7}  ; 
                 \clabel{\assign{j}{j + 1}}^{8}\\
              \eif(\clabel{ j \neq k - 2}^{9}, 
              \clabel{ \assign{p}{0}}^{10} ,\clabel{\eskip}^{10})
         \Big);\\
            \end{array}
    $
    \caption{}
    \end{centering}
    \end{subfigure}
    \begin{subfigure}{.6\textwidth}
        \begin{centering}
        \begin{tikzpicture}[scale=\textwidth/28cm,samples=150]
    % Variables Initialization
    \draw[] (-5, 1) circle (0pt) node{{ $z^1: {}^{w_1}_{1}$}};
    \draw[] (-5, 7) circle (0pt) node{{$p^2: {}^{w_1}_{0}$}};
    \draw[] (-5, 4) circle (0pt) node{{ $y^4: {}^{w_1}_{1}$}};
    % Variables Inside the Loop
     \draw[] (0, 6) circle (0pt) node{{ $p^7: {}^{w_k}_{1}$}};
     \draw[] (0, 2) circle (0pt) node{{ $p^{10}: {}^{w_k}_{0}$}};
     % Counter Variables
     \draw[] (5, 6) circle (0pt) node {{$j^0: {}^{w_1}_{0}$}};
     \draw[] (5, 2) circle (0pt) node {{ $j^8: {}^{w_k}_{0}$}};
     %
     % Value Dependency Edges:
     \draw[ thick, -Straight Barb] (1.4, 1.6) arc (120:-200:1);
     \draw[ ultra thick, -Straight Barb, densely dotted,] (0.8, 7) arc (220:-100:1);
     \draw[ thick, -latex] (-1.5, 6)  to  [out=-130,in=130]  (-1.5, 2);
     % Value Dependency Edges on Initial Values:
     \draw[ ultra thick, -latex, densely dotted,] (-5, 3.5)  -- (-5, 1.5) ;
     \draw[ thick, -latex,] (-1.5, 6)  -- (-4, 7) ;
     \draw[  ultra thick, -latex, densely dotted,] (-1.5, 6)  -- (-4, 4.7) ;
     %
     % Value Dependency For Control Variables:
     \draw[ thick, -Straight Barb] (6.5, 2.5) arc (150:-150:1);
    %  \draw[ ultra thick, -latex, densely dotted,] (-0.5, 1.5)  to  [out=-250,in=250]  (-0.5, 7);
     % Control Dependency
     \draw[ thick, -latex] (5, 2.5)  -- (5, 5.5) ;
     \draw[ thick,-latex] (1.5, 6)  -- (3.5, 6) ;
     \draw[ thick,-latex] (1.5, 1.8)  -- (3.5, 6) ;
     \draw[ thick,-latex] (1.5, 6)  -- (3.5, 2) ;
    %  \draw[ thick,-latex] (1.5, 4)  -- (4, 6) ;
     \draw[ thick,-latex] (1.5, 1.8)  -- (3.5, 2) ; 
    \end{tikzpicture}
     \caption{}
        \end{centering}
        \end{subfigure}
    % \end{wrapfigure}
    % \end{equation*}
    % \vspace{-0.4cm}
     \caption{(a) The multi rounds single example
     (b) The execution-based dependency graph.}
    \label{fig:overdefn_example}
    % \vspace{-0.5cm}
    \end{figure}
        }
    \end{example}
%%
% \subsubsection{ Methodology}
% \label{sec:refine-exe-analysis}
% % In terms of techniques, our work relies on ideas from both static analysis and dynamic analysis. 
We discuss closely related work in both areas.


The program's adaptivity in the formal model through the execution-based analysis in Section~\ref{sec:dynamic}
% which we define over the program's execution-based dependency graph from the dynamic 
% analysis 
(specifically in Definition~\ref{def:trace_adapt}), isn't precise enough w.r.t. the intuitive adaptivity rounds.
It comes across as an over-approximation 
% on the program's
% intuitive adaptivity rounds.
as shown in the Example~\ref{ex:multipleRoundsSingle_example}.
% It is 
In this example, the formalized adaptivity by execution-based analysis in Section~\ref{sec:dynamic} 
is the initial value of the input variable $k$.
However, the intuitive \emph{adaptivity} is only $2$ given whatever initial trace.
This is resulted from the
%  difference 
disconnection between the 
% dependency Depth 
dependency quantity analysis in Section~\ref{sec:dynamic-reachability} and 
the data dependency analysis in \emph{variable may-dependency} definition in Section~\ref{sec:dynamic-datadep}.
It occurs when the 
% weight 
dependency quantity is computed over the traces different from the traces used in 
% witness 
analyzing the \emph{variable may-dependency} relation.
% As shown in the Example~\ref{ex:multipleRoundsSingle_example}.
\begin{example}[Over-Defined Adaptivtiy Example]
    \label{ex:overdefined_adapt}
    The program's adaptivity in our formal model,
    % which we define over the program's execution-based dependency graph from the dynamic 
    % analysis 
    in Definition~\ref{def:trace_adapt} also
     comes across an over-approximation on the program's
     intuitive adaptivity rounds.
    It is resulted from difference between its weight calculation and the \emph{variable may-dependency} definition.
    It occurs when the weight is computed over the traces different from the traces used in 
    witness the \emph{variable may-dependency} relation.
    % control flow can be decided in a particular way in front of conditional branches, while the static analysis fails to witness. 
    
    % We use one example to show the over-approximated definition, 
    As the program in Figure~\ref{fig:overdefn_example}(a),
    % This example is the variant of the multiple rounds strategy, 
    % we call it a multiple rounds odd iteration algorithm.
    % This example is still 
    which is a variant of the multiple rounds strategy, 
    % we call it a multiple rounds single iteration algorithm, 
    named $\kw{multipleRoundSingle(k)}$ with input $k$.
    % as the input variable.
    In this algorithm, 
    at line 7 of every iteration, 
    a query $\query(\chi[y] + p)$ based on previous query results stored in $p$ and $y$ is asked by the analyst like in the multiple rounds strategy. 
    The difference is that only the query answers from the one single iterations ($j = k - 2 $) are 
    % used to $b$. 
    used in this query $\query(\chi[y] + p)$.
    Because the execution trace updates 
    %   $b$ using the query answers at odd iterations, so the answers from even iterations do not affect the queries at odd iterations. From the query-based dependency graph in Figure~\ref{fig:overappr_example}(b), we can see that there is no edge from queries at odd iterations (such as $q_1,q_3,q_5$) to queries at even iteration(such as $q_2,q_4$). The longest path is dashed with a length $3$.  However, {\THESYSTEM} fails to realize that odd iteration will always execute then branch and even iteration means else branch, so its dependency graph considers both branches for every iteration. In this sense, the dependency graph by {\THESYSTEM} is similar to the one in the multiple rounds strategy. We show the estimated graph in Figure~\ref{fig:overappr_example}(c). The estimated upper bound is then, $5$, instead of $3$. 
    $p$ using the constant $0$ for all the iterations where ($j \neq k - 2$) at line $10$ after the 
    query request at line $7$.
    In this way, all the query answers stored in $p$ will not be accessed in next query request at line $7$ in the iterations 
    where  ($j \neq k - 2$).
    Only query answer at one single iteration where ($j = k - 2 $) will be used in next query request
    $\query(\chi[y] + p)$ at line $7$.
    So the adaptivity for this example is $2$. 
    % so the answers from odd iterations do not affect the queries at even iterations. 
    % However, from the execution-based dependency graph in Figure~\ref{fig:overappr_example}(b), 
    However, our adaptivity model fails to realize that there is only dependency relation 
    between $p^7$ and $p^7$ in one single iteration, 
    not the others. 
    % there is no edge from queries at odd iterations (such as $q_1,q_3,q_5$) to queries at even iteration(such as $q_2,q_4$). The longest path is dashed with a length $3$.  
    As shown in the execution-based dependency graph in Figure~\ref{fig:overdefn_example}(b), 
    there is an edge from $p^7$ to itself representing the existence of \emph{Variable May-Dependency} from $p^7$ on itself,
    and the visiting times of labeled variable $p^7$ is 
    $w_k(\trace_0)$ with a initial trace $\trace_0$. 
    % will always execute then branch and even iteration means else branch, so 
    % % its dependency 
    % it considers both branches for every iteration. 
    % In this sense, the weight estimated for $y^6$ and $w^6$ are both 
    % $k$.
    As a result, the walk with the longest query length 
    is
    $p^7  \to \cdots \to p^7 \to y^4  \to z^1 $ with the vertex $p^7$ visited $w_k(\trace_0)$,
    as the dotted arrows. 
    The adaptivity 
    % the Program-Based Dependency graph from {\THESYSTEM} by finding 
    based on
    this walk
    % walk with the longest query length 
    is $2 + w(\trace_0)$, instead of $2$. 
    % %
    % T% estimated from the Program-Based Dependency graph from by finding the walk with the longest query length 
    % is $1 + 2 * k$, instead of $1 + K$.
    Though the $\THESYSTEM$ is able to give us $2 + k$,  as an accurate bound w.r.t this definition.
    %  we show the estimated graph in Figure~\ref{fig:overappr_example}(c). 
    
        {\small
    \begin{figure}
     \centering
    %}
    \quad
    \begin{subfigure}{.35\textwidth}
    \begin{centering}
    $
        \begin{array}{l}
            \kw{multipleRoundsSingle(k)}\\
               \clabel{ \assign{j}{0}}^{0} ; 
                \clabel{\assign{z}{\query(0)} }^{1} ;             
                \clabel{\assign{p}{0} }^{2} ; \\
                \eif(\clabel{ k = 0}^{3}, 
                \clabel{ \assign{y}{\query(z)}}^{4}, \clabel{\eskip}^5);\\
                \ewhile ~ \clabel{j \neq k}^{6} ~ \edo ~ \\
                \Big(
                 \clabel{\assign{p}{\query(\chi[y]+p)} }^{7}  ; 
                 \clabel{\assign{j}{j + 1}}^{8}\\
              \eif(\clabel{ j \neq k - 2}^{9}, 
              \clabel{ \assign{p}{0}}^{10} ,\clabel{\eskip}^{10})
         \Big);\\
            \end{array}
    $
    \caption{}
    \end{centering}
    \end{subfigure}
    \begin{subfigure}{.6\textwidth}
        \begin{centering}
        \begin{tikzpicture}[scale=\textwidth/28cm,samples=150]
    % Variables Initialization
    \draw[] (-5, 1) circle (0pt) node{{ $z^1: {}^{w_1}_{1}$}};
    \draw[] (-5, 7) circle (0pt) node{{$p^2: {}^{w_1}_{0}$}};
    \draw[] (-5, 4) circle (0pt) node{{ $y^4: {}^{w_1}_{1}$}};
    % Variables Inside the Loop
     \draw[] (0, 6) circle (0pt) node{{ $p^7: {}^{w_k}_{1}$}};
     \draw[] (0, 2) circle (0pt) node{{ $p^{10}: {}^{w_k}_{0}$}};
     % Counter Variables
     \draw[] (5, 6) circle (0pt) node {{$j^0: {}^{w_1}_{0}$}};
     \draw[] (5, 2) circle (0pt) node {{ $j^8: {}^{w_k}_{0}$}};
     %
     % Value Dependency Edges:
     \draw[ thick, -Straight Barb] (1.4, 1.6) arc (120:-200:1);
     \draw[ ultra thick, -Straight Barb, densely dotted,] (0.8, 7) arc (220:-100:1);
     \draw[ thick, -latex] (-1.5, 6)  to  [out=-130,in=130]  (-1.5, 2);
     % Value Dependency Edges on Initial Values:
     \draw[ ultra thick, -latex, densely dotted,] (-5, 3.5)  -- (-5, 1.5) ;
     \draw[ thick, -latex,] (-1.5, 6)  -- (-4, 7) ;
     \draw[  ultra thick, -latex, densely dotted,] (-1.5, 6)  -- (-4, 4.7) ;
     %
     % Value Dependency For Control Variables:
     \draw[ thick, -Straight Barb] (6.5, 2.5) arc (150:-150:1);
    %  \draw[ ultra thick, -latex, densely dotted,] (-0.5, 1.5)  to  [out=-250,in=250]  (-0.5, 7);
     % Control Dependency
     \draw[ thick, -latex] (5, 2.5)  -- (5, 5.5) ;
     \draw[ thick,-latex] (1.5, 6)  -- (3.5, 6) ;
     \draw[ thick,-latex] (1.5, 1.8)  -- (3.5, 6) ;
     \draw[ thick,-latex] (1.5, 6)  -- (3.5, 2) ;
    %  \draw[ thick,-latex] (1.5, 4)  -- (4, 6) ;
     \draw[ thick,-latex] (1.5, 1.8)  -- (3.5, 2) ; 
    \end{tikzpicture}
     \caption{}
        \end{centering}
        \end{subfigure}
    % \end{wrapfigure}
    % \end{equation*}
    % \vspace{-0.4cm}
     \caption{(a) The multi rounds single example
     (b) The execution-based dependency graph.}
    \label{fig:overdefn_example}
    % \vspace{-0.5cm}
    \end{figure}
        }
    \end{example}

Based on the execution-based dependency analysis in Section~\ref{sec:dynamic}, the improvement on the accuracy 
of formalizing the
\emph{adaptivity} is planned to develop in following three-steps.
\begin{enumerate}
\item In the first stage of the execution-based analysis, 
I will give an alternative variable \emph{may-dependency} definition 
by referring to the analysis methodology in \cite{Cousot19a}.
%
% Specifically, I will define the variables' dependency relation over two witness traces and an initial trace. Comparing to 
% the existing \emph{may-dependency} definition, which quantifies overall possible execution traces, the alternative
% the definition explicitly relies on two specific witness traces from execution.
% This externalization helps in analyzing the dependency quantity through the same 
% witness traces as the \emph{may-dependency} relation. In this way, the over-approximation as illustrated above
% can be reduced.
%
\item In the second stage of the execution-based analysis, 
based on the new \emph{may-dependency} definition,
I will compute the weight of every edge constructed from 
\emph{may-dependency} relation in the execution-based dependency graph, w.r.t. to the witness traces.
%
\item Then in the third stage in Section~\ref{sec:refine-exe-adapt}, 
I formalize the \emph{adaptivity} as the 
length of the longest finite walk with newly definition. 
% Differently from the previous one, I restrict 
% the occurrence of every edge in a finite walk to no more than its weight as well.
\end{enumerate}

\paragraph{Accurate Execution-Based Data Dependency Analysis}
\label{sec:refine-exe-datadep}
In the first stage of the execution-based analysis, 
I will give an alternative variable \emph{may-dependency} definition 
by referring to the analysis methodology in \cite{Cousot19a}.

Specifically, I will define the variables dependency relation over two witness traces and an initial trace. Comparing to 
the existing \emph{may-dependency} definition, which quantifies overall possible execution traces, the alternative
definition explicitly relies on two specific witness traces from execution.

In order to define this two witness traces based \emph{may-dependency} definition, we 
need to observe the evaluation variation differently.
New definitions on value difference is necessary in order to observe the evaluation variation in a different way, formally as follows.
\highlight{
\begin{defn}[Value Sequence $\seq(\trace, x^l)$]
  \label{def:vseq}
  \[
\begin{array}{l}
  \seq(\trace :: (x, l, v, \bullet), x^l) \triangleq \seq(\trace)::v  \qquad
  \seq(\trace :: (x, l, v, \qval), x^l) \triangleq \seq(\trace):: \qval \qquad
  \seq([]) \triangleq []\\
  \seq(\trace :: (y, j, \_, \_), x^l) \triangleq \seq(\trace) \quad y \neq x \lor j \neq l 
\end{array}
\]
\end{defn}
%
\begin{defn}[Difference Sequence $\sdiff(\trace_1, \trace_2, x^l )$]
  \label{def:diffseq}
  Let $ s_1 = \seq(\trace_1, x^l) \land s_2 = \seq(\trace_2, x^l)$ be the value sequence of $x^l$ 
  on $\trace_1$ and $\trace_2$, and $s^l$ be the sequence with longer length and $s^t$ the 
  shorter one,
  then their difference sequence is defined as follows,
  \[
    \sdiff(\trace_1, \trace_2, x^l) \triangleq
    \begin{array}{l}
      \{ (s^t[k], s^l[k]) ~|~ 
      % \land 
      % \land 
      s^t[k] \neq s^l[k], k = 0, \ldots, len(s^t)
      \}
      \\
      \cup 
      \{ (\cdot, s^l[k]) ~|~ 
      \len(s^t) \leq \len(s^l)k = len(s^t), \ldots \len(s^l)
      \}
    \end{array}
    \]
\end{defn}
}
Then, alternative variable \emph{may-dependency} is formally defined as follows,
\begin{defn}[Variable May-Dependency]
    \label{def:improved_var_dep}
\highlight{
    A labeled variable $y^j \in \lvar(c)$ is in the \emph{may-dependency} relation with another
    labeled variable $x^i \in \lvar(c)$ in a program ${c}$, w.r.t. an initial trace $\trace_0 \in \mathcal{T}_0(c)$
    and two witness traces $\trace_1, \trace_2 \in \mathcal{T}$,
    denoted as 
    %
    $\dep(x^i, y^j, \trace_1, \trace_2, \trace_0, {c})$, if an only if
    \[
      \begin{array}{l}
    \exists 
    D \in \dbdom, 
    \trace_0' \in \mathcal{T} \sthat
    (\forall z \neq x \sthat   \env(\trace_0 ) z =   \env(\trace_0') z )
    \\ \quad \land 
     \config{{c}, \vtrace_0} \rightarrow^{*} 
      \config{\clabel{\eskip}^l, \vtrace_0  \tracecat \trace_1 } 
      \land 
      % \config{{c}_1, \vtrace_0' \tracecat [\event_1']}  \rightarrow^{*} 
      % \config{\clabel{\eskip}^l,  
      % \vtrace_0' \tracecat [\event_1] \tracecat \trace_2 }   
      \config{{c}, \vtrace_0'} \rightarrow^{*} 
      % \config{{c}_1, \vtrace_0' \tracecat [\event_1]}  \rightarrow^{*} 
        \config{\clabel{\eskip}^l, \vtrace_0'  \tracecat \trace_2} 
      \land 
        \sdiff(\trace_1, \trace_2, y^j ) \neq \emptyset
      \end{array}
    \]  
  }
    \end{defn}
This externalization helps in analyzing the dependency quantity through the same 
witness traces as the \emph{may-dependency} relation. In this way, the over-approximation as illustrated above
can be reduced.
% \paragraph*{Challenge}
In the data analysis model our programming framework supports, 
%  an \emph{analyst} asks a sequence of queries to the mechanism, and receives the answers to these queries from the mechanism. In this model, the adaptivity we are interested in is the length of the longest sequence of such adaptively chosen queries, among all the queries the data analyst asks. 
  we define that a query is adaptively chosen when it is affected by answers of previous queries. The next thing is to decide how do we define whether one query is "affected" by previous answers, with the limited information we have? As a reminder, 
 when the analyst asks a query, the only known information will be the answers to previous queries and the current execution trace of the program.


There are two possible situations that a query will be "affected",  
either when the query expression directly uses the results of previous queries (data dependency), or when the control flow of the program with respect to a query (whether to ask this query or not) depends on the results of previous queries (control flow dependency).
% As a first step, we give a definition of when one query may depend on a previous query, which is supposed to consider both control dependency and data dependency. We first look at two possible candidates:
% \begin{enumerate}
%     \item One query may depend on a previous query if and only if a change of the answer to the previous query may also change the result of the query.
%     \item One query may depend on a previous query if and only if a change of the answer to the previous query may also change the appearance of the query.
% \end{enumerate}


Since the the results of previous queries can be stored or used in variables
which aren't associated to the query request,
it is necessary to track the dependency between queries, through all the program's variables,  
and then we can distinguish variables which are assigned with query requests.
 We give a definition of when one variable \emph{may-depend} on a previous variable with two candidates.
{
\begin{enumerate}
    \item One variable may depend on a previous variable if and only if a change of the value assigned to the previous variable may also change the value assigned to the variable.
    \item One variable may depend on a previous variable if and only if a change of the value assigned to the previous variable may also change the appearance of the assignment command to this variable 
    % in\wq{during?} 
    during execution.
\end{enumerate}
}
%   The first candidate works well by witnessing the result of one query according to the change of the answer of another query. We can easily find that the two queries have nothing to do with each other in a simple example   

{   
% The first situations works well by witnessing the result assigned to variable 
% according to the change of the value assigned to another query. 
% We can easily find that the two queries have nothing to do with each other in a simple example 
% In the first one, by defining the dependency as
The first definition is defined as
% witnessing 
% the query expressions equivalence (or the value equality for non-query assignment )
the witness of a variation on the value assigned to the same variable through two executions,
% assigned to the same variable through two executions, 
according to the change of the value assigned to another variable in pre-trace.
% the situation of data-dependency works well. \wq{long sentence, make it short?}
In particular for query requests, the variation we observe is on the query value instead of on the query requesting results.
% We can find that two queries 
% % have nothing to do with each other in this simple example 
% % depends on each other\wq{not each other, one direction.} 
% satisfy this definition
In 
%this 
the simple program $c_1 =\assign{x}{\query(\chi[2])} ;\assign{y}{\query(\chi[3] + x)}$.
 %
 From our perspective, $\query(\chi[1])$ is different from $\query(\chi[2]))$. Informally, we think $\query(\chi[3] + x)$ may depend on the query $\query(\chi[2]))$, because equipped function of the former $\chi[3] + x$ may depend on the data stored in x assigned with the result of $\query(\chi[2]))$, according to this definition. }
%
% in this example: $c_1 = \assign{x}{\query(0)}; \assign{z}{\query(\chi[x])}$.
% This candidate definition works well 
Nevertheless, the first definition fails to catch control dependency because it just monitors the changes to a query, but misses the appearance of the query when the answers of its previous queries change. 
For instance, it fails to handle $
      c_2 = \assign{x}{\query(\chi[1])} ; \eif( x > 2 , \assign{y}{\query(\chi[2])}, \eskip )
   $, but the second definition can. However, it only considers the control dependency and misses the data dependency. This reminds us to define a \emph{may-dependency} relation between labeled variables by combining the two definitions to capture the two situations.
%
%
%
%
\paragraph{Dependency}
 To define the may dependency relation on two labeled variables, we rely on the limited information at hand - the trace generated by the operational semantics. In this end, we first define the \emph{may-dependency} between events, and use it as a foundation of the variable may-dependency relation.
\begin{defn}[Events Different up to Value ($\diff$)]
  Two events $\event_1, \event_2 \in \eventset$ are  \emph{Different up to Value}, 
  denoted as $\diff(\event_1, \event_2)$ if and only if:
  \[
    \begin{array}{l}
  \pi_1(\event_1) = \pi_1(\event_2) 
  \land  
  \pi_2(\event_1) = \pi_2(\event_2) \\
  \land  
  \big(
    (\pi_3(\event_1) \neq \pi_3(\event_2)
  \land 
  \pi_{4}(\event_1) = \pi_{4}(\event_2) = \bullet )
  % \qquad \qquad 
  \lor 
  (\pi_4(\event_1) \neq \bullet
  \land 
  \pi_4(\event_2) \neq \bullet
  \land 
  \pi_{4}(\event_1) \neq_q \pi_{4}(\event_2)) 
  \big)
  \end{array}
  \]
  \end{defn}
 %
 We compare two events by defining $\diff(\event_1, \event_2)$. We use $\qexpr_1 =_{q} \qexpr_2$ and $\qexpr_1 \neq_{q} \qexpr_2$ to notate query expression equivalence and in-equivalence, distinct from standard equality. A program $c$'s
%  , its 
 labeled variables 
%  and assigned variables are subsets of 
is a subset of
the labeled variables $\mathcal{LV}$, denoted by $\lvar(c) \in \mathcal{P}(\mathcal{VAR} \times \mathcal{L}) \subseteq \mathcal{LV}$.
% annotated by a label. 
% We use  
%$\mathcal{LVAR} = \mathcal{VAR} \times \mathcal{L} $ 
% $\mathcal{LV}$ represents the universe of all the labeled variables and 
% $\avar(c) \in \mathcal{P}(\mathcal{VAR} \times \mathbb{N}) \subset \mathcal{LV}$ and 
% $\lvar(c) \in \mathcal{P}(\mathcal{VAR} \times \mathcal{L}) \subseteq \mathcal{LV}$ for them. 
We also define the set of query variables for a program $c$, $\qvar: \cdom \to 
\mathcal{P}(\mathcal{LV})$.

A program $c$'s query variables is a subset of 
its labeled variables, $\qvar(c) \subseteq \lvar(c)$. We have the operator $\tlabel : \mathcal{T} \to \ldom$, which gives the set of labels in every event belonging to the trace.
Then we introduce a counting operator $\vcounter : \mathcal{T} \to \mathbb{N} \to \mathbb{N}$, 
% \wq{which counts the occurrence of of a variable in the trace,} 
which counts the occurrence of a labeled variable in the trace,
whose behavior is defined as follows,
\[
\begin{array}{ll}
\vcounter(\trace :: (\_, l, \_, \_), l ) \triangleq \vcounter(\trace, l) + 1
&
\vcounter(\trace  ::(b, l, v, \bullet), l) \triangleq \vcounter(\trace, l) + 1
\\
\vcounter(\trace  :: (x, l, v, \qval), l) \triangleq \vcounter(\trace, l) + 1
&
% \vcounter(\trace :: (\_, l', \_, \_), l ) \triangleq \vcounter(\trace, l), l' \neq l 
% &
\vcounter(\trace  :: (x, l', v, \bullet), l) \triangleq \vcounter(\trace, l), l' \neq l
\\
\vcounter(\trace  :: (b, l', v, \bullet), l) \triangleq \vcounter(\trace, l), l' \neq l
&
\vcounter(\trace  :: (x, l', v, \qval), l) \triangleq \vcounter(\trace, l), l' \neq l
\\
\vcounter({[]}, l) \triangleq 0
\end{array}
\]
The full definitions of these above operators can be found in the appendix.
\begin{defn}[Event May-Dependency].
\label{def:event_dep}
\\ 
  An event $\event_2$ is in the \emph{event may-dependency} relation with an assignment
  event $\event_1 \in \eventset^{\asn}$ in a program ${c}$
  with a hidden database $D$ and a trace $\trace \in \mathcal{T}$ denoted as 
  %
  $\eventdep(\event_1, \event_2, [\event_1 ] \tracecat \trace \tracecat [\event_2], c, D)$, iff
  %
  \[
    \begin{array}{l}
  \exists \vtrace_0,
  \vtrace_1, \vtrace' \in \mathcal{T},\event_1' \in \eventset^{\asn}, {c}_1, {c}_2  \in \cdom  \sthat
  \diff(\event_1, \event_1') \land 
      \\ \quad
      (
        \exists  \event_2' \in \eventset \sthat 
    \left(
    \begin{array}{ll}   
   & \config{{c}, \vtrace_0} \rightarrow^{*} 
  \config{{c}_1, \vtrace_1 \tracecat [\event_1]}  \rightarrow^{*} 
    \config{{c}_2,  \vtrace_1 \tracecat [\event_1] \tracecat \vtrace \tracecat [\event_2] } 
    % 
   \\ 
   \bigwedge &
    \config{{c}_1, \vtrace_1 \tracecat [\event_1']}  \rightarrow^{*} 
    \config{{c}_2,  \vtrace_1 \tracecat[ \event_1'] \tracecat \vtrace' \tracecat [\event_2'] } 
  \\
  \bigwedge & 
  \diff(\event_2,\event_2' ) \land 
  \vcounter(\vtrace, \pi_2(\event_2))
  = 
  \vcounter(\vtrace', \pi_2(\event_2'))\\
  \end{array}
  \right)
  \\ \quad
  \lor 
  \exists \vtrace_3, \vtrace_3'  \in \mathcal{T}, \event_b \in \eventset^{\test} \sthat 
  \\ \quad
  \left(
  \begin{array}{ll}   
    & \config{{c}, \vtrace_0} \rightarrow^{*} 
      \config{{c}_1, \vtrace_1 \tracecat [\event_1]}  \rightarrow^{*} 
      \config{c_2,  \vtrace_1 \tracecat [\event_1] \tracecat \trace \tracecat [\event_b] \tracecat  \trace_3} 
    \\ 
    \bigwedge &
    \config{{c}_1, \vtrace_1 \tracecat [\event_1']}  \rightarrow^{*} 
    \config{c_2,  \vtrace_1 \tracecat [\event_1'] \tracecat \trace' \tracecat [(\neg \event_b)] \tracecat \trace_3'} 
    \\
    \bigwedge &  \tlabel_{\trace_3} \cap \tlabel_{\trace_3'} = \emptyset
     \land \vcounter(\trace', \pi_2(\event_b)) = \vcounter(\trace, \pi_2(\event_b)) 
    %   \land \event_2 \eventin \trace_3
    % \land \event_2 \not\eventin \trace_3'
    \land \event_2 \in \trace_3
    \land \event_2 \not\in \trace_3'
  \end{array}
  \right)
  )
\end{array}
   \]
% , where ${\tt label}(\event_2) = \pi_2(\event_2)$.
  %  
%
\end{defn}
% \todo{add explnanation}
% \jl{
Our event \emph{may-dependency} relation of 
two events $\event_1 \in \eventset^{\asn}$ and $\event_2 \in \eventset$, 
for a program $c$ and hidden database $D$ is w.r.t to
a trace $[\event_1 ] \tracecat \trace \tracecat [\event_2]$.
The $\event_1 \in \eventset^{\asn}$ is an assignment event because only a change on an assignment event will affect the execution trace, according to our operational semantics.
In order to observe the changes of $\event_2$ under the modification of $\event_1$, this trace 
$[\event_1 ] \tracecat \trace \tracecat [\event_2]$
starts with $\event_1$ and ends with $\event_2$.
% }
{The \emph{may-dependency} relation considers both the value dependency and value control dependency as discussed in Section~\ref{sec:design_choice}. The relation can be divided into two parts naturally in Definition~\ref{def:event_dep} (line $2-4$, $5-8$ respectively, starting from line $1$). The idea of the event $\event_1$ may depend on $\event_2$ can be briefly described:
we have one execution of the program as reference (See line $2$ and $6$, for the two kinds of dependency). 
When the value assigned to the 
% first variable 
first variable in $\event_1$ is modified, the reference trace $\trace_1 \tracecat [\event_1]$ is modified correspondingly to $\trace_1 \tracecat [\event_1']$.
We use $\diff(\event_1, \event_1')$ at line $1$ to express this modification, which guarantees that $\event_1$ and $\event_1'$ only differ in their assigned values and are equal on variable name and label. We perform a second run of the program by continuing the execution of the same program from the same execution point, 
but with the modified trace $\trace_1 \tracecat [\event_1']$ (See line $3$, $7$). 
The expected may dependency will be caught by observing two different possible changes (See line $4, 8$ respectively) when comparing the second execution with the reference one (similar definitions as in \cite{Cousot19a}). 

% \wq{
% In the first situation, we are witnessing 
In the first part (line $2-4$ of Definition~\ref{def:event_dep}), we witness
% that the value assigned to the second variable in $\event_2$
the appearance of $\event_2'$ in the second execution, and
% a variation in $\event_2$, which changes into $\event_2'$.
a variation between $\event_2$ and $\event_2'$ on their values.
% changes in $\event_2'$.
% \jl{
We have special requirement $\diff(\event_2, \event_2')$, which guarantees that they
have the same variable name and label but only differ 
% % in their assigned value. 
in their evaluated values.
% assigned to the same variable. 
In particularly for queries, if $\event_2$ and $\event_2'$ are 
% query assignment events, then 
generated from query requesting, then $\diff(\event_2, \event_2')$ guarantees that
they differ in their query values rather than the 
% query requesting value. 
query requesting results. 
Additionally, in order to handle multiple occurrences of the same event through iterations of the while loop,
 where  $\event_2$ and $\event_2'$ could be 
in different while loops,
we restrict the same occurrence of $\event_2$'s label in $\trace$ from the first execution with  the occurrence of $\event_2'$'s label in $\trace'$ from the second execution,
through $\vcounter(\vtrace, \pi_2(\event_2))
= 
\vcounter(\vtrace', \pi_2(\event_2'))$ at line $4$.
% }
% }

% \wq{
In the second part (line $5-8$ of Definition~\ref{def:event_dep}), we 
% are witnessing 
witness
the disappearance of $\event_2$ through observing the change of a testing event $\event_b$.
% In order to change the appearance of 
% % and event, the command that generating $\event_2$ must not be executed in 
% 5yhan event, 
To witness
the disappearance, the command that generates $\event_2$ must not be executed in 
the second execution. 
The only way to control whether a command will be executed, is through the change of a guard's 
evaluation result in an if or while command, which generates a testing event $\event_b$ in the first place.
So we observe when
$\event_b$ changes into $\neg \event_b$ in the second execution firstly, 
whether it follows with the disappearance of $\event_2$ in the second trace. We restrict the occurrence of $\event_b$'s label in the two traces being the same
}
% s to the occurrence times of $\event_2'$'s label in the second trace,
through $\vcounter(\trace', \pi_2(\event_b)) = \vcounter(\trace, \pi_2(\event_b))$ to handle the while loop.
% changes in $\event_2'$, have the same variable and label and only differ in their assigned value. 
Again, for queries, we observe the disappearance based on the query value equivalence.
% if $\event_2$ and $\event_2'$ are query assignment events, then 
% they differ in their query value rather than the assigned value. 
% }
%
% \mg{I don't understand this explanation. What are the ``assignment commands associated to the two labelled variables''}
% \jl{revised but need more think}
% Explanation: 

{Considering 
% a program's all possible executions 
all events generated during a program's executions
under an initial trace,
% among all events generated during these executions
% and the variables and labels of these events are 
% corresponding to the two labeled variables,
% evaluations of the assignment commands associated to the two labelled variables respectively, 
as long as there is one pair of events satisfying the \emph{event may-dependency} relation in Definition~\ref{def:event_dep}, 
 we say the two 
related
variables satisfy the \emph{variable may-dependency} relation, in Definition~\ref{def:var_dep}.
}

\begin{defn}[Variable May-Dependency].
  \label{def:var_dep}
  \\
  A variable ${x}_2^{l_2} \in \lvar(c)$ is in the \emph{variable may-dependency} relation with another
  variable ${x}_1^{l_1} \in \lvar(c)$ in a program ${c}$, denoted as 
  %
  $\vardep({x}_1^{l_1}, {x}_2^{l_2}, {c})$, if and only if.
\[
  \begin{array}{l}
\exists \event_1, \event_2 \in \eventset^{\asn}, \trace \in \mathcal{T} , D \in \dbdom \sthat
% (\pi_{1}{(\event_1)}, \pi_{2}{(\event_1)}) = ({x}_1, l_1)
% \land
% (\pi_{1}{(\event_2)}, \pi_{2}{(\event_2)}) = ({x}_2, l_2)
\pi_{1}{(\event_1)}^{\pi_{2}{(\event_1)}} = {x}_1^{l_1}
\land
\pi_{1}{(\event_2)}^{\pi_{2}{(\event_2)}} = {x}_2^{l_2}% \\ \quad 
\land 
\eventdep(\event_1, \event_2, \trace, c, D) 
  \end{array}
\]  %
\end{defn}

\paragraph{Accurate Data Dependency Quantity Analysis}
\label{sec:refine-exe-reachability}
% For a program $c$, there are two data \emph{dependency quantities} we are considering.
The first quantity is the reachability times of each labeled variable during the program execution.
The second quantity is the reachability time for every pair of labeled variables with variable \emph{may-dependency} relation.
% \paragraph*{Variable Reachability}
\paragraph{The Dependency Quantity for Labeled Variables}
The reachability time of a labeled variable indicates the evaluation times of the assignment command assigning a value to this variable.  
\begin{defn}[Reachability Time of Labeled Variable]
  \label{def:adapt-var_reachability}
The reachability for every labeled variable overall $c$'s execution traces,
w.r.t. an initial trace $\vtrace \in \mathcal{T}_0(c)$ is defined as follows,
\[
  rb(x^l) \triangleq \forall \vtrace \in \mathcal{T}_0(c), \trace' \in \mathcal{T} \sthat \config{{c}, \trace} \to^{*} \config{\eskip, \trace\tracecat\vtrace'} 
  \implies w(\trace) = \vcounter(\vtrace', l) 
  \]
\end{defn}
%
$(x^l, w) \in \mathcal{LV} \times (\mathcal{T} \to \mathbb{N})$,
with a labeled variable as first component and
its weight $w$ the second component.
Weight $w$ for
% a labeled variable 
$x^l$ is a function $w : \mathcal{T} \to \mathbb{N}$
mapping from a starting trace to a natural number.
When program executes under this starting trace $\trace$,
$\config{{c}, \trace} \to^{*} \config{\eskip, \trace\tracecat\vtrace'} $, it generates an execution trace $\trace'$.
This natural number is the evaluation times of the labeled command corresponding to the vertex, 
computed by the counter operator $w(\trace) = \vcounter(\vtrace', l)$.


In most data analysis programs $c$ we are interested, there are usually some user input variables, such as $k$ in $\kw{twoRounds}$. 
We denote $\mathcal{T}_0(c)$ as the set of initial traces in which all the input variables in $c$ are initialized, it is also reflected in $\traceW({c})$.    
%
\paragraph{Dependency Quantity for the Pair of Labeled Variables}
% \paragraph*{Dependent Variables Reachability}
%
% For a program $c$ I compute the reachability bound for every labeled variable overall $c$'s execution traces,
% w.r.t. an initial trace as follows,
\begin{defn}[Reachability Time of Dependent Variables]
  \label{def:adapt-depvar_reachability}
  The execution-based reachability time for every pair of 
  labeled in the
  \emph{may-dependency} relation w.r.t. an initial trace. Formally as follows,
    \[
    \begin{array}{l}
        rb(x^i, y^j) \triangleq 
%   x^i, y^j \in \lvar(c)
%   \land w \in \mathcal{P}( \mathcal{T}_0(c) \to \mathbb{N})
%   \land 
%   \exists \trace \in \mathcal{T}_0(c), 
%   \trace_1, \trace_2 \in \mathcal{T} \sthat \dep(x^i, y^j,\trace_1, \trace_2, \trace_0, c)
%   \\
%   \land 
\forall \trace_0 \in \mathcal{T}_0(c) \sthat
  w (\trace_0) = \max \left\{ | \sdiff(\trace_1, \trace_2, y)|
  ~\middle\vert~
  \forall \trace_1, \trace_2 \in \mathcal{T} \sthat \dep(x^i, y^j,\trace_1, \trace_2, \trace_0, c) \right\}
\end{array}
\]
\end{defn}
%
For any pair of labeled variable $(x^i, y^j) \in \ldom$, 
$ rb(x^i, y^j)$ is a function $w: \mathcal{T}_0(c) \to \mathbb{N}$,
    where given an initial trace $\trace_0$,
    it is the maximum length of the difference sequence between all pairs of the witness traces $\trace_1, \trace_2$ 
    satisfying the dependency relation.

    \highlight{\paragraph*{Improvements Analysis}
    Previous works do not have any quantity analysis on the dependency relation.
    Comparing to them, this part is stronger in following senses.
    % It is more scalable to general program, and it provides the program with preciser formal definition for \emph{Adaptivity} than previous definition,
    % specifically as follows.
    % language and operational semantics design improves the expressiveness, efficiency, and the accuracy to a large extend.
    \todo{Add details}
    \begin{itemize}
      \item \textbf{Improvements on Efficiency}
      \\
      It is also efficient.
      \item \textbf{Improvements on Accuracy}
      This quantity analysis can help to improve the precision of the adaptivity formalization.
      \end{itemize}
      }

\paragraph*{The Dependency Quantity through The Two Rounds Example}
\begin{example}[Variable \emph{May-Dependency} Quantity in The Two Rounds Data Analysis Example Program]
    In the same $\kw{towRounds(k)}$ example Program,    the analyst asks in total $k+1$ queries to the mechanism in two phases.
    %
    \[         \begin{array}{l}
      \kw{towRounds(k)} \triangleq \\
             \clabel{ \assign{a}{0}}^{0} ;
              \clabel{\assign{j}{k} }^{1} ;\\
              \ewhile ~ \clabel{j > 0}^{2} ~ \edo ~
              \Big(
               \clabel{\assign{x}{\query(\chi[j] \cdot \chi[k])} }^{3}  ;
               \clabel{\assign{j}{j-1}}^{4} ;
              \clabel{\assign{a}{x + a}}^{5}       \Big);\\
              \clabel{\assign{l}{\query(\chi[k]*a)} }^{6}
          \end{array}
          \]    %
    % Queries are of the form $q(e)$ where $e$ is an expression with a special variable $\chi$ representing a possible row. Mainly $e$ represents a function from $X$ to some domain $U$, for example $U$ could be $[-1,1]$ or $[0,1]$. This function characterizes the linear query I are interested in running. As an example, $x \leftarrow q(\chi[2])$ computes an approximation, according to the used mechanism, of the empirical mean of the second attribute, identified by $\chi[2]$. Notice that I don't materialize the mechanism but I assume that it is implicitly run when I execute the query. 
    % \jl{We use $\chi$ to abstract a possible row in the database and }
    % queries are of the form $\query(\qexpr)$, where $\qexpr$ is a special expression 
    With the initial trace
    $[(k, in, 2, \bullet)]$ and following execution trace, 
    \\
    $
    \trace_1 \triangleq 
    \left[\begin{array}{l}
    % \trace_0 \tracecat
     (a, 0, 0, \bullet),
    (j, 1, 2, \bullet),
    (j>0, 2, \etrue, \bullet),
    (x, 3, v_1, \chi[2]*\chi[2]),
    (j, 4, 1, \bullet),
    (a, 5, v_1, \bullet),\\
    (j>0, 2, \etrue, \bullet),
    (x, 3, v_2, \chi[1]*\chi[2]),
    (j, 4, 0, \bullet),
    (a, 5, v_1 + v_2, \bullet),
    (j>0, 4, \efalse, \bullet),\\
    (l, 6, v_3, \chi[2]*( v_1 + v_2))
    \end{array} \right]
    $.
    Based on these observations, we analyze the \emph{may-dependency} quantity for every labeled variable,
    and pairs of dependent variables as follows.
\begin{itemize}
    \item \textbf{The Dependency Quantity for Labeled Variables}
    \\
    For the specific two execution traces above,
    the \emph{may-dependency} quantity for every variable
    is computed as follows,
   %   where $k$ is the 
   %  initial value of input variable $k$ given by user,
   %  we observe the execution trace as
   \\
   $rb(a^0) ((k, in, 2, \bullet))  = \vcounter(\trace_1) = 1$
   \\
   $\cdots$
  \\
   $rb(x^3) ((k, in, 2, \bullet))  = \vcounter(\trace_1) = 2$
    \\
    $\cdots$
    \\

    Then, for arbitrary initial trace $\trace_0 \in \mathcal{T}_0(\kw{twoRounds(k)})$,
    the \emph{may-dependency} quantity for every variable under $\trace_0$ is a function
    as follows,
    \\
    $rb(a^0) (\trace_0)  = 1$
    \\
    $\cdots$
    \\
    $rb(x^3) (\trace_0)  = \max\{0, \env(\trace_0) k \} $
    \item \textbf{Dependency Quantity for the Pair of Labeled Variables}
    For the specific two execution traces above,
    the \emph{may-dependency} quantity for every variable
    is computed as follows,
   %   where $k$ is the 
   %  initial value of input variable $k$ given by user,
   %  we observe the execution trace as
   \\
   $rb(a^0) ((k, in, 2, \bullet))  = \vcounter(\trace_1) = 1$
   \\
   $\cdots$
  \\
   $rb(x^3) ((k, in, 2, \bullet))  = \vcounter(\trace_1) = 2$
    \\
    $\cdots$
    \\

    Then, for arbitrary initial trace $\trace_0 \in \mathcal{T}_0(\kw{twoRounds(k)})$,
    the \emph{may-dependency} quantity for every variable under $\trace_0$ is a function
    as follows,
    \\
    $rb(a^0) (\trace_0)  = 1$
    \\
    $\cdots$
    \\
    $rb(x^3) (\trace_0)  = \max\{0, \env(\trace_0) k \} $
\end{itemize}
    % \\
    % We modify the value assigned to $x$ when evaluating the command $ \clabel{\assign{x}{\query(\chi[j] \cdot \chi[k])} }^{3}$
    % in the first iteration.
    % By manipulating the event in the trace, 
    % the event $(x, 3, v_1, \chi[2]*\chi[2])$
    % is modified into $(x, 3, v_1', \chi[2]*\chi[2])$ where $v_1 \neq v_1'$.
    % Then, through executing the program from the execution point after executing line $3$, we observe another execution trace as follows,
    % \\
    % $
    % \left[\begin{array}{l}
    % % \trace_0 \tracecat
    %  (a, 0, 0, \bullet),
    % (j, 1, 2, \bullet),
    % (j>0, 2, \etrue, \bullet),
    % \highlight{(x, 3, v_1', \chi[2]*\chi[2])},
    % (j, 4, 1, \bullet),
    % (a, 5, v_1, \bullet),\\
    % (j>0, 2, \etrue, \bullet),
    % (x, 3, v_2, \chi[1]*\chi[2]),
    % (j, 4, 0, \bullet),
    % \highlight{(a, 5, v_1' + v_2, \bullet),}
    % (j>0, 4, \efalse, \bullet),\\
    % (l, 6, v_3, \chi[2]*( v_1' + v_2))
    % \end{array} \right]
    % $.   
    % \\
    % In this trace, the event $(a, 5, v_1' + v_2, \bullet),$  is different from $(a, 5, v_1 + v_2, \bullet),$ in the first 
    % trace.
    % \\
    % This change satisfies the Definition~\ref{def:event_dep}, so there exists the variable \emph{may-dependency} relation 
    % between variable $x^3$ and $a^5$.
\end{example}%
In the second stage of the execution-based analysis, 
based on the new \emph{may-dependency} definition,
I compute the execution-based reachability bound for every pair of 
labeled in the
\emph{may-dependency} relation w.r.t. an initial trace. Formally as follows,
%
% For a program $c$ I compute the reachability bound for every labeled variable overall $c$'s execution traces,
% w.r.t. an initial trace as follows,
\[
    \begin{array}{l}
        rb(x^i, y^j) \triangleq 
%   x^i, y^j \in \lvar(c)
%   \land w \in \mathcal{P}( \mathcal{T}_0(c) \to \mathbb{N})
%   \land 
%   \exists \trace \in \mathcal{T}_0(c), 
%   \trace_1, \trace_2 \in \mathcal{T} \sthat \dep(x^i, y^j,\trace_1, \trace_2, \trace_0, c)
%   \\
%   \land 
\forall \trace_0 \in \mathcal{T}_0(c) \sthat
  w (\trace_0) = \max \left\{ | \sdiff(\trace_1, \trace_2, y)|
  ~\middle\vert~
  \forall \trace_1, \trace_2 \in \mathcal{T} \sthat \dep(x^i, y^j,\trace_1, \trace_2, \trace_0, c) \right\}
\end{array}
\]
%
For any pair of labeled variable $(x^i, y^j) \in \ldom$, 
$ rb(x^i, y^j)$ is a function $w: \mathcal{T}_0(c) \to \mathbb{N}$,
    where given an initial trace $\trace_0$,
    it is the maximum length of the difference sequence between all pairs of the witness traces $\trace_1, \trace_2$ 
    satisfying the dependency relation.
%
\paragraph{Accurate Execution-Based Adaptivity Analysis}
\label{sec:refine-exe-adapt}
% 
Based on the variable \emph{may-dependency} relation in Section~\ref{subsec:dynamic-datadep} and 
the dependency quantity analysis in Section~\ref{subsec:dynamic-reachability}.
% gives us the edges, 
I firstly define the execution-based dependency graph, then formalize the \emph{adaptivity} in this section.
% \wq{Just a few sentences here, some overview of this subsection. See 4.2 for instance.}
\paragraph{Execution Based Dependency Graph}
\label{para:execution-base-graph-def}
Based on the variable \emph{may-dependency} relation,
% gives us the edges, 
we define the execution-based dependency graph.
% \wq{Just a few sentences here, some overview of this subsection. See 4.2 for instance.}
\begin{defn}[Execution Based Dependency Graph]
\label{def:trace_graph}
Given a program ${c}$,
its \emph{execution-based dependency graph} 
$\traceG({c}) = (\traceV({c}), \traceE({c}), \traceW({c}), \traceF({c}))$ is defined as follows,
{
  \small
\[
\begin{array}{rlcl}
  \text{Vertices} &
  \traceV({c}) & := & \left\{ 
  x^l \in \mathcal{LV}
  ~ \middle\vert ~ x^l \in \lvar(c)
  \right\}
  \\
  \text{Directed Edges} &
  \traceE({c}) & := & 
  \left\{ 
  (x^i, y^j) 
%   \in \mathcal{LV} \times \mathcal{LV}
  ~ \middle\vert ~
  x^i, y^j \in \lvar(c) \land \vardep(x^i, y^j, c) 
  % \text{\mg{$\land$ instead of ,}}
  \right\}
  \\
  \text{Weights} &
  \traceW({c}) & := & 
%   \left
  \{ 
  (x^l, w) 
  % \in \mathcal{LV} \times \mathbb{N}
  ~ \vert ~ 
  w : \mathcal{T} \to \mathbb{N}
  \land
  x^l \in \lvar(c) 
  \\ & & &
  \land
  % n = \max \left\{ 
    % ~ \middle\vert~
  \forall \vtrace \in \mathcal{T}_0(c), \trace' \in \mathcal{T} \sthat \config{{c}, \trace} \to^{*} \config{\eskip, \trace\tracecat\vtrace'} 
  \implies w(\trace) = \vcounter(\vtrace', l) 
  %  \right\}
%   \right
\}
  \\
  % \text{Query Label} &
  \text{Query Annotation} &
  \traceF({c}) & := & 
\left\{(x^l, n)  
% \in  \mathcal{LV}\times \{0, 1\} 
~ \middle\vert ~
 x^l \in \lvar(c) \land
n = 1 \Leftrightarrow x^l \in \qvar(c) \land n = 0 \Leftrightarrow  x^l \notin \qvar(c)
\right\}
\end{array}.
\]
}
\end{defn}
%
There are four components of the execution-based dependency graph. 
The vertices $\traceV(c)$ is the set of program $c$'s labeled variables $\lvar(c)$,
which are statically collected.
The query annotation is 
a set of pairs $\traceF(c) \in \mathcal{P}(\mathcal{LV} \times \{0, 1\} )$ 
mapping each $x^l \in \traceV(c)$ to $0$ or $1$, 
indicating whether this labeled variable is in program $c$'s query variable set $\qvar(c)$.
{
The weights is a set of pairs, $(x^l, w) \in \mathcal{LV} \times (\mathcal{T} \to \mathbb{N})$,
with a labeled variable as first component and
its weight $w$ the second component.
Weight $w$ for
% a labeled variable 
$x^l$ is a function $w : \mathcal{T} \to \mathbb{N}$
mapping from a starting trace to a natural number.
When program executes under this starting trace $\trace$,
$\config{{c}, \trace} \to^{*} \config{\eskip, \trace\tracecat\vtrace'} $, it generates an execution trace $\trace'$.
This natural number is the evaluation times of the labeled command corresponding to the vertex, 
computed by the counter operator $w(\trace) = \vcounter(\vtrace', l)$.
We can see in the execution-based dependency graph of $\kw{twoRounds}$ in Figure~\ref{fig:overview-example}(b), the weight of vertices in the while loop is  $\env(\trace) k$, which depends on the value of the user input $k$ specified in the starting trace $\tau$.
The directed edges $\traceE({c})$ is also a set of pairs with two labeled variables $ (x^i, y^j) \in \mathcal{LV} \times \mathcal{LV}$, from $x^i$ pointing to $y^j$ in the graph.
The edges are constructed directly from our variable may-dependency relation. 
For any two vertices $x^{i}$ and $y^{j}$ in $\traceV(c)$, if they satisfy the variable may-dependency relation $\vardep(x^i, y^j, c)$, there is a direct edge between the two vertices in our execution-based dependency graph for program $c$.
} 
In most data analysis programs $c$ we are interested, there are usually some user input variables, such as $k$ in $\kw{twoRounds}$. 
We denote $\mathcal{T}_0(c)$ as the set of initial traces in which all the input variables in $c$ are initialized, it is also reflected in $\traceW({c})$.    

\paragraph{Trace-based Adaptivity}

% \wq{
% Given 
% a program $c$'s execution-based dependency graph 
% % $G_{trace}(c)(\trace) = (\vertxs, \edges, \weights, \qflag)$,
% $\traceG({c}) = (\traceV({c}), \traceE({c}), \traceW({c}), \traceF({c}))$
% we define adaptivity 
% with respect to $\trace$ by the finite walk in the graph, which has the most query requests along the walk.
% }

Given 
a program $c$'s execution-based dependency graph 
% $G_{trace}(c)(\trace) = (\vertxs, \edges, \weights, \qflag)$,
$\traceG({c})$,
we define adaptivity 
with respect to an initial trace $\trace_0 \in \mathcal{T}_0(c)$ by the finite walk in the graph, which has the most query requests along the walk.
We show the definition of a finite walk as follows.
%
% The query length of a walk $k$ is the number of vertices which correspond to query variables in the vertices sequence of this walk. 
% Instead of counting all 
% the vertices in $k$'s vertices sequence, i

\begin{defn}[Finite Walk (k)].
  \label{def:finitewalk}
  \\
%   Given a program $c$'s execution-based dependency graph $\traceG({c})(\trace)$, 
%   a \emph{finite walk} $fw$ in $\traceG({c})(\trace)$ is a sequence of edges $(e_1 \ldots e_{n - 1})$ 
%   for which there is a sequence of vertices $(v_1, \ldots, v_{n})$ such that:
%   \begin{itemize}
%       \item $e_i = (v_{i},v_{i + 1})$ for every $1 \leq i < n$.
%       \item every vertex $v \in \traceV({c}) $ appears in $(v_1, \ldots, v_{n})$ at most 
%       \wq{$\traceW({c})(\trace)$} times.  
%   \end{itemize}
%   %
%   The length of $fw$ is the number of vertices in its vertex sequence, i.e., $\len(k) = n$.
  Given the execution-based dependency graph $\traceG({c}) = (\traceV({c}), \traceE({c}), \traceW({c}), \traceF({c}))$ of a program $c$,
  a \emph{finite walk} $k$ in $\traceG({c})$ is a 
  function $k: \mathcal{T} \to $ sequence of edges.
  For a initial trace $\trace_0 \in \mathcal{T}_0(c)$, 
  $k(\trace_0)$ is a sequence of edges $(e_1 \ldots e_{n - 1})$ 
  for which there is a sequence of vertices 
  $(v_1, \ldots, v_{n})$ such that:
  \begin{itemize}
      \item $e_i = (v_{i},v_{i + 1}) \in \traceE(c)$ for every $1 \leq i < n$.
      \item every $v_i \in \traceV(c)$
      and $(v_i, w_i) \in \traceW(c)$, 
       $v_i$ appears in $(v_1, \ldots, v_{n})$ at most 
    %   \wq{$\traceW({c})(\trace)$} 
    $w(\trace_0)$
      times.  
  \end{itemize}
  %
  The length of $k(\trace_0)$ is the number of vertices in its vertices sequence, i.e., $\len(k)(\trace_0) = n$.
 \end{defn}

We use $\walks(\traceG(c))$ to denote 
% \mg{``the set'', not ``a set''}a set containing all finite walks $k$ in $G$;
the set containing all finite walks $k$ in $\traceG(c)$;
and $k_{v_1 \to v_2} \in \walks(\traceG(c))$ with $v_1, v_2 \in \traceV(c)$ denotes the walk from vertex $v_1$ to $v_2$ . 
\\
We are interested in queries, so we need to recover the 
variables corresponding to queries from the walk. We define the query length of a walk, 
instead of counting all 
the vertices in $k$'s vertices sequence, we just count the number of vertices which correspond to query variables in this sequence.
%
% \mg{I don't understand this definition. Is wrt a single query?if yes, who is chosing the query? Or is it any query?}
% \jl{It is for any query, as long as the vertex is a query variable, in another worlds, this length just counting the number of query variables in the walk, instead of counting all 
% the vertices.}
% \todo{Make the definition clear}
\begin{defn}[Query Length of the Finite Walk($\qlen$)].
\label{def:qlen}
\\
% Given 
% % labelled weighted graph $G = (\vertxs, \edges, \weights, \qflag)$, 
% a program $c$'s execution-based dependency graph $\traceG(c)(\trace)$
%  and a \emph{finite walk} $k$ in $\traceG(c)(\trace)$ with its vertex sequence $(v_1, \ldots, v_{n})$, 
% %  the length of $k$ w.r.t query is defined as:
% The query length of $k$ is the number of vertices which correspond to query variables in $(v_1, \ldots, v_{n})$ as follows, 
% \[
%   \qlen(k) = \len\big( v \mid v \in (v_1, \ldots, v_{n}) \land \qflag(v) = 1 \big)
% \]
% , where $\big(v \mid v \in (v_1, \ldots, v_{n}) \land \qflag(v) = 1 \big)$ is a subsequence of $(v_1, \ldots, v_{n})$.
Given 
% labelled weighted graph $G = (\vertxs, \edges, \weights, \qflag)$, 
the execution-based dependency graph $\traceG({c}) = (\traceV({c}), \traceE({c}), \traceW({c}), \traceF({c}))$ of a program $c$,
 and a \emph{finite walk} 
%  $k$ in $\traceG(c)(\trace)$
 $k \in \walks(\traceG(c))$. 
%  with its vertex sequence $(v_1, \ldots, v_{n})$, 
%  the length of $k$ w.r.t query is defined as:
The query length of $k$ is a function $\qlen(k): \mathcal{T} \to \mathbb{N}$, such that with an initial trace  $\trace_0 \in \mathcal{T}_0(c)$, $\qlen(k)(\trace_0)$ is
the number of vertices which correspond to query variables in the vertices sequence of the walk $k(\trace_0)$
$(v_1, \ldots, v_{n})$ as follows, 
\[
  \qlen(k)(\trace_0) = |\big( v \mid v \in (v_1, \ldots, v_{n}) \land \qflag(v) = 1 \big)|.
\]
\end{defn}
The definition of adaptivity is then presented in Def~\ref{def:trace_adapt} below.

\begin{defn}
  [Adaptivity of a Program].
  \label{def:trace_adapt}
  \\
  Given a program ${c}$, 
  its adaptivity $A(c)$ is function 
  $A(c) : \mathcal{T} \to \mathbb{N}$ such that for an
  % with respect to a starting trace $\trace$ 
  initial trace $\trace_0 \in \mathcal{T}_0(c)$, 
  % is defined as follows:
  %
 $$
  A(c)(\trace_0) = \max \big 
  \{ \qlen(k)(\trace_0) \mid k \in \walks(\traceG(c)) \big \} $$
  \end{defn}%
%
In the third stage, the execution-based dependency graph is defined in a different way as follows,
\begin{defn}[Execution Based Dependency Graph]
    \label{def:trace_graph}
    Given a program ${c}$,
    its \emph{Execution-Base Dependency Graph} 
    $\traceG({c}) = (\traceV({c}), \traceE({c}), \traceW({c}), \traceF({c}))$ is defined as follows,
    % over all possible traces,
    %
    \highlight{\small
    \[
    \begin{array}{lcl}
      % \text{Vertices} &
      \traceV({c}) & := & 
      \{ 
      (x^l, w) 
      % \in \mathcal{LV} \times \mathbb{N}
      ~ \vert ~ 
      w : \mathcal{T} \to \mathbb{N}
      \land
      x^l \in \lvar(c) 
      \\ & &
      \land
      % n = \max \left\{ 
        % ~ \middle\vert~
      \forall \trace \in \mathcal{T}_0(c), \trace' \in \mathcal{T} \sthat \config{{c}, \trace} \to^{*} \config{\eskip, \trace\tracecat\vtrace'} 
      \implies w(\trace) = \vcounter(\vtrace', l) 
      %  \right\}
    %   \right
    \}
      \\
      % \text{Edges} &
      \traceE({c}) & := & 
      \{ 
      (x^i, w, y^j) 
    %   \in \mathcal{LV} \times \mathcal{LV}
      ~ \vert ~
      x^i, y^j \in \lvar(c)
      \land w \in \mathcal{P}( \mathcal{T}_0(c) \to \mathbb{N})
      \land 
      \exists \trace \in \mathcal{T}_0(c), 
      \trace_1, \trace_2 \in \mathcal{T} \sthat \dep(x^i, y^j,\trace_1, \trace_2, \trace_0, c)
      \\ & &
      \land \forall \trace_0 \in \mathcal{T}_0(c) \sthat
      w (\trace_0) = \max \left\{ | \sdiff(\trace_1, \trace_2, y)|
      ~\middle\vert~
      \forall \trace_1, \trace_2 \in \mathcal{T} \sthat \dep(x^i, y^j,\trace_1, \trace_2, \trace_0, c) \right\}
      \}
    \end{array}
    \]
    }
    \end{defn}
    There are two components of the execution-based dependency graph. 
    \\
    \highlight{
    The vertices $\traceV(c)$  is a set of pairs, $(x^l, w) \in \mathcal{LV} \times (\mathcal{T} \to \mathbb{N})$,
    with a labeled variable as first component and
    its weight $w$ the second component.
    Weight $w$ for
    % a labeled variable 
    $x^l$ is a function $w : \mathcal{T} \to \mathbb{N}$
    mapping from a starting trace to a natural number.
    When program executes under this starting trace $\trace$,
    $\config{{c}, \trace} \to^{*} \config{\eskip, \trace\tracecat\vtrace'} $, it generates an execution trace $\trace'$.
    This natural number is the evaluation times of the labeled command corresponding to the vertex, 
    computed by the counter operator $w(\trace) = \vcounter(\vtrace', l)$.
    We can see in the execution-based dependency graph of $\kw{twoRounds}$ in
     Figure~3(b) in main paper, the weight of vertices in the while loop is  $\env(\trace) k$, which depends on the value of the user input $k$ specified in the starting trace $\tau$.
    \\
    The directed edges $\traceE({c})$ is a set of triples $ (x^i, w, y^j) \in \mathcal{LV} \times \mathcal{P}(\mathcal{T}_0(c) \to \mathbb{N}) \times \mathcal{LV}$,
     with two labeled variables (from $x^i$ pointing to $y^j$) and a weight $w$ for this edge.
    % comes 
    The edges are constructed directly from our variable may-dependency relation. 
    For any two vertices $x^{i}$ and $y^{j}$ in $\traceV(c)$, if there exists two witness traces $\trace_1, \trace_2$ and an initial trace $\trace_0 \in \mathcal{T}_0$ such that,
    they satisfy the variable may-dependency relation 
    $\dep(x^i, y^j, \trace_1, \trace_2, \trace_0, c)$ , 
    there is a direct edge. 
    The weight of the edge is a function $w: \mathcal{T}_0(c) \to \mathbb{N}$,
    where given an initial trace $\trace_0$,
    it is the maximum length of the difference sequence between all pairs of the witness traces $\trace_1, \trace_2$ 
    satisfying the dependency relation.
    }
    Then, the \emph{adaptivity} is still formalized as the 
length of the longest finite walk. 
Differently from the previous one, I restrict 
the occurrence of every edge in a finite walk no more than its weight as well. Formally as follows,
\begin{defn}[Finite Walk (k)].
    \label{def:finitewalk}
    \\
    Given the execution-based dependency graph $\traceG({c}) = (\traceV({c}), \traceE({c}), \traceW({c}), \traceF({c}))$ of a program $c$,
    a \emph{finite walk} $k$ in $\traceG({c})$ is a 
    function $k: \mathcal{T} \to $ sequence of edges.
    For a initial trace $\trace_0 \in \mathcal{T}_0(c)$, 
    $k(\trace_0)$ is a sequence of edges $(e_1 \ldots e_{n - 1})$ 
    for which there is a sequence of vertices 
    $(v_1, \ldots, v_{n})$ such that:
    \begin{itemize}
        \item \highlight{
          $e_i = (v_{i}, w_i, v_{i + 1}) \in \traceE(c)$ for every $1 \leq i < n$, 
          and $e_i$ appears in $(e_1 \ldots e_{n - 1})$  at most $w_i(\trace_0)$
          times.
        }
        \item every $(v_i, w_i) \in \traceV(c)$
        % and $(v_i, w_i) \in \traceW(c)$, 
         and $v_i$ appears in $(v_1, \ldots, v_{n})$ at most 
      %   \wq{$\traceW({c})(\trace)$} 
      $w_i(\trace_0)$
        times.  
    \end{itemize}
    %
    The length of $k(\trace_0)$ is the number of vertices in its vertices sequence, i.e., $\len(k)(\trace_0) = n$.
   \end{defn}
  
  We use $\walks(\traceG(c))$ to denote 
  % \mg{``the set'', not ``a set''}a set containing all finite walks $k$ in $G$;
  the set containing all finite walks $k$ in $\traceG(c)$;
  and $k_{v_1 \to v_2} \in \walks(\traceG(c))$ with $v_1, v_2 \in \traceV(c)$ denotes the walk from vertex $v_1$ to $v_2$ . 
  \\
  We are interested in queries, so we need to recover the 
  variables corresponding to queries from the walk. We define the query length of a walk, 
  instead of counting all 
  the vertices in $k$'s vertices sequence, we just count the number of vertices which correspond to query variables in this sequence.
  %
  % \mg{I don't understand this definition. Is wrt a single query?if yes, who is chosing the query? Or is it any query?}
  % \jl{It is for any query, as long as the vertex is a query variable, in another worlds, this length just counting the number of query variables in the walk, instead of counting all 
  % the vertices.}
  % \todo{Make the definition clear}
  \begin{defn}[Query Length of the Finite Walk($\qlen$)].
  \label{def:qlen}
  \\
  Given 
  the execution-based dependency graph 
  $\traceG({c}) = (\traceV({c}), \traceE({c}))$ of a program $c$,
   and a \emph{finite walk} 
  %  $k$ in $\traceG(c)(\trace)$
   $k \in \walks(\traceG(c))$. 
  %  with its vertex sequence $(v_1, \ldots, v_{n})$, 
  %  the length of $k$ w.r.t query is defined as:
  The query length of $k$ is a function $\qlen(k): \mathcal{T} \to \mathbb{N}$, such that with an initial trace  $\trace_0 \in \mathcal{T}_0(c)$, $\qlen(k)(\trace_0)$ is
  the number of vertices which correspond to query variables in the vertices sequence of the walk $k(\trace_0)$
  $(v_1, \ldots, v_{n})$ as follows, 
  \[
    \qlen(k)(\trace_0) = |\big( v \mid v \in (v_1, \ldots, v_{n}) \land v \in \qvar(c) \big)|.
  \]
  % , where $\trace_0 \in \mathcal{T}$ is the initial trace and $\big(v \mid v \in (v_1, \ldots, v_{n}) \land \qflag(v) = 1 \big)$ is a subsequence of $(v_1, \ldots, v_{n})$.
  %  $k$'s vertex sequence.
  % \mg{If I understand where you want to go, why don't you just use the cardinality of the set above, rather than taking the length of a subsequence?}
  % \jl{because the same vertex could have multiple occurrence in the sequence, and we will count all the occurrence instead of just once.
  % So the cardinality of set doesn't work.}
  \end{defn}
%
With the alternative finite walk and query length definition, the adaptivity in execution-based analysis 
is formalized in the same way as in Definition~\ref{def:trace_adapt}.
\begin{defn}
    [Adaptivity of a Program].
    \\
    Given a program ${c}$, 
    its adaptivity $A(c)$ is function 
    $A(c) : \mathcal{T} \to \mathbb{N}$ such that for an
    % with respect to a starting trace $\trace$ 
    initial trace $\trace_0 \in \mathcal{T}_0(c)$, 
    % is defined as follows:
    %
   $$
    A(c)(\trace_0) = \max \big 
    \{ \qlen(k)(\trace_0) \mid k \in \walks(\traceG(c)) \big \} $$
    \end{defn}

\subsection{Accurate Static Adaptivity Analysis}
\label{sec:refine-static}

% In static program analysis framework $\THESYSTEM$, specifically on the dependency quantity, 
% I adopt the reachability bound analysis technique to estimate this dependency quantity.
% % In existing static reachability bound analysis, 
% However, it isn't precise enough w.r.t. the execution-based reachability bound on every program command.
% It comes across an over-approximation on the estimation due to its path-insensitive nature. 
% It occurs when the control flow can be decided in a particular way in front of conditional branches, 
% while the static analysis fails to witness. 
% As shown in Example~\ref{ex:overapproximate}.
% \begin{example}
[Multiple Rounds Odds Algorithm]
\label{ex:overapproximate}
The $\THESYSTEM$ comes across an over-approximation on the estimation due to its path-insensitive nature. 
It occurs when the control flow can be decided in a particular way in front of conditional branches, while the static analysis fails to witness. 

We show the over-approximation, in Figure~\ref{fig:overappr_example}(a),
we call it a multiple rounds odd iteration algorithm. In this algorithm, at line 5 of every iteration, 
a query $\query(\chi[x])$ based on previous query results stored in $x$ is asked by the analyst like in the multiple rounds strategy. The difference is that only the query answers from the even iterations ($i =0, 2, \cdots $) are 
% used to $b$. 
used in the query 
in line 7, $\query(\chi[\ln(y)])$.
  Because the execution trace only updates 
%   $b$ using the query answers at odd iterations, so the answers from even iterations do not affect the queries at odd iterations. From the query-based dependency graph in Figure~\ref{fig:overappr_example}(b), we can see that there is no edge from queries at odd iterations (such as $q_1,q_3,q_5$) to queries at even iteration(such as $q_2,q_4$). The longest path is dashed with a length $3$.  However, {\THESYSTEM} fails to realize that odd iteration will always execute then branch and even iteration means else branch, so its dependency graph considers both branches for every iteration. In this sense, the dependency graph by {\THESYSTEM} is similar to the one in the multiple rounds strategy. We show the estimated graph in Figure~\ref{fig:overappr_example}(c). The estimated upper bound is then, $5$, instead of $3$. 
$x$ using the query answers in even iterations, so the answers from odd iterations do not affect the queries in even iterations. 
From the execution-based dependency graph in Figure~\ref{fig:overappr_example}(b), 
we can see that the weight for the vertex $y^5$ is 
$w_k/2$. a function which takes any initial trace $\trace_0$, return the value of $k/2$ evaluated in $\trace_0$.  
However, {\THESYSTEM} fails to realize that odd iteration will always execute the then branch and even iteration means else branch, so 
% its dependency 
it considers both branches for every iteration. 
In this sense, the weight estimated for $y^5$ and $p^6$ are both 
$k$ as in Figure~\ref{fig:overappr_example}(c).
As a result, {\THESYSTEM}  estimates the longest walk from Figure~\ref{fig:overappr_example}(c),
$y^5  \to x^7  \to y^5  \to \cdots \to x^7  $ with each vertex visited $k$ times,
as the dotted arrows. 
And the adaptivity computed 
% estimated from the program-based dependency graph graph from by finding the walk with the longest query length 
is $1 + 2 * k$, instead of $1 + k$. 
% We omitted the estimated graph, which is identical to the graph in Figure~\ref{fig:overappr_example}(b). 
%

{ \small
\begin{figure}
\centering
    \begin{subfigure}{0.33\textwidth}
\centering
\small{
    \[
    %
    \begin{array}{l}
        \kw{multipleRoundsOdd}(k) \triangleq \\
        \clabel{ \assign{j}{k}}^{0} ; 
        \clabel{ \assign{x}{\query(\chi[0])} }^{1} ; \\
            \ewhile ~ \clabel{j > 0}^{2} ~ \edo ~ 
            \Big(
             \clabel{\assign{j}{j-1}}^{3} ;\\
             \eif(\clabel{j \% 2 == 0}^{4}, \\
             \clabel{\assign{y}{\chi[x]}}^{5}, 
             \clabel{\assign{p}{\chi[x]}}^{6});\\                            
             \clabel{\assign{x}{\query(\chi(\ln(y)))} }^{7} \Big)
        \end{array}
    \]
}
 \caption{}
    \end{subfigure}
%
\begin{subfigure}{.31\textwidth}
    \begin{centering}
    \begin{tikzpicture}[scale=\textwidth/11cm,samples=200]
% Variables Initialization
\draw[] (5, 1) circle (0pt) node{{ $x^1: {}^{w_1}_{1}$}};
% Variables Inside the Loop
 \draw[] (0, 7) circle (0pt) node{{ $y^5: {}^{w_k/2}_{1}$}};
 \draw[] (0, 4) circle (0pt) node{{ $p^6: {}^{w_k/2}_{1}$}};
 \draw[] (0, 1) circle (0pt) node{{ $x^7: {}^{w_k}_{1}$}};
 % Counter Variables
 \draw[] (5, 7) circle (0pt) node {{$j^0: {}^{w_1}_{0}$}};
 \draw[] (5, 4) circle (0pt) node {{ $j^3: {}^{w_k}_{0}$}};
 %
 % Value Dependency Edges:
 \draw[ thick, -latex,]  (0, 3.5) -- (0, 1.5) ;
%  \draw[ thick, -Straight Barb] (1, 4.2) arc (220:-100:1);
 \draw[ thick, -Straight Barb] (6.5, 4.5) arc (150:-150:1);
 \draw[ thick, -latex] (5, 4.5)  -- (5, 6.5) ;
%  \draw[ thick, -Straight Barb] (1., 1.5) arc (120:-200:1);
 % Value Dependency Edges on Initial Values:
 \draw[ thick, -latex,] (1.5, 1)  -- (4, 1) ;
 %
 \draw[ ultra thick, -latex, densely dotted,] (-0.6, 1.5)  to  [out=-220,in=220]  (-0.5, 6.5);
\draw[ ultra thick, -latex, densely dotted,]  (0.5, 6.5) to  [out=-30,in=30] (0.6, 1.6) ;
%  \draw[ ultra thick, -latex, densely dotted,]  (0.5, 10)  to  [out=-50,in=50] (0.5, 4);
 % Control Dependency
 \draw[ thick,-latex] (1.5, 7)  -- (4, 6) ;
 \draw[ thick,-latex] (1.5, 4)  -- (4, 6) ;
 \draw[ thick,-latex] (1.5, 1)  -- (4, 6) ;
%  \draw[ thick,-latex] (1.5, 10)  -- (4, 6) ;
 \end{tikzpicture}
 \caption{}
    \end{centering}
    \end{subfigure}
    \begin{subfigure}{.31\textwidth}
        \begin{centering}
        \begin{tikzpicture}[scale=\textwidth/11cm,samples=200]
    % Variables Initialization
    \draw[] (5, 1) circle (0pt) node{{ $x^1: {}^1_{1}$}};
    % Variables Inside the Loop
     \draw[] (0, 7) circle (0pt) node{{ $y^5: {}^{k}_{1}$}};
     \draw[] (0, 4) circle (0pt) node{{ $\mathbf{p^6: {}^{k}_{1}}$}};
     \draw[] (0, 1) circle (0pt) node{{ $\mathbf{x^7: {}^{k}_{1}}$}};
     % Counter Variables
     \draw[] (5, 7) circle (0pt) node {{$j^0: {}^{1}_{0}$}};
     \draw[] (5, 4) circle (0pt) node {{ $j^3: {}^{k}_{0}$}};
     %
% Value Dependency Edges:
 \draw[ thick, -latex,]  (0, 3.5) -- (0, 1.5) ;
%  \draw[ thick, -Straight Barb] (1, 4.2) arc (220:-100:1);
 \draw[ thick, -Straight Barb] (6.5, 4.5) arc (150:-150:1);
 \draw[ thick, -latex] (5, 4.5)  -- (5, 6.5) ;
%  \draw[ thick, -Straight Barb] (1., 1.5) arc (120:-200:1);
 % Value Dependency Edges on Initial Values:
 \draw[ thick, -latex,] (1.5, 1)  -- (4, 1) ;
 %
 \draw[ ultra thick, -latex, densely dotted,] (-0.6, 1.5)  to  [out=-220,in=220]  (-0.5, 6.5);
\draw[ ultra thick, -latex, densely dotted,]  (0.5, 6.5) to  [out=-30,in=30] (0.6, 1.6) ;
%  \draw[ ultra thick, -latex, densely dotted,]  (0.5, 10)  to  [out=-50,in=50] (0.5, 4);
 % Control Dependency
 \draw[ thick,-latex] (1.5, 7)  -- (4, 6) ;
 \draw[ thick,-latex] (1.5, 4)  -- (4, 6) ;
 \draw[ thick,-latex] (1.5, 1)  -- (4, 6) ;
%  \draw[ thick,-latex] (1.5, 10)  -- (4, 6) ;
     \end{tikzpicture}
     \caption{}
        \end{centering}
        \end{subfigure}
        \vspace{-0.4cm}
\caption{(a) The multiple rounds odd example 
(b) The execution-based dependency graph
(c) The program-based dependency graph graph from $\THESYSTEM$.}
    \label{fig:overappr_example}
    % \vspace{-0.5cm}
\end{figure}
}
%
\end{example}

\paragraph*{Motivation}
In static program analysis framework $\THESYSTEM$, specifically on the dependency quantity, 
I adopt the reachability bound analysis technique to estimate this dependency quantity.
% In existing static reachability bound analysis, 
However, it isn't precise enough w.r.t. the execution-based reachability bound on every program command.
It comes across an over-approximation on estimation due to its path-insensitive nature,
as shown in the Example~\ref{ex:multipleRoundsOdd}.
% It is 
In this example, the estimated adaptivity by static analysis in Chapter~\ref{ch:static} 
is $1 + 2*k$ where $k$ is the initial value of the input variable.
However, the formal \emph{adaptivity} is only $1 + k$ from the execution-based analysis.

This is resulted from the  path-insensitive of the dependency quantity in
second stage of the static analysis in Section~\ref{sec:static-adapfun}
It occurs when the control flow can be decided in a particular way in front of conditional branches, 
while the static analysis fails to witness. 
% As shown in Example~\ref{ex:overapproximate}.
\begin{example}
[Multiple Rounds Odds Algorithm]
\label{ex:overapproximate}
The $\THESYSTEM$ comes across an over-approximation on the estimation due to its path-insensitive nature. 
It occurs when the control flow can be decided in a particular way in front of conditional branches, while the static analysis fails to witness. 

We show the over-approximation, in Figure~\ref{fig:overappr_example}(a),
we call it a multiple rounds odd iteration algorithm. In this algorithm, at line 5 of every iteration, 
a query $\query(\chi[x])$ based on previous query results stored in $x$ is asked by the analyst like in the multiple rounds strategy. The difference is that only the query answers from the even iterations ($i =0, 2, \cdots $) are 
% used to $b$. 
used in the query 
in line 7, $\query(\chi[\ln(y)])$.
  Because the execution trace only updates 
%   $b$ using the query answers at odd iterations, so the answers from even iterations do not affect the queries at odd iterations. From the query-based dependency graph in Figure~\ref{fig:overappr_example}(b), we can see that there is no edge from queries at odd iterations (such as $q_1,q_3,q_5$) to queries at even iteration(such as $q_2,q_4$). The longest path is dashed with a length $3$.  However, {\THESYSTEM} fails to realize that odd iteration will always execute then branch and even iteration means else branch, so its dependency graph considers both branches for every iteration. In this sense, the dependency graph by {\THESYSTEM} is similar to the one in the multiple rounds strategy. We show the estimated graph in Figure~\ref{fig:overappr_example}(c). The estimated upper bound is then, $5$, instead of $3$. 
$x$ using the query answers in even iterations, so the answers from odd iterations do not affect the queries in even iterations. 
From the execution-based dependency graph in Figure~\ref{fig:overappr_example}(b), 
we can see that the weight for the vertex $y^5$ is 
$w_k/2$. a function which takes any initial trace $\trace_0$, return the value of $k/2$ evaluated in $\trace_0$.  
However, {\THESYSTEM} fails to realize that odd iteration will always execute the then branch and even iteration means else branch, so 
% its dependency 
it considers both branches for every iteration. 
In this sense, the weight estimated for $y^5$ and $p^6$ are both 
$k$ as in Figure~\ref{fig:overappr_example}(c).
As a result, {\THESYSTEM}  estimates the longest walk from Figure~\ref{fig:overappr_example}(c),
$y^5  \to x^7  \to y^5  \to \cdots \to x^7  $ with each vertex visited $k$ times,
as the dotted arrows. 
And the adaptivity computed 
% estimated from the program-based dependency graph graph from by finding the walk with the longest query length 
is $1 + 2 * k$, instead of $1 + k$. 
% We omitted the estimated graph, which is identical to the graph in Figure~\ref{fig:overappr_example}(b). 
%

{ \small
\begin{figure}
\centering
    \begin{subfigure}{0.33\textwidth}
\centering
\small{
    \[
    %
    \begin{array}{l}
        \kw{multipleRoundsOdd}(k) \triangleq \\
        \clabel{ \assign{j}{k}}^{0} ; 
        \clabel{ \assign{x}{\query(\chi[0])} }^{1} ; \\
            \ewhile ~ \clabel{j > 0}^{2} ~ \edo ~ 
            \Big(
             \clabel{\assign{j}{j-1}}^{3} ;\\
             \eif(\clabel{j \% 2 == 0}^{4}, \\
             \clabel{\assign{y}{\chi[x]}}^{5}, 
             \clabel{\assign{p}{\chi[x]}}^{6});\\                            
             \clabel{\assign{x}{\query(\chi(\ln(y)))} }^{7} \Big)
        \end{array}
    \]
}
 \caption{}
    \end{subfigure}
%
\begin{subfigure}{.31\textwidth}
    \begin{centering}
    \begin{tikzpicture}[scale=\textwidth/11cm,samples=200]
% Variables Initialization
\draw[] (5, 1) circle (0pt) node{{ $x^1: {}^{w_1}_{1}$}};
% Variables Inside the Loop
 \draw[] (0, 7) circle (0pt) node{{ $y^5: {}^{w_k/2}_{1}$}};
 \draw[] (0, 4) circle (0pt) node{{ $p^6: {}^{w_k/2}_{1}$}};
 \draw[] (0, 1) circle (0pt) node{{ $x^7: {}^{w_k}_{1}$}};
 % Counter Variables
 \draw[] (5, 7) circle (0pt) node {{$j^0: {}^{w_1}_{0}$}};
 \draw[] (5, 4) circle (0pt) node {{ $j^3: {}^{w_k}_{0}$}};
 %
 % Value Dependency Edges:
 \draw[ thick, -latex,]  (0, 3.5) -- (0, 1.5) ;
%  \draw[ thick, -Straight Barb] (1, 4.2) arc (220:-100:1);
 \draw[ thick, -Straight Barb] (6.5, 4.5) arc (150:-150:1);
 \draw[ thick, -latex] (5, 4.5)  -- (5, 6.5) ;
%  \draw[ thick, -Straight Barb] (1., 1.5) arc (120:-200:1);
 % Value Dependency Edges on Initial Values:
 \draw[ thick, -latex,] (1.5, 1)  -- (4, 1) ;
 %
 \draw[ ultra thick, -latex, densely dotted,] (-0.6, 1.5)  to  [out=-220,in=220]  (-0.5, 6.5);
\draw[ ultra thick, -latex, densely dotted,]  (0.5, 6.5) to  [out=-30,in=30] (0.6, 1.6) ;
%  \draw[ ultra thick, -latex, densely dotted,]  (0.5, 10)  to  [out=-50,in=50] (0.5, 4);
 % Control Dependency
 \draw[ thick,-latex] (1.5, 7)  -- (4, 6) ;
 \draw[ thick,-latex] (1.5, 4)  -- (4, 6) ;
 \draw[ thick,-latex] (1.5, 1)  -- (4, 6) ;
%  \draw[ thick,-latex] (1.5, 10)  -- (4, 6) ;
 \end{tikzpicture}
 \caption{}
    \end{centering}
    \end{subfigure}
    \begin{subfigure}{.31\textwidth}
        \begin{centering}
        \begin{tikzpicture}[scale=\textwidth/11cm,samples=200]
    % Variables Initialization
    \draw[] (5, 1) circle (0pt) node{{ $x^1: {}^1_{1}$}};
    % Variables Inside the Loop
     \draw[] (0, 7) circle (0pt) node{{ $y^5: {}^{k}_{1}$}};
     \draw[] (0, 4) circle (0pt) node{{ $\mathbf{p^6: {}^{k}_{1}}$}};
     \draw[] (0, 1) circle (0pt) node{{ $\mathbf{x^7: {}^{k}_{1}}$}};
     % Counter Variables
     \draw[] (5, 7) circle (0pt) node {{$j^0: {}^{1}_{0}$}};
     \draw[] (5, 4) circle (0pt) node {{ $j^3: {}^{k}_{0}$}};
     %
% Value Dependency Edges:
 \draw[ thick, -latex,]  (0, 3.5) -- (0, 1.5) ;
%  \draw[ thick, -Straight Barb] (1, 4.2) arc (220:-100:1);
 \draw[ thick, -Straight Barb] (6.5, 4.5) arc (150:-150:1);
 \draw[ thick, -latex] (5, 4.5)  -- (5, 6.5) ;
%  \draw[ thick, -Straight Barb] (1., 1.5) arc (120:-200:1);
 % Value Dependency Edges on Initial Values:
 \draw[ thick, -latex,] (1.5, 1)  -- (4, 1) ;
 %
 \draw[ ultra thick, -latex, densely dotted,] (-0.6, 1.5)  to  [out=-220,in=220]  (-0.5, 6.5);
\draw[ ultra thick, -latex, densely dotted,]  (0.5, 6.5) to  [out=-30,in=30] (0.6, 1.6) ;
%  \draw[ ultra thick, -latex, densely dotted,]  (0.5, 10)  to  [out=-50,in=50] (0.5, 4);
 % Control Dependency
 \draw[ thick,-latex] (1.5, 7)  -- (4, 6) ;
 \draw[ thick,-latex] (1.5, 4)  -- (4, 6) ;
 \draw[ thick,-latex] (1.5, 1)  -- (4, 6) ;
%  \draw[ thick,-latex] (1.5, 10)  -- (4, 6) ;
     \end{tikzpicture}
     \caption{}
        \end{centering}
        \end{subfigure}
        \vspace{-0.4cm}
\caption{(a) The multiple rounds odd example 
(b) The execution-based dependency graph
(c) The program-based dependency graph graph from $\THESYSTEM$.}
    \label{fig:overappr_example}
    % \vspace{-0.5cm}
\end{figure}
}
%
\end{example}
% as follo
\paragraph*{Overview}
The accurate static adaptivity analysis is planned to be improved in following two steps.
\begin{enumerate}
    \item \textbf{Path Sensitive Reachability Bound Algorithm Design}
    The imprecision comes from the second stage of the static program analysis.
    The algorithm in this stage is to statically estimate the 
    reachability bound for every location of program.
    Specifically, this reachability bound analysis algorithm isn't path sensitive. 
    There are many works in the program complexity analysis area, estimating the path-sensitive loop bound 
    or reachability bound
    \cite{GustafssonEL05, HumenbergerJK18}, 
    \cite{BrockschmidtEFFG16,AlbertAGP08,AliasDFG10,Flores-MontoyaH14}, 
    \cite{GulwaniZ10, SinnZV17,GulwaniJK09, GulwaniMC09, abs-2203-04243}. 
    But they aren't analyzing the reachability
    bound in a path-sensitive way.
    \\
    Motivated by this observation, 
    in Section~\ref{sec:refine-static} 
    I first design path-sensitive reachability bound analysis algorithm computing the 
    reachability bounds for every labeled command taking the different paths inside while loop into consideration.
% Compared to just computing the reachability bound for the while loop command, the new methodology improves the accuracy of the 
% reachability bound for every labeled command.
% \\
    \item \textbf{Accurate $\THESYSTEM$ via Path Sensitive Reachability Bound Algorithm}

    Then in order to improve the static program adaptivity analysis, specifically
    the $\THESYSTEM$ framework presented in Chapter~\ref{ch:static}, I apply the 
    path sensitive reachability bound algorithm into it in Section~\ref{sec:refine-static}.
    % it. 
    Following the same static program analysis $\THESYSTEM$ framework
    %  in Section~\ref{sec:static},
% and design a new algorithm for this stage.
% Then, I will 
this improvement is designed in following three steps.
% result will be applied to estimate the dependency quantity as follows,
\begin{itemize}
    \item Improved $\THESYSTEM$ constructs different abstract control flow graph based on the requirement of the newly-designed
    path-sensitive reachability bound algorithm in Section~\ref{sec:refine-static-datadep}
    \item Then it performs the same algorithm for analyzing the dependency relation in the first stage, still in Section~\ref{sec:refine-static-datadep}.
    \item But in the second stage in Section~\ref{sec:refine-static-reachability}, the improved $\THESYSTEM$ 
    apply the newly-designed  path-sensitive reachability bound algorithm for analyzing the dependency quantity.
    The result from the new algorithm 
    give a tighter approximation on the dependency quantity,
    specifically on the reachability times of the program's every location.
    \item Based on the two results analyzed from the steps above, 
    i.e., the estimated dependency relation and the tighter reachability bound, 
    the improved $\THESYSTEM$ will construct
a more accurate program-based data dependency graph.
%  constructed in Section~\ref{sec:static-adapt}
Then, by applying the same algorithm as in Section~\ref{sec:static-adapt}, 
the improved $\THESYSTEM$ will compute a tighter \emph{adaptivity} for a program.
This is presented in Section~\ref{sec:refine-static-adapt}%
\end{itemize}
\end{enumerate}

\subsubsection{Path Sensitive Reachability Bound Algorithm Design}
\label{sec:refine-static-psreachability}
The imprecision comes from the second stage of the static program analysis.
The algorithm in this stage is to statically estimate the 
reachability bound for every location of program.
Specifically, this reachability bound analysis algorithm isn't path sensitive. 
There are many works in the program complexity analysis area, estimating the path-sensitive loop bound 
or reachability bound
\cite{GustafssonEL05, HumenbergerJK18}, 
\cite{BrockschmidtEFFG16,AlbertAGP08,AliasDFG10,Flores-MontoyaH14}, 
\cite{GulwaniZ10, SinnZV17,GulwaniJK09, GulwaniMC09, abs-2203-04243}. 
But they aren't analyzing the reachability
bound in a path-sensitive way.
\\
Motivated by this observation, I will first design path-sensitive reachability bound analysis algorithm computing the 
reachability bounds for every labeled command taking the different paths inside while loop into consideration.
% Comparing to just compute the reachability bound for the while loop command, new methodology improves the accuracy of the 
% reachability bound for every labeled command.

\todo{Path Sensitive Reachability Bound Analysis Algorithm And Papers}
\\
\subsubsection{Accurate $\THESYSTEM$ via Path Sensitive Reachability Bound Algorithm}
\label{sec:refine-static}
Then follow the $\THESYSTEM$ framework,
% and design new algorithm for this stage.
% Then, I will 
this improved analysis result will be applied to estimate the dependency quantity.
Specifically, the newly designed path sensitive reachability bound analysis will 
give tighter bound for program's every execution location.
Then, based on the tighter bound,
the program-based data dependency graph constructed in Section~\ref{sec:static-adapt}
will have a more accurate estimated weight, for each vertex.
%
As shown in Figure~\ref{fig:multipleRoundsOdd_example}, 
through the newly-designed path sensitive reachability bound analysis,
ideally the weight for vertex $p^6$ and $x^7$ will be $\frac{k}{2}$.

With extension as follows,
\paragraph{Extended Static Data Dependency Analysis}
\label{sec:refine-static-datadep}
by using the results of Reaching definition analysis results, specifically $\live(l, c)$ for every label in a program $c$, we refine the vertices and edges in the $\absG$ graph 
by generating the set of feasible data-flow between labeled variables as follows,
\begin{defn}[Feasible Data-Flow]
  \label{def:feasible_flowsto}
  Given a program $c$ and two labeled variables $x^i, y^j$  in this program, 
  $\flowsto(x^i, y^j, c)$ is 
    {\footnotesize
    \[
   \begin{array}{ll}
    \flowsto(x^i, y^j, \clabel{\assign{x}{\expr}}{}^l)  & \triangleq (x^i, y^j) \in \{ (y^i, x^l) | y \in \mathsf{FV}(\expr) 
    % \land (y,i) \in \live(l, \clabel{\assign{x}{\expr}}^l) \}  \\
    \land y^i \in \live(l, \clabel{\assign{x}{\expr}}^l) \}  \\
    \flowsto(x^i, y^j, \clabel{\assign{x}{\query(\qexpr)}}{}^l)  & \triangleq (x^i, y^j) \in \{ (y^i, x^l) | y \in \mathsf{FV}(\qexpr) 
    % \land (y,i) \in \live(l,\clabel{\assign{x}{\query(\qexpr)}}^l) \}  \\
    \land y^i \in \live(l,\clabel{\assign{x}{\query(\qexpr)}}^l) \}  \\
    \flowsto(x^i, y^j, [\eskip]^{l}) & \triangleq \efalse \\
    \flowsto(x^i, y^j, \eif ([b]^l, c_1, c_2))  & \triangleq \flowsto(x^i, y^j, c_1) \lor \flowsto(x^i, y^j, c_2) \\ 
        & \lor (x^i, y^j) \in
       \{(x^i,y^j) | x \in \mathsf{FV}(b) \land 
      %  (x,i) 
      x^i \in \live(l, \eif ([b]^l, c_1, c_2)) \land  y^j \in \lvar(c_1) \\
       &\lor (x^i, y^j) \in \{(x^i,y^j) | x \in \mathsf{FV}(b) \land 
      %  (x,i) 
      x^i\in \live(l, \eif ([b]^l, c_1, c_2))  \land  y^j \in \lvar(c_2) \\
       \flowsto(x^i, y^j, \ewhile [b]^l \edo c_w)  & \triangleq  \flowsto(x^i, y^j, c_w)  \lor
       \\ & 
       (x^i, y^j) \in  \{(x^i,y^j) | x \in \mathsf{FV}(b) \land 
      %  (x,i) 
      x^i \in \live(l,   \ewhile [b]^l \edo c_w) \land  y^j \in \lvar(c_w) \\
      \flowsto(x^i, y^j, c_1 ;c_2)  & \triangleq \flowsto(x^i, y^j, c_1) \lor \flowsto(x^i, y^j, c_2) \\
      {\highlight{\flowsto(x^i, y^j, \clabel{\efun}^l: f ~ (r, x_1, \ldots, x_n) := c) }}
       & \triangleq \efalse\\
       {\highlight{\flowsto(x^i, y^j, \clabel{\assign{x}{\ecall(f, e_1, \ldots, e_n)}}^l )} } 
       &     
       \triangleq
       \flowsto(x^i, y^j, \clabel{\assign{x_i}{e_i}}^{(l,i)}) \lor
       \flowsto(x^i, y^j, \clabel{c^{+n}}^l) 
       \\ & \quad
       \lor
       \left(\flowsto(x^i, y^j, \clabel{\assign{x}{r}}^{l}) 
       \land f(r, x_1, \ldots, x_n) := c\in \live(l, c) \right)
   \end{array}
   \]
   }
   \end{defn}
%
We prove that this \emph{Feasible Data-Flow} relation is a sound approximation 
of the \emph{Variable May-Dependency} relation over labeled variables for every program,
in Appendix~\ref{apdx:flowsto_soundness_extend}.
%
\paragraph{Extended Static Data Dependency Quantity Analysis}
\label{sec:refine-static-reachability}
\paragraph*{Data Dependency Quantity over pair of Labeled Variables }
Then we define the estimated directed edges
% for each vertex in $\progV(c)$,
between vertices $({x}_1^{i}, w_1)$  
and $({x}_2^{j}, w_2)$ 
where ${x}_1^{i}, {x}_2^{j} \in \lvar(c)$,
as a set of triples 
% $\progW(c) \in \mathcal{P}(\mathcal{LV} \times \mathcal{LV} \times EXPR(\constdom))$ 
$\progE(c) \in \mathcal{P}(\mathcal{LV} \times \mathcal{A}_{\lin} \times \mathcal{LV})$
% is the set of pairs 
% The weight for each vertex in $\progV(c)$ is computed 
indicating a directed edge from the first vertex to the second one in each pair
as follows,
\highlight{
  \[
    \progE^0(c) \triangleq 
    \left\{ 
    ({x}_1^{i}, w, {x}_2^{j}) \in \mathcal{LV} \times 
    \mathcal{A}_{\kw{in}} \times \mathcal{LV}
    ~ \middle\vert ~
    \begin{array}{l}
      {x}_1^{i}, {x}_2^{j} \in \lvar(c)
    \land
      % \\
      \exists n \in \mathbb{N}, z_1^{r_1}, \cdots, z_n^{r_n} \in \lvar_{{c}} \sthat 
      n \geq 0 \land
      \\
      \flowsto(x^i,  z_1^{r_1}, c) 
      \land \cdots \land \flowsto(z_n^{r_n}, y^j, c) 
    \end{array}
    \right\}
    \]
}
with compute the weight for each edge in $\progE(c)$ computed above,
% % as a set of pairs $\progW(c) \in \mathcal{P}(\mathcal{LV} \times \mathcal{LV} \times EXPR(\constdom))$ 
% as a set of pairs 
% % is the set of pairs 
% % The weight for each vertex in $\progV(c)$ is computed 
% mapping each $x^l \in \progV(c)$ to a symbolic expression over $\constdom$. Since symbolic expression 
% over $\constdom$ is a subset of arithmetic expressions,
% we use $\mathcal{A}_{in}$ denotes the arithmetic expression 
% over $\mathcal{N}$ and input variable and $\progW(c) \in \mathcal{P}(\mathcal{LV} \times \mathcal{A}_{in})$ 
% as follows,
\highlight{
% :
% \\
 \[
   \progE(c) \triangleq
   \left\{ (x^i, w, y^j) 
\mid
(x^i, w, y^j) \in \progE^0(c) \land 
% w = \max\limits_{\absevent = (i, \_, j)} \{ \absclr(\absevent)\} 
w = \max \left\{ \absclr(\absevent) ~\mid~ \absevent \in \absflow(c) \land \absevent = (i, \_, j) \right\} 
\right\}.
\]
}
%
% Since 
We prove that this 
% symbolic expression is the upper bound for $x^l$'s 
symbolic expression $w$ for edge $(x^i, w, y^j) \in \progE(c)$
 is a sound upper bound of 
the weight for the same edge $(x^i, w', y^j)$ in Program's execution-based dependency graph in Appendix~\ref{apdx:edgeweight_soundness}.
% The maximum visiting times of $x^l$ over all execution traces of $c$ in Appendix~\ref{apdx:reachability_soundness}. 
%
\begin{thm}[Soundness of the Edge Weight Estimation]
  \label{thm:edgeweight_soundness}
Given a program ${c}$ with its program-based dependency graph 
$\progG = (\progV, \progE)$,
$\traceG = (\traceV, \traceE)$, we have:
%
\[
\forall (x^l, w_{t}) \in \traceW,
(x^l, w_{p}) \in \progW, \vtrace \in \mathcal{T} \sthat
\config{{c}, \trace} \to^{*} \config{\eskip, \trace_0 \tracecat \vtrace'} 
\land 
\config{w_{p}, \trace} \earrow v
\implies
% \right\} 
\leq 
w_{t}(\trace) \leq v
\]
\end{thm}

% this improved analysis result will be applied to estimate the dependency quantity as follows,
% \begin{itemize}
%     \item $\THESYSTEM$ will construct different abstract control flow graph based on the requirement of the newly-designed
%     path-sensitive reachability bound algorithm.

%     \item The improved $\THESYSTEM$ will perform the same algorithm for analyzing the dependency relation in the first stage.
%     \item But in the second stage, the improved $\THESYSTEM$ 
%     apply the newly-designed  path-sensitive reachability bound algorithm for analyzing the dependency quantity.
%     The result from the new algorithm 
%     give a tighter approximation on the dependency quantity,
%     specifically on the reachability times of the program's every location.
%     \item Based on the two results analyzed from the steps above, 
%     i.e., the estimated dependency relation and the tighter reachability bound, the improved $\THESYSTEM$ will construct
% a more accurate program-based data dependency graph.
% %  constructed in Section~\ref{sec:static-adapt}
% Then, by applying the same algorithm as in Section~\ref{sec:static-adapt}, 
% the improved $\THESYSTEM$ will compute a tighter \emph{adaptivity} for a program.%
% \end{itemize}
\paragraph{Extended Static Adaptivity Analysis}
\label{sec:refine-static-adapt}
we build the estimated data dependency graph based on the above program static analysis as follows:
\\
\highlight{
  \[
    % \progG(c) = (\progV(c), \progE(c), \progW(c), \progF(c))
    \progG(c) = (\progV(c), \progE(c))
    \]
}
with $\progV(c)$ and  $\progE(c)$
as computed in each steps above.
%
This program-based graph program-based graph has a similar topology structure as 
% the one
% of 
the Execution-Based Dependency Graph. It has the same
vertices 
% and query annotations, 
but approximated edges and weights.  
% The algorithm computation is 
It is formally defined in Definition~\ref{def:prog_graph}.
% Through the reachable definition set on every label,
% we remove the edges between labels where the variables associated to that labeled command isn't reachable from the second location.
%\absG(c) =(\absV, \absE, \absW)
\begin{defn}
  [Program-Based Dependency Graph]
  \label{def:improved_prog_graph}
  % [Program-Based Weighted Data Dependency Graph Generation Algorithm]
% \label{def:analyz_dcfg}
Given a program $c$, with its abstract weighted control flow graph $\absG(c) = (\absV, \absE, \absW)$ and 
feasible data flow relation $\flowsto(x^i, y^j, c)$ for every $x^i, y^j \in \lvar_c$, its Program-Based Weighted Data Dependency Graph
$\progG(c) = (\progV, \progE)$,
is generated as follows,
{\footnotesize
\[
\begin{array}{lcl}
\progV(c) & \triangleq &
% \bigcup
% \begin{array}{l}
\left\{ (x^l, w) \in  \mathcal{LV} \times \mathcal{A}_{in}
\mid
x^l \in \lvar_{{c}} \land (l, w) \in \absW(c)
\right\}
\\
\progE(c) & \triangleq &
   \Big\{ (x^i, w, y^j) \in \mathcal{LV} \times 
   \mathcal{A}_{\kw{in}} \times \mathcal{LV}
~\mid~
  \\ & & \quad 
x^i, y^j \in \lvar(c) \land \flowsto(x^i, y^j, c) \land
  \exists n \in \mathbb{N}, z_1^{r_1}, \cdots, z_n^{r_n} \in \lvar_{{c}} \sthat 
  n \geq 0 
  % \\ & & \quad 
  % \flowsto(x^i,  z_1^{r_1}, c) 
  \land \cdots \land \flowsto(z_n^{r_n}, y^j, c) 
  \\ & & \quad 
  \land
  w = \max \left\{ \absclr(\absevent) ~\mid~ \absevent \in \absflow(c) \land \absevent = (i, \_, j) \right\} 
\Big\}.
\end{array}
\] }
\end{defn}
%  from refined weighted-labeled data-flow graph}
Then, we define the adaptivity upper bound for a program $c$ under the extended definition.
% Given a program ${c}$, we generate
\\
With
% its 
$c$'s program-based data dependency graph $\progG({c})$ approximated above,
%
its adaptivity upper bound 
% Defined in Definition~\ref{def:prog_adapt} as 
%
is estimated as
% Then the adaptivity bound based on program analysis for ${c}$ 
% is the number of query vertices on a finite walk in $\progG({c})$. This finite walk satisfies:
% \begin{itemize}
% \item the number of query vertices on this walk is maximum
% \item the visiting times of each vertex $v$ on this walk is bound by its reachability bound $\weights(v)$.
% \end{itemize}
the maximum query length over all finite walks in $\walks(\progG({c}))$ formally in Definition~\ref{def:prog_adapt}, 
and computed 
% is computed as the maximum query length over all finite walks in $\walks(\progG({c}))$, and computed 
in Algorithm~\ref{alg:adpt_alg}.
%
% It is formally defined in \ref{def:prog_adapt}.
% defined formally as follows.
%
% %
% \begin{defn}
% [{Program-Based Adaptivity}].
% \label{def:prog_adapt}
% \\
% {
% Given a program ${c}$ and its program-based graph 
% $\progG({c})$
% %  = (\vertxs, \edges, \weights, \qflag)$,
% %
% the program-based adaptivity for $c$ is defined as%
% \[
% \progA({c}) 
% \triangleq \max
% \left\{ \qlen(k)\ \mid \  k\in \walks(\progG({c}))\right \}.
% \]
% }
% \end{defn} 

% We use $\walks(\progG(c))$ represents the walks over the program-based dependency graph for $c$.
Different from the finite walk on a program $c$'s execution based graph,
%  $\traceG(c)$, 
% $k \in \walks(\progG(c))$ 
the finite walk in $\progG(c)$ doesn't rely on initial trace.
The occurrence times of every $v_i $ in $k$'s vertex sequence is bound by 
an arithmetic expression $w_i$ where $(v_i, w_i) \in \progV(c)$, is $v_i$'s estimated weight. 
% Then $\qlen(k) \in \mathcal{A}_{in}$ as well. 
% The full definition for $\walks(\progG(c))$ and $\qlen$ over $\walks(\progG(c))$ is in Apdix.
%
Formally defined as follows.
\begin{defn}[Finite Walk on Program-Based Dependency Graph ($k$)].
  \label{def:prog_finitewalk}
  \\
%   Given a program $c$'s execution-based dependency graph $\traceG({c})(\trace)$, 
%   a \emph{finite walk} $fw$ in $\traceG({c})(\trace)$ is a sequence of edges $(e_1 \ldots e_{n - 1})$ 
%   for which there is a sequence of vertices $(v_1, \ldots, v_{n})$ such that:
%   \begin{itemize}
%       \item $e_i = (v_{i},v_{i + 1})$ for every $1 \leq i < n$.
%       \item every vertex $v \in \traceV({c}) $ appears in $(v_1, \ldots, v_{n})$ at most 
%       \wq{$\traceW({c})(\trace)$} times.  
%   \end{itemize}
%   %
%   The length of $fw$ is the number of vertices in its vertex sequence, i.e., $\len(k) = n$.
  Given a program $c$'s program-based dependency graph 
  $\progG({c}) = (\progV(c), \progE(c))$
  % , \progW(c), \progF(c))$, 
  a \emph{finite walk} $k$ in $\traceG({c})$ is
  % function $k: \mathcal{T} \to $ 
  % sequence of edges.
  % For a initial trace $\trace_0 \in \mathcal{T}$, 
  % $k(\trace_0)$ is
  a sequence of edges $(e_1 \ldots e_{n - 1})$ 
  for which there is a sequence of vertices 
  $(v_1, \ldots, v_{n})$ such that:
  \begin{itemize}
      \item 
      \highlight{
        $e_i = (v_{i},w_i, v_{i + 1}) \in \progE(c)$ for every $1 \leq i < n$,
        and occurrence times of $e_i$ smaller than $w_i$.
        }
      \item 
      \highlight{
        every vertex $(v_i, w_i) \in \progV(c)$,
       $v_i$ appears in $(v_1, \ldots, v_{n})$ at most 
    %   \wq{$\traceW({c})(\trace)$} 
    $w_i$
      times. } 
  \end{itemize}
  %
  The length of $k$ is the number of vertices in its vertex sequence, i.e., $\len(k) = a$.
 \end{defn}
  We abuse the notation $\walks(\progG(c))$ represents the walks over the program-based dependency graph for $c$.
Different from the walks on a program $c$'s execution based graph,
 $k \in \walks(\traceG(c))$, 
$k \in \walks(\progG(c))$ doesn't rely on initial trace.
The occurrence times of every $v_i $ in $k$'s vertex sequence is bound by 
an arithmetic expression $w_i$ where $(v_i, w_i) \in \progV(c)$, is $v_i$'s estimated weight. 
% Notice here, for a walk in $\progG(c)$, the occurrence times of every vertex in vertices sequence, 
%  and its 
 The length of a finite walk $k \in \walks(\progG(c))$ is an arithmetic expression
 as well, i.e., $\len(k) \in \mathcal{A}_{in}$

 Then the query length of a finite walk in  $\progG(c)$ is an arithmetic expression as well as follows,
%  $\qlen(k) \in \mathcal{A}_{in}$ as well. 
% The adaptivity upper bound 
% is estimated as
% Then the adaptivity bound based on program analysis for ${c}$ 
% is the number of query vertices on a finite walk in $\progG({c})$. This finite walk satisfies:
% \begin{itemize}
% \item the number of query vertices on this walk is maximum
% \item the visiting times of each vertex $v$ on this walk is bound by its reachability bound $\weights(v)$.
% \end{itemize}
\begin{defn}[Query Length of the Finite Walk on Program-Based Dependency Graph ($\qlen$)]
  \label{def:qlen}
  % Given 
  % % labelled weighted graph $G = (\vertxs, \edges, \weights, \qflag)$, 
  % a program $c$'s execution-based dependency graph $\traceG(c)(\trace)$
  %  and a \emph{finite walk} $k$ in $\traceG(c)(\trace)$ with its vertex sequence $(v_1, \ldots, v_{n})$, 
  % %  the length of $k$ w.r.t query is defined as:
  % The query length of $k$ is the number of vertices which correspond to query variables in $(v_1, \ldots, v_{n})$ as follows, 
  % \[
  %   \qlen(k) = \len\big( v \mid v \in (v_1, \ldots, v_{n}) \land \qflag(v) = 1 \big)
  % \]
  % , where $\big(v \mid v \in (v_1, \ldots, v_{n}) \land \qflag(v) = 1 \big)$ is a subsequence of $(v_1, \ldots, v_{n})$.
  Given 
  % labelled weighted graph $G = (\vertxs, \edges, \weights, \qflag)$, 
  a program $c$'s execution-based dependency graph 
  $\progG({c}) = (\progV(c), \progE(c))$, 
   and a \emph{finite walk} $k \in \walks(\progG(c))$,
  %  $k$ in $\traceG(c)(\trace)$
  %  $k \in \walks(\traceG(c))$. 
  %  with its vertex sequence $(v_1, \ldots, v_{n})$, 
  %  the length of $k$ w.r.t query is defined as:
  The query length of $k$, $\qlen(k) \in \mathcal{A}_{in}$ 
  % is a function $\qlen(k): \mathcal{T} \to \mathbb{N}$, such that with an initial trace  $\trace_0 \in \mathcal{T}$, 
  % $\qlen(k)(\trace_0)$ 
  is the number of vertices which correspond to query variables in the vertices sequence of the this walk $k$
  $(v_1, \ldots, v_{n})$ as follows, 
  \[
    \qlen(k) = |\big( v \mid v \in (v_1, \ldots, v_{n}) \land v \in \qvar(c) \big)|.
  \]
  \end{defn}
% is computed as the maximum query length over all finite walks in $\walks(\progG({c}))$, and computed 
%
% It is formally defined in \ref{def:prog_adapt}.
% defined formally as follows.
%
%
\begin{defn}
[{Program-Based Adaptivity}].
\label{def:prog_adapt}
\\
{
Given a program ${c}$ and its program-based graph 
$\progG({c})$
%  = (\vertxs, \edges, \weights, \qflag)$,
%
the program-based adaptivity for $c$ is 
% a function $\progA({c}): \mathcal{T} \to\mathbb{N} $,
% for an initial trace $\trace_0 \in \mathcal{T}$,
defined as%
\[
\progA({c})
\triangleq \max
\left\{ \qlen(k) \ \mid \  k \in \walks(\progG(c))\right \}.
\]
}
\end{defn}
% \subsection{Static Adaptivity Computation towards Completeness}
% \label{sec:furthers-adaptcomplete}
% The Algorithm is conditional completeness as proved in appendix, but Algorithm~\ref{alg:adaptscc} isn't.
% In the algorithm design at line: in Algorithm~\ref{alg:adaptscc}, an over-approximation happens here. 

% As  following motivating example shows.
% \subsubsection{Proposed Methodology}
% \label{sec:furthers-adaptcomplete-methodology}
% %
% 1. looking into more over-approximated example and summarize the common properties of these examples.
% \\
% 2. Modify the Algorithm~\ref{alg:adaptscc}, targeting the line: 12 of the algorithm. 
% The goal is to reduce the over-approximation in computing adaptivity statically.

% \subsection{Accurate Full-Spectrum Adaptivity Analysis through Examples}
% \label{sec:improved-examples}
% Through the three steps above, I give a more accurate formalization of the intuitive \emph{adaptivity}.
% \begin{example}[\highlight{Adapativity for Multiple Rounds Single Example}]
    \label{ex:multipleRoundSingle}
    % The program's adaptivity in our formal model,
    % % which we define over the program's execution-based dependency graph from the dynamic 
    % % analysis 
    % in Definition~\ref{def:trace_adapt} also
    %  comes across an over-approximation on the program's
    %  intuitive adaptivity rounds.
    % It is resulted from difference between its weight calculation and the \emph{variable may-dependency} definition.
    % It occurs when the weight is computed over the traces different from the traces used in 
    % witness the \emph{variable may-dependency} relation.
    % % control flow can be decided in a particular way in front of conditional branches, while the static analysis fails to witness. 
    % 
    % We use one example to show the over-approximated definition, 
    \highlight{As the program in Figure~\ref{fig:multipleRoundsSingle}(a),
    % This example is the variant of the multiple rounds strategy, 
    % we call it a multiple rounds odd iteration algorithm.
    % This example is still 
    which is a variant of the multiple rounds strategy, 
    % we call it a multiple rounds single iteration algorithm, 
    $\kw{multipleRoundSingle(k)}$ with input $k$.
    % as the input variable.
    In this algorithm, 
    at line 3 of every iteration, 
    a query $\query(\chi[z] + y)$ based on previous query results stored in $z$ and $y$ is asked by the analyst 
    as the $\kw{multipleRounds}$ strategy. 
    The difference is that only the query answers from the one single iterations ($j = 2 $) are 
    % used to $b$. 
    used in this query $\query(\chi[z] + y)$.
    Because the execution trace updates 
    %   $b$ using the query answers at odd iterations, so the answers from even iterations do not affect the queries at odd iterations. From the query-based dependency graph in Figure~\ref{fig:overappr_example}(b), we can see that there is no edge from queries at odd iterations (such as $q_1,q_3,q_5$) to queries at even iteration(such as $q_2,q_4$). The longest path is dashed with a length $3$.  However, {\THESYSTEM} fails to realize that odd iteration will always execute then branch and even iteration means else branch, so its dependency graph considers both branches for every iteration. In this sense, the dependency graph by {\THESYSTEM} is similar to the one in the multiple rounds strategy. We show the estimated graph in Figure~\ref{fig:overappr_example}(c). The estimated upper bound is then, $5$, instead of $3$. 
    $y$ by the constant $0$ for all the iterations where ($j \neq 2$) at line $5$ after the 
    query request at line $3$.
    In this way, all the query answers stored in $y$ will not be accessed in next query request at line $3$ in the iterations 
    where  ($j \neq 2$).
    Only query answer at one single iteration where ($j = 2 $) will be used in next query request
    $\query(\chi[z] + y)$ at line $3$.
    So the adaptivity for this example is $2$. 
    % so the answers from odd iterations do not affect the queries at even iterations. 
    % However, from the execution-based dependency graph in Figure~\ref{fig:overappr_example}(b), 
    However, our adaptivity model fails to realize that there is only dependency relation 
    between $y^3$ and $y^3$ in one single iteration, 
    not the others. 
    % there is no edge from queries at odd iterations (such as $q_1,q_3,q_5$) to queries at even iteration(such as $q_2,q_4$). The longest path is dashed with a length $3$.  
    As shown in the execution-based dependency graph in Figure~\ref{fig:multipleRoundsSingle}(b), 
    there is an edge from $y^3$ to itself representing the existence of \emph{Variable May-Dependency} from $y^3$ on itself,
    and the visiting times of labeled variable $y^3$ is 
    $w_k(\trace_0)$ with a initial trace $\trace_0$. 
    % will always execute then branch and even iteration means else branch
    % $k$.
    As a result, the walk with the longest query length 
    is
    $y^3  \to \cdots \to y^3 \to z^1 $ with the vertex $y^3$ visited $w_k(\trace_0)$,
    as the dotted arrows. 
    The adaptivity 
    % the Program-Based Dependency graph from {\THESYSTEM} by finding 
    based on
    this walk
    % walk with the longest query length 
    is $2$.
    % %
    The $\THESYSTEM$ is able to give us $2$,  as an accurate bound w.r.t this definition.}
        \begin{figure}
     \centering
    \quad
    \begin{subfigure}{.35\textwidth}
    \begin{centering}
    $
        \begin{array}{l}
            \kw{multiRoundsSin(k)}\\
               \clabel{ \assign{j}{k}}^{0} ; \\
                \clabel{\assign{z}{\query(0)} }^{1} ;   \\          
                \ewhile ~ \clabel{j > 0}^{2} ~ \edo ~ \\
                \Big(
                 \clabel{\assign{y}{\query(\chi[z]+y)} }^{3}  ; \\
              \eif(\clabel{ j \neq 2}^{4}, \\
              \qquad \clabel{ \assign{y}{0}}^{5},\clabel{\eskip}^{6});\\
              \clabel{\assign{j}{j - 1}}^{7}
         \Big);\\
            \end{array}
    $
    \caption{}
    \end{centering}
    \end{subfigure}
    \begin{subfigure}{.6\textwidth}
        \begin{centering}
        \begin{tikzpicture}[scale=\textwidth/15cm,samples=150]
    % Variables Initialization
    % \draw[] (-5, 1) circle (0pt) node{{ $z^1: {}^{w_1}_{1}$}};
    % \draw[] (-5, 7) circle (0pt) node{{$p^2: {}^{w_1}_{0}$}};
    \draw[] (-5, 4) circle (0pt) node{{ $z^1: {}^{w_1}_{1}$}};
    % Variables Inside the Loop
     \draw[] (0, 6) circle (0pt) node{{ $y^3: {}^{w_k}_{1}$}};
     \draw[] (0, 2) circle (0pt) node{{ $y^{5}: {}^{w_k}_{0}$}};
     % Counter Variables
     \draw[] (5, 6) circle (0pt) node {{$j^0: {}^{w_1}_{0}$}};
     \draw[] (5, 2) circle (0pt) node {{ $j^8: {}^{w_k}_{0}$}};
     %
     % Value Dependency Edges:
     \draw[ ultra thick, -Straight Barb, densely dotted,] (0.8, 7) arc (220:-100:1);
     % The Weight for this edge
     \draw[](1.2, 9.5) node 
     {\highlight{\footnotesize
            $\trace_0 \to 
            \left\{\begin{array}{ll}
               \env(\trace_0) k & \env(\trace_0) k  \leq 1 \\
           2 & \env(\trace_0) k \geq 2
            \end{array}\right\}
            $}};
     \draw[ thick, -latex] (-1, 6)  to  [out=-130,in=130]  
    % The Weight for this edge
    node [] {\highlight{$\trace_0 \to 1 $}} (-1, 2);
     % Value Dependency Edges on Initial Values:
     \draw[ ultra thick, -latex, densely dotted,] (-1.5, 6)  -- 
    % The Weight for this edge
    node [left] {\highlight{$\trace_0 \to \env(\trace_0) k $}} (-4, 4.7) ;
     %
     % Value Dependency For Control Variables:
     \draw[ thick, -Straight Barb] (6.5, 2.5) arc  (150:-150:1);
    % The Weight for this edge
    \draw[](8, 2) node [] {\highlight{$\trace_0 \to \env(\trace_0) k  $}};
     % Control Dependency
     \draw[ thick, -latex] (5, 2.5)  -- 
    % The Weight for this edge
    node [right] {\highlight{$\trace_0 \to \env(\trace_0) k $}} (5, 5.5);
     \draw[ thick,-latex] (1.5, 6)  -- (3.5, 6) ;
     \draw[ thick,-latex] (1.5, 1.8)  -- 
    % The Weight for this edge
    node [] {\highlight{$\trace_0 \to \env(\trace_0) k $}} (3.5, 6) ;
     \draw[ thick,-latex] (1.5, 6)  -- (3.5, 2) ;
     \draw[ thick,-latex] (1.5, 1.8)  -- (3.5, 2) ; 
    \end{tikzpicture}
     \caption{}
        \end{centering}
        \end{subfigure}
    % \end{wrapfigure}
    % \end{equation*}
    \vspace{-0.4cm}
     \caption{(a) The multi rounds single example
     (b) The execution-based dependency graph.}
    \label{fig:multipleRoundsSingle}
    \vspace{-0.5cm}
    \end{figure}
    \end{example}
% \begin{example}
[Multiple Rounds Odds Algorithm]
\label{ex:overapproximate}
The $\THESYSTEM$ comes across an over-approximation on the estimation due to its path-insensitive nature. 
It occurs when the control flow can be decided in a particular way in front of conditional branches, while the static analysis fails to witness. 

We show the over-approximation, in Figure~\ref{fig:overappr_example}(a),
we call it a multiple rounds odd iteration algorithm. In this algorithm, at line 5 of every iteration, 
a query $\query(\chi[x])$ based on previous query results stored in $x$ is asked by the analyst like in the multiple rounds strategy. The difference is that only the query answers from the even iterations ($i =0, 2, \cdots $) are 
% used to $b$. 
used in the query 
in line 7, $\query(\chi[\ln(y)])$.
  Because the execution trace only updates 
%   $b$ using the query answers at odd iterations, so the answers from even iterations do not affect the queries at odd iterations. From the query-based dependency graph in Figure~\ref{fig:overappr_example}(b), we can see that there is no edge from queries at odd iterations (such as $q_1,q_3,q_5$) to queries at even iteration(such as $q_2,q_4$). The longest path is dashed with a length $3$.  However, {\THESYSTEM} fails to realize that odd iteration will always execute then branch and even iteration means else branch, so its dependency graph considers both branches for every iteration. In this sense, the dependency graph by {\THESYSTEM} is similar to the one in the multiple rounds strategy. We show the estimated graph in Figure~\ref{fig:overappr_example}(c). The estimated upper bound is then, $5$, instead of $3$. 
$x$ using the query answers in even iterations, so the answers from odd iterations do not affect the queries in even iterations. 
From the execution-based dependency graph in Figure~\ref{fig:overappr_example}(b), 
we can see that the weight for the vertex $y^5$ is 
$w_k/2$. a function which takes any initial trace $\trace_0$, return the value of $k/2$ evaluated in $\trace_0$.  
However, {\THESYSTEM} fails to realize that odd iteration will always execute the then branch and even iteration means else branch, so 
% its dependency 
it considers both branches for every iteration. 
In this sense, the weight estimated for $y^5$ and $p^6$ are both 
$k$ as in Figure~\ref{fig:overappr_example}(c).
As a result, {\THESYSTEM}  estimates the longest walk from Figure~\ref{fig:overappr_example}(c),
$y^5  \to x^7  \to y^5  \to \cdots \to x^7  $ with each vertex visited $k$ times,
as the dotted arrows. 
And the adaptivity computed 
% estimated from the program-based dependency graph graph from by finding the walk with the longest query length 
is $1 + 2 * k$, instead of $1 + k$. 
% We omitted the estimated graph, which is identical to the graph in Figure~\ref{fig:overappr_example}(b). 
%

{ \small
\begin{figure}
\centering
    \begin{subfigure}{0.33\textwidth}
\centering
\small{
    \[
    %
    \begin{array}{l}
        \kw{multipleRoundsOdd}(k) \triangleq \\
        \clabel{ \assign{j}{k}}^{0} ; 
        \clabel{ \assign{x}{\query(\chi[0])} }^{1} ; \\
            \ewhile ~ \clabel{j > 0}^{2} ~ \edo ~ 
            \Big(
             \clabel{\assign{j}{j-1}}^{3} ;\\
             \eif(\clabel{j \% 2 == 0}^{4}, \\
             \clabel{\assign{y}{\chi[x]}}^{5}, 
             \clabel{\assign{p}{\chi[x]}}^{6});\\                            
             \clabel{\assign{x}{\query(\chi(\ln(y)))} }^{7} \Big)
        \end{array}
    \]
}
 \caption{}
    \end{subfigure}
%
\begin{subfigure}{.31\textwidth}
    \begin{centering}
    \begin{tikzpicture}[scale=\textwidth/11cm,samples=200]
% Variables Initialization
\draw[] (5, 1) circle (0pt) node{{ $x^1: {}^{w_1}_{1}$}};
% Variables Inside the Loop
 \draw[] (0, 7) circle (0pt) node{{ $y^5: {}^{w_k/2}_{1}$}};
 \draw[] (0, 4) circle (0pt) node{{ $p^6: {}^{w_k/2}_{1}$}};
 \draw[] (0, 1) circle (0pt) node{{ $x^7: {}^{w_k}_{1}$}};
 % Counter Variables
 \draw[] (5, 7) circle (0pt) node {{$j^0: {}^{w_1}_{0}$}};
 \draw[] (5, 4) circle (0pt) node {{ $j^3: {}^{w_k}_{0}$}};
 %
 % Value Dependency Edges:
 \draw[ thick, -latex,]  (0, 3.5) -- (0, 1.5) ;
%  \draw[ thick, -Straight Barb] (1, 4.2) arc (220:-100:1);
 \draw[ thick, -Straight Barb] (6.5, 4.5) arc (150:-150:1);
 \draw[ thick, -latex] (5, 4.5)  -- (5, 6.5) ;
%  \draw[ thick, -Straight Barb] (1., 1.5) arc (120:-200:1);
 % Value Dependency Edges on Initial Values:
 \draw[ thick, -latex,] (1.5, 1)  -- (4, 1) ;
 %
 \draw[ ultra thick, -latex, densely dotted,] (-0.6, 1.5)  to  [out=-220,in=220]  (-0.5, 6.5);
\draw[ ultra thick, -latex, densely dotted,]  (0.5, 6.5) to  [out=-30,in=30] (0.6, 1.6) ;
%  \draw[ ultra thick, -latex, densely dotted,]  (0.5, 10)  to  [out=-50,in=50] (0.5, 4);
 % Control Dependency
 \draw[ thick,-latex] (1.5, 7)  -- (4, 6) ;
 \draw[ thick,-latex] (1.5, 4)  -- (4, 6) ;
 \draw[ thick,-latex] (1.5, 1)  -- (4, 6) ;
%  \draw[ thick,-latex] (1.5, 10)  -- (4, 6) ;
 \end{tikzpicture}
 \caption{}
    \end{centering}
    \end{subfigure}
    \begin{subfigure}{.31\textwidth}
        \begin{centering}
        \begin{tikzpicture}[scale=\textwidth/11cm,samples=200]
    % Variables Initialization
    \draw[] (5, 1) circle (0pt) node{{ $x^1: {}^1_{1}$}};
    % Variables Inside the Loop
     \draw[] (0, 7) circle (0pt) node{{ $y^5: {}^{k}_{1}$}};
     \draw[] (0, 4) circle (0pt) node{{ $\mathbf{p^6: {}^{k}_{1}}$}};
     \draw[] (0, 1) circle (0pt) node{{ $\mathbf{x^7: {}^{k}_{1}}$}};
     % Counter Variables
     \draw[] (5, 7) circle (0pt) node {{$j^0: {}^{1}_{0}$}};
     \draw[] (5, 4) circle (0pt) node {{ $j^3: {}^{k}_{0}$}};
     %
% Value Dependency Edges:
 \draw[ thick, -latex,]  (0, 3.5) -- (0, 1.5) ;
%  \draw[ thick, -Straight Barb] (1, 4.2) arc (220:-100:1);
 \draw[ thick, -Straight Barb] (6.5, 4.5) arc (150:-150:1);
 \draw[ thick, -latex] (5, 4.5)  -- (5, 6.5) ;
%  \draw[ thick, -Straight Barb] (1., 1.5) arc (120:-200:1);
 % Value Dependency Edges on Initial Values:
 \draw[ thick, -latex,] (1.5, 1)  -- (4, 1) ;
 %
 \draw[ ultra thick, -latex, densely dotted,] (-0.6, 1.5)  to  [out=-220,in=220]  (-0.5, 6.5);
\draw[ ultra thick, -latex, densely dotted,]  (0.5, 6.5) to  [out=-30,in=30] (0.6, 1.6) ;
%  \draw[ ultra thick, -latex, densely dotted,]  (0.5, 10)  to  [out=-50,in=50] (0.5, 4);
 % Control Dependency
 \draw[ thick,-latex] (1.5, 7)  -- (4, 6) ;
 \draw[ thick,-latex] (1.5, 4)  -- (4, 6) ;
 \draw[ thick,-latex] (1.5, 1)  -- (4, 6) ;
%  \draw[ thick,-latex] (1.5, 10)  -- (4, 6) ;
     \end{tikzpicture}
     \caption{}
        \end{centering}
        \end{subfigure}
        \vspace{-0.4cm}
\caption{(a) The multiple rounds odd example 
(b) The execution-based dependency graph
(c) The program-based dependency graph graph from $\THESYSTEM$.}
    \label{fig:overappr_example}
    % \vspace{-0.5cm}
\end{figure}
}
%
\end{example}