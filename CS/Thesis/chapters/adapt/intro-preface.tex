% %%%% Benifit of reasoning about 
% \subsection{Motivation of Reasoning about Adaptivity}
% %%%% Benefit of reasoning about 
% \subsection{Motivation of Reasoning about Adaptivity}
% % 
% The adaptive data analysis designed for identifying  properties for unknown populations / distributions 
% through data samples is widely 
% used in research and industrial areas.
% % , including machine learning areas, etc.. 
% When generalizing the result from data sample to the unknown populations, 
% the generalization error is a key issue researchers focusing on reducing.
% From existing research, the \emph{adaptivity} in the analysis plays a key role in reducing the generalization error.

In this chapter, 
I firstly introduce the adaptive data analysis, and the cruciality for reasoning about the \emph{adaptivity} quantity property 
for adaptive data analysis program in Section~\ref{sec:adapt-background}.
% analyzing 
Motivated by the significance of \emph{adaptivity},
in order to analyze this \emph{adaptivity} property for the adaptive data analysis, there are 3 challenges
% problems encountered.
introduced in Section~\ref{sec:adapt-motivation}.
% I introduce these three problems
% and the full-spectrum analysis methodologies developed according to these problems 
Targeting to the three challenges, I introduce my full-spectrum analysis methodologies accordingly.
Concretely, the full-spectrum analysis is developed through the language formalization,
the execution-based analysis and the static-based program analysis.
%  Data analyses are usually designed to identify some property of the population from which the data are drawn, 
%  generalizing beyond the specific data sample. For this reason, data analyses are often designed in a way that guarantees that they produce a low generalization error.
%   That is, they are designed so that the result of a data analysis run on sample 
%   data does not differ too much from the result one would achieve by running the analysis over the entire population. 
Then I give the outline of this chapter with the contribution of this work.
% An adaptive data analysis can be seen as a process composed of multiple queries interrogating some data, where the choice of which query to run next may rely on the results of previous queries. 

% \end{enumerate}% \\

% \subsection{Further Works}
% \subsubsection{Towards Accurate Full-Spectrum Adaptivity Analysis}
% \label{sec:intro-improve}

% The work of{\ADAPTSYSTEM} is still under preparation.


