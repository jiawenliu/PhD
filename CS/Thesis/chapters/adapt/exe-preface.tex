% In this section, I present my 
% execution-based adaptivity analysis as the first part of the full-spectrum adaptivity analysis as in Figure~\ref{fig:structure}. 
In this section, 
I formally present a new formal adaptivity model and definition
%  analysis 
by analyzing the program's execution based on the language and the trace-based operational semantics introduced above.
As in Figure~\ref{fig:structure}, this is the second major part of this adaptivity analysis framework
built on the language design. 
It is more advanced than previous works
in both the accuracy and efficiency aspects.
This new adaptivity formal model defines the intuitive \emph{adaptivity} through three steps.
In Section~\ref{sec:dynamic-datadep}, I first give the new data dependency definition.
Then Section~\ref{sec:dynamic-reachability} presents the dependency quantity analysis for the dependency relation.
The  Section~\ref{sec:dynamic-adapt} gives the formal \emph{adaptivity} definition.
% ,~\ref{sec:dynamic-reachability}
% % I  an execution based program analysis in this section.
% and~\ref{sec:dynamic-adapt}.
Through an example in Section~\ref{sec:dynamic-examples}, I show that the formalized
% can give the 
adaptivity matches the program's intuitive \emph{adaptivity} more precisely and efficiently than previous works.

\paragraph{Adaptivity Formal Model Outline}
% To formalize this intuition as a quantitative program property, I develop an
This improved execution-based analysis is developed
% I first consider all the possible evaluations of a program --- I do this by 
% I use a trace semantics recording the execution history of programs on some given input --- and I create a dependency graph, where the dependency between different variables (query is also assigned to a variable) is explicit and track which variable is associated with a query request. 
% I then enrich this graph with weights describing the maximal number of times each variable is evaluated in a program evaluation starting with an initial state. The adaptivity is then defined as the length of the walk visiting most query-related variables on this graph. 
% Through two aspects: the execution-based analysis and static-based program analysis.
% In the execution-based analysis, I will formalize the intuitive notion of \emph{adaptivity} as a quantitative 
% property of programs. This analysis is developed 
 in three steps as follows,
 \begin{enumerate}
 \item The first step on \emph{dependency relation} analysis is presented in 
 Section~\ref{sec:dynamic-datadep}.
 In this step, I define the variable \emph{may-dependency} relation based on the trace semantics in Section~\ref{sec:language-os}.
%   to analyze the \emph{dependency relation} between every query, 
%  through the methodology of semantic data dependency analysis.
%  %
%  Specifically through a trace semantics recording the execution history of programs on given input,
%  % --- and I create a dependency graph, 
%  the dependency between different variables (query is also assigned to a variable) is explicitly tracked and 
%  analyzed.
%   and 
%   which variable is associated with a query request. 
% I then enrich this graph with weights describing the maximal number of times each variable is evaluated in a program evaluation starting with an initial state. The adaptivity is then defined as the length of the walk visiting most query-related variables on this graph. 
% In the execution-based analysis, I will formalize the intuitive notion of \emph{adaptivity} as a quantitative 
% property of programs. This analysis is developed 
% \\
 \item In the second step in Section~\ref{sec:dynamic-reachability}, I analyze the \emph{dependency quantity} through the methodology of execution-based reachability bound analysis.
%  As 
% %  analysis, 
% based on the \emph{dependency relation} above.
% This analysis is developed through the methodology of execution-based reachability bound analysis.
% \\
 \item The last step is the intuitive \emph{adaptivity} quantity formalization presented in Section~\ref{sec:dynamic-adapt}.
 According to the two analysis results above, specifically \emph{dependency relation} and \emph{dependency quantity},
 I define the formal \emph{adaptivity} model in definition~\ref{def:trace_adapt} through 
 construct a dependency graph.
%  This analysis is developed through the formal \emph{adaptivity} definition. \\
%  Specifically, I create a dependency graph, where the dependency between different variables (query is also assigned to a variable) is explicit and track which variable is associated with a query request. 
%  I then enrich this graph with weights describing the maximal number of times each variable is evaluated in a program evaluation starting with an initial state. 
%  The adaptivity is then defined as the length of the walk visiting most query-related variables on this graph. 
 \end{enumerate}