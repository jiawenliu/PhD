In this section, I present the new 
adaptivity analysis based on static program analysis as the third part of 
the adaptivity analysis framework as in Figure~\ref{fig:structure}. 
It is more advanced than previous works in all the accuracy, efficiency and generalization aspects.
It is also fully automated and implemented.


\paragraph{{Static Analysis for Adaptivity Outline}}
%
In order to give a sound and precise approximation of the formal adaptivity from Section~\ref{sec:adapt-exe} efficiently, 
the new static program analysis framework is developed through three steps analysis
similar to the execution-based analysis.
% In order to give a sound approximating this quantity, I design a static program analysis framework through three steps analysis
% similar to the execution-based analysis in Section~\ref{sec:dynamic}.
% In this static program analysis, the program will be analysed in the same 3 aspects as the execution-based analysis 
%    while through static program analysis techniques, and a sound estimated result will be given in each aspect.
%    \\
% 	a. The data dependency relation analysis through the static data flow analysis technique.
%    \\
% 	b. The dependency quantity analysis through the static program reachability bound analysis techniques.
%    \\
% 	c. Combining the two analysis result above, I build a program-based dependency graph for approximating
%     the execution-based dependency graph. Then, I design an algorithm computing the adaptivity upper bound soundly 
%    and accurately on the program-based dependency graph.
\begin{itemize}
   \item In Section~\ref{sec:static-dep},
   the data \emph{dependency relation} is analyzed through the static data flow analysis technique.
   \item The Section~\ref{sec:static-quantity} presents the \emph{dependency quantity} analysis through
   the static reachability bound analysis techniques developed from the first part of this thesis.
   \item The static analysis for estimating program's adaptivity is presented in Section~\ref{sec:static-adapt}.
   % the program adaptivity estimation, 
   In this analysis, I construct a program-based dependence graph for approximating the execution-based graph in Section~\ref{sec:dynamic-adapt}.
   Then, based on this graph, I design an algorithm
   %  based on the results estimated above, 
   computing the adaptivity upper bound soundly 
   and accurately.
   \end{itemize}