\paragraph*{Chapter Outline}
This chapter includes following parts. 
\begin{enumerate}
   \item Previous Works on Adaptive Data Analysis
\item A while-like language extended with query request feature, named {\tt Query While} Language, 
used to implement 
the adaptive data analysis in Chapter~\ref{ch:language}. 
\item A formal adaptivity model through execution-based adaptivity analysis in Chapter~\ref{ch:dynamic}.
\item A static program analysis algorithm, named {\THESYSTEM} through static adaptivity analysis in Chapter~\ref{ch:static}.
\item An extended full-spectrum program adaptivity analysis in Chapter~\ref{ch:improved}.
\item An accurate full-spectrum program resource cost analysis, 
generalized from {\THESYSTEM} with implementation in Chapter~\ref{ch:generalization}.
\item Proposed future works for solving the CFL-reachability problem via reduction into the {\THESYSTEM} framework in
Chapter~\ref{ch:furthers},
expected to be started off before final defense and developing further after.
\end{enumerate}


\paragraph{Contributions}
This chapter has the following contributions.
\begin{enumerate}
   \item A programming framework for adaptive data analyses where the program represents an analyst that can query a generalization-preserving mechanism mediating the access to some data. 
   This language is extended from existing framework with inputs and inter-procedure call.
   % \item A trace-based operational semantics for the loop language, specific to dependency between queries.
   \item 
   % {A formal definition of the notion of adaptivity under the analyst-mechanism model. This definition is built on a query-based dependency graph built out of all the possible program execution traces.}
   A formal definition of the notion of adaptivity under the analyst-mechanism model. 
   This definition is built on a variable-based dependency graph that is constructed using sets of program execution traces.
   % \item A transformation between the {\tt Loop} language and the ssa language, with the soundness of the transformation.
   \item 
   % A program analysis algorithm {\THESYSTEM} which provides an upper bound on the adaptivity via a variable-based dependency graph.
   A static program analysis algorithm {\THESYSTEM} combining data flow, control flow and  reachability bound analysis in order to provide tight bounds on the adaptivity of a program. 
   %
   \item A soundness proof of the program analysis showing that the adaptivity estimated by {\THESYSTEM} bounds the true adaptivity of the program. 
   \item An implementation of {\THESYSTEM} and an experimental evaluation the bounds this implementation provides on several examples.
   % \item A path-sensitive program reachability bound analysis algorithm designed for program with application beyond adaptivity analysis
   %  with implementation.
   % \item An accurate full-spectrum program resource cost analysis, 
   % generalized from {\THESYSTEM} with implementation.
\end{enumerate}
%
\todo{combine}
\paragraph*{Contributions and Improvements}
Given the limitation of existing reachability bound analysis techniques, I designed a path-sensitive reachability bound 
analysis algorithm

Based on the implementation and experimental results of the basic full-spectrum analysis,
% $\THESYSTEM$,
I 
% also focus on the two further features can be 
extend it in Section~\ref{ch:improved} in following 3 aspects.
% \begin{enumerate}
%  \item The precision of formalizing the intuitive \emph{adaptivity} 
% %  in the formal  model 
% through the execution-based program analysis.

% \item In static program analysis, I will give a tighter estimated upper bound on dependency quantity through 
% path sensitive reachability bound analysis techniques. 

% % \item In the third step of static program analysis, I will improve the accuracy of the adaptivity computation algorithm,
% % compute a tighter adaptivity upper bound as well.
% \end{enumerate}
\begin{enumerate}
   \item The {\tt Query While} Language is extended in Section~\ref{sec:refine-exe-language} with inter-procedure call.
   \item In the execution-based \emph{adaptivity} analysis part,
   the precision of formalizing the intuitive \emph{adaptivity} is improved, with extension on the
%  in the formal  model 
% in the meantime extend this analysis with 
inter-procedure call in Section~\ref{sec:refine-exe}.
\item The static program \emph{adaptivity} analysis is extended inter-procedure call as well, 
and improved by
giving a tighter estimated upper bound on \emph{adaptivity} in Section~\ref{sec:refine-static}.
%  give a tighter estimated upper bound 
Especially in order to improve the accuracy of the estimated result, a
% I improve the accuracy of the static \emph{dependency quantity} analysis in the second step through 
% designing a
path sensitive reachability bound analysis algorithm is designed in Section~\ref{sec:refine-static-psreachability}.
Based on this newly designed algorithm,
I improve the accuracy of the static \emph{dependency quantity} analysis in the second step 
is improved, and a tighter upper bound on \emph{adaptivity} is computed correspondingly.
% \item In the third step of static program analysis, I will improve the accuracy of the adaptivity computation algorithm,
% compute a tighter adaptivity upper bound as well.
\end{enumerate}

%%%%% To reason about%
