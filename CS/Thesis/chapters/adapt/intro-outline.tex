\paragraph*{Chapter Outline}
The rest parts of this chapter is organized as follows. 
\begin{enumerate}
   \item The previous works on adaptive data analysis are introduced in Section~\ref{sec:prework}.
   \item The new program analysis framework for the adaptivity of Adaptive Data Analysis is presented 
   in Section~\ref{sec:adapt-analysis}.
   This new program analysis framework has three major components:
   \begin{enumerate}
      \item A while-like language extended with query request feature, named {\tt Query While} Language, 
      used to implement the adaptive data analysis in Section~\ref{sec:adapt-language};
      \item A formal adaptivity model through analyzing the program's execution semantically in Section~\ref{sec:adapt-exe};
      \item A static program analysis algorithm, named {\THESYSTEM} through static adaptivity analysis in Section~\ref{sec:adapt-static}.
   \end{enumerate}
   \item The manual examples demonstrating the new adaptivity analysis framework,
   and the evaluation results from the implementation
   of this analysis framework is presented in Section~\ref{sec:adapt-implementation}.
   \item Section~\ref{sec:adapt-relatedwork} presents the related works in the program analysis and the data analysis areas
   related to our adaptivity analysis framework.
\end{enumerate}


\paragraph{Contributions}
This chapter has the following contributions.
\begin{enumerate}
   \item A programming framework for adaptive data analyses where programs represent analysts that can query generalization-preserving mechanisms mediating the access to some data. 
   \item 
   A formal definition of the notion of adaptivity under the analyst-mechanism model. 
   This definition is built on a variable-based dependency graph that is constructed using sets of program execution traces.
   \item 
   A static program analysis algorithm {\THESYSTEM} combining data flow, control flow and  reachability bound analysis in order to provide tight bounds on the adaptivity of a program.
   \item A soundness proof of the program analysis showing that the adaptivity estimated by {\THESYSTEM} bounds the true adaptivity of the program. 
   \item An implementation of {\THESYSTEM} and an experimental evaluation of the bounds this implementation provides on several examples.
\end{enumerate}