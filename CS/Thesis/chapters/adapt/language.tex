In this chapter, 
I formally introduce the language I will focus on for writing data analyses.  
This is an extended standard while language with some primitives for calling queries.
It also extends the expressiveness of previous language design for adaptivity analysis.
After defining the syntax of the language and showing an example, 
I will define its trace-based operational semantics. 
This is the main technical ingredient I will use to define the program's adaptivity.
\subsection{Syntax of {\tt Query While} Language}
\label{sec:language-syntax}
The syntax is shown as follows,
\[
\begin{array}{llll}
\mbox{Arithmetic Operators} 
& \oplus_a & ::= & + ~|~ - ~|~ \times 
%
~|~ \div ~|~ \max ~|~ \min\\  
% \mbox{Unary Operators} 
% & \oplus_a & ::= & + ~|~ - ~|~ \times 
% %
% ~|~ \div \\  
\mbox{Boolean Operators} 
& \oplus_b & ::= & \lor ~|~ \land
\\
%
\mbox{Relational Operators} 
& \sim & ::= & < ~|~ \leq ~|~ == 
\\  
%
\mbox{Label} 
& l & \in & \mathbb{N} \cup \{\lin, \lex\} 
\\ 
%
\mbox{Arithmetic Expression} 
& \aexpr & ::= & 
n ~|~ {x} ~|~ \aexpr \oplus_a \aexpr  
% \\
% &  &  & 
 ~|~ \elog \aexpr  ~|~ \esign \aexpr
\\
%
\mbox{Boolean Expression} & \bexpr & ::= & 
%
\etrue ~|~ \efalse  ~|~ \neg \bexpr
 ~|~ \bexpr \oplus_b \bexpr
%
~|~ \aexpr \sim \aexpr 
\\
%
\mbox{Expression} & \expr & ::= & v ~|~ \aexpr ~|~ \bexpr ~|~ [\expr, \dots, \expr]
\\  
%
\mbox{Value} 
& v & ::= & { n ~|~ \etrue ~|~ \efalse ~|~ [] ~|~ [v, \dots, v]}  
\\
%
\highlight{\mbox{Query Expression}  }
& {\qexpr} & ::= 
& \highlight{ \qval ~|~ \aexpr ~|~ \qexpr \oplus_a \qexpr ~|~ \chi[\aexpr]}
\\
%
\highlight{\mbox{Query Value} }& \qval & ::= 
& \highlight{n ~|~ \chi[n] ~|~ \qval \oplus_a  \qval ~|~ n \oplus_a  \chi[n]
    ~|~ \chi[n] \oplus_a  n}
    \\
% &&& \text{\mg{I don’t think this is what I want. Isn’t $\chi[n+1]$ a query value?}}\\
% &&& \text{\mg{What about $\chi[i] + \chi[i] + \chi[i]$? They are not in the grammar}}
% \\
% &&& \text{\jl{ $\chi[i] + \chi[i] + \chi[i]$ and $\chi[n+1]$ are both expressions, they will be evaluated to a value 
% }}
% \\%
\mbox{Labeled Command} 
& {c} & ::= &   [\assign {{x}}{ {\expr}}]^{l} ~|~  \highlight{[\assign {{x} } {{\query(\qexpr)}}]^{l}}
~|~ {\ewhile [ \bexpr ]^{l} \edo {c} }
\\
&&&
~|~ {c};{c}  
~|~ \eif([\bexpr]{}^l , {c}, {c}) 
~|~ [\eskip]^l
\\ 
\mbox{Event} 
& \event & ::= & 
    ({x}, l, v, \bullet) ~|~ ({x}, l, v, \qval)  ~~~~~~~~~~~ \mbox{Assignment Event} \\
&&& ~|~(\bexpr, l, v, \bullet)   ~~~~~~~~~~~~~~~~~~~~~~~~~~~~~~~~~~ \mbox{Testing Event}
\\
\mbox{Trace} & \trace
& ::= & [] ~|~ \trace :: \event
\\
\end{array}
\]
% \[
% \begin{array}{llll}
% \mbox{Arithmetic Operators} 
% & \oplus_a & ::= & + ~|~ - ~|~ \times 
% %
% ~|~ \div ~|~ \max ~|~ \min\\  
% % ~|~ \div \\  
% \mbox{Boolean Operators} 
% & \oplus_b & ::= & \lor ~|~ \land
% \\
% %
% \mbox{Relational Operators} 
% & \sim & ::= & < ~|~ \leq ~|~ == 
% \\  
% %
% \mbox{Arithmetic Expression} 
% & \aexpr & ::= & 
% n ~|~ {x} ~|~ \aexpr \oplus_a \aexpr  
%  ~|~ \elog \aexpr  ~|~ \esign \aexpr
% \\
% %
% \mbox{Boolean Expression} & \bexpr & ::= & 
% %
% \etrue ~|~ \efalse  ~|~ \neg \bexpr
%  ~|~ \bexpr \oplus_b \bexpr
% %
% ~|~ \aexpr \sim \aexpr 
% \\
% %
% \mbox{Expression} & \expr & ::= & v ~|~ \aexpr ~|~ \bexpr ~|~ [\expr, \dots, \expr] ~|~ \highlight{\fname}
% \\  
% %
% \mbox{Value} 
% & v & ::= & { n ~|~ \etrue ~|~ \efalse ~|~ [] ~|~ [v, \dots, v]}  
% \\ 
% &&&
% \highlight
% {
% ~|~ (x_0, x_1, \ldots, x_n) := c
% }
% \\
% %
% \highlight{\mbox{Query Expression}} 
% & {\qexpr} & ::= 
% & \highlight{ \qval ~|~ \aexpr ~|~ \qexpr \oplus_a \qexpr ~|~ \chi[\aexpr]} 
% \\
% %
% \mbox{Query Value} & \qval & ::= 
% & \highlight{n ~|~ \chi[n] ~|~ \qval \oplus_a  \qval ~|~ n \oplus_a  \chi[n]
%     ~|~ \chi[n] \oplus_a  n}
% \\
% % \\%
% \mbox{Label} 
% & l & ::= & (n \in \mathbb{N} \cup \{\lin, \lex\}) ~|~ (l, n)
% \\ 
% %
% \mbox{Labeled Command} 
% & {c} & ::= &  
% \clabel{\assign{x}{\expr}}^l 
% ~|~ \clabel{\assign{x}{\query(\qexpr)}}^l
% ~|~  \clabel{\eskip}^l
% ~|~ \ewhile \clabel{\bexpr}^{l} \edo {c}
% ~|~ \eif(\clabel{\bexpr}^{l} , {c}, {c}) 
% \\ 
% &&&
% \highlight
% {
% ~|~ \clabel{\efun}^l: \fname (x_0, x_1, \ldots, x_n) := c
% ~|~ \clabel{\assign{x}{\ecall(x, e_1, \ldots, e_n)}}^l
% }
% ~|~ {c};{c}  
% \\ 
% % \\
% \mbox{Event} 
% & \event & ::= & 
%     ({x}, l, v, \bullet) ~|~ ({x}, l, v, \qval) ~|~ (\fname, l, v, \qval)  ~~~~~~~~~~~ \mbox{Assignment Event} \\
% &&& ~|~(\bexpr, l, v, \bullet)   ~~~~~~~~~~~~~~~~~~~~~~~~~~~~~~~~~~ \mbox{Testing Event}
% \\
% % &&& \text{\mg{I think it would be better to use quadruples for events, where the}}\\
% % &&& \text{\mg{first element is either a variable or a boolean expression and }}\\
% % &&& \text{\mg{the last is either a query value or some default value $\bullet$}}\\
% %
% % \mbox{Trace} & \trace
% % & ::= & \cdot | \trace \cdot \event | \trace \tracecat \trace 
% % \\
% %
% % \mbox{Trace} & \trace
% % & ::= & [] ~|~ \event:: \trace ~|~ \trace \tracecat \trace  \\
% \mbox{Trace} & \trace
% & ::= & [] ~|~ \trace :: \event\\
% % &&& \text{\mg{I don't understand why you need both :: and ++ as constructors.}}\\
% % &&& \text{\jl{Because append is to the left but we are adding element to the left in the OS}}\\
% % &&& \text{\jl{I was too sticky to the convention, it is a good idea to append to the left and just use $::$}}
% % %
% % \mbox{Event Signature} & \sig
% % & ::= & (x, l, n) | (x, l, n, \query) | (b, l, n)
% % \\
% % %
% \end{array}
% \]
For clarity, the following notations are used to represent the set of corresponding terms:
\[
\begin{array}{lll}
\mathcal{VAR} & : & \mbox{Set of Variables}  
\\ 
%
\mathcal{VAL} & : & \mbox{Set of Values} 
\\ 
%
\mathcal{QVAL} & : & \mbox{Set of Query Values} 
\\ 
%
\cdom & : & \mbox{Set of Commands} 
\\ 
%
\mathcal{LV} & : & \mbox{Set of Labeled Variables}
\\
%
\eventset  & : & \mbox{Set of Events}  
\\
%
\eventset^{\asn}  & : & \mbox{Set of Assignment Events}  
\\
%
\eventset^{\test}  & : & \mbox{Set of Testing Events}  
\\
%
\ldom  & : & \mbox{Set of Labels}  
\\
%%
\mathcal{VAL}  & : & \mbox{Set of Labeled Variables}  
\\
%%
\dbdom  & : & \mbox{{Set of Databases}} 
\\
%
{\mathcal{T}} & : & \mbox{Set of Traces}
\\
%
% \qdom = {[-1,1]} & : & \mbox{{Domain of Query Results}}\\
\qdom & : & \mbox{{Domain of Query Results}}\\
% &&\text{\mg{I don't think you need to hard code [-1,1] here}}\\
\end{array}
\]
\paragraph*{Standard Expression}
The expressions are either the standard one or the extended one.
A standard expression is
% can be 
either a standard arithmetic expression or a boolean expression, or a list of expressions.
An arithmetic expression can be a constant $n$ denoting integer, a variable $x$ from some countable set $\mathcal{VAR}$, binary operation $\oplus_a$ such as addition, product, subtraction, etc, over arithmetic expressions, and also log and sign operation. 
%
A boolean expression can be either {\tt true} or {\tt false}, basic boolean connectives such as logical negation, logical and and or denoted by $\oplus_b$, and basic comparison $sym$ between arithmetic expressions, e.g., $\leq,=,<,$ etc.
Additionally, I also introduce list in expression.
Our language supports primitives for queries, 
where a specific query is specified by a query expression $\qexpr$. 
A query expression contains the necessary information for a query request, for example, 
$\chi[\aexpr]$ represents the values at a certain index $\aexpr$ in a row $\chi$ of the database. 
Query expressions combine access to the database with other expressions, 
for example, $\chi[3] + 5$ represents a query which asks the value from the column 3 of each database raw $\chi$, adds 5 to each of these values, 
and then computes the average of these values.
\paragraph*{Query Expression}
The key extension is
%  language supports 
the primitive for queries, where a specific query is specified by a query expression $\qexpr$. 
A query expression contains the necessary information for a query request, 
for example, $\chi[\qexpr]$ represents the values at a certain index $\qexpr$ in a row $\chi$ of the database. 
When this expression is encapsulated by the symbol $\query$,
 $ \query(\chi[\qexpr]) $ computes the average value at certain index over each row of the database as follows,
 \[
  \query(\chi[\qexpr]) = \frac{1}{n}\sum\limits_{i = 0}^{n}\chi_i[\qexpr]
  \]
Query expressions combine access to the database with other expressions, 
for example, 
$\chi[3] + 5$ represents a query that asks the value from column 3 of each database raw $\chi$, 
adds 5 to each of these values, and then computes the average of these values as follows, where $n$ is 
data base $\chi$'s number of raw.
%
\[
  \query(\chi[3] + 5) = \frac{1}{n}\sum\limits_{i = 0}^{n}\chi_i[3] + 5
  \]

% the expression also includes the special variable $\chi$ representing a row of the database, and access to values at a certain index in $\chi$, as $\chi[\aexpr]$. Additionally, list over expressions is supported and $[]$ stands for the empty list. The access to elements in the list can be achieved through $x[\aexpr]$ when variable $x$ is referred to a list. The value $v$ now contains the natural number $n$, the boolean primitives $\etrue$ and $\efalse$, the special row $\chi$ and access to it $\chi[v]$, the empty list $[]$ and non-empty list $[v, \dots, v]$.
% 
% Another extension is the inter-procedure call and function definition.
% In the function define command $\clabel{\efun}^l: \fname (x_0, x_1, \ldots, x_n) := c$,
% the function body $c$ is assigned to the function of name $\fname$, $x_1, \ldots, x_n$ is the function
% arguments and the first element $x_0$ in the arguments is the return variable.
% We only support the first-order function definition and function call. 

% %
\paragraph*{Labeled Command}
 A labeled command $c$ is just a command with a label --- I assume that labels are unique, so that they can help to identify uniquely every subexpression. 
%  I have $\eskip$, assignment $\assign{x}{\expr}$, the composition of two commands $c;c$, an if statement $\eif(\bexpr, c, c)$, a while statement  $\ewhile \bexpr \edo {c} $.
 The main novelty of the syntax is the query request command $\assign{x}{\query(\qexpr)}$. 
 For instance, if a data analyst wants to ask a simple linear query which returns the first element of the row, 
 they can simply use the command $ \assign{x}{\query(\chi[1])}$ in their data analysis program.
%  \wq{Shall I distinguish command and labeled command, they are now both $c$. }
%  \jl{I'm not sure, I don't want to programmer to add the label when writing the program. The label is just added by us for analysis. but I'm worried it is too complicate if use two notations for command and labeled command }
%
% \[
% \begin{array}{llll}
% \mbox{Label} 
% & l & \in & \mathbb{N} \cup \{in, ex\} \\
% \mbox{Labeled Commands} 
% & {c} & ::= &   [\assign {{x}}{ {\expr}}]^{l} ~|~  [\assign {{x} } {{\query(\qexpr)}}]^{l}
% ~|~ {\ewhile [ \bexpr ]^{l} \edo {c} }
%  \\
%  &&&
% ~|~ {c};{c}  
% ~|~ \eif([\bexpr]{}^l , {c}, {c}) 
% ~|~ [\eskip]^l 
% \end{array}
% \]
\paragraph*{Labeled Variables}
The labeled variables and assigned variables are set of variables annotated by a label. 
We use  
%$\mathcal{LVAR} = \mathcal{VAR} \times \mathcal{L} $ 
$\mathcal{LV}$ represents the universe of all the labeled variables and 
$\avar_c \in \mathcal{P}(\mathcal{VAR} \times \mathbb{N}) \subset \mathcal{LV}$ and 
$\lvar_c \in \mathcal{P}(\mathcal{VAR} \times \mathcal{L}) \subseteq \mathcal{LV}$,
represents the the set of assigned variables and labeled variables for a labeled command $c$,
defined in Definition~\ref{def:lvar} and \ref{def:avar}.
%
% \\
$FV: \expr \to \mathcal{P}(\mathcal{VAR})$, computes the set of free variables in an expression. To be precise,
$FV(\aexpr)$, $FV(\bexpr)$ and $FV(\qexpr)$ represent the set of free variables in arithmetic
expression $\aexpr$, boolean expression $\bexpr$ and query expression $\qexpr$ respectively.
Labeled variables in $c$ is the set of assigned variables and all the free variables
showing up in $c$ with a default label $in$. 
The free variables
showing up in $c$, which aren't defined before be used, are actually the input variables of this program.
%
%
\begin{defn}[Assigned Variables (
% $\avar_{c} \subseteq \mathcal{VAR} \times \mathbb{N}$ or 
$\avar : \cdom \to \mathcal{P}(\mathcal{VAR} \times \mathbb{N})$)]
% labelled Variables 
% (
% % $\lvar_{c} \subseteq \mathcal{VAR} \times \mathbb{N}$ or 
% $\lvar : \cdom \to \mathcal{P}(\mathcal{VAR} \times \mathcal{L})$
\label{def:avar}
$$ \avar_{c} \triangleq
  \left\{
  \begin{array}{ll}
      \{{x}^l\}                   
      & {c} = [{\assign x e}]^{l} 
      \\
      \{{x}^l\}                   
      & {c} = [{\assign x \query(\qexpr)}]^{l} 
      \\
      \avar_{{c_1}} \cup \avar_{{c_2}}  
      & {c} = {c_1};{c_2}
      \\
      \avar_{{c}} \cup \avar_{{c_2}} 
      & {c} =\eif([\bexpr]^{l}, c_1, c_2) 
      \\
      \avar_{{c}'}
      & {c}   = \ewhile ([\bexpr]^{l}, {c}')
\end{array}
\right.
$$
\end{defn}
%
%
\begin{defn}[labelled Variables 
(
% $\lvar_{c} \subseteq \mathcal{VAR} \times \mathbb{N}$ or 
$\lvar : \cdom \to \mathcal{P}(\mathcal{LV})$]
\label{def:lvar}
$$
  \lvar_{c} \triangleq
  \left\{
  \begin{array}{ll}
      \{{x}^l\} \cup FV(\expr)^{in}                  
      & {c} = [{\assign x e}]^{l} 
      \\
      \{{x}^l\}   \cup FV(\qexpr)^{in}                
      & {c} = [{\assign x \query(\qexpr)}]^{l} 
      \\
      \lvar_{{c_1}} \cup \lvar_{{c_2}}  
      & {c} = {c_1};{c_2}
      \\
      \lvar_{{c}} \cup \lvar_{{c_2}} \cup FV(\bexpr)^{in}
      & {c} =\eif([\bexpr]^{l}, c_1, c_2) 
      \\
      \lvar_{{c}'} \cup FV(\bexpr)^{in}
      & {c}   = \ewhile ([\bexpr]^{l}, {c}')
\end{array}
\right.
$$
\end{defn}
%
%
%
% is a subset of the program's assigned variables, where every variable in this set is assigned by a query in the program.
% \mg{The set of query variables of a program is the set of variables set to the result of a query in the program.}\\
% In the same way, in order to 
\paragraph*{Query Variables}
Distinctively, a key definition for the extension of the query primitives 
is the set of query variables for a program $c$.
This definition is the key point to track the query requests in the Following full-spectrum adaptivity analysis.
% track the I also defined the set of query variables for a program $c$.
It is defined as the set of variables,
which are assigned by the result of a query request in the program formally in Definition~\ref{def:qvar}.
% \mg{In the next definition, why do you call it a vector? It seems that you define it as a set.}\\
% \jl{fixed}\\
%
% \begin{defn}[Query Variables ($\qvar_{c} \subseteq \mathcal{VAR} \times \mathbb{N}$)].
  % \\
\begin{defn}[Query Variables ($\qvar: \cdom \to \mathcal{P}(\mathcal{LV})$)] 
  \label{def:qvar}
Given a program $c$, its query variables 
% \mg{it seems you are missing the $_c$ subscript. Also, this is a minor point but I don't think it is a good idea to use a subscript, cannot you just use $\qvar(c)$.}
$\qvar(c)$ is the set of variables set to the result of a query in the program.
% \jl{fixed}
It is defined as follows:
{
$$
  % \qvar_{{c}} \triangleq
  \qvar(c) \triangleq
  \left\{
  \begin{array}{ll}
      % \{\}                  
      % & {c} = [{\assign x e}]^{(l, w)} 
      % \\
      % \{{x}^l\}                  
      % & {c} = [{\assign x \query(\qexpr)}]^{(l, w)} 
      % \\
      % \qvar_{{c_1}} \cup \qvar_{{c_2}}  
      % & {c} = {c_1};{c_2}
      % \\
      % \qvar_{{c_1}} \cup \qvar_{{c_2}} 
      % & {c} =\eif([\bexpr]^{l}, c_1, c_2) 
      % \\
      % \qvar_{{c}'}
      % & {c}   = \ewhile ([\bexpr]^{l}, {c}')
      \{\}                  
      & {c} = [{\assign x \expr}]^{l} 
      \\
      \{{x}^l\}                  
      & {c} = [{\assign x \query(\qexpr)}]^{l} 
      \\
      \qvar(c_1) \cup \qvar(c_2)  
      & {c} = {c_1};{c_2}
      \\
      \qvar(c_1) \cup \qvar(c_2) 
      & {c} =\eif([\bexpr]^{l}, c_1, c_2) 
      \\
      \qvar(c')
      & {c}   = \ewhile ([\bexpr]^{l}, {c}')
\end{array}
\right.
$$
}
\end{defn}
%
It is easy to see that a program $c$'s query variables is a subset of 
its labeled variables, $\qvar(c) \subseteq \lvar(c)$.
%
% \mg{In this definition as well as in others, I have the impression that you assume that the labelled variables are unique in the program. For example, it would not make sense to assign a query to the same labelled variable over and over. If this is the case, I need to make this very explicit in the paper.}
% \jl{TODO}
%
Every labeled variable in a program is unique, formally as follows with proof in Appendix~\ref{apdx:lemma_sec123}.
\begin{lem}[Uniqueness of the Labeled Variables]
  \label{lem:lvar_unique}
  For every program $c \in \cdom$ and every two labeled variables such that
  $x^i, y^j \in \lvar(c)$, then $x^i \neq y^j$.
  \[
    \forall c \in \cdom, x^i, y^j \in \mathcal{L} \sthat x^i, y^j \in \lvar(c)\implies x^i \neq y^j.
    \]
\end{lem}

\highlight{\paragraph*{Improvements through Examples}
It is expressive in two following aspects.
\begin{itemize}
  \item \textbf{Improvements from Standard While Language}
  \\
  It also extends the standard while language with query requests. 
  The general data analysis program with query requests on data  are supported in this {\tt Query While} language.
  The program can access the database through a special  interface $\chi$ encapsulated by the identifier $\query$ (for example the program below) in the new language.
  \[
    {\assign{x}{20}};
    \assign{y}{\query(\chi[2])};
    \ewhile (x < 100) \edo 
    \{
      \assign{x}{x + 1};
      \assign{y}{\query(\chi[x]*\chi[n])};
      \}\}
    \] 
%
    \item \textbf{Improvements from Previous Works}
  \\
This {\tt Query While} language is also more expressive than the language designed in previous works.
The previous language only supports the data analysis with constant number of loop iterations.
Comparing to it, in the new language design,
the general data analysis program with non-deterministic loop iterations
(for example the program below as shown in Section~\ref{sec:prework-language})
is supported.
\[
  {\assign{x}{20}};
  \assign{y}{40};
  \ewhile (x < y) \edo 
  \{
    \assign{x}{x + 1};
    \assign{y}{y - 2};
    \}\}
  \] 
Previous work does not support data analysis program with user inputs, which is supported in the new language as well.
\end{itemize}
}
\subsection{Trace-based Operational Semantics}
\label{sec:language-os}
The operational semantics is defined based on the event and trace, which are introduced firstly as follows.
% \\
\paragraph*{Event}
An event tracks useful information about each step of the evaluation, as a quadruple. Its first element is either 
an assigned variable (from an assignment command) or a boolean expression (from the guard of if or while command), follows by 
 the label associated with this event, the value evaluated either from the expression assigned to the variable,
or the boolean expression in the guard.
 The last element stores the query information, which is a query value whose default is $\bullet$. I declare event projection operator $\pi_i$ which projects the $i$th element from an event.
\[
\begin{array}{llll}
\mbox{Event} 
& \event & ::= & 
 ({x}, l, v, \bullet) ~|~ ({x}, l, v, \qval) ~~~~~~~~~~~ \mbox{Assignment Event} 
~|~(\bexpr, l, v, \bullet) 
~~~~
\mbox{Testing Event}
% \mbox{Trace} & \trace
% & ::= & [] ~|~ \trace :: \event
\end{array}
\]
% \input{event}
% To distinguish if a query's choice is affected by previous values, 
% % \jl{we need to be able to identify whether two queries are equivalent or not so that when we change the result of one query, another query is affected. For the equivalence of queries, } 
% we need to be able to identify whether two queries are equivalent or not, so that when we change the result of one query, whether or not another query is affected. 
% To define equivalence of queries,
% quite different from the equality between the evaluation results as the regular assignment results, 
% we are neither observing the syntactic equivalence between the two query expressions,
% nor two results return from the database. 
% Instead, we define the equivalence of query expression by quantifying over all values returned from the database on a certain form of query value, formally as follows.
% \begin{defn}[Equivalence of Query Expression]
% %
% \label{def:query_equal}
% % \mg{Two} \sout{2} 
% Two query expressions $\qexpr_1$, $\qexpr_2$ are equivalent, denoted as $\qexpr_1 =_{q} \qexpr_2$, if and only if
% % $$
% % \begin{array}{l} 
% % \exists \qval_1, \qval_2 \in \mathcal{QVAL} \st \forall \trace \in \mathcal{T} \st
% % (\config{\trace, \qexpr_1} \qarrow \qval_1 \land \config{\trace, \qexpr_2 } \qarrow \qval_2) 
% % \\
% % \quad \land (\forall D \in \dbdom, r \in D \st 
% % \exists v \in \mathcal{VAL} \st 
% % \config{\trace, \qval_1[r/\chi]} \aarrow v \land \config{\trace, \qval_2[r/\chi] } \aarrow v) 
% % \end{array}.
% % $$
% $$
% \begin{array}{l} 
% \forall \trace \in \mathcal{T} \st \exists \qval_1, \qval_2 \in \mathcal{QVAL} \st
% (\config{\trace, \qexpr_1} \qarrow \qval_1 \land \config{\trace, \qexpr_2 } \qarrow \qval_2) 
% \\
% \quad \land (\forall D \in \dbdom, r \in D \st 
% \exists v \in \mathcal{VAL} \st 
% \config{\trace, \qval_1[r/\chi]} \aarrow v \land \config{\trace, \qval_2[r/\chi] } \aarrow v) 
% \end{array}.
% $$
% % \mg{$$
% % \begin{array}{l} 
% % \forall \trace \in \mathcal{T} \st \exists \qval_1, \qval_2 \in \mathcal{QVAL} \st
% % (\config{\trace, \qexpr_1} \qarrow \qval_1 \land \config{\trace, \qexpr_2 } \qarrow \qval_2) 
% % \\
% % \quad \land (\forall D \in \dbdom, r \in D \st 
% % \exists v \in \mathcal{VAL} \st 
% % \config{\trace, \qval_1[r/\chi]} \aarrow v \land \config{\trace, \qval_2[r/\chi] } \aarrow v) 
% % \end{array}.
% % $$
% % }
% %
% where $r \in D$ is a record in the database domain $D$. 
% I denote by $\qexpr_1 \neq_{q} \qexpr_2$ the negation of the equivalence relation.
% % \\ 
% % where $r \in D$ is a record in the database domain $D$,
% % \mg{is $FV(\qexpr)$ being defined here? If yes, I suggest putting it in a different place, rather than in the middle of another definition.} 
% % $FV(\qexpr)$ is the set of free variables in the query expression $\qexpr$.
% % \sout{$\qexpr_1 \neq_{q}^{\trace} \qexpr_2$ is defined vice versa.}
% % \mg{As usual, we will denote by $\qexpr_1 \neq_{q}^{\trace} \qexpr_2$ the negation of the equivalence.}
% %
% \end{defn}
%
% \mg{In the next definition you don’t need the subscript e, it is clear that it is an equivalence of events by the fact that the elements on the two sides of = are events. That is also true for query expressions. Also, I am confused by this definition. What happens for two query events?}
% \\
% \jl{The last component of the event is equal based on Query equivalence, $\pi_{4}(\event_1) =_q \pi_{4}(\event_2)$.
% In the previous version, the query expression is in the third component and I defined $v \neq \qexpr$ for all $v$ that isn't a query value.}
% \begin{defn}[Event Equivalence $\eventeq$]
% Two events $\event_1, \event_2 \in \eventset$ \mg{are equivalent, \sout{is in \emph{Equivalence} relation,}} denoted as $\event_1 \eventeq \event_2$ if and only if:
% \[
% \pi_1(\event_1) = \pi_1(\event_2) 
% \land 
% \pi_2(\event_1) = \pi_2(\event_2) 
% \land
% \pi_{3}(\event_1) = \pi_{3}(\event_2)
% \land 
% \pi_{4}(\event_1) =_q \pi_{4}(\event_2)
% \]
% %
% % \sout{The $\event_1 \eventneq \event_2$ is defined as vice versa.}
% % \mg{As usual, we will denote by $\event_1 \eventneq \event_2$ the negation of the equivalence.}
% \end{defn}
% \wq{Now we can compare two events by defining the event equivalence and difference relation.}
% Now we can compare two events by defining the event equivalence and difference relation based on the query equivalence.
% \begin{defn}[Event Equivalence]
% \label{def:event_eq}
% Two events $\event_1, \event_2 \in \eventset$ are equivalent, 
% % denoted as $\event_1 \eventeq \event_2$ 
% denoted as $\event_1 = \event_2$ 
% if and only if:
% \[
% \pi_1(\event_1) = \pi_1(\event_2) 
% \land 
% \pi_2(\event_1) = \pi_2(\event_2) 
% \land
% \pi_{3}(\event_1) = \pi_{3}(\event_2)
% \land 
% \pi_{4}(\event_1) =_q \pi_{4}(\event_2)
% \]
% %
% As usual, we will denote by $\event_1 \neq \event_2$ the negation of the equivalence.
% % As usual, we will denote by $\event_1 \eventneq \event_2$ the negation of the equivalence.
% % When it is clear from the context, we omit the subscript $\kw{e}$ and use 
% % $\event_1 = \event_2$ (and $\event_1 \neq \event_2$) for event equivalent
% \end{defn}
% %
% %
% % \begin{defn}[Signature Equivalence of Events $\sigeq$]
% % Two events $\event_1, \event_2 \in \eventset$ is in \emph{signature equivalence} relation, denoted as $\event_1 \sigeq \event_2$ if and only if:
% % \[
% % \forall i \in \{1, 2, 3\} \st \pi_{\sig}(\event_1) = \pi_{\sig}(\event_2) 
% % \]
% % The $\event_1 \signeq \event_2$ is defined as vice versa.
% % \end{defn}
% %
% % \begin{defn}[Events Different up to Value ($\diff$)]
% % Two events $\event_1, \event_2 \in \eventset$ \mg{are \sout{is}} \emph{Different up to Value}, 
% % denoted as $\diff(\event_1, \event_2)$ if and only if:
% % \[
% % \pi_1(\event_1) = \pi_1(\event_2) 
% % \land 
% % \pi_2(\event_1) = \pi_2(\event_2) 
% % \land 
% % \pi_3(\event_1) \neq_q \pi_3(\event_2)
% % \]
% % \end{defn}
% \begin{defn}[Events Different up to Value ($\diff$)]
% Two events $\event_1, \event_2 \in \eventset$ are \emph{Different up to Value}, 
% denoted as $\diff(\event_1, \event_2)$ if and only if:
% \[
% \begin{array}{l}
% \pi_1(\event_1) = \pi_1(\event_2) 
% \land 
% \pi_2(\event_1) = \pi_2(\event_2) \\
% \land 
% \big(
% (\pi_3(\event_1) \neq \pi_3(\event_2)
% \land 
% \pi_{4}(\event_1) = \pi_{4}(\event_2) = \bullet )
% % \qquad \qquad 
% \lor 
% (\pi_4(\event_1) \neq \bullet
% \land 
% \pi_4(\event_2) \neq \bullet
% \land 
% \pi_{4}(\event_1) \neq_q \pi_{4}(\event_2)) 
% \big)
% \end{array}
% \]
% \end{defn}
% %
% %
\paragraph*{Trace}
A trace $\trace \in \mathcal{T} $ is a list of events, 
collecting the events generated along the program execution. $\mathcal{T} $ represents the set of traces. There are some useful operators: the trace concatenation operator $\tracecat: \mathcal{T} \to \mathcal{T} \to \mathcal{T}$, combines two traces.
The belongs to operator $\in : \eventset \to \mathcal{T} \to \{\etrue, \efalse \} $ and its opposite $\not\in$
express whether or not an event belongs to a trace.
Another operator $\llabel : \mathcal{T} \to \mathcal{VAR} \to \{\mathbb{N}\}\cup \{\bot\}$,
takes a trace and a variable as input and returns the label of the latest assignment event which assigns value to that variable. 
% I also have the operator $\tlabel : \mathcal{T} \to \ldom$, which gives the set of labels in every event belonging to a trace. 
% The full definitions of these above operators can be found in the appendix.
% \[
% \begin{array}{llll}
% \mbox{Trace} & \trace
% & ::= & [] ~|~ \trace :: \event
% \end{array}
% \]
%
A trace can be regarded as the program history, which records queries asked by the analyst during the execution of the program. I collect the trace with a trace-based operational semantics based on transitions of the form $ \config{c, \trace} \to \config{c', \trace'} $. It states that a configuration $\config{c, \trace}$, which consists of a command $c$ to be evaluated and a starting trace $\trace$, evaluates to another configuration with the trace updated along with the evaluation of the command $c$ to the normal form of the command $\eskip$.
% \jl{I introduce some operations here: the trace concatenation $\tracecat: \mathcal{T} \to \mathcal{T} \to \mathcal{T}$, which combines two traces; they belong to operator $\in$ so that an event $\event \in \eventset$ belongs to a trace $\trace$ is notated by $\event \in \trace$. 
% As usual, we denote by $\event \notin \trace$ that the event $\event$ doesn't belong to the trace $\trace$. 
% Another operator $\llabel : \mathcal{T} \to \mathcal{VAR} \to \{\mathbb{N}\}\cup \{\bot\}$,
% takes a trace and a variable and returns the label of the latest assignment event which assigns value to that variable. I also have the operator $\tlabel : \mathcal{T} \to \mathcal{P}{(\mathbb{N})}$, which gives the set of labels in every event belonging to a trace. The full definitions of these above operators can be found in the appendix.
% }
% \wq{It seems trace concatenation and event belonging to a trace do not deserve so much space here:-)}
%\jl{I agree}

% \\
% I also introduce a counting operator $\vcounter : \mathcal{T} \to \mathbb{N} \to \mathbb{N}$, 
% % \wq{which counts the occurrence of a variable in the trace,} 
% which counts the occurrence of a labeled variable in the trace,
% whose behavior is defined as follows,
% % \[
% % \begin{array}{lll}
% % \vcounter(\trace :: (x, l, v, \bullet) ) l \triangleq \vcounter(\trace) l + 1
% % &
% % \vcounter(\trace ::(b, l, v, \bullet) ) l \triangleq \vcounter(\trace) l + 1
% % &
% % \vcounter(\trace :: (x, l, v, \qval) ) l \triangleq \vcounter(\trace) l + 1
% % \\
% % \vcounter(\trace :: (x, l', v, \bullet) ) l \triangleq \vcounter(\trace ) l, l' \neq l
% % &
% % \vcounter(\trace :: (b, l', v, \bullet) ) l \triangleq \vcounter(\trace ) l, l' \neq l
% % &
% % \vcounter(\trace :: (x, l', v, \qval)) l \triangleq \vcounter(\trace ) l, l' \neq l
% % \\
% % \vcounter({[]}) l \triangleq 0
% % &&
% % \end{array}
% % \]
% \[
% \begin{array}{lll}
% \vcounter(\trace :: (x, l, v, \bullet), l ) \triangleq \vcounter(\trace, l) + 1
% &
% \vcounter(\trace ::(b, l, v, \bullet), l) \triangleq \vcounter(\trace, l) + 1
% &
% \vcounter(\trace :: (x, l, v, \qval), l) \triangleq \vcounter(\trace, l) + 1
% \\
% \vcounter(\trace :: (x, l', v, \bullet), l) \triangleq \vcounter(\trace, l), l' \neq l
% &
% \vcounter(\trace :: (b, l', v, \bullet), l) \triangleq \vcounter(\trace, l), l' \neq l
% &
% \vcounter(\trace :: (x, l', v, \qval), l) \triangleq \vcounter(\trace, l), l' \neq l
% \\
% \vcounter({[]}, l) \triangleq 0
% &&
% \end{array}
% \]
% \input{trace}
%%% trace, queries
% A memory is standard, a map from variables to values. Queries can be uniquely annotated as $\mathcal{AQ}$, and the annotation $(l,w)$ considers the location of the query by line number $l$ and which iteration the query is at when it appears in a loop statement, specified by $w$. A trace $t$ is a list of annotated queries accumulated along the execution of the program. 



\paragraph*{Environment}
The function $\env : {\mathcal{T}} \to \mathcal{VAR} \to \mathcal{VAL} \cup \{\bot\}$, which maps a trace and a variable to the latest value assigned to this variable on the trace is defined as follows.
% \wq{Question: Seem $\env$ is a function that looks up in the input trace and returns you the latest value of the variable. I have a question, in the two-round example, I see $env(\tau)(k)$ while $k$ is not defined(it is input), so in our two-round example in Overview, the value is stored in the second event is $\bot$? Also, another important, $\env$ relies on the input trace, so it will not appear in the trace, or config, is it precise?}
% \jl{yes, it is precise. 
% I have initial trace and everything belonging is defined over all possible initial traces.
%in the two-round example, there is an initial trace where the value of k is defined there. It is worth explaining this here.
% }
\[
\begin{array}{lll}
\env(\trace \traceadd (x, l, v, \bullet)) x \triangleq v
&
\env(\trace \traceadd (y, l, v, \bullet)) x \triangleq \env(\trace) x, y \neq x
&
\env(\trace \traceadd (b, l, v, \bullet)) x \triangleq \env(\trace) x
\\
\env(\trace \traceadd (x, l, v, \qval)) x \triangleq v
&
\env(\trace \traceadd (y, l, v, \qval)) x \triangleq \env(\trace) x, y \neq x
&
\env({[]} ) x \triangleq \bot
\end{array}
\]
 %% trace

%
% figure, evaluation rules.
% {\footnotesize
% \begin{figure}
% \begin{mathpar}
% \boxed{ \config{m, c, t,w} \xrightarrow{} \config{m', c', t', w'} \; }
% \and
% %
% {\inferrule
% {
% \valr_N > 0
% }
% {
% \config{m, \eloop ~ [\valr_N]^{l} ~ \edo ~ c , t, w }
% \xrightarrow{} \config{m, c ; \eloop ~ [(\valr_N-1)]^{l} ~ \edo ~ c , t, (w + l) }
% }
% ~\textbf{low-loop}
% }
% %
% \and
% %
% \inferrule
% {
% }
% {
% \config{m, [\eskip]^{l} ; c_2, t,w} \xrightarrow{} \config{m, c_2, t,w}
% }
% ~\textbf{low-seq2}
% %
% \quad
% %
% {
% \inferrule
% {
% \valr_N = 0
% }
% {
% \config{m, \eloop ~ [\valr_N]^{l} ~ \edo ~ c , t, w }
% \xrightarrow{} \config{m, [\eskip]^{l} , t, (w \setminus l) }
% }
% ~\textbf{low-loop-exit}
% }
% \and
% %
% \inferrule
% {
% }
% {
% \config{m, \eif([\efalse]^{l}, c_1, c_2), t,w} 
% \xrightarrow{} \config{m, c_2, t,w}
% }
% ~\textbf{low-if-f}
% %
% ~~
% % { Memory \times Com \times Trace \times WhileMap \Rightarrow^{} Memory \times Com \times Trace \times WhileMap}
% \inferrule
% {
% \config{m,\expr} \to \expr'
% }
% {
% \config{m, [\assign{x}{q(\expr)}]^l, t, w} \xrightarrow{} \config{m, [\assign{x}{q(\expr')}]^l, t, w}
% }
% ~\textbf{low-query-e}
% %
% \and
% %
% %
% \inferrule
% {
% \config{m, c_1, t,w} \xrightarrow{} \config{m', c_1', t',w'}
% }
% {
% \config{m, c_1; c_2, t,w} \xrightarrow{} \config{m', c_1'; c_2, t',w'}
% }
% ~\textbf{low-seq1}
% ~~
% \inferrule
% {
% q(v) = v_q
% }
% {
% \config{m, [\assign{x}{q(v)}]^l, t, w} \xrightarrow{} \config{m[ v_q/ x], \eskip, t \mathrel{++} [q(v)^{(l,w )}],w }
% }
% ~\textbf{low-query-v}
% %
% % \inferrule
% % {
% % }
% % {
% % \config{m, [\assign x v]^{l}, t,w} \xrightarrow{} \config{m[v/x], [\eskip]^{l}, t,w}
% % }
% % ~\textbf{low-assn}
% %
% %
% %
% \and
% %
% \inferrule
% {
% \config{ m, \bexpr} \barrow \bexpr'
% }
% {
% \config{m, \eif([\bexpr]^{l}, c_1, c_2), t,w} 
% \xrightarrow{} \config{m, \eif([\bexpr']^{l}, c_1, c_2), t,w}
% }
% ~\textbf{low-if}
% %
% ~~~~
% %
% \inferrule
% {
% }
% {
% \config{m, \eif([\etrue]^{l}, c_1, c_2),t,w} 
% \xrightarrow{} \config{m, c_1, t,w}
% }
% ~\textbf{low-if-t}
% %
% % %
% %
% \end{mathpar}
% \vspace{-0.3cm}
% \caption{Trace-based operational semantics}
% \label{fig:evaluation}
% \vspace{-0.5cm}
% \end{figure}
% }
%
% explanation of rules

%
\begin{figure}
 \begin{mathpar}
 \boxed{
 \mbox{Command $\times$ Trace}
 \xrightarrow{}
 \mbox{Command $\times$ Trace}
 }
 \and
 \boxed{\config{{c, \trace}}
 \xrightarrow{} 
 \config{{c', \trace'}}
 }
 \\
 % \inferrule
 % {
 % \empty
 % }
 % {
 % \config{\clabel{\eskip}^l, \trace } 
 % \xrightarrow{} 
 % \config{\clabel{\eskip}^l, \trace}
 % }
 % ~\textbf{skip}
 %
 % \and
 %
 \inferrule
 {
 \config{\trace, \expr} \earrow v 
 \and
 \event = ({x}, l, v, \bullet)
 }
 {
 \config{[\assign{{x}}{\expr}]^{l}, \trace } 
 \xrightarrow{} 
 \config{\clabel{\eskip}^l, \trace \traceadd \event}
 }
 ~\textbf{assn}
 %
 \and
 %
 \highlight{
 \inferrule
 {
\config{ \trace, \qexpr} \qarrow \qval
 \and 
 \query(\qval) = v
 \and 
 \event = ({x}, l, v, \qval)
 }
 {
 \config{{[\assign{x}{\query(\qexpr)}]^l, \trace}}
 \xrightarrow{} 
 \config{{\clabel{\eskip}^l, \trace \traceadd \event} }
 }
 ~\textbf{query}
 }
 %
 \and
 %
 \inferrule
 {
\config{ \trace, b} \barrow \etrue
 \and 
 \event = (b, l, \etrue, \bullet)
 }
 {
 \config{{\ewhile [b]^{l} \edo c, \trace}}
 \xrightarrow{} 
 \config{{
 c; \ewhile [b]^{l} \edo c),
 \trace \traceadd \event}}
 }
 ~\textbf{while-t}
 %
 %
 \quad
 %
 \inferrule
 {
 \config{\trace, b} \barrow \efalse
 \and 
 \event = (b, l, \efalse, \bullet)
 }
 {
 \config{{\ewhile [b]^{l}, \edo c, \trace}}
 \xrightarrow{} 
 \config{{
 \clabel{\eskip}^l,
 \trace \traceadd \event}}
 }
 ~\textbf{while-f}
 %
 %
 \and
 %
 %
 \inferrule
 {
 \config{{c_1, \trace}}
 \xrightarrow{}
 \config{{\clabel{\eskip}^l, \trace'}}
 \and 
 \config{{\clabel{\eskip}^l; c_2, \trace'}} \xrightarrow{} \config{{ \clabel{\eskip}^l, \trace''}}
 }
 {
 \config{{c_1; c_2, \trace}} 
 \xrightarrow{} 
 \config{{\clabel{\eskip}^l, \trace''}}
 }
 ~\textbf{seq}
 %
 % \and
 % %
 % \inferrule
 % {
 % \config{{c_2, \trace}}
 % \xrightarrow{}
 % \config{{c_2', \trace'}}
 % }
 % {
 % \config{{\clabel{\eskip}^l; c_2, \trace}} \xrightarrow{} \config{{ c_2', \trace'}}
 % }
 % ~\textbf{seq2}
 %
 \quad
 %
 %
 \inferrule
 {
 \trace, b \barrow \etrue
 \and 
 \event = (b, l, \etrue, \bullet)
 }
 {
 \config{{
 \eif([b]^{l}, c_1, c_2), 
 \trace}}
 \xrightarrow{} 
 \config{{c_1, \trace \traceadd \event}}
 }
 ~\textbf{if-t}
 %
 % \and
 % %
 % \inferrule
 % {
 % \trace, b \barrow \efalse
 % \and 
 % \event = (b, l, \efalse, \bullet)
 % }
 % {
 % \config{{\eif([b]^{l}, c_1, c_2), \trace}}
 % \xrightarrow{} 
 % \config{{c_2, \trace \traceadd \event}}
 % }
 % ~\textbf{if-f}
 \end{mathpar}
 % \end{subfigure}
 \vspace{-0.5cm}
 \caption{Trace-based Operational Semantics for Language.}
 \label{fig:os}
 \end{figure}
 %

% {The big step trace-based operational semantics has the form of $ \config{c, \trace} \xrightarrow{} { \config{c', \trace'}}$. It reads that the configuration $(c, \trace)$ with labeled command $c$ and trace $\trace$, will be evaluated to another configuration, in which $c$ is evaluated to $c'$ and the trace is updated during the evaluation, to $\trace'$. 
% }
% The step trace-based operational semantics has the form of $ \config{c, \trace} \xrightarrow{} { \config{c', \trace'}}$. 
% It reads the configuration $\config{c, \trace}$ consisting of a labeled command $c$ and a pre-trace $\trace$, 
% and evaluates it to another configuration, 
% in which $c$ is evaluated to $c'$ and trace $\trace$ is updated to $\trace'$. 
% is updated during the evaluation,
\paragraph*{Operational Semantics}
I give a selection of rules of the trace-based operational semantics in Figure~\ref{fig:os}. 

% \todo{Make sure the operational semantics is a big step and correct assn rules.}
% \jl{
The rule $\textbf{assn}$ evaluates a standard assignment $\assign{x}{\expr}$, the expression $\expr$ is first evaluated by our expression evaluation $\config{\trace, \expr} \earrow v $, presented below. And the result $v$ of evaluating $\expr$ is used to construct a new event $\event = (x, l, v,\bullet)$ and attach it to the previous trace. 
\begin{mathpar}
% \boxed{ \config{\trace, \expr} \earrow v \, : \, \mbox{Trace $\times$ Expression $\Rightarrow$ Value} }
% \\
\inferrule{ 
 \config{\trace, \aexpr} \aarrow v
}{
 \config{\trace, \aexpr} 
 \earrow v
}
\and
\inferrule{ 
 \config{\trace, \bexpr} \barrow v
}{
 \config{\trace, \bexpr} 
 \earrow v
}
\and
\inferrule{ 
 \config{\trace, \expr_1} \earrow v_1
 \cdots
 \config{\trace, \expr_n} \earrow v_n
}{
 \config{\trace, [\expr_1, \cdots, \expr_n]} 
 \earrow [v_1, \cdots, v_n]
}
\and
\inferrule{ 
 \empty
}{
 \config{\trace, v} 
 \earrow v
}
\end{mathpar}
The expression evaluation rules also rely on the evaluation of arithmetic expressions $\config{\trace,\aexpr} \aarrow v $ and boolean expressions $\config{\trace, \bexpr} \barrow v $. The full rules can be found in the appendix.
% \begin{mathpar}
% \boxed{ \config{\trace,\aexpr} \aarrow v \, : \, \mbox{Trace $\times$ Arithmetic Expr $\Rightarrow$ Arithmetic Value} }
% % \text{\mg{Missing. Without these rules, it is difficult to understand why we need a trace to evaluate expressions.}}
% \\
% \inferrule{ 
% \empty
% }{
% \config{\trace, n} 
% \aarrow n
% }
% \and
% \inferrule{ 
% \env(\trace) x = v
% }{
% \config{\trace, x} 
% \aarrow v
% }
% \and
% \inferrule{ 
% \config{\trace, \aexpr_1} \aarrow v_1
% \and 
% \config{\trace, \aexpr_2} \aarrow v_2
% \and 
% v_1 \oplus_a v_2 = v
% }{
% \config{\trace, \aexpr_1 \oplus_a \aexpr_2} 
% \aarrow v
% }
% % \and
% % \inferrule{ 
% % \config{\trace, \aexpr} \aarrow v'
% % \and 
% % \elog v' = v
% % }{
% % \config{\trace, \elog \aexpr} 
% % \aarrow v
% % }
% % \and
% % \inferrule{ 
% % \config{\trace, \aexpr} \aarrow v'
% % \and 
% % \esign v' = v
% % }{
% % \config{\trace, \esign \aexpr} 
% % \aarrow v
% % }
% \\
% \boxed{ \config{\trace, \bexpr} \barrow v \, : \, \mbox{Trace $\times$ Boolean Expr $\Rightarrow$ Boolean Value} }
% % \text{\mg{Missing. Without these rules, it is difficult to understand why we need a trace to evaluate expressions.}}
% \\
% % \inferrule{ 
% % \empty
% % }{
% % \config{\trace, \efalse} 
% % \barrow \efalse
% % }
% % \and 
% % \inferrule{ 
% % \empty
% % }{
% % \config{\trace, \etrue} 
% % \barrow \etrue
% % }
% % \and 
% \inferrule{ 
% \config{\trace, \bexpr} \barrow v'
% \\ 
% \neg v' = v
% }{
% \config{\trace, \neg \bexpr} 
% \barrow v
% }
% \and 
% \inferrule{ 
% \config{\trace, \bexpr_1} \barrow v_1
% \\ 
% \config{\trace, \bexpr_2} \barrow v_2
% \\ 
% v_1 \oplus_b v_2 = v
% }{
% \config{\trace, \bexpr_1 \oplus_b \bexpr_2} 
% \barrow v
% }
% \and 
% \inferrule{ 
% \config{\trace, \aexpr_1} \aarrow v_1
% \\ 
% \config{\trace, \aexpr_2} \aarrow v_2
% \\ 
% v_1 \sim v_2 = v
% }{
% \config{\trace, \aexpr_1 \sim \aexpr_2} 
% \barrow v
% }
% \end{mathpar}
% % }


Distinguished from the standard assignment evaluation, 
the rule $\textbf{query}$ 
evaluates a query requesting command $\clabel{\assign{x}{\query(\qexpr)}}^l$ in two steps.
The query expression $\qexpr$ is first evaluated into a query value $\qval$ by following the rules below.
Then, by sending this query request $\query(\qval)$ to a hidden mechanism, this query is evaluated to a result value returned from it, $v = \query(\qval)$.
% by sending this query request $\query(\qval)$ to it.
Also, the generated event stores both the query value $\alpha$ here, and the result value of the query request.

\begin{mathpar}
% \boxed{ \config{\trace, \qexpr} \qarrow \qval \, : \, \mbox{Trace $\times$ Query Expr $\Rightarrow$ Query Value} }
% \\
\inferrule{ 
 \config{\trace, \aexpr} \aarrow n
}{
 \config{\trace, \aexpr} 
 \qarrow n
}
\and
\inferrule{ 
 \config{\trace, \qexpr_1} \qarrow \qval_1
 \and
 \config{\trace, \qexpr_2} \qarrow \qval_2
}{
 \config{\trace, \qexpr_1 \oplus_a \qexpr_2} 
 \qarrow \qval_1 \oplus_a \qval_2
}
\and
\inferrule{ 
 \config{\trace, \aexpr} \aarrow n
}{
 \config{\trace, \chi[\aexpr]} \qarrow \chi[n]
}
\and
\inferrule{ 
 \empty
}{
 \config{\trace, \qval} 
 \qarrow \qval
}
 \end{mathpar}
% }
% \wq{The rules for if hand while both have two versions, when the guard evaluates to true and false, respectively. In these rules, the evaluation of the guard also generates a testing event and our trace is updated as well. }
The rules for if and while both have two versions 
when the boolean expressions in the guards are evaluated to true and false, respectively. 
In these rules, the evaluation of the guard generates a testing event and the trace is updated as well by appending this event.
% The rule $\textbf{query}$ evaluates the argument of a query request to a normal form and obtains the answer $v_q$ of the query $\query(v)$ from the mechanism. 
% Then the trace is expanded by appending the query expression $\query(v)$ with the current annotation $(l,w)$. 

% The rule for assignment is standard and the trace remains unchanged. The sequence rule keeps tracking the modification of the trace, and the evaluation rule for if conditional 

% \jl{If we observe the operational semantics rules, we can find that no rule will shrink the trace.} 
% If we observe the operational semantics rules, we can find that no rule will shrink the trace. It is proved in the appendix.
% So we have the Lemma~\ref{lem:tracenondec}, specifically, the trace has the property that its length never decreases during the program execution.

% \begin{lem}
% [Trace Non-Decreasing]
% \label{lem:tracenondec}
% For every program $c \in \cdom$ and traces $\trace, \trace' \in \mathcal{T}$, if 
% $\config{c, \trace} \rightarrow^{*} \config{\eskip, \trace'}$,
% then there exists a trace $\trace'' \in \mathcal{T}$ with $\trace \tracecat \trace'' = \trace'$
% %
% $$
% \forall \trace, \trace' \in \mathcal{T}, c \st
% \config{c, \trace} \rightarrow^{*} \config{\eskip, \trace'} 
% \implies \exists \trace'' \in \mathcal{T} \st \trace \tracecat \trace'' = \trace'
% $$
% \end{lem}
% %
% % \mg{This corollary needs some explanation. In particular, we should stress that $\event$ and $\event'$ may differ in the query value.}
% % Since the equivalence over two events is defined over the query value equivalence, 
% % when there is an event 
% % belonging to a trace, 
% % it is possible that the event showing up in this trace has a different form of query value, but they are equivalent by Definition~\ref{def:query_equal}.
% Since the equivalence over two events is defined over the query value equivalence, 
% when there is an event belonging to a trace, 
% if this event is a query assignment event, 
% it is possible that 
% the event showing up in this trace has a different form of query value, 
% but they are equivalent by Definition~\ref{def:query_equal}.
% So we have the following Corollary~\ref{coro:aqintrace} with proof in Appendix.
% % ~\ref{apdx:lemma_sec123}.
% % \todo{we should stress that $\event$ and $\event'$ may differ in the query value.}
% \begin{coro}
% \label{coro:aqintrace}
% For every event and a trace $\trace \in \mathcal{T}$,
% if $\event \in \trace$, 
% then there exist another event $\event' \in \eventset$ and traces $\trace_1, \trace_2 \in \mathcal{T}$
% such that $\trace_1 \tracecat [\event'] \tracecat \trace_2 = \trace $
% with 
% $\event$ and $\event'$ equivalent but may differ in their query value.
% \[
% \forall \event \in \eventset, \trace \in \mathcal{T} \st
% \event \in \trace \implies \exists \trace_1, \trace_2 \in \mathcal{T}, 
% \event' \in \eventset \st (\event = \event') \land \trace_1 \tracecat [\event'] \tracecat \trace_2 = \trace 
% \]
% \end{coro}



\subsection{{\tt Query While} Language through Examples}
\label{sec:language-examples}
\begin{example}[Evaluation Trace for A Two Rounds Data Analysis Example Program]
    In the following example program $\kw{towRounds(k)}$, the same one as Figure~\ref{fig:overview_tworounds},
    the analyst asks in total $k+1$ queries to the mechanism in two phases.
    %
    \[         \begin{array}{l}
      \kw{towRounds(k)} \triangleq \\
             \clabel{ \assign{a}{0}}^{0} ;
              \clabel{\assign{j}{k} }^{1} ;\\
              \ewhile ~ \clabel{j > 0}^{2} ~ \edo ~
              \Big(
               \clabel{\assign{x}{\query(\chi[j] \cdot \chi[k])} }^{3}  ;
               \clabel{\assign{j}{j-1}}^{4} ;
              \clabel{\assign{a}{x + a}}^{5}       \Big);\\
              \clabel{\assign{l}{\query(\chi[k]*a)} }^{6}
          \end{array}
          \]
    In the first phase, the analyst asks $k$ queries and stores the answers that are provided by the mechanism. 
    In the second phase, the analyst constructs a new query based on the results of the previous $k$ queries and sends this query to the mechanism. More specifically, I assume that, in this example, the domain $\dbdom$ 
    contains at least $k$ numeric attributes, which I index just by natural numbers. 
    The queries inside the while loop correspond to the first phase and compute an approximation of 
    the product of the empirical mean of the first $k$ attributes. 
    The query outside the loop corresponds to the second phase and computes an approximation of the empirical mean where each record is weighted by the sum of the empirical mean of the first $k$ attributes.
    %
    With the initial trace
    $\trace_0 = [(k, in, 2, \bullet)]$,
    the trace generated from this example is 
   %   where $k$ is the 
   %  initial value of input variable $k$ given by user,
   %  we observe the execution trace as
   \\
    $
    \left[\begin{array}{l}
    % \trace_0 \tracecat
     (a, 0, 0, \bullet),
    (j, 1, 2, \bullet),
    (j>0, 2, \etrue, \bullet),
    (x, 3, v_1, \chi[2]*\chi[2]),
    (j, 4, 1, \bullet),
    (a, 5, v_1, \bullet),\\
    (j>0, 2, \etrue, \bullet),
    (x, 3, v_2, \chi[1]*\chi[2]),
    (j, 4, 0, \bullet),
    (a, 5, v_1 + v_2, \bullet),
    (j>0, 4, \efalse, \bullet),\\
    (l, 6, v_3, \chi[2]*( v_1 + v_2))
    \end{array} \right]
    $.
\end{example}
