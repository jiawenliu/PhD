Previous works formalized the adaptive data analysis into a while-like language named loop language with  limited expressiveness.
It is presented in {Thesis~\cite{weihao22}}.
\subsection*{Syntax}
Figure~\ref{fig:prework_syntax} is a selection of the syntax from their loop language.
{\small
\begin{figure}
\[
\begin{array}{llll}
    % \mbox{Values } & v & ::= & n \sep \etrue \sep \efalse \sep \chi \sep [] ~|~ [v, \dots, v] ~|~ \chi[v] \\
 \mbox{Arithmetic Operators} & \oplus_a & ::= & + ~|~ - ~|~ \times 
%
~|~ \div \\  
  \mbox{Boolean Operators} & \oplus_b & ::= & \lor ~|~ \land ~|~ \neg\\
  %
   \mbox{Relational Operators} & \sim & ::= & < ~|~ \leq ~|~ == \\  
%  \mbox{Label} & l & := & \mathbb{N} \\ 
%  \mbox{Loop Maps} & w & \in & \mbox{Label} \times \mathbb{N} \\
\mbox{Arithmetic Expressions} & \aexpr & ::= & 
	%
	n ~|~ x ~|~ \aexpr \oplus_a \aexpr  \\
% 	\sep \pi (l , \aexpr, \aexpr) \\
    %
\mbox{Boolean Expressions} & \bexpr & ::= & 
	%
	\etrue ~|~ \efalse  ~|~ \neg \bexpr
	 ~|~ \bexpr \oplus_b \bexpr
	%
	~|~ \aexpr \sim \aexpr \\
\mbox{Expressions} & \expr & ::= & \aexpr ~|~ \bexpr ~|~ [] ~|~ [\expr, \dots, \expr] \\	
\mbox{Values} & v & ::= & n ~|~ \etrue ~|~ \efalse ~|~ [] ~|~ [v, \dots, v] \\
\mbox{Query expressions} & \expr_q & ::= & \aexpr ~|~ \chi ~|~ \chi[\aexpr] ~|~ \expr_q \oplus_a \expr_q \\
\mbox{Query Values} & v_q & ::= & n ~|~ \chi ~|~ \chi[n] ~|~ v_q \oplus_a  v_q \\
% \mbox{Labelled commands} & c & ::= & 
% [\assign x \expr]^{l} ~|~  [\assign x q(e_q)]^{l}
%  ~|~  \eloop ~ [\aexpr]^{l} ~ \edo ~ c  ~|~ c;c \\
%  & & & ~|~ \eif([\bexpr]^l, c, c) 	 ~|~ [\eskip]^{l} \\
\mbox{Commands} & c & ::= &  \eskip  ~|~  \assign x \expr ~|~  \assign{x}{ q(\expr_q)}
%
~|~ \eloop ~ \aexpr  ~ \edo ~ c  \\ &&& ~|~ c;c  ~|~ \eif(\bexpr, c, c)
\end{array}
\]
 \caption{Syntax of loop language.}
    \label{fig:prework_syntax}
\end{figure}
}
The expressions can be arithmetic expressions, boolean expressions or query expression.
The arithmetic expressions boolean expressions are standard.
They extend the expression with the query expression in order to support the query requests in the data analysis.
The special variable $\chi$ represents a row of the database,
and access to values at a certain index in $\chi$, as $\chi[\aexpr]$.
But it neither  support while loop with non-deterministic iterations, nor any user input.
%
\subsection*{Trace-Based Operational Semantics}
The previous operational semantics is defined based on labeled command and trace with some special operators.
Below is a summary of their designs.
\paragraph*{Labeled Commands}
They first annotate each command with a label $l$,
a natural number standing for the line of code where the command appears.
They associate the label $l$ to the conditional predicate $\bexpr$ in the if statement,
and to the loop counter $\aexpr$ in the loop statement.
\[
\begin{array}{llll}
     \mbox{Labeled commands} & c & ::= &   [\assign x \expr]^{l} ~|~  [\assign x q(e_q)]^{l}
 ~|~  \eloop ~ [\aexpr]^{l} ~ \edo ~ c  ~|~ c;c \\
 & & & ~|~ \eif([\bexpr]^l, c, c) 	 ~|~ [\eskip]^{l} \\
\end{array}
\]
% Each command is now labeled 
\paragraph*{Loop Map}
Then, they have the Loop map, which is a map from the label $l$ to the iteration number $n$.
%   Because statements in the loop share the same line number,  varied iterations , the label $l$ is not enough to distinguish statements.
  A mapping $[k \to n]$ gives accurate information on which loop a statement is in by its key $k$ (label at loop counter),
  and which iteration $n$ the statement belongs to.
  % For example, the loop map $w=[3:1, 4:2]$ indicates that the statement is currently in a nested loop, the outer loop starting from label $3$ and in its first iteration, the statement is now in the inner loop starting from label $4$ and in the second iteration. We use $\emptyset$ to represent an empty map, indicating the statement is not in any loop.
\[
\begin{array}{llll}
 \mbox{Loop Map} & w & \in & \mbox{Label} \to \mathbb{N} \\
\mbox{Annotated Query} & \mathcal{AQ}  & ::= & \{ q(v_q)^{(l,w)}  \} \\
\end{array}
\begin{array}{llll}
    \mbox{Memory} & m & ::= & [] ~|~ m[x \to v] \\
\mbox{Trace} & t & ::= & [] ~|~ q(v_q)^{(l, w) } :: t \\
\end{array}
\]
%  Then, they have the Loop map, which is a map from the label $l$ to the iteration number $n$.
% %   Because statements in the loop share the same line number,  varied iterations , the label $l$ is not enough to distinguish statements.
%   A mapping $[k \to n]$ gives accurate information on which loop a statement is in by its key $k$ (label at loop counter),
%   and which iteration $n$ the statement belongs to.
%   % For example, the loop map $w=[3:1, 4:2]$ indicates that the statement is currently in a nested loop, the outer loop starting from label $3$ and in its first iteration, the statement is now in the inner loop starting from label $4$ and in the second iteration. We use $\emptyset$ to represent an empty map, indicating the statement is not in any loop.

 \paragraph*{Annotated Query} 
 They 
distinctively design the annotated query, in order to analyze the relation between queries. This is a key extension for analyzing the \emph{adaptivity}
Through this annotation, queries can be uniquely annotated as $\mathcal{AQ}$,
  and the annotation $(l,w)$ considers the location of the query
  by line number $l$ and which iteration the query is at when it appears in a loop statement, specified by $w$.
  %% trace
\paragraph{Trace} 
A trace $t$ is a list of annotated queries accumulated along with the execution of the program. 
%   A trace can be regarded as the program history, where this history 
  It consists of the queries asked by the analyst during the execution of the program,
%   We 
collected through
  a trace-based small-step operational semantics based on transitions of the form $ \config{m,c, t, w} \to \config{m', \eskip, t', w'} $.
  % \paragraph{Memory}
  % The memory in their language is standard, which is a map from variables to values.
  \paragraph*{Operational Semantics Rules}
Figure~\ref{fig:evaluation} is a selection of rules of their trace-based operational semantics from {Thesis~\cite{weihao22}}.
Only the rules related to query requests and the while loop evaluations are selected here.
The rule $\textbf{l-query-e}$ evaluates the argument $\expr_q$ of a query request $q(\expr_q)$ using the query evaluation $\qarrow$.
When the query expression is in the normal form, this query will be answered.
The rule $\textbf{l-query-v}$ modifies the starting memory $m$ to $m[v_q/x]$ using the answer $v$ of the query $q(v_q)$ from the mechanism, with the trace expanded by appending the query $q(v_q)$ with the current annotation $(l,w)$.
The rule for assignment is standard and the trace remains unchanged.
The sequence rule keeps tracking the modification of the trace, and the evaluation rule for if conditional goes into one branch based on the result of the conditional predicate $\bexpr$. 
The rule \textbf{l-loop-a} first evaluates the loop counter $\aexpr$, when the loop counter is a number, then the evaluation will start to execute the loop body.
The rules for loop modify the loop map $w$. In the rule $\textbf{l-loop}$, the loop map $w$ is updated by $w + l$ because the execution goes into another iteration when the condition $v_N >0$ is satisfied.
% When $v_N$ reaches $0$, the loop exits and the loop map $w$ eliminates the label $l$ of this loop statement by $w \setminus l$ in the rule $\textbf{l-loop-exit}$. 
% 
\begin{figure}
{\footnotesize
  \begin{mathpar}
  % \boxed{ \config{m, c, t,w} \xrightarrow{} \config{m', c',  t', w'} \; }
  % \\
  % \inferrule
  % {
  %  \config{m, \expr } \xrightarrow{}  \config{m, \expr' }
  % }
  % {
  % \config{m, [\assign x \expr]^{l},  t,w} \xrightarrow{} \config{m, [\assign x \expr']^{l}, t,w}
  % }
  % ~\textbf{l-assn1}
  % \and
  % %
  % \inferrule
  % {
  % }
  % {
  % \config{m, [\assign x v]^{l},  t,w} \xrightarrow{} \config{m[v/x], [\eskip]^{l}, t,w}
  % }
  % ~\textbf{l-assn2}
  % %
  % \and
  {\inferrule
  {
    \config{m, \aexpr} \aarrow \config{m, \aexpr'}
  }
  {
  \config{m, \eloop ~ [\aexpr]^{l}  ~ \edo ~ c ,  t, w }
  \xrightarrow{} \config{m, \eloop ~ [\aexpr']^{l} ~ \edo ~ c ,  t, (w + l) }
  }
  ~\textbf{l-loop-a}
  }
  %
  \and
  %
  {\inferrule
  {
    \valr_N > 0
  }
  {
  \config{m, \eloop ~ [\valr_N]^{l}  ~ \edo ~ c ,  t, w }
  \xrightarrow{} \config{m, c ;  \eloop ~ [(\valr_N-1)]^{l} ~ \edo ~ c ,  t, (w + l) }
  }
  ~\textbf{l-loop}
  }
  %
  % \and
  % %
  % {
  % \inferrule
  % {
  %   \valr_N = 0
  % }
  % {
  % \config{m,  \eloop ~ [\valr_N]^{l} ~ \edo ~ c  ,  t, w }
  % \xrightarrow{} \config{m, [\eskip]^{l} ,  t, (w \setminus l) }
  % }
  % ~\textbf{l-loop-exit}
  % }
  %
  \and
  % {  Memory \times Com  \times Trace \times WhileMap \Rightarrow^{} Memory \times Com  \times Trace \times WhileMap}
  \inferrule
  {
  \config{m,\expr_q} \qarrow \config{m,\expr_q'}
  }
  {
  \config{m, [\assign{x}{q(\expr_q)}]^l, t, w} \xrightarrow{}  \config{m, [\assign{x}{q(\expr_q')}]^l, t, w}
  }
  ~\textbf{l-query-e}
  \and
  \inferrule
  {
  q(v_q) = v
  }
  {
  \config{m, [\assign{x}{q(v_q)}]^l, t, w} \xrightarrow{} \config{m[ v/ x], \eskip,  t \mathrel{++} [q(v_q)^{(l,w )}],w }
  }
  ~\textbf{l-query-v}
  %
  \and
  %
  %
  % \inferrule
  % {
  % \config{m, c_1,  t,w} \xrightarrow{} \config{m', c_1',  t',w'}
  % }
  % {
  % \config{m, c_1; c_2,  t,w} \xrightarrow{} \config{m', c_1'; c_2, t',w'}
  % }
  % ~\textbf{l-seq1}
  % %
  % \and
  % %
  % \inferrule
  % {
  % }
  % {
  % \config{m, [\eskip]^{l} ; c_2,  t,w} \xrightarrow{} \config{m, c_2,  t,w}
  % }
  % ~\textbf{l-seq2}
  % %
  % %
  % \and
  % %
  % \inferrule
  % {
  % \config{ m, \bexpr} \barrow \bexpr'
  % }
  % {
  % \config{m, \eif([\bexpr]^{l}, c_1, c_2),  t,w} 
  % \xrightarrow{} \config{m,  \eif([\bexpr']^{l}, c_1, c_2),  t,w}
  % }
  % ~\textbf{l-if}
  % %
  % \and
  %
  % \inferrule
  % {
  % }
  % {
  % \config{m, \eif([\etrue]^{l}, c_1, c_2),t,w} 
  % \xrightarrow{} \config{m, c_1,  t,w}
  % }
  % ~\textbf{l-if-t}
  % \and
  % %
  % \inferrule
  % {
  % }
  % {
  % \config{m,  \eif([\efalse]^{l}, c_1, c_2),  t,w} 
  % \xrightarrow{} \config{m, c_2,  t,w}
  % }
  % ~\textbf{l-if-f}
  %
  % %
  %
  \end{mathpar}
  }
        \caption{Trace-based operational semantics of loop language.}
        \label{fig:evaluation}
    \end{figure}
    %
    % explanation of rules
 \highlight{
\paragraph*{Limitations of The Trace-Based Operational Semantics}
This operational semantics is limited in the following aspects:
\begin{itemize}
 \item \textbf{Expressiveness Limitation}
 \\
 The operational-semantics causes a similar limitation as their syntax.
 The loop iteration number can only be a constant number or an arithmetic expression evaluated to a nature number.
 This is caused by their operational semantics rule design, trace design, and their annotated query design.
 In the rule \rname{l-loop}, a nature number $v_N$ is required on the premise of tracking the iteration times.
 This requires that the loop iteration number has to be a nature number or evaluated to a nature number in advance to execute the loop.
 Then in their trace and annotated query design,
 % generated through the operational semantics rules.
 the trace tracks the annotated query, which requires an integer annotation explicitly indicating the iteration number of the loop.
 % annotatation of the the query request executed in the program
 % with integer indicating the while loops. 
 \\
 For example, in the following program with a simple while loop,
 \[
 {\assign{x}{20}};
 \assign{y}{100};
 \ewhile (x < y) \edo 
 \{
 \assign{x}{x + 1};
 \assign{y}{y - 2};
 \}\}
 \] 
 the number of iterations cannot be evaluated to a nature number in advance of entering this loop. 
 The iteration number is only
 able to be decided during executing this while loop body.
 This program represents a class of data analysis programs with non-constant loop iterations
 which is very common in data analysis. However, it isn't supported by their design.
 \item \textbf{Accuracy Limitation}
 \\
 This operational semantics design causes in-precision in formalizing the \emph{adaptivity}.
 Because there is information loss in their trace generated through the operational semantic rules.
 The trace only tracks the query requests. This lost the information of the variables
 which are assigned by query values even if they are not assigned by query requests. However, these variables
 are critical in analyzing the \emph{adaptivity}.
 This will be analyzed in detail in the limitation in Section~\ref{sec:prework-formalization}.
 \item \textbf{Efficiency Limitation}
 \\
 There are four components in their configuration in order to evaluate the program. 
 The update operations for this quadruple configuration are low-efficient, especially the update operation of the while map.
\end{itemize}
}   