We are interested in defining a notion of dependencies between program variables since assigned variables are a good proxy to study dependencies between queries---we can recover query requests from variables associated with queries. We consider dependencies that can be generated by either data or control flow.
% as follows,
% \begin{enumerate}
For example, in the program 
\[c_1 =[\assign{x}{\query(\chi[2])}]^1 ;[\assign{y}{\query(\chi[3] + x)}]^2\]
the query $\query(\chi[3] + x)$  depends on the query $\query(\chi[2]))$ through a \emph{value dependency} via  $x^1$.
% ), because $\chi[3] + x$ may depend on the data stored in x assigned by the result of $\query(\chi[2]))$. 
% From our perspective, $\query(\chi[1])$ is different from $\query(\chi[2]))$.
\\
% \\
% (2). One query may depend on a previous query if and only if a change of the value returned
%     to the previous query request may also change the appearance of this query quest.
%     This captures the control influence.
Conversely, in the program
\[c_2 = [\assign{x}{\query(\chi[1])}]^1 ; \eif( [x > 2]^2 , [\assign{y}{\query(\chi[2])}]^3, [\eskip]^4 )\] 
the query $\query(\chi[2])$ 
depends on the query $\query(\chi[1])$ via the \emph{control dependency} of the guard of the if command involving the labeled variable $x^1$.

To define dependency between program variables we will consider two events that are generated from the same command, hence they have the same variable name or boolean expression and label, but have either different value or different query expression, captured by the following definition. 

\begin{defn}[Events Different in the Value]
\label{def:diff}
Two events $\event_1, \event_2 \in \eventset$ differ in their value, or query value,
denoted as $\diff(\event_1, \event_2)$, if and only if:
{\small
\begin{subequations}
\begin{align}
& \pi_1(\event_1) = \pi_1(\event_2) 
  \land  
  \pi_2(\event_1) = \pi_2(\event_2) \\
& \land  
  \big(
   (\pi_3(\event_1) \neq \pi_3(\event_2)
  \land 
  \pi_{4}(\event_1) = \pi_{4}(\event_2) = \bullet )
  \lor 
  (\pi_4(\event_1) \neq \bullet
  \land 
  \pi_4(\event_2) \neq \bullet
  \land 
  \pi_{4}(\event_1) \neq_q \pi_{4}(\event_2)) 
  \big)
\end{align}
\label{eq:diff}
\end{subequations}
}
where $\qexpr_1 =_{q} \qexpr_2$ denotes the semantics equivalence between query values\footnote{The formal definition is in the supplementary material},
and $\pi_i$ projects the $i$-th element from the quadruple of an event.
\end{defn}
\jl{
$\pi_1(\event_1) = \pi_1(\event_2) 
  \land  
  \pi_2(\event_1) = \pi_2(\event_2)$ at Eq.\ref{eq:diff}(a)
requires that $\event_1$ and $\event_2$ have the same variable name and label. 
This guarantees that $\event_1$ and $\event_2$ are generated from the same labeled command.
In Eq.\ref{eq:diff}(b),
two kinds of comparisons between the third and fourth element are for the non-query assignment and query request separately.
For events generated from the non-query assignments (via checking
$\pi_{4}(\event_1) =_q \pi_{4}(\event_2) = \bullet$), we only compare their assigned values through $\pi_3(\event_1) \neq \pi_3(\event_2)$.
But for these from query requests (via checking
$\pi_{4}(\event_1) \neq \bullet \land \pi_{4}(\event_2) \neq \bullet$),
we are comparing their query expressions by $\pi_{4}(\event_1) \neq_q \pi_{4}(\event_2)$ rather than the assigned value computed from the unknown database server.
This matches the intuitive data dependency between queries, where one query is influenced by others as long as the query request is changed.
}

{Below is the \emph{event may-dependency} between events based on formally observing their differences via $\diff$.}
\begin{defn}[Event May-Dependency]
\label{def:event_dep}
An event $\event_2$ is in the \emph{event may-dependency} relation with an assignment event $\event_1 \in \eventset^{\asn}$ in a program ${c}$  with a hidden database $D$ and a witness trace $\trace \in \mathcal{T}$,
$\eventdep(\event_1, \event_2, [\event_1 ] \tracecat \trace \tracecat [\event_2], c, D)$ if and only if
\begin{subequations}
\begin{align}
&  
\exists \trace_0, \trace_1, \trace' \in \mathcal{T},\event_1' \in \eventset^{\asn}, {c}_1, {c}_2  \in \cdom  \sthat \diff(\event_1, \event_1') \land \\
& 
\quad (\exists  \event_2' \in \eventset \sthat
\left(
\begin{array}{ll}   
  & \config{{c}, \trace_0} \rightarrow^{*} 
  \config{{c}_1, \trace_1 \tracecat [\event_1]}  \rightarrow^{*} 
  \config{{c}_2,  \trace_1 \tracecat [\event_1] \tracecat \trace \tracecat [\event_2] } 
   \\ 
   \bigwedge &
   \config{{c}_1, \trace_1 \tracecat [\event_1']}  \rightarrow^{*}
    \config{{c}_2,  \trace_1 \tracecat[ \event_1'] \tracecat \trace' \tracecat [\event_2'] } 
  \\
  \bigwedge & 
  \diff(\event_2,\event_2' ) \land 
  \vcounter(\trace, \pi_2(\event_2))
  = 
  \vcounter(\trace', \pi_2(\event_2'))\\
  \end{array}
  \right)\\ 
  & 
  \quad
  \lor 
  \left(
  \begin{array}{l} 
  \exists \trace_3, \trace_3'  \in \mathcal{T}, \event_b \in \eventset^{\test} \sthat  
  \\
   \quad \config{{c}, \trace_0} \rightarrow^{*} \config{{c}_1, \trace_1 \tracecat [\event_1]}  \rightarrow^{*}
   \config{c_2,  \trace_1 \tracecat [\event_1] \tracecat
   \trace \tracecat [\event_b] \tracecat  \trace_3} 
\\ \quad \land
\config{{c}_1, \trace_1 \tracecat [\event_1']}  \rightarrow^{*} 
\config{c_2,  \trace_1 \tracecat [\event_1'] \tracecat \trace' \tracecat [(\neg \event_b)] \tracecat \trace_3'} 
\\
\quad \land \tlabel({\trace_3}) \cap \tlabel({\trace_3'})
= \emptyset
\land \vcounter(\trace', \pi_2(\event_b)) = \vcounter(\trace, \pi_2(\event_b)) 
    \land \event_2 \in \trace_3
    \land \event_2 \not\in \trace_3'
  \end{array}
  \right)
  ),
\end{align}
\label{eq:eventdep}
\end{subequations}
where $\tlabel(\trace) \subseteq \ldom$ is the set of the labels in all the events from trace $\trace$ and $\event_2 \in \trace_3$ or $\event_2 \notin \trace_3$ denotes that $\event_2$ belongs to $\trace_3$ or not.
\end{defn}
The first line in Eq.~\ref{eq:eventdep}(a) requires that $\event_1$ comes from an assignment command and then modifies its assigned value via $\diff(\event_1, \event_1')$.

\jl{Then, the following two parts in Eq~\ref{eq:eventdep}(b) and (c) capture the intuitive value dependency and control dependency respectively. 
As in the literature on non-interference, and following~\cite{Cousot19a}, we formulate these dependencies as relational properties, i.e. in terms of two different traces of execution. 
We force these two traces to differ by using the event $\event_1$ in one and $\event_1'$ in the other. 
Both parts execute the program two times w.r.t. the different values in $\event_1$ (as line:1 in Eq~\ref{eq:eventdep}(b) and line:2 in Eq~\ref{eq:eventdep}(c))
and $\event_1'$ (as line:2 in Eq~\ref{eq:eventdep}(b) and line:3 in Eq~\ref{eq:eventdep}(c)), 
but observe the difference in the newly generated traces in different ways (via $3$rd line in Eq~\ref{eq:eventdep}(b) and $4$th line in Eq~\ref{eq:eventdep}(c)). This idea is similar to the dependency definition from \cite{Cousot19a}.
}

\jl{For the value dependency we check whether the change also create a change in the value of $\event_2$ or not.
In Eq~\ref{eq:eventdep}(b) line:2, if the newly generated trace, $\trace' ++ [\event_2']$ still contains $\event_2$ as $\event_2'$, we check the difference on their value in line:3.
We additionally check that the two events we consider appear the same number of times in the two traces - this to make sure that if the events are generated by assignments in a loop, we consider the same iterations. 
If they only differ in their assigned values, i.e., $\diff(\event_2, \event_2')$ and
they are in the same loop iteration (via $\vcounter(\trace, \pi_2(\event_2)) = \vcounter(\trace', \pi_2(\event_2'))$),
then we say there is a value \emph{may-dependency} relation between $\event_1$ and $\event_2$.}

\jl{The Eq~\ref{eq:eventdep}(c) captures the control dependency through observing the disappearance $\event_2$ from newly generated traces, $\trace' \tracecat [(\neg \event_b)] \tracecat \trace_3'$ in the second execution (line:3).
$\event_2 \in \trace_3 \land \event_2 \not\in \trace_3'$ in Eq~\ref{eq:eventdep}(c) line:4 specifies this disappearance.
$\vcounter(\trace', \pi_2(\event_b)) = \vcounter(\trace, \pi_2(\event_b))$ is used to make sure the two executions are
in the same loop iteration as well.
Different from Eq~\ref{eq:eventdep}(b) line:3,
we use a testing event, $\event_b$ here because
$\vcounter(\trace, \pi_2(\event_2)) = \vcounter(\trace', \pi_2(\event_2'))$ cannot guarantee the disappearance
when there is no influence through control dependency. Checking only $\event_2$'s occurrence causes false positive.
And the presence of a test event whose value is affected by the change in $\event_1$
can guarantee that the computation goes through a control flow guard.
This is correct because the control dependency can only be passed through the guard of if or while command,
and this guard must be evaluated into two different values ($\event_b$ and $\neg \event_b$) in the two executions.
}

\jl{
  We can now extend the dependency relation to variables by considering all the assignment events generated during the program’s execution. 
}
\begin{defn}[Variable May-Dependency]
  \label{def:var_dep}
A variable ${x}_2^{l_2} \in \lvar(c)$  \emph{may-dependend} on the 
  variable ${x}_1^{l_1} \in \lvar(c)$ in a program ${c}$,
  %
  $\vardep({x}_1^{l_1}, {x}_2^{l_2}, {c})$, iff
\begin{center}
$
{
\begin{array}{l}
\exists \event_1, \event_2 \in \eventset^{\asn}, \trace \in \mathcal{T} \sthat
\pi_{1}{(\event_1)}^{\pi_{2}{(\event_1)}} = {x}_1^{l_1}
\land
\pi_{1}{(\event_2)}^{\pi_{2}{(\event_2)}} = {x}_2^{l_2}% \\ \quad 
\land 
\eventdep(\event_1, \event_2, \trace, c) 
  \end{array}
}%
$
\end{center}
  \end{defn}
\jl{From the definition, a labeled variable $x_2^{l_2}$ may depend on another labeled variable $x_1^{l_1}$ in a program $c$ under the hidden database $D$, 
as long as there exist two assignment events $\event_1$ (for $x_1^{l_1}$) and $\event_2$ for $x_2^{l_2}$
satisfy the \emph{event may-dependency} relation under a witness trace $\trace$.  
Notice that in the definition above we can also have that the two variables are the same,
this allow us to capture self-dependencies
}
