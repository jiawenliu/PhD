The new data dependency relation analysis algorithm
is performed on basis of the standard control flow graph.
\highlight{Improved from previous analysis,
this new analysis algorithm combines the reaching definition analysis into the standard data flow analysis.
It designs an improved version of data flow analysis named feasible 
data-flow, estimating a tighter bound on the data dependency relation. 
}
%  of the program, called abstract control flow graph.
In this part, I first show how to generate the standard abstract control flow graph in
Section~\ref{sec:static-cfg}
Then in Section~\ref{sec:static-feasible_flowsto}, 
I present the new feasible 
data-flow analysis
%  reachability-bound analysis algorithm adapted from the newly-designed
% technique in Chapter~\ref{sec:reachability-analysis} 
for estimating the data dependency relation.
Section~\ref{sec:static-dep_compute}
uses the reachability-bound analysis results estimating the dependency quantity
over labeled variables.
%  and
% Section~\ref{sec:static-quantity-edge} uses the same information
% estimating the dependency quantity for the pairs of dependent variables.
In Appendix~\ref{apdx:flowsto_soundness_extend},
this static algorithm is proved as the sound approximation of the variable \emph{may-dependency} relation
in the execution-based analysis in Section~\ref{sec:dynamic-datadep}.
%  before the introduction of  the edge and weight estimation.  

\subsubsection{Standard Control Flow graph}
\label{sec:static-cfg}
% Another operator \mathsf{blocks} 
The standard control flow graph is generated through following steps.
\paragraph*{Vertices Construction}
The vertices of this graph is the set of $c$'s labels , 
\[ 
  \cfgV(c) = labels(c)
\]
%
% Performing a feasible data-flow analysis through the reachable definition algorithm. 
%  By generating set of all the reachable variables at location of label $l$ in the program $c$.
% For every labelled variable $x^l$ in this set, 
% the value assigned to that variable
% in the assignment command associated to that label is reachable at the entry point of  executing the command of label $l$.
\paragraph*{Edges Construction}
The edges set is constructed in following steps.
\\
\textbf{Initial State:}
%
$\cfginit(c) \in \ldom$.
\\
The \emph{initial state} for a program $c$ is the initial label of this program.
This label corresponds to the first labeled command of this program 
when executing this program.
\\
Given a program $c$, its abstract initial state is computed as follows,
%
\[
  \begin{array}{ll}
    \cfginit(\clabel{\assign{x}{\expr}}{}^l)  & = l  \\
    \cfginit(\clabel{\assign{x}{\query(\qexpr)}}{}^l)  & = l \\
    \cfginit(\clabel{\eskip}^{l})  & = l \\
    \cfginit(\eif [b]^l \ethen c_1 \eelse c_2)  & = l \\
    \cfginit(\ewhile [b]^l \edo c)  & = l \\
    \cfginit(c_1 ; c_2)  & = \cfginit(c_1) \\
 \end{array}
 \]
%

\textbf{Final State}: $\cfgfinal(c) \in \mathcal{P}(\ldom)$
The \emph{Final State} of the program $c$, 
$\cfgfinal(c) \in \mathcal{P}(\ldom)$
is a set of labels.
Every label in $\cfgfinal(c)$ corresponds to the exist point of $c$.
\\
Given a program $c$, its final state is computed as follows,
% computes the set of Final State for the command. 
 \[
  \begin{array}{ll}
    \cfgfinal(\clabel{\assign{x}{\expr}}{}^l)  & = \{l\}  \\
     \cfgfinal(\clabel{\assign{x}{\query(\qexpr)}}{}^l)  & = \{l\}  \\
     \cfgfinal(\clabel{\eskip}^{l})  & = \{l\} \\
     \cfgfinal(\eif [b]^l \ethen c_1 \eelse c_2)  & = \cfgfinal(c_1) \cup \cfgfinal(c_2) \\
     \cfgfinal(\ewhile [b]^l \edo c)  & = \{l\} \\
     \cfgfinal(c_1 ; c_2)  & =  \cfgfinal(c_2) 
 \end{array}
 \]

The control flow graph is generated by edges between labels. 
Define $\cfgflow(c): \mathcal{P}(\ldom \times \ldom)$.
%
\[
 \begin{array}{ll}
    \cfgflow(\clabel{\assign{x}{\expr}}{}^l)  & = \emptyset  \\
    \cfgflow(\clabel{\assign{x}{\query(\qexpr)}}{}^l)   & = \emptyset  \\
    \cfgflow(\clabel{\eskip}^{l})  & = \emptyset \\
    \cfgflow(\eif [b]^l \ethen c_1 \eelse c_2) & =  \cfgflow(c_1) \cup \cfgflow(c_2)\cup \{(l, \cfginit(c_1)) , (l, \cfginit(c_2)) \} \\
    \cfgflow(\ewhile [b]^l \edo c)  & =  \cfgflow(c) \cup \{(l, \cfginit(C)) \} \cup \{(l', l)| l' \in \cfgfinal(c) \} \\
    \cfgflow(c_1 ; c_2)  & = \cfgflow(c_1) \cup  \cfgflow(c_2) \cup \{ (l, \cfginit(c_2)) | l \in \cfgfinal(c_1) \} \\
 \end{array}
 \]

The edges for the control flow graph is generated as follows.
Then, I build the edge for $c$'s abstract control flow graph as follows,
\[
  \cfgE(c) = \{(l_1, l_2) | (l_1, l_2) \in \cfgflow(c)\}
  \]

%  I have a pre-processing algorithm to go through the programs and returns the list of labels associating with a loop and whose visiting times need to be analyzed.
%


\paragraph{Control Flow Graph} 
With the vertices $\cfgV(c)$ and edges $\cfgE(c)$ ready, I construct the abstract control flow graph, formally 
% Through a program $c$'s abstract execution trace, its abstract control flow graph is computed 
defined in 
Definition~\ref{def:abs_cfg}.
% Given program $c$ with its abstract control flow $\cfgflow(c)$, the Abstract Control Flow Graph:
% \\
\begin{defn}[Control Flow Graph]
\label{def:cfg_graph}
Given a program $c$, 
with its abstract control flow $\cfgflow(c)$
its abstract control flow graph $\cfgG(c) =(\cfgV(c), \cfgE(c))$ is defined as follows,
\\
% \highlight{
% :
%
% \\
$\cfgE(c) = \{(l_1, dc, l_2) | (l_1, dc, l_2) \in \cfgflow(c)\}$,
\\
$\cfgV(c) = labels(c)$
% }
% \\
% , where the weight of every label to be computed in the next step.
\end{defn}




\subsubsection{Feasible Data-Flow Analysis}
\label{sec:static-feasible_flowsto}
I develop a variant of data flow analysis, called Feasible Data-Flow Generation, which 
considers both the control flow and data flow and
is a sound approximation of the edges in the execution based dependency graph.

% \wq{
\highlight{Improved from previous analysis,
this new analysis algorithm combines the reaching definition analysis on the control flow graph in feasible 
data-flow generation to have a more precise approximation. 
}
% }

% In this step, through 
% % the vertices and edges in 
% $c$'s abstract contrl flow graph $\cfgG(c)$,
%  $\THESYSTEM$ performs a feasible data-flow analysis 
%  using the reachable definition algorithm,
% %  and then Then I generated the set of feasible data-flow between labeled variables based on that.
% and generates the 
% %set of 
% feasible data-flow relation between labeled variables.
% \\
%  By generating set of all the reachable variables at location of label $l$ in the program $c$.
% For every labelled variable $x^l$ in this set, 
% the value assigned to that variable
% in the assignment command associated to that label is reachable at the entry point of  executing the command of label $l$.
% \\
It first defines some operations and then
% In the first step, 
% it 
performs the standard reaching definition analysis given a program $c$, 
on 
% its every label $l$
every label in $\cfgV(c)$.  
% \\
% it performs the standard reaching definition analysis given a program $c$, on its every label $l$.
% \\
% Another operator \mathsf{blocks} 
%
% Performing a feasible data-flow analysis through the reachable definition algorithm. 
%  By generating set of all the reachable variables at location of label $l$ in the program $c$.
% For every labelled variable $x^l$ in this set, 
% the value assigned to that variable
% in the assignment command associated to that label is reachable at the entry point of  executing the command of label $l$.
% \paragraph{Generate CFG}
%  \begin{def}
%   \label{def:init_label}
%   Define $\mathsf{init}$: Command -> label, which returns the initial label of the statement. 
% \[
%  \begin{array}{ll}
%     init([x := e]^{l})  & = l  \\
%      init([x := q(e)]^{l})  & = l \\
%      init([skip]^{l})  & = l \\
%      init([if [b]^l then C_1 else C_2]^{l})  & = l \\
%      init([while [b]^l do C]^{l})  & = l \\
%      init(C_1 ; C_2)  & = init(C_1) \\
%  \end{array}
%  \]
% \end{def}
%   Define $\mathsf{final}$: Command -> Powerset(label), which returns the final labels of the statement. 
%  \[
%  \begin{array}{ll}
%     final([x := e]^{l})  & = \{l\}  \\
%      final([x := q(e)]^{l})  & = \{l\}  \\
%      final([skip]^{l})  & = \{l\} \\
%      final([if [b]^l then C_1 else C_2]^{l})  & = final(C_1) \cup final(C_2) \\
%      final([while [b]^l do C]^{l})  & = \{l\} \footnote{while terminates after b evaluates to false} \\
%      final(C_1 ; C_2)  & =  final(C_2) \\
%  \end{array}
%  \]
% \paragraph*{Blocks and Defs}
%  Define block B to be either the command of the form of assignment, skip, or test of the form of $[b]^{l}$.\\
%  Define $\mathsf{blocks}$ : command -> Powerset(Block)
%  \[
%  \begin{array}{ll}
%     blocks([x := e]^{l})  & = \{[x := e]^{l}\}  \\
%      block([x := q(e)]^{l})  & = \{[x := q(e)]^{l}\}  \\
%      blocks([skip]^{l})  & = \{[skip]^{l}\} \\
%      blocks([if [b]^l then C_1 else C_2]^{l})  & = {[b]^{l}} \cup blocks(C_1) \cup blocks(C_2) \\
%      blocks([while [b]^l do C]^{l})  & = \{[b]^{l}\} \cup blocks(C) \\
%      blocks(C_1 ; C_2)  & = blocks(C_1) \cup  blocks(C_2) \\
%  \end{array}
%  \]
%  Define $\mathsf{labels}$ to get the labels of blocks.
%  \[
%    labels(C) = \{l | [B]^{l} \in blocks(C) \}
%  \]  

% The control flow graph is generated by edges between labels. Define $\mathsf{flow}$: command -> P (label $\times$ label ).

% \[
%  \begin{array}{ll}
%     flow([x := e]^{l})  & = \emptyset  \\
%      flow([x := q(e)]^{l})  & = \emptyset  \\
%      flow([skip]^{l})  & = \emptyset \\
%      flow([if [b]^l then C_1 else C_2)  & =  flow(C_1) \cup flow(C_2)\cup \{(l, init(C_1)) , (l, init(C_2)) \} \\
%      flow([while [b]^l do C)  & =  flow(C) \cup \{(l, init(C)) \} \cup \{(l', l)| l' \in final(C) \} \\
%      flow(C_1 ; C_2)  & = flow(C_1) \cup  flow(C_2) \cup \{ (l,init(C_2)) | l \in final(C_1) \} \\
%  \end{array}
%  \]
 
%  Define block B to be either the command of the form of assignment, skip, or test of the form of $[b]^{l}$.\\
%  Define $\mathsf{blocks}$ : command -> Powerset(Block)
\paragraph*{Blocks and Defs}
The block, 
is either the command of the form of assignment, skip, or a test of the form of $[b]^{l}$, 
% and $block$ of program $c$ is 
denoted by $\mathsf{blocks}(c)$
the set of all the blocks 
in program $c$, where  $\mathsf{blocks}: \cdom \to \mathcal{P}(\cdom \cup \clabel{\bexpr}^{l})$.
Then it generates the set of feasible data-flow between labeled variables with detail in Definition~\ref{def:feasible_flowsto}, 
based on $\live(l, c)$ for every label in a program $c$ and its blocks $\mathsf{blocks}$.
\\
The details are as follows.

The operator $\mathsf{blk} : \cdom \to \mathsf{blocks}$ gives all the blocks in program $c$.
\\
 \[
 \begin{array}{ll}
  \mathsf{blk}(\clabel{\assign{x}{\expr}}{}^l)  & = \{(\clabel{\assign{x}{\expr}}{}^l)\}  \\
  \mathsf{blk}(\clabel{\assign{x}{\query(\qexpr)}}{}^l)  & = \{(\clabel{\assign{x}{\query(\qexpr)}}{}^l)\}  \\
  \mathsf{blk}([\eskip]^{l})  & = \{[\eskip]^{l}\} \\
  \mathsf{blk}(\eif [b]^l \ethen c_1 \eelse c_2)  & = {[b]^{l}} \cup \mathsf{blk}(c_1) \cup \mathsf{blk}(c_2) \\
  \mathsf{blk}(\ewhile [b]^l \edo c)  & = \{[b]^{l}\} \cup \mathsf{blk}(c) \\
  \mathsf{blk}(C_1 ; C_2)  & = \mathsf{blk}(c_1) \cup  \mathsf{blk}(c_2) \\
 \end{array}
 \]
%  $label^{?}$ is label $\cup \{?\}$.\\
%  Define $\mathsf{kill}$: $blocks \to \mathcal{P}(\mathcal{VAR} \times LABEL \cup \{?\})$, which produces the set of labelled variables of assignment destroyed by the block.
\paragraph*{Kills and Gens}
The operator $\mathsf{kill}$: $\mathsf{blocks} \to \mathcal{P}(\mathcal{VAR} \times \ldom \cup \{?\})$ produces the set of labelled variables of assignment destroyed by the block.
 \\
  % Define $\mathsf{gen}$: $blocks \to \mathcal{P}(\mathcal{VAR} \times LABEL \cup \{?\})$, which generates the set of labelled variables generated by the block.
  The operator $\mathsf{gen}$: $\mathsf{blocks} \to \mathcal{P}(\mathcal{VAR} \times \ldom \cup \{?\})$ generates the set of labelled variables generated by the block.
  \\
  % Define $defs(x)(c): \mathcal{VAR} \to LABEL$, gives all the labels where assigns value to variable x in the target program $c$. 
  % The operator  $defs(c): \mathcal{VAR} \to \ldom$ gives all the labels where assigns value to variable in $c$. 
 \[
 \begin{array}{ll}
  \mathsf{kill}(\clabel{\assign{x}{\expr}}{}^l)  & = \{ (x, ?) \} \cup \{ (x, l') | l' \in defs(x) \} \\
  \mathsf{kill}(\clabel{\assign{x}{\query(\qexpr)}}{}^l)  & = \{ (x, ?) \} \cup \{ (x, l') | l' \in defs(x) \}  \\
  \mathsf{kill}([\eskip]^{l})  & = \emptyset \\
  \mathsf{kill}([ [b]^l ]^{l})  & =  \emptyset
 \end{array}
 ~~
  \begin{array}{ll}
    \mathsf{gen}(\clabel{\assign{x}{\expr}}{}^l)  & = \{ (x, l) \}  \\
    \mathsf{gen}(\clabel{\assign{x}{\query(\qexpr)}}{}^l)  & = \{ (x, l) \}  \\
    \mathsf{gen}([\eskip]^{l})  & = \emptyset \\
    \mathsf{gen}([ [b]^l ]^{l})  & =  \emptyset 
 \end{array}
 \]
 The $?$ is a placeholder for the unknown label need to be comptued.
%  Define $in(l)$, $out(l)$: LABEL$ \to \mathcal{VAR} \times LABEL \cup \{?\}$ for every block in program $c$ is computed as follows,
%  \[
%  \begin{array}{lll}
%     in(l)  & = \{ (x, ?) | x^l \in \lvar_c \land  l = \cfginit(c) \}  
%     \cup \{ out(l')|  | (l',\_, l) \in \cfgE \land  l \neq \cfginit(c)\}  \\
%      out(l)  & =  gen(B^{l}) \cup \{ in(l) \setminus kill(B^l)  \} & B^l \in blocks(c)   
%  \end{array}
%  \]
%  computing $in(l)$ and $out(l)$ for every $B^l \in blocks(c) $, and repeating these two step
% until the $in(l)$ and $out(l)$ are stabilized for every $B^l \in blocks(c) $
%  I use $\live_{in}(l,c)$ and $\live_{out}(l, c)$ denote the stabilized results for the command of label $l$ in program $c$. 
%  Define $defs(x)(c): \mathcal{VAR} \to LABEL$, gives all the labels where assigns value to variable x in the target program $c$.
% Define $defs(x)(c): \mathcal{VAR} \to \ldom$, gives all the labels where assigns value to variable x in the target program $c$.
% \\
%  Define $in(l)$, $out(l)$: $ \ldom \to \mathcal{VAR} \times LABEL \cup \{?\}$ for every block in program $c$ is computed as follows,
\paragraph*{In and Out}
The operator  $in(l)$, $out(l)$: $ \ldom \to \mathcal{LV} \cup \{?\}$ for every block in program $c$ is defined as follows,
 \[
 \begin{array}{ll}
    % in(l)  & = \{ (x, ?) | x^l \in \lvar_c \land  l = \cfginit(c) \}  
    in(l)  & = \{ x^{?} | x^l \in \lvar_c \land  l = \cfginit(c) \}  
    \cup \{ out(l')|  | (l',\_, l) \in \cfgE(c) \land  l \neq \cfginit(c)\}  \\
     out(l)  & =  \mathsf{gen}(B^{l}) \cup \{ in(l) \setminus \mathsf{kill}(B^l)  \}  
 \end{array}
 \]
computing $in(l)$ and $out(l)$ for every $B^l \in \mathsf{blk}(c) $, and repeating these two steps
until the $in(l)$ and $out(l)$ are stabilized for every $B^l \in \mathsf{blk}(c) $
%  I use $\live_{in}(l,c)$ and $\live_{out}(l, c)$ denote the stabilized results for the command of label $l$ in program $c$. 
 I use $\live(l,c)$ to represent 
% $\live_{in}(l,c)$ in the other part of the paper.
denote the stabilized result of $in(l)$ at label $l$ in program $c$. 
% The $\live_{in}(l,c)$ and $\live_{out}(l, c)$ is computed by the Standard worklist algorithm. (For simplicity, I use $\live(l,c)$ to represent $\live_{in}(l,c)$ in the other part of the paper.}
\\
% The $\live_{in}(l,c)$ and $\live_{out}(l, c)$ 
\paragraph{Reaching Definition Computation}
This step generates set of all the reachable variables at location of label $l$ in the program $c$.
The $\live(l, c)$ represent the analysis result, which is the set of 
reachable labeled variables in program $c$ at the location of label $l$.
For every labelled variable $x^l$ in this set, 
the value assigned to that variable
in the assignment command associated to that label is reachable at the entry point of  executing the command of label $l$.

The stabilized $in(l)$ and $out(l)$ for program $c$, as well as $\live(l, c)$,
is computed by the standard worklist algorithm with detail as below. 
% For simplicity, I use $\live(l,c)$ to represent $\live_{in}(l,c)$ in the other part of the paper.
\begin{enumerate}
    \item Initializing in[l]=out[l]=$\emptyset$
    \item Initializing in[entry label] = $\emptyset$
    \item Initializing a work queue, contains all the blocks in $c$
    \item while |W| != 0 \\
         pop l in W\\
          old = out[l]\\
          in(l) =  out(l') where $(l',\_, l) \in \cfgE(c)$\\
           out(l) = $\mathsf{gen}$($b^l$) $\cup$ (in(l) - $\mathsf{kill}$($b^l$) ) where $b^l$ in $\mathsf{blk}(c)$   \\
          if (old != out(l)) W= W $\cup$ \{l'| (l,l') in $(l',\_, l) \in \cfgE(c)$\}\\
          end while
\end{enumerate}
%
% computing $in(l)$ and $out(l)$ for every $B^l \in blocks(c) $, and repeating these two step
% until the $in(l)$ and $out(l)$ are stabilized for every $B^l \in blocks(c) $
%  I use $\live_{in}(l,c)$ and $\live_{out}(l, c)$ denote the stabilized results for the command of label $l$ in program $c$. 
% The $\live_{in}(l,c)$ and $\live_{out}(l, c)$ is computed by the Standard worklist algorithm. (For simplicity, I use $\live(l,c)$ to represent $\live_{in}(l,c)$ in the other part of the paper.
%%
\paragraph{Feasible Data-Flow Generation}
By using the results of Reaching definition analysis results, specifically $\live(l, c)$ for every label in a program $c$, I refine the vertices and edges in the $\cfgG$ graph 
by generating the set of feasible data-flow between labeled variables as follows,
%
%   \[
%  \begin{array}{ll}
%     dcdg([x := e]^{l})  & = \{ (y^i, x^l) | y \in VAR(e) \land (y,i) \in \live_{in}(l) \}  \\
%      dcdg([x := q(e)]^{l})  & = \{ (y^i, x^l) | y \in VAR(e) \land (y,i) \in \live_{in}(l) \}  \\
%      dcdg([skip]^{l})  & = \emptyset \\
%      dcdg([if [b]^l then C_1 else C_2)  & =  dcdg(c_1) \cup dcdg(c_2)\\ & \cup \{(x^i,y^j) | x \in VAR(b) \land (x,i) \in \live_{in}(l) \land ([y = \_]^j) \in blocks(c_1) \} \\
%      &\cup \{(x^i,y^j) | x \in VAR(b) \land (x,i) \in \live_{in}(l) \land ([y = \_]^j) \in blocks(c_2) \} \\
%      dcdg([while [b]^l do c)  & =  dcdg(c) \cup \{(x^i,y^j) | x \in VAR(b) \land (x,i) \in \live_{in}(l) \land ([y = \_]^j) \in blocks(C) \} \\
%      dcdg(c_1 ;c_2)  & = dcdg(c_1) \cup  dcdg(c_2) \\
%  \end{array}
%  \]
%
\begin{defn}[Feasible Data-Flow]
  \label{def:feasible_flowsto}
  Given a program $c$ and two labeled variables $x^i, y^j$  in this program, 
  $\flowsto(x^i, y^j, c)$ is 
    {\footnotesize
    \[
   \begin{array}{ll}
    \flowsto(x^i, y^j, \clabel{\assign{x}{\expr}}{}^l)  & \triangleq (x^i, y^j) \in \{ (y^i, x^l) | y \in \mathsf{FV}(\expr) 
    % \land (y,i) \in \live(l, \clabel{\assign{x}{\expr}}^l) \}  \\
    \land y^i \in \live(l, \clabel{\assign{x}{\expr}}^l) \}  \\
    \flowsto(x^i, y^j, \clabel{\assign{x}{\query(\qexpr)}}{}^l)  & \triangleq (x^i, y^j) \in \{ (y^i, x^l) | y \in \mathsf{FV}(\qexpr) 
    % \land (y,i) \in \live(l,\clabel{\assign{x}{\query(\qexpr)}}^l) \}  \\
    \land y^i \in \live(l,\clabel{\assign{x}{\query(\qexpr)}}^l) \}  \\
    \flowsto(x^i, y^j, [\eskip]^{l})  & = \emptyset \\
    \flowsto(x^i, y^j, \eif ([b]^l, c_1, c_2))  & \triangleq \flowsto(x^i, y^j, c_1) \lor \flowsto(x^i, y^j, c_2) \\ 
        & \lor (x^i, y^j) \in
       \{(x^i,y^j) | x \in \mathsf{FV}(b) \land 
      %  (x,i) 
      x^i \in \live(l, \eif ([b]^l, c_1, c_2)) \land  y^j \in \lvar(c_1) \\
    %   ([y = \_]^j) \in \mathsf{blk}(c_1) \} \\
       &\lor (x^i, y^j) \in \{(x^i,y^j) | x \in \mathsf{FV}(b) \land 
      %  (x,i) 
      x^i\in \live(l, \eif ([b]^l, c_1, c_2))  \land  y^j \in \lvar(c_2) \\
    %   \land ([y = \_]^j) \in \mathsf{blk}(c_2) \} \\
       \flowsto(x^i, y^j, \ewhile [b]^l \edo c_w)  & \triangleq  \flowsto(x^i, y^j, c_w)  \lor
       \\ & 
       (x^i, y^j) \in  \{(x^i,y^j) | x \in \mathsf{FV}(b) \land 
      %  (x,i) 
      x^i \in \live(l,   \ewhile [b]^l \edo c_w) \land  y^j \in \lvar(c_w) \\
    %   ([y = \_]^j) \in \mathsf{blk}(c_w) \} \\
       \flowsto(x^i, y^j, c_1 ;c_2)  & \triangleq \flowsto(x^i, y^j, c_1) \lor \flowsto(x^i, y^j, c_2) \\
   \end{array}
   \]
   }
   \end{defn}
%
This \emph{Feasible Data-Flow} relation is a sound approximation 
of the variable \emph{May-Dependency} relation over labeled variables for every program.
The soundness is proved
in Appendix~\ref{apdx:flowsto_soundness}.
%
\subsubsection{Data Dependency Relation Estimation}
\label{sec:static-dep_compute}
%
For any program $c$ and two labeled variables $x^i, y^j$  in this program, 
$x^i, y^j$  is in the estimated dependency relation if and only if they satisfy
$\flowsto(x^i, y^j, c)$.

As introduce in the third step in Section~\ref{sec:static-overview},
this set is used in Section~\ref{sec:static-adapt} to estimate
the execution-based dependency graph.
As introduced in third step of this analysis framework in Section~\ref{sec:static-overview},
this information will be used in constructing the program-based dependency graph, to estimate the execution-based dependency graph for the program.
For easy understanding,
I reveal the estimated directed edges construction below
% vertex and weight construction of this estimated graph below
and presents the formal definition later. 
% for each vertex in $\progV(c)$,
This estimated directed edges set is defined as a set of triples 
% $\progW(c) \in \mathcal{P}(\mathcal{LV} \times \mathcal{LV} \times EXPR(\constdom))$ 
$\progE(c) \in \mathcal{P}(\mathcal{LV} \times \mathcal{A}_{\lin} \times \mathcal{LV})$,
Each directed edge is defined between vertices $({x}_1^{i}, w_1)$  
and $({x}_2^{j}, w_2)$ 
where ${x}_1^{i}, {x}_2^{j} \in \lvar(c)$, and they are in the estimated dependency relation
% is the set of pairs 
% The weight for each vertex in $\progV(c)$ is computed 
% indicating a directed edge from the first vertex to the second one in each pair
as follows,
\highlight{
  \[
    \progE^0(c) \triangleq 
    \left\{ 
    ({x}_1^{i}, w, {x}_2^{j}) \in \mathcal{LV} \times 
    \mathcal{A}_{\kw{in}} \times \mathcal{LV}
    ~ \middle\vert ~
    \begin{array}{l}
      {x}_1^{i}, {x}_2^{j} \in \lvar(c)
    \land
      % \\
      \exists n \in \mathbb{N}, z_1^{r_1}, \cdots, z_n^{r_n} \in \lvar_{{c}} \sthat
      n \geq 0 \land
      \\
      \flowsto(x^i,  z_1^{r_1}, c) 
      \land \cdots \land \flowsto(z_n^{r_n}, y^j, c) 
    \end{array}
    \right\}
    \]
}
The weight $w$ for every edge in this definition is a placeholder.
It will be computed in Section~\ref{sec:static-quantity}.
We prove that this estimated directed edge set $\progE(c)$ is a sound approximation of the 
edge set in $c$'s Execution-Based Dependency Graph 
in Appendix~\ref{apdx:adapt_soundness}.

% \paragraph*{Edges Estimation}
% Then I define the estimated directed edges
% % for each vertex in $\progV(c)$,
% between vertices in $\progV(c)$,
% as a set of pairs 
% % $\progW(c) \in \mathcal{P}(\mathcal{LV} \times \mathcal{LV} \times EXPR(\constdom))$ 
% $\progE(c) \in \mathcal{P}(\in \mathcal{LV} \times \mathcal{LV})$
% % is the set of pairs 
% % The weight for each vertex in $\progV(c)$ is computed 
% indicating a directed edge from the first vertex to the second one in each pair
% as follows,
% \highlight{
%   \[
%     \progE(c) \triangleq 
%     \left\{ 
%     ({x}_1^{i}, {x}_2^{j}) \in \mathcal{LV} \times \mathcal{LV}
%     ~ \middle\vert ~
%     \begin{array}{l}
%       {x}_1^{i}, {x}_2^{j} \in \progV(c)
%     \land
%       % \\
%       \exists n \in \mathbb{N}, z_1^{r_1}, \cdots, z_n^{r_n} \in \lvar_{{c}} \sthat  
%       n \geq 0 \land
%       \\
%       \flowsto(x^i,  z_1^{r_1}, c) 
%       \land \cdots \land \flowsto(z_n^{r_n}, y^j, c) 
%     \end{array}
%     \right\}
%     \]
% }
% This estimated directed edge set $\progE(c)$ is a sound approximation of the 
% edge set in $c$'s execution-based dependency graph, which is proved 
% in Appendix~\ref{apdx:adapt_soundness}.
%  \begin{defn}[Feasible Data-Flow]
%   \label{def:feasible_flowsto}
%     {\footnotesize
%     \[
%    \begin{array}{ll}
%       dcdg(\clabel{\assign{x}{\expr}}{}^l)  & = \{ (y^i, x^l) | y \in FV(e) \land (y,i) \in \live_{in}(l, \clabel{\assign{x}{\expr}}^l) \}  \\
%        dcdg(\clabel{\assign{x}{\query(\qexpr)}}{}^l)  & = \{ (y^i, x^l) | y \in FV(e) \land (y,i) \in \live_{in}(l,\clabel{\assign{x}{\query(\qexpr)}}^l) \}  \\
%        dcdg([\eskip]^{l})  & = \emptyset \\
%        dcdg([\eif [b]^l \ethen c_1 \eelse c_2)  & =  dcdg(c_1) \cup dcdg(c_2)\\ & \cup 
%        \{(x^i,y^j) | x \in FV(b) \land (x,i) \in \live_{in}(l) \land ([y = \_]^j) \in blocks(c_1) \} \\
%        &\cup \{(x^i,y^j) | x \in FV(b) \land (x,i) \in \live_{in}(l) \land ([y = \_]^j) \in blocks(c_2) \} \\
%        dcdg([\ewhile [b]^l \edo c)  & =  dcdg(c) \cup \{(x^i,y^j) | x \in FV(b) \land (x,i) \in \live_{in}(l) \land ([y = \_]^j) \in blocks(C) \} \\
%        dcdg(c_1 ;c_2)  & = dcdg(c_1) \cup  dcdg(c_2) \\
%    \end{array}
%    \]
%    }
%    \end{defn}
%    For any two labeled variables $x^i, y^j$ in a program $c$, 
%   %  it is easy to see that there is a one-on-one correspondence between 
%   %  $\flowsto$ relation of the two variables, and the $dcdg$ analysis result on $c$.
%   I use $\flowsto()$ denote if they have a feasible data-flow relation in Definition~\ref{def:flowsto}.
%    \begin{defn}[Feasible Data-Flow ($\flowsto$)]
%    \label{def:flowsto}
%    \[
%    \forall c \in \cdom, x^i, y^j \in \lvar_c \sthat  
%    \flowsto(x^i, y^j, c) \iff (x^i, y^j) \in dcdg(c)
%    \]
%    \end{defn}
  %  This soundness is proved in Proof~\ref{pf:rd_soundness} in Appendix~\ref{apdx:rd_soundness}.
  %  For any two labeled variables in a program $c$, it is easy to see that there is a one-on-one correspondence between 
  %  $\flowsto$ relation of the two variables, and the $dcdg$ analysis result on $c$.
  %  \begin{thm}[Soundness of the Feasible Data-Flow Analysis]
  %  \label{thm:rd_soundness}
  %  \[
  %  \forall c \in \cdom, x^i, y^j \in \lvar_c \sthat  
  %  \flowsto(x^i, y^j, c) \iff (x^i, y^j) \in dcdg(c)
  %  \]
  %  \end{thm}
  %  This soundness is proved in Proof~\ref{pf:rd_soundness} in Appendix~\ref{apdx:rd_soundness}.
  \paragraph*{Example}
% Still looking at the Figure~\ref{fig:adapfun_tworound}(c), 
% and taking the edge $(l^6, a^5)$ for example.
% By $\flowsto(l^6, a^5, c)$, I can see $a$ is used directly in the query expression $\chi[k]*a$,
% in the assignment command $\clabel{\assign{l}{\query(\chi[k]*a)}}^l$,
% i.e., $a \in FV(\chi[k]*a)$.
% Also, from the Reaching definition analysis, I know $a^5 \in \live(6, two-round)$.
% Then I have $\flowsto(l^6, a^5, c)$ and construct the edge $(l^6, a^5)$.
% And same way for constructing the rest edges.
%
The edge $(l^6, a^5)$ in Figure~\ref{fig:twoRounds_example}(c) is constructed by this definition.
% and take  for example.
By $\flowsto(l^6, a^5, c)$, I can see $a$ is used directly in the query expression $\chi[k]*a$,
in the assignment command $\clabel{\assign{l}{\query(\chi[k]*a)}}^l$,
i.e., $a \in FV(\chi[k]*a)$.
Also, from the reaching definition analysis, I know $a^5 \in \live(6, \kw{twoRounds(k)})$.
Then I have $\flowsto(l^6, a^5, c)$ and construct the edge $(l^6, a^5)$.
And the same way for constructing the rest edges. Also, the edge $(x^3,j^5)$ in the same graph represents the control flow, 
caught by the $\flowsto$ relation.
% Still looking at the Figure~3(c) in main paper, 
% and taking the edge $(l^6, a^5)$ for example.
% By $\flowsto(l^6, a^5, c)$, I can see $a$ is used directly in the query expression $\chi[k]*a$,
% in the assignment command $\clabel{\assign{l}{\query(\chi[k]*a)}}^l$,
% i.e., $a \in FV(\chi[k]*a)$.
% Also, from the Reaching definition analysis, I know $a^5 \in \live(6, two-round)$.
% Then I have $\flowsto(l^6, a^5, c)$ and construct the edge $(l^6, a^5)$.
% And same way for constructing the rest edges. Also, the edge $(x^3,j^5)$ in the same graph represents the control flow, caught by our $\flowsto$ relation.
%
  \subsubsection{Improvements Analysis}
  \label{sec:static-dep-improvements}
  \highlight{
    This new data dependency estimation algorithm is more precise and efficient than previous static analysis.
% language and operational semantics design improves the expressiveness, efficiency, and the accuracy to a large extend.
% \todo{Add details}
\begin{itemize}
  %   \item \textbf{Improvements on Expressiveness}
  %   \\
  % This language is extended over the standard while language. 
  % In this sense, it supports all the general data analysis.
  \item \textbf{Improvements on Efficiency}
  \\
  New analysis architecture does not rewrite the program into SSA version.
\\
  New analysis technique does not pre-collect the variables generated during the program.
\\
  New analysis technique does unfold the while loop and perform analysis for each iteration of the loop, 
  instead, it only performs the analysis on the while body once.
  \\
In the three senses above, the efficiency is improved significantly.
  \item \textbf{Improvements on Accuracy}
  Comparing to previous dependency estimation method,
  new analysis technique uses the reachable definition algorithm.
  This algorithm improves the accuracy 
  on the approximating the data dependency relation.
  Let us see a simple example, a program $ [x = 0]^{1}; [x=2]^{2};  [y = x+1]^{3}$. 
The standard data flow analysis 
tells us that both the labeled variable $x^{1}$ and $x${2} may flow to $y^{3}$, which will result in an unnecessary edge ($x^{1}, y^{3}$). The result of reaching definition 
can help us eliminate this kind of edge by telling us, at line $3$, only variable $x^{2}$ is reachable. 
  \end{itemize}
  }