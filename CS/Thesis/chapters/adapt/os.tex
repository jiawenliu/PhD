

We use a trace based operational semantics tracking the history of programs execution.
This operational semantics is defined based on the event and trace with an efficient 
environment operator as follows.


\paragraph{Event}
We have two kinds of events: \emph{assignment events} and \emph{testing events}. 
Each event consists of a quadruple,
and we use $\eventset^{\asn}$ and $\eventset^{\test}$ to denote the set of all assignment events and testing events, respectively.
\begin{center}
$ \begin{array}{lllll}
\mbox{Event} 
& \event & ::= & 
({x}, l, v, \bullet) ~|~ ({x}, l, v, \qval)  & \mbox{Assignment Event} \\
&&& ~|~(\bexpr, l, v, \bullet)  & \mbox{Testing Event}
\\
\end{array}$
\end{center}
An assignment event tracks the execution of an assignment  or a query request and consists of the assigned variable, the label of the command that generates it, the value assigned to the variable, and the normal form of the query expression, $\qval$ if this command is a query request, otherwise a default value $\bullet$.
A testing event tracks the execution of if and while commands and consists of the guard of the command, the label of the command, the result of evaluating the guard, while the last element is $\bullet$.

The event projection operators $\pi_i$ projects the $i^{th}$ element from an event:
$ \pi_i : 
\eventset \to \mathcal{VAR}\cup \mbox{Boolean Expression}  \cup \mathbb{N} \cup \mathcal{VAL} \cup \mathcal{QVAL} $.
Its computation is as follows,
\\
$\pi_1(x, l, v, \bullet) = x $ \qquad  $\pi_1({x}, l, v, \qval) = x$ \qquad $ \pi_1(\bexpr, l, v, \bullet) = \bexpr$
\\
$\pi_2(\_, l, \_, \_ ) = l$
\\
$\pi_3(\_, \_, v, \_) = v$
\\
$\pi_4(\_, \_, \_, \qval) = \qval$ \qquad $\pi_4(\_, \_, \_, \bullet) = \bullet$

\paragraph{Trace}
Based on the event recording all the program's evaluations for expressions, 
 the trace designed in the new trace-based operational semantics
 now tracks all these events.

The trace $\trace \in \mathcal{T} $ is a list of events, 
collecting the events generated along the program execution. 
\[
\begin{array}{llll}
\mbox{Trace} & \trace
& ::= & [] ~|~ \trace :: \event
\end{array}
\]
$\mathcal{T} $ represents the set of traces. 
A trace can be regarded as the program history, 
which records all the evaluations for assignment commands and guards in $\eif$ and $\ewhile$ command.
This includes queries asked by the analyst during the execution of the program as well. 
I collect the trace with a trace-based operational semantics based on transitions 
of the form $ \config{c, \trace} \to \config{c', \trace'} $. 
It states that a configuration $\config{c, \trace}$,
which consists of a command $c$ to be evaluated and a starting trace $\trace$, 
evaluates to another configuration with the trace updated along with the evaluation of the command $c$ to the normal form of the command $\eskip$.
%
\\
Some useful operators are defined below w.r.t. the trace.
\\
The \emph{Trace Concatenation} operator $\tracecat: \mathcal{T} \to \mathcal{T} \to \mathcal{T}$ concatenate two traces
as the standard list concatenation formally
in Definition~\ref{def:trace_cat}.
The concatenation preserves the order of every element in the two traces.
%  combines two traces.
% I also have the operator $\tlabel : \mathcal{T} \to \ldom$, which gives the set of labels in every event belonging to a trace. 
% The full definitions of these above operators can be found in the appendix.
\begin{defn}[Trace Concatenation: $\tracecat: \mathcal{T} \to \mathcal{T} \to \mathcal{T}$]
  \label{def:trace_cat}
  Given two traces $\trace_1, \trace_2 \in \mathcal{T}$, the trace concatenation operator 
  $\tracecat$ is defined as:
  \[
    \trace_1 \tracecat \trace_2 \triangleq
    \left\{
    \begin{array}{ll} 
       \trace_1 & \trace_2 = [] \\
       (\trace_1  \tracecat \trace_2')  :: \event & \trace_2 = \trace_2' :: \event
    \end{array}
    \right.
  \]
  \end{defn}
  
The \emph{Belongs To} operator $\in : \eventset \to \mathcal{T} \to \{\etrue, \efalse \} $ and its opposite $\not\in$
express whether an event belongs to a trace, formally as follows in Definition~\ref{def:event_belong}.
\begin{defn}[An Event Belongs to A Trace]
  \label{def:event_belong}
    An event $\event \in \eventset$ belongs to a trace $\trace$, i.e., $\event \in \trace$ are defined as follows:
  %
  \begin{equation}
    \event \in \trace  
    \triangleq \left\{
    \begin{array}{ll} 
      \etrue                  & \trace =  \trace' :: \event'
       \land \event = \event'
                                \\
      \event \in \trace' & \trace =  \trace' :: \event'
      \land \event \neq \event' \\ 
      \efalse                 & \trace = []
    \end{array}
    \right.
  \end{equation}
  As usual, we denote by $\event \notin \trace$ that the event $\event$ doesn't belong to the trace $\trace$.
  \end{defn}
  %
  % An event $\event \in \eventset$ belongs to a trace $\trace$ up to value, 
  % i.e., $\event \sigin \trace$ are defined as follows:
  %   %
  % \begin{equation}
  %   \event \sigin \trace  
  %   \triangleq \left\{
  %   \begin{array}{ll} 
  %     \etrue                  & \trace =  (\trace' \tracecate \event')                          \land \pi_1(\event_1) = \pi_1(\event_2) 
  %                               \land  \pi_2(\event_1) = \pi_2(\event_2)  
  %                               % \land \vcounter(\trace \event) = \vcounter()
  %                               \\
  %     \event \sigin \trace'   & \trace =  (\trace' \tracecate \event') 
  %                               \land 
  %                               (\pi_1(\event_1) \neq \pi_1(\event_2) 
  %                               \lor  \pi_2(\event_1) \neq \pi_2(\event_2)) 
  %                               \\ 
  %     \efalse                 & o.w.
  %   \end{array}
  %   \right.
  % \end{equation}
  %
  % % \mg{Why the previous definition used :: and now you switch to ++? Cannot this just be defined using ::? I am trying to anticipate places where a reader might be confused. Also, this definition would be much simpler if we defined event in a more uniform way. Ideally, we want to distinguish three cases, we don't need to distinguish 7 cases.}\\
  % % \jl{My bad, I was really too sticky to the convention.
  % I though the list appending  $::$ can only append element on the left side.}
  % \mg{We introduce a counting operator $\vcounter : \mathcal{T} \to \mathbb{N} \to \mathbb{N}$ whose behavior is defined as follows, \sout{Counter $\vcounter : \mathcal{T} \to \mathbb{N} \to \mathbb{N}$ }}.
  % \wq{The operator counter actually provides the number of times a specific label appears in a trace. Only a number, the position of label is ignored.}
  % \[
  % \begin{array}{lll}
  % \vcounter((x, l, v) :: \trace ) l \triangleq \vcounter(\trace) l + 1
  % &
  % \vcounter((b, l, v):: \trace ) l \triangleq \vcounter(\trace) l + 1
  % &
  % \vcounter((x, l, \qval, v):: \trace ) l \triangleq \vcounter(\trace) l + 1
  % \\
  % \vcounter((x, l', v):: \trace ) l \triangleq \vcounter(\trace ) l
  % &
  % \vcounter((b, l', v):: \trace ) l \triangleq \vcounter(\trace ) l
  % &
  % \vcounter((x, l', \qval, v):: \trace ) l \triangleq \vcounter(\trace ) l
  % \\
  % \vcounter({[]}) l \triangleq 0
  % &&
  % \end{array}
  % \]
  % \[
  % \begin{array}{lll}
  % \vcounter(\trace  \tracecat [(x, l, v)] ) l \triangleq \vcounter(\trace) l + 1
  % &
  % \vcounter(\trace  \tracecat [(b, l, v)] ) l \triangleq \vcounter(\trace) l + 1
  % &
  % \vcounter(\trace  \tracecat [(x, l, \qval, v)] ) l \triangleq \vcounter(\trace) l + 1
  % \\
  % \vcounter(\trace  \tracecat [(x, l', v)] ) l \triangleq \vcounter(\trace ) l
  % &
  % \vcounter(\trace  \tracecat [(b, l', v)] ) l \triangleq \vcounter(\trace ) l
  % &
  % \vcounter(\trace  \tracecat [(x, l', \qval, v)]) l \triangleq \vcounter(\trace ) l
  % \\
  % \vcounter({[]}) l \triangleq 0
  % &&
  % \end{array}
  % \]
The \emph{Occurrence Counter} $\vcounter : \mathcal{T} \to \mathbb{N} \to \mathbb{N}$ counts the 
occurrence of a label in a trace,
% whose behavior is defined 
formally as follows in Definition~\ref{def:trace_cnt}.
  % \[
  % \begin{array}{lll}
  % \vcounter(\trace :: (x, l, v, \bullet) ) l \triangleq \vcounter(\trace) l + 1
  % &
  % \vcounter(\trace  ::(b, l, v, \bullet) ) l \triangleq \vcounter(\trace) l + 1
  % &
  % \vcounter(\trace  :: (x, l, v, \qval) ) l \triangleq \vcounter(\trace) l + 1
  % \\
  % \vcounter(\trace  :: (x, l', v, \bullet) ) l \triangleq \vcounter(\trace ) l, l' \neq l
  % &
  % \vcounter(\trace  :: (b, l', v, \bullet) ) l \triangleq \vcounter(\trace ) l, l' \neq l
  % &
  % \vcounter(\trace  :: (x, l', v, \qval)) l \triangleq \vcounter(\trace ) l, l' \neq l
  % \\
  % \vcounter({[]}) l \triangleq 0
  % &&
  % \end{array}
  % \]
  \begin{defn}[Occurrence Counter]
    \label{def:trace_cnt}
    \[
      \begin{array}{ll}
      \vcounter(\trace :: (\_, l, \_, \_), l ) \triangleq \vcounter(\trace, l) + 1
      &
      \vcounter(\trace  ::(b, l, v, \bullet), l) \triangleq \vcounter(\trace, l) + 1
      \\
      \vcounter(\trace  :: (x, l, v, \qval), l) \triangleq \vcounter(\trace, l) + 1
      &
      % \vcounter(\trace :: (\_, l', \_, \_), l ) \triangleq \vcounter(\trace, l), l' \neq l 
      % &
      \vcounter(\trace  :: (x, l', v, \bullet), l) \triangleq \vcounter(\trace, l), l' \neq l
      \\
      \vcounter(\trace  :: (b, l', v, \bullet), l) \triangleq \vcounter(\trace, l), l' \neq l
      &
      \vcounter(\trace  :: (x, l', v, \qval), l) \triangleq \vcounter(\trace, l), l' \neq l
      \\
      \vcounter({[]}, l) \triangleq 0
      \end{array}
      \]
\end{defn}
The \emph{Latest Label}
operator $\llabel : \mathcal{T} \to \mathcal{VAR} \to \{\mathbb{N}\}\cup \{\bot\}$,
takes a trace and a variable as input and returns the label of the latest assignment event which assigns value to that variable. 
  % We introduce an operator $\llabel : \mathcal{T} \to \mathcal{VAR} \to \ldom \cup \{\bot\}$, which 
  % takes a trace and a variable and returns the label of the latest assignment event which assigns value to that variable.
  Its behavior is defined as follows,
  \begin{defn}[Latest Label]
    \[
      % \begin{array}{lll}
    \llabel(\trace  :: (x, l, \_, \_)) x \triangleq l
    ~~~
    \llabel(\trace  :: (y, l, \_, \_)) x \triangleq \llabel(\trace ) x, y \neq x
    % &
    ~~~
    \llabel(\trace :: (b, l, v, \bullet)) x \triangleq \llabel(\trace) x
    % &
    % \\
    % \llabel(\trace  :: (y, l, v, \bullet)) x \triangleq \llabel(\trace ) x
    % &
    % \llabel(\trace :: (y, l, v, \qval)) x \triangleq \llabel(\trace ) x
    % &
    ~~~
    \llabel({[]}) x \triangleq \bot
    % \end{array}
    \]
  \end{defn}
  %
    The operator $\tlabel : \mathcal{T} \to \mathcal{P}{(\ldom)}$ gives the set of labels in every event belonging to 
    a trace, whose behavior is defined as follows,
  \[
    % \begin{array}{llll}
  \tlabel{(\trace  :: (\_, l, \_, \_)])} \triangleq \{l\} \cup \tlabel{(\trace )}
  ~~~
  \tlabel({[ ]}) \triangleq \{\}
  % \end{array}
  \]
  %
%  
\paragraph{Environment}
{
  The environment is simply an efficient operator, $\env$ where $\env(\trace) x$ fetches the latest value assigned to $x$ in the trace $\trace$
  Since the trace records all the program's evaluations for assignments, 
  the value assigned to variables can be recovered from trace directly.
  In this sense, it is more efficiently by reading value from trace directly rather than using a separate memory recording the variable values.
}
Formally, the environment operator $\env : {\mathcal{T}}  \to \mathcal{VAR} \to \mathcal{VAL} \cup \{\bot\}$, 
which maps a trace and a variable to the latest value assigned to this variable on the trace is defined as follows.
\[
\begin{array}{lll}
\env(\trace  \traceadd (x, l, v, \bullet)) x \triangleq v
&
\env(\trace \traceadd (y, l, v, \bullet)) x \triangleq \env(\trace) x, y \neq x
&
\env(\trace \traceadd (b, l, v, \bullet)) x \triangleq \env(\trace) x
\\
\env(\trace \traceadd (x, l, v, \qval)) x \triangleq v
&
\env(\trace \traceadd (y, l, v, \qval)) x \triangleq \env(\trace) x, y \neq x
&
\env({[]} ) x \triangleq \bot
\end{array}
\]

 \paragraph*{Operational Semantics}
 The trace based operational semantics is described in terms of a small step evaluation relation
 $\config{c, \trace} \to \config{c', \trace}'$  describing how a configuration program-trace evaluates to another
 configuration program-state. 
 The rules for the operational semantics are described in Fig.~\ref{fig:os}.
 The rules for assignment and query generate assignment events, while the rules for while and if generate testing events. 
 The rules for the standard while language constructs correspond to the usual rules extended to deal with traces. 

The rule $\textbf{assn}$ evaluates a standard assignment $\assign{x}{\expr}$, the expression $\expr$ is first evaluated by our expression evaluation $\config{\trace, \expr} \earrow v $, presented below.
{Different from standard rule, the result $v$ of evaluating $\expr$ is used to construct a new event $\event = (x, l, v,\bullet)$
and attach it to the previous trace. }
We have relations $\config{\trace, \expr} \earrow v $  and $\config{\trace, \bexpr} \barrow v $  to evaluate expressions and boolean expressions, respectively in Figure~\ref{fig:expr_eval}.
% \begin{mathpar} 

The rules for if and while both have two versions, 
when the boolean expressions in the guards are evaluated to true and false, respectively. 
In these rules, the evaluation of the guard generates a testing event and the trace is updated as well by appending this event.

The only rule that is non-standard is the $\textbf{query}$ rule. When evaluating a query, the query expression $\qexpr$ is first simplified to its normal form $\alpha$ using an evaluation relation 
$\config{\trace, \qexpr} \qarrow \qval$. 
Then normal form $\qval$ characterize the linear query that is run against the database. 
The query result $v$ is the expected value of the function $\lambda \chi.\qval$ applied to each row of the dataset. 
We summarize this process with the notation $\query(\qval) = v$ which we use in the rule $\textbf{query}$. 
Once the answer of the query is computed, the rules record all the needed information in the trace. 
As usual, we will use $\to^*$ for the reflexive and transitive closure of $\to$. 
 
The query expression evaluation relation  $\config{\trace, \qexpr} \qarrow \qval$ is defined by the following rules which reduce a query expression to its normal form.
{\small
\begin{mathpar}
\inferrule{ 
  \config{\trace, \aexpr} \aarrow n
}{
 \config{\trace,  \aexpr} 
 \qarrow n
}
\and
\inferrule{ 
  \config{\trace, \qexpr_1} \qarrow \qval_1
  \and
  \config{\trace, \qexpr_2} \qarrow \qval_2
}{
 \config{\trace,  \qexpr_1 \oplus_a \qexpr_2} 
 \qarrow \qval_1 \oplus_a \qval_2
}
\and
\inferrule{ 
  \config{\trace, \aexpr} \aarrow n
}{
 \config{\trace, \chi[\aexpr]} \qarrow \chi[n]
}
\and
\inferrule{ 
  \empty
}{
 \config{\trace,  \qval} 
 \qarrow \qval
}
 \end{mathpar}
 }

 \begin{figure}
  \begin{mathpar}
  \boxed{ \config{\trace,\aexpr} \aarrow v \, : \, \mbox{Trace  $\times$ Arithmetic Expr $\Rightarrow$ Arithmetic Value} }
  \\
  % \\
  \inferrule{ 
    \empty
  }{
   \config{\trace,  n} 
   \aarrow n
  }
  \and
  \inferrule{ 
    \env(\trace) x = v
  }{
   \config{\trace,  x} 
   \aarrow v
  }
  \and
  \inferrule{ 
    \config{\trace, \aexpr_1} \aarrow v_1
    \and 
    \config{\trace, \aexpr_2} \aarrow v_2
    \and 
     v_1 \oplus_a v_2 = v
  }{
   \config{\trace,  \aexpr_1 \oplus_a \aexpr_2} 
   \aarrow v
  }
  \and
  \inferrule{ 
    \config{\trace, \aexpr} \aarrow v'
    \and 
    \elog v' = v
  }{
   \config{\trace,  \elog \aexpr} 
   \aarrow v
  }
  \and
  \inferrule{ 
    \config{\trace, \aexpr} \aarrow v'
    \and 
    \esign v' = v
  }{
   \config{\trace,  \esign \aexpr} 
   \aarrow v
  }
  \\
  \boxed{ \config{\trace, \bexpr} \barrow v \, : \, \mbox{Trace $\times$ Boolean Expr $\Rightarrow$ Boolean Value} }
  \\
  % \\
  \inferrule{ 
    \empty
  }{
   \config{\trace,  \efalse} 
   \barrow \efalse
  }
  \and 
  \inferrule{ 
    \empty
  }{
   \config{\trace,  \etrue} 
   \barrow \etrue
  }
  \and 
  \inferrule{ 
    \config{\trace, \bexpr} \barrow v'
    \and 
    \neg v' = v
  }{
   \config{\trace,  \neg \bexpr} 
   \barrow v
  }
  \and 
  \inferrule{ 
    \config{\trace, \bexpr_1} \barrow v_1
    \and 
    \config{\trace, \bexpr_2} \barrow v_2
    \and 
     v_1 \oplus_b v_2 = v
  }{
   \config{\trace,  \bexpr_1 \oplus_b \bexpr_2} 
   \barrow v
  }
  \and 
  \inferrule{ 
    \config{\trace, \aexpr_1} \aarrow v_1
    \and 
    \config{\trace, \aexpr_2} \aarrow v_2
    \and 
     v_1 \sim v_2 = v
  }{
   \config{\trace,  \aexpr_1 \sim \aexpr_2} 
   \barrow v
  }
  \\
  \boxed{ \config{\trace, \expr} \earrow v \, : \, \mbox{Trace $\times$ Expression $\Rightarrow$ Value} }
  \\
  \inferrule{ 
    \config{\trace, \aexpr} \aarrow v
  }{
   \config{\trace,  \aexpr} 
   \earrow v
  }
  \and
  \inferrule{ 
    \config{\trace, \bexpr} \barrow v
  }{
   \config{\trace,  \bexpr} 
   \earrow v
  }
  \and
  \inferrule{ 
    \config{\trace, \expr_1} \earrow v_1
    \cdots
    \config{\trace, \expr_n} \earrow v_n
  }{
   \config{\trace,  [\expr_1, \cdots, \expr_n]} 
   \earrow [v_1, \cdots, v_n]
  }
  \and
  \inferrule{ 
    \empty
  }{
   \config{\trace,  v} 
   \earrow v
  }
   \end{mathpar}
   \caption{Arithmetic and Boolean Evaluation Rules}
   \label{fig:expr_eval}
  \end{figure}


%
%
\begin{figure}
  {
  \begin{mathpar}
  \boxed{
  \mbox{Command $\times$ Trace}
  \xrightarrow{}
  \mbox{Command $\times$ Trace}
  }
  \and
  \boxed{\config{{c, \trace}}
  \xrightarrow{} 
  \config{{c',  \trace'}}
  }
  \\
  \inferrule
  {
  \empty
  }
  {
  \config{\clabel{\eskip}^l,  \trace } 
  \xrightarrow{} 
  \config{\clabel{\eskip}^l, \trace}
  }
  ~\textbf{skip}
  %
  \and
  %
  \inferrule
  {
  \event = ({x}, l, v, \bullet)
  }
  {
  \config{[\assign{{x}}{\aexpr}]^{l},  \trace } 
  \xrightarrow{} 
  \config{\clabel{\eskip}^l, \trace \traceadd \event}
  }
  ~\textbf{assn}
  %
  \and
  %
  {
  \inferrule
  {
   \trace, \qexpr \qarrow \qval
   \and 
  \query(\qval) = v
  \and 
  \event = ({x}, l, v, \qval)
  }
  {
  \config{{[\assign{x}{\query(\qexpr)}]^l, \trace}}
  \xrightarrow{} 
  \config{{\clabel{\eskip}^l,  \trace \traceadd \event} }
  }
  ~\textbf{query}
  }
  %
  \and
  %
  \inferrule
  {
   \trace, b \barrow \etrue
   \and 
   \event = (b, l, \etrue, \bullet)
  }
  {
  \config{{\ewhile [b]^{l} \edo c, \trace}}
  \xrightarrow{} 
  \config{{
  c; \ewhile [b]^{l} \edo c),
  \trace \traceadd \event}}
  }
  ~\textbf{while-t}
  %
  %
  \and
  %
  \inferrule
  {
   \trace, b \barrow \efalse
   \and 
   \event = (b, l, \efalse, \bullet)
  }
  {
  \config{{\ewhile [b]^{l}, \edo c, \trace}}
  \xrightarrow{} 
  \config{{
    \clabel{\eskip}^l,
  \trace \traceadd \event}}
  }
  ~\textbf{while-f}
  %
  %
  \and
  %
  %
  \inferrule
  {
  \config{{c_1, \trace}}
  \xrightarrow{}
  \config{{c_1',  \trace'}}
  }
  {
  \config{{c_1; c_2, \trace}} 
  \xrightarrow{} 
  \config{{c_1'; c_2, \trace'}}
  }
  ~\textbf{seq1}
  %
  \and
  %
  \inferrule
  {
    \config{{c_2, \trace}}
    \xrightarrow{}
    \config{{c_2',  \trace'}}
  }
  {
  \config{{\clabel{\eskip}^l; c_2, \trace}} \xrightarrow{} \config{{ c_2', \trace'}}
  }
  ~\textbf{seq2}
  %
  \and
  %
  %
  \inferrule
  {
     \trace, b \barrow \etrue
   \and 
   \event = (b, l, \etrue, \bullet)
  }
  {
   \config{{
  \eif([b]^{l}, c_1, c_2), 
  \trace}}
  \xrightarrow{} 
  \config{{c_1, \trace \traceadd \event}}
  }
  ~\textbf{if-t}
  %
  \and
  %
  \inferrule
  {
   \trace, b \barrow \efalse
   \and 
   \event = (b, l, \efalse, \bullet)
  }
  {
  \config{{\eif([b]^{l}, c_1, c_2), \trace}}
  \xrightarrow{} 
  \config{{c_2, \trace \traceadd \event}}
  }
  ~\textbf{if-f}
  % %
  %
  %
  \end{mathpar}
  }
      \caption{Trace-based Operational Semantics for Language.}
      \label{fig:os}
  \end{figure}

\paragraph*{Properties and Lemmas}

In order to distinguish if a query's choice is affected by previous values, 
we need to be able to identify whether two queries are the equivalent or not, so that when we change the result of one query, 
whether another query is affected. 
To define equivalence of queries,
quiet different from the equality between the evaluation results as the regular assignment results, 
we are neither observing the syntactic equivalence between the two query expressions,
nor two results return from the database. 
Instead, we define the equivalence of query expression by quantifying over all values returned from database on a certain form of query value, formally as follows.

\begin{defn}[Equivalence of Query Expression]
%
\label{def:query_equal}
Two query expressions $\qexpr_1$, $\qexpr_2$ are equivalent, denoted as $\qexpr_1 =_{q} \qexpr_2$, if and only if
$$
 \begin{array}{l} 
   \forall \trace \in \mathcal{T} \sthat \exists \qval_1, \qval_2 \in \mathcal{QVAL} \sthat
    (\config{\trace,  \qexpr_1} \qarrow \qval_1 \land \config{\trace,  \qexpr_2 } \qarrow \qval_2) 
    \\
    \quad \land (\forall D \in \dbdom, r \in D \sthat 
    \exists v \in \mathcal{VAL} \sthat 
          \config{\trace, \qval_1[r/\chi]} \aarrow v \land \config{\trace,  \qval_2[r/\chi] } \aarrow v)  
  \end{array}.
$$
 where $r \in D$ is a record in the database domain $D$. 
 As usual, we will denote by $\qexpr_1 \neq_{q} \qexpr_2$  the negation of the equivalence.
\end{defn}
%
\begin{defn}[Event Equivalence]
  Two events $\event_1, \event_2 \in \eventset$ are equivalent, 
  % denoted as $\event_1 \eventeq \event_2$ 
  denoted as $\event_1 = \event_2$ 
  if and only if:
  \[
  \pi_1(\event_1) = \pi_1(\event_2) 
  \land  
  \pi_2(\event_1) = \pi_2(\event_2) 
  \land
  \pi_{3}(\event_1) = \pi_{3}(\event_2)
  \land 
  \pi_{4}(\event_1) =_q \pi_{4}(\event_2)
  \]
  %
  As usual, we will denote by $\event_1 \neq \event_2$  the negation of the equivalence.
\end{defn}

  
  If we observe the operational semantics rules, we can find that no rule will shrink the trace. 
So we have the Lemma~\ref{lem:tracenondec} with proof in Appendix~\ref{apdx:lemma_sec123}, 
specifically the trace has the property that its length never decreases during the program execution.
\begin{lem}
[Trace Non-Decreasing]
\label{lem:tracenondec}
For every program $c \in \cdom$ and traces $\trace, \trace' \in \mathcal{T}$, if 
$\config{c, \trace} \rightarrow^{*} \config{\eskip, \trace'}$,
then there exists a trace $\trace'' \in \mathcal{T}$ with $\trace \tracecat \trace'' = \trace'$
%
$$
\forall \trace, \trace' \in \mathcal{T}, c \sthat 
\config{c, \trace} \rightarrow^{*} \config{\eskip, \trace'} 
\implies \exists \trace'' \in \mathcal{T} \sthat \trace \tracecat \trace'' = \trace'
$$
\end{lem}
Since the equivalence over two events is defined over the query value equivalence, 
when there is an event belonging to a trace, 
if this event is a query assignment event, 
it is possible that 
the event showing up in this trace has a different form of query value, 
but they are equivalent by Definition~\ref{def:query_equal}.
So we have the following Corollary~\ref{coro:aqintrace} with proof in Appendix~\ref{apdx:lemma_sec123}.
\begin{coro}
\label{coro:aqintrace}
For every event and a trace $\trace \in \mathcal{T}$,
if $\event \in \trace$, 
then there exist another event $\event' \in \eventset$ and traces $\trace_1, \trace_2 \in \mathcal{T}$
such that $\trace_1 \tracecat [\event'] \tracecat \trace_2 = \trace $
with 
$\event$ and $\event'$ equivalent but may differ in their query value.
\[
  \forall \event \in \eventset, \trace \in \mathcal{T} \sthat 
\event \in \trace \implies \exists \trace_1, \trace_2 \in \mathcal{T}, 
\event' \in \eventset \sthat (\event \in \event') \land \trace_1 \tracecat [\event'] \tracecat \trace_2 = \trace  
\]
\end{coro}