\paragraph{language}

\highlight{
 This {\tt Query While} language enriches previous language design with standard while loop and user inputs.
 It is more expressive than the language designed in previous works.
The previous language only supports the data analysis with the constant number of loop iterations.
Compared to it, in the new language design,
the general data analysis program with non-deterministic loop iterations
(for example the program below as shown in Section~\ref{sec:prework-language})
is supported.
\[
 {\assign{x}{20}};
 \assign{y}{40};
 \ewhile (x < y) \edo 
 \{
 \assign{x}{x + 1};
 \assign{y}{y - 2};
 \}\}
 \] 
Previous work does not support a data analysis program with user inputs, which is supported in the new language as well.
}

\paragraph*{Language}
\highlight{\paragraph{Improvements through Examples}
It is expressive in two following aspects.
\begin{itemize}
  \item \textbf{Improvements from Standard While Language}
  \\
  It also extends the standard while language with query requests. 
  The general data analysis program with query requests on data  are supported in this {\tt Query While} language.
  The program can access the database through a special  interface $\chi$ encapsulated by the identifier $\query$ (for example the program below) in the new language.
  \[
    {\assign{x}{20}};
    \assign{y}{\query(\chi[2])};
    \ewhile (x < 100) \edo 
    \{
      \assign{x}{x + 1};
      \assign{y}{\query(\chi[x]*\chi[n])};
      \}\}
    \] 
%
    \item \textbf{Improvements from Previous Works}
  \\
This {\tt Query While} language is also more expressive than the language designed in previous works.
The previous language only supports the data analysis with constant number of loop iterations.
Comparing to it, in the new language design,
the general data analysis program with non-deterministic loop iterations
(for example the program below as shown in Section~\ref{sec:prework-language})
is supported.
\[
  {\assign{x}{20}};
  \assign{y}{40};
  \ewhile (x < y) \edo 
  \{
    \assign{x}{x + 1};
    \assign{y}{y - 2};
    \}\}
  \] 
Previous work does not support data analysis program with user inputs, which is supported in the new language as well.
\end{itemize}
}


\paragraph*{Semantics}
\highlight{\paragraph*{Improvements}
Comparing to previous works,
this language and operational semantics design improves the expressiveness, efficiency, and the accuracy to a large extend.
\begin{itemize}
    \item \textbf{Improvements on Expressiveness}
    \\
  Based on the expressiveness improvements from the syntax of the {\tt Query While} language,
  the new operational semantics design is also improved with the same features.
  The general data analysis with nondeterministic iterations and user inputs is fully supported.
  \item \textbf{Improvements on Accuracy}
  \\
  In previous operational semantics design, trace tracking only the annotated query.
  Key information is lost for analyzing the adaptivity, which causes the in-accuracy in formalizing the adaptivity.
  Comparing to it, the new trace-based operational semantics tracks all the labeled variables with its
  assigned value in the trace.
  In this sense, all the information is preserved for analyzing and formalizing the adaptivity.
  \item \textbf{Improvements on Efficiency}
  \\
  The new operational semantics rules only update the configuration by appending event with $O(1)$ complexity.
  Comparing to the exponential complexity in previous operational semantics rules, this improved the efficiency
  significantly.
  % Built on the new trace design and environment design, the configuration in the
  % new operational semantics contains only the trace and program.
\end{itemize}
  }