\paragraph{Challenge}
\label{para:exe-dep-challenge}
In the data analysis model the programming framework supports, 
%  an \emph{analyst} asks a sequence of queries to the mechanism, and receives the answers to these queries from the mechanism. In this model, the adaptivity I are interested in is the length of the longest sequence of such adaptively chosen queries, among all the queries the data analyst asks. 
  I define that a query is adaptively chosen when it is affected by answers of previous queries. The next thing is to decide how do I define whether one query is "affected" by previous answers, with the limited information I have? As a reminder, 
 when the analyst asks a query, the only known information will be the answers to previous queries and the current execution trace of the program.


There are two possible situations that a query will be "affected",  
either when the query expression directly uses the results of previous queries (data dependency), or when the control flow of the program with respect to a query (whether to ask this query or not) depends on the results of previous queries (control flow dependency).
% As a first step, I give a definition of when one query may depend on a previous query, which is supposed to consider both control dependency and data dependency. We first look at two possible candidates:
% \begin{enumerate}
%     \item One query may depend on a previous query if and only if a change of the answer to the previous query may also change the result of the query.
%     \item One query may depend on a previous query if and only if a change of the answer to the previous query may also change the appearance of the query.
% \end{enumerate}


Since the results of previous queries can be stored or used in variables
which aren't associated to the query request,
it is necessary to track the dependency between queries, through all the program's variables.
In the meantime, the trace-based operational semantics tracks the information,
distinguishing variables which are assigned with query requests.
% and then I can distinguish variables which are assigned with query requests.
In this way, the \emph{may-dependency} relation between queries can be recovered through 
the \emph{may-dependency} relation over all the program's variables.
According to this, I focus on the analysis on the variable \emph{may-dependency} relation in the following of this section.

I give a definition of when one variable \emph{may-depend} on a previous variable with two candidates.
{
\begin{enumerate}
    \item One variable may depend on a previous variable if and only if a change of the value assigned to the previous variable may also change the value assigned to the variable.
    \item One variable may depend on a previous variable if and only if a change of the value assigned to the previous variable may also change the appearance of the assignment command to this variable 
    % in\wq{during?} 
    during execution.
\end{enumerate}
}
%   The first candidate works well by witnessing the result of one query according to the change of the answer of another query. We can easily find that the two queries have nothing to do with each other in a simple example   

{   
% The first situations works well by witnessing the result assigned to variable 
% according to the change of the value assigned to another query. 
% We can easily find that the two queries have nothing to do with each other in a simple example 
% In the first one, by defining the dependency as
The first definition is defined as
% witnessing 
% the query expressions equivalence (or the value equality for non-query assignment )
the witness of a variation on the value assigned to the same variable through two executions,
% assigned to the same variable through two executions, 
according to the change of the value assigned to another variable in pre-trace.
% the situation of data-dependency works well. \wq{long sentence, make it short?}
In particular for query requests, the variation I observe is on the query value instead of on the query requesting results.
% We can find that two queries 
% % have nothing to do with each other in this simple example 
% % depends on each other\wq{not each other, one direction.} 
% satisfy this definition
In 
%this 
the simple program $c_1 =\assign{x}{\query(\chi[2])} ;\assign{y}{\query(\chi[3] + x)}$.
 %
 From our perspective, $\query(\chi[1])$ is different from $\query(\chi[2]))$. Informally, I think $\query(\chi[3] + x)$ may depend on the query $\query(\chi[2]))$, because equipped function of the former $\chi[3] + x$ may depend on the data stored in x assigned with the result of $\query(\chi[2]))$, according to this definition. }
%
% in this example: $c_1 = \assign{x}{\query(0)}; \assign{z}{\query(\chi[x])}$.
% This candidate definition works well 
Nevertheless, the first definition fails to catch control dependency because it just monitors the changes to a query, but misses the appearance of the query when the answers of its previous queries change. 
For instance, it fails to handle $
      c_2 = \assign{x}{\query(\chi[1])} ; \eif( x > 2 , \assign{y}{\query(\chi[2])}, \eskip )
   $, but the second definition can. However, it only considers the control dependency and misses the data dependency. 
  This reminds me to define a \emph{may-dependency} relation between labeled variables by combining the two definitions to capture the two situations.
%
%
% I define the According to the trace-based operational semantics,
%
\paragraph*{Value Difference on Traces}
%
\highlight{
  To define the \emph{may-dependency} relation on two labeled variables, I rely on the limited information at hand - the trace generated by the operational semantics. In this end, 
% So I first define some operations on the trace.
% In order to define the \emph{may-dependency} relation, 
Specifically, we need to be able to observe the difference of the values for each variable
through traces, i.e., through the events contained in traces.
Followed by this, it is necessary to give formal definition on observing this difference through traces.
Given the limitation of previous definition based on limited language,
I present the new definitions on observing the value difference as follows.
These new definitions help in formally capturing the intuitive \emph{adaptivity} precisely, concisely, and efficiently.}
%  in order to observe the evaluation variation in a different way, formally as follows.
\highlight{
\begin{defn}[Value Sequence $\seq(\trace, x^l)$]
  \label{def:vseq}
  \[
\begin{array}{l}
  \seq(\trace :: (x, l, v, \bullet), x^l) \triangleq \seq(\trace)::v  \qquad
  \seq(\trace :: (x, l, v, \qval), x^l) \triangleq \seq(\trace):: \qval \qquad
  \seq([]) \triangleq []\\
  \seq(\trace :: (y, j, \_, \_), x^l) \triangleq \seq(\trace) \quad y \neq x \lor j \neq l 
\end{array}
\]
\end{defn}
%
\begin{defn}[Difference Sequence $\sdiff(\trace_1, \trace_2, x^l )$]
  \label{def:diffseq}
  Let $ s_1 = \seq(\trace_1, x^l) \land s_2 = \seq(\trace_2, x^l)$ be the value sequence of $x^l$ 
  on $\trace_1$ and $\trace_2$, and $s^l$ be the sequence with longer length and $s^t$ the 
  shorter one,
  then their difference sequence is defined as follows,
  \[
    \sdiff(\trace_1, \trace_2, x^l) \triangleq
    \begin{array}{l}
      \{ (s^t[k], s^l[k]) ~|~ 
      % \land 
      % \land 
      s^t[k] \neq s^l[k], k = 0, \ldots, len(s^t)
      \}
      \\
      \cup 
      \{ (\cdot, s^l[k]) ~|~ 
      \len(s^t) \leq \len(s^l)k = len(s^t), \ldots \len(s^l)
      \}
    \end{array}
    \]
\end{defn}
}
% A program $c$'s query variables is a subset of 
% its labeled variables, $\qvar(c) \subseteq \lvar(c)$. We have the operator $\tlabel : \mathcal{T} \to \ldom$, which gives the set of labels in every event belonging to the trace.
% Then I introduce a counting operator $\vcounter : \mathcal{T} \to \mathbb{N} \to \mathbb{N}$, 
% % \wq{which counts the occurrence of of a variable in the trace,} 
% which counts the occurrence of a labeled variable in the trace,
% whose behavior is defined as follows,
% \[
% \begin{array}{ll}
% \vcounter(\trace :: (\_, l, \_, \_), l ) \triangleq \vcounter(\trace, l) + 1
% &
% \vcounter(\trace  ::(b, l, v, \bullet), l) \triangleq \vcounter(\trace, l) + 1
% \\
% \vcounter(\trace  :: (x, l, v, \qval), l) \triangleq \vcounter(\trace, l) + 1
% &
% % \vcounter(\trace :: (\_, l', \_, \_), l ) \triangleq \vcounter(\trace, l), l' \neq l 
% % &
% \vcounter(\trace  :: (x, l', v, \bullet), l) \triangleq \vcounter(\trace, l), l' \neq l
% \\
% \vcounter(\trace  :: (b, l', v, \bullet), l) \triangleq \vcounter(\trace, l), l' \neq l
% &
% \vcounter(\trace  :: (x, l', v, \qval), l) \triangleq \vcounter(\trace, l), l' \neq l
% \\
% \vcounter({[]}, l) \triangleq 0
% \end{array}
% \]
  \paragraph{Event Dependency}
\todo{
 To define the \emph{may-dependency} relation on two labeled variables, I rely on the limited information at hand - the trace generated by the operational semantics. In this end, 
 I first define the \emph{may-dependency} between events, and use it as a foundation of the variable may-dependency relation.
\begin{defn}[Events Different up to Value ($\diff$)]
  Two events $\event_1, \event_2 \in \eventset$ are  \emph{Different up to Value}, 
  denoted as $\diff(\event_1, \event_2)$ if and only if:
  \[
    \begin{array}{l}
  \pi_1(\event_1) = \pi_1(\event_2) 
  \land  
  \pi_2(\event_1) = \pi_2(\event_2) \\
  \land  
  \big(
    (\pi_3(\event_1) \neq \pi_3(\event_2)
  \land 
  \pi_{4}(\event_1) = \pi_{4}(\event_2) = \bullet )
  % \qquad \qquad 
  \lor 
  (\pi_4(\event_1) \neq \bullet
  \land 
  \pi_4(\event_2) \neq \bullet
  \land 
  \pi_{4}(\event_1) \neq_q \pi_{4}(\event_2)) 
  \big)
  \end{array}
  \]
  \end{defn}
 %
 We compare two events by defining $\diff(\event_1, \event_2)$. We use $\qexpr_1 =_{q} \qexpr_2$ and $\qexpr_1 \neq_{q} \qexpr_2$ to notate query expression equivalence and in-equivalence, distinct from standard equality. A program $c$'s
%  , its 
 labeled variables 
%  and assigned variables are subsets of 
is a subset of
the labeled variables $\mathcal{LV}$, denoted by $\lvar(c) \in \mathcal{P}(\mathcal{VAR} \times \mathcal{L}) \subseteq \mathcal{LV}$.
% annotated by a label. 
% We use  
%$\mathcal{LVAR} = \mathcal{VAR} \times \mathcal{L} $ 
% $\mathcal{LV}$ represents the universe of all the labeled variables and 
% $\avar(c) \in \mathcal{P}(\mathcal{VAR} \times \mathbb{N}) \subset \mathcal{LV}$ and 
% $\lvar(c) \in \mathcal{P}(\mathcal{VAR} \times \mathcal{L}) \subseteq \mathcal{LV}$ for them. 
We also define the set of query variables for a program $c$, $\qvar: \cdom \to 
\mathcal{P}(\mathcal{LV})$.
\\
A program $c$'s query variables is a subset of 
its labeled variables, $\qvar(c) \subseteq \lvar(c)$. We have the operator $\tlabel : \mathcal{T} \to \ldom$, which gives the set of labels in every event belonging to the trace.
Then I introduce a counting operator $\vcounter : \mathcal{T} \to \mathbb{N} \to \mathbb{N}$, 
% \wq{which counts the occurrence of of a variable in the trace,} 
which counts the occurrence of a labeled variable in the trace,
whose behavior is defined as follows,
\[
\begin{array}{ll}
\vcounter(\trace :: (\_, l, \_, \_), l ) \triangleq \vcounter(\trace, l) + 1
&
\vcounter(\trace  ::(b, l, v, \bullet), l) \triangleq \vcounter(\trace, l) + 1
\\
\vcounter(\trace  :: (x, l, v, \qval), l) \triangleq \vcounter(\trace, l) + 1
&
% \vcounter(\trace :: (\_, l', \_, \_), l ) \triangleq \vcounter(\trace, l), l' \neq l 
% &
\vcounter(\trace  :: (x, l', v, \bullet), l) \triangleq \vcounter(\trace, l), l' \neq l
\\
\vcounter(\trace  :: (b, l', v, \bullet), l) \triangleq \vcounter(\trace, l), l' \neq l
&
\vcounter(\trace  :: (x, l', v, \qval), l) \triangleq \vcounter(\trace, l), l' \neq l
\\
\vcounter({[]}, l) \triangleq 0
\end{array}
\]
% The full definitions of these above operators can be found in the appendix.
\begin{defn}[Event May-Dependency].
\label{def:event_dep}
\\ 
  An event $\event_2$ is in the \emph{event may-dependency} relation with an assignment
  event $\event_1 \in \eventset^{\asn}$ in a program ${c}$
  with a hidden database $D$ and a trace $\trace \in \mathcal{T}$ denoted as 
  %
  $\eventdep(\event_1, \event_2, [\event_1 ] \tracecat \trace \tracecat [\event_2], c, D)$, iff
  %
  \[
    \begin{array}{l}
  \exists \vtrace_0,
  \vtrace_1, \vtrace' \in \mathcal{T},\event_1' \in \eventset^{\asn}, {c}_1, {c}_2  \in \cdom  \sthat
  \diff(\event_1, \event_1') \land 
      \\ \quad
      (
        \exists  \event_2' \in \eventset \sthat 
    \left(
    \begin{array}{ll}   
   & \config{{c}, \vtrace_0} \rightarrow^{*} 
  \config{{c}_1, \vtrace_1 \tracecat [\event_1]}  \rightarrow^{*} 
    \config{{c}_2,  \vtrace_1 \tracecat [\event_1] \tracecat \vtrace \tracecat [\event_2] } 
    % 
   \\ 
   \bigwedge &
    \config{{c}_1, \vtrace_1 \tracecat [\event_1']}  \rightarrow^{*} 
    \config{{c}_2,  \vtrace_1 \tracecat[ \event_1'] \tracecat \vtrace' \tracecat [\event_2'] } 
  \\
  \bigwedge & 
  \diff(\event_2,\event_2' ) \land 
  \vcounter(\vtrace, \pi_2(\event_2))
  = 
  \vcounter(\vtrace', \pi_2(\event_2'))\\
  \end{array}
  \right)
  \\ \quad
  \lor 
  \exists \vtrace_3, \vtrace_3'  \in \mathcal{T}, \event_b \in \eventset^{\test} \sthat 
  \\ \quad
  \left(
  \begin{array}{ll}   
    & \config{{c}, \vtrace_0} \rightarrow^{*} 
      \config{{c}_1, \vtrace_1 \tracecat [\event_1]}  \rightarrow^{*} 
      \config{c_2,  \vtrace_1 \tracecat [\event_1] \tracecat \trace \tracecat [\event_b] \tracecat  \trace_3} 
    \\ 
    \bigwedge &
    \config{{c}_1, \vtrace_1 \tracecat [\event_1']}  \rightarrow^{*} 
    \config{c_2,  \vtrace_1 \tracecat [\event_1'] \tracecat \trace' \tracecat [(\neg \event_b)] \tracecat \trace_3'} 
    \\
    \bigwedge &  \tlabel_{\trace_3} \cap \tlabel_{\trace_3'} = \emptyset
     \land \vcounter(\trace', \pi_2(\event_b)) = \vcounter(\trace, \pi_2(\event_b)) 
    %   \land \event_2 \eventin \trace_3
    % \land \event_2 \not\eventin \trace_3'
    \land \event_2 \in \trace_3
    \land \event_2 \not\in \trace_3'
  \end{array}
  \right)
  )
\end{array}
   \]
% , where ${\tt label}(\event_2) = \pi_2(\event_2)$.
  %  
%
\end{defn}
% \todo{add explnanation}
% \jl{
Our event \emph{may-dependency} relation of 
two events $\event_1 \in \eventset^{\asn}$ and $\event_2 \in \eventset$, 
for a program $c$ and hidden database $D$ is w.r.t to
a trace $[\event_1 ] \tracecat \trace \tracecat [\event_2]$.
The $\event_1 \in \eventset^{\asn}$ is an assignment event because only a change on an assignment event will affect the execution trace, according to our operational semantics.
In order to observe the changes of $\event_2$ under the modification of $\event_1$, this trace 
$[\event_1 ] \tracecat \trace \tracecat [\event_2]$
starts with $\event_1$ and ends with $\event_2$.
% }
{The \emph{may-dependency} relation considers both the value dependency and value control dependency as discussed in 
analysis challenges in Paragraph~\ref{para:exe-dep-challenge}. The relation can be divided into two parts naturally in Definition~\ref{def:event_dep} (line $2-4$, $5-8$ respectively, starting from line $1$). The idea of the event $\event_1$ may depend on $\event_2$ can be briefly described:
I have one execution of the program as reference (See line $2$ and $6$, for the two kinds of dependency). 
When the value assigned to the 
% first variable 
first variable in $\event_1$ is modified, the reference trace $\trace_1 \tracecat [\event_1]$ is modified correspondingly to $\trace_1 \tracecat [\event_1']$.
We use $\diff(\event_1, \event_1')$ at line $1$ to express this modification, which guarantees that $\event_1$ and $\event_1'$ only differ in their assigned values and are equal on variable name and label. We perform a second run of the program by continuing the execution of the same program from the same execution point, 
but with the modified trace $\trace_1 \tracecat [\event_1']$ (See line $3$, $7$). 
The expected may dependency will be caught by observing two different possible changes (See line $4, 8$ respectively) when comparing the second execution with the reference one (similar definitions as in \cite{Cousot19a}). 
\\
% \wq{
% In the first situation, I are witnessing 
In the first part (line $2-4$ of Definition~\ref{def:event_dep}), I witness
% that the value assigned to the second variable in $\event_2$
the appearance of $\event_2'$ in the second execution, and
% a variation in $\event_2$, which changes into $\event_2'$.
a variation between $\event_2$ and $\event_2'$ on their values.
% changes in $\event_2'$.
% \jl{
We have special requirement $\diff(\event_2, \event_2')$, which guarantees that they
have the same variable name and label but only differ 
% % in their assigned value. 
in their evaluated values.
% assigned to the same variable. 
In particularly for queries, if $\event_2$ and $\event_2'$ are 
% query assignment events, then 
generated from query requesting, then $\diff(\event_2, \event_2')$ guarantees that
they differ in their query values rather than the 
% query requesting value. 
query requesting results. 
Additionally, in order to handle multiple occurrences of the same event through iterations of the while loop,
 where  $\event_2$ and $\event_2'$ could be 
in different while loops,
I restrict the same occurrence of $\event_2$'s label in $\trace$ from the first execution with  the occurrence of $\event_2'$'s label in $\trace'$ from the second execution,
through $\vcounter(\vtrace, \pi_2(\event_2))
= 
\vcounter(\vtrace', \pi_2(\event_2'))$ at line $4$.
% }
% }
\\
% \wq{
In the second part (line $5-8$ of Definition~\ref{def:event_dep}), I 
% are witnessing 
witness
the disappearance of $\event_2$ through observing the change of a testing event $\event_b$.
% In order to change the appearance of 
% % and event, the command that generating $\event_2$ must not be executed in 
% 5yhan event, 
To witness
the disappearance, the command that generates $\event_2$ must not be executed in 
the second execution. 
The only way to control whether a command will be executed, is through the change of a guard's 
evaluation result in an if or while command, which generates a testing event $\event_b$ in the first place.
So I observe when
$\event_b$ changes into $\neg \event_b$ in the second execution firstly, 
whether it follows with the disappearance of $\event_2$ in the second trace. We restrict the occurrence of $\event_b$'s label in the two traces being the same
}
% s to the occurrence times of $\event_2'$'s label in the second trace,
through $\vcounter(\trace', \pi_2(\event_b)) = \vcounter(\trace, \pi_2(\event_b))$ to handle the while loop.
% changes in $\event_2'$, have the same variable and label and only differ in their assigned value. 
Again, for queries, I observe the disappearance based on the query value equivalence.
% if $\event_2$ and $\event_2'$ are query assignment events, then 
% they differ in their query value rather than the assigned value. 
% }
%
% \mg{I don't understand this explanation. What are the ``assignment commands associated to the two labelled variables''}
% \jl{revised but need more think}
% Explanation: 
}
\paragraph*{Variable Dependency}
{Considering 
% a program's all possible executions 
all events generated during a program's executions
under an initial trace,
% among all events generated during these executions
% and the variables and labels of these events are 
% corresponding to the two labeled variables,
% evaluations of the assignment commands associated to the two labelled variables respectively, 
as long as there is one pair of events satisfying the \emph{event may-dependency} relation in Definition~\ref{def:event_dep}, 
 I say the two 
related
variables satisfy the \emph{variable may-dependency} relation, in Definition~\ref{def:var_dep}.
}
\begin{defn}[Variable May-Dependency]
    \label{def:var_dep}
\highlight{
    A labeled variable $y^j \in \lvar(c)$ is in the \emph{may-dependency} relation with another
    labeled variable $x^i \in \lvar(c)$ in a program ${c}$, w.r.t. an initial trace $\trace_0 \in \mathcal{T}_0(c)$
    and two witness traces $\trace_1, \trace_2 \in \mathcal{T}$,
    denoted as 
    %
    $\dep(x^i, y^j, \trace_1, \trace_2, \trace_0, {c})$, if an only if
    \[
      \begin{array}{l}
    \exists 
    D \in \dbdom, 
    \trace_0' \in \mathcal{T} \sthat
    (\forall z \neq x \sthat   \env(\trace_0 ) z =   \env(\trace_0') z )
    \\ \quad \land 
     \config{{c}, \vtrace_0} \rightarrow^{*} 
      \config{\clabel{\eskip}^l, \vtrace_0  \tracecat \trace_1 } 
      \land 
      % \config{{c}_1, \vtrace_0' \tracecat [\event_1']}  \rightarrow^{*} 
      % \config{\clabel{\eskip}^l,  
      % \vtrace_0' \tracecat [\event_1] \tracecat \trace_2 }   
      \config{{c}, \vtrace_0'} \rightarrow^{*} 
      % \config{{c}_1, \vtrace_0' \tracecat [\event_1]}  \rightarrow^{*} 
        \config{\clabel{\eskip}^l, \vtrace_0'  \tracecat \trace_2} 
      \land 
        \sdiff(\trace_1, \trace_2, y^j ) \neq \emptyset
      \end{array}
    \]  
  }
    \end{defn}
    Specifically, I define the variables dependency relation over two witness traces and an initial trace. Comparing to 
    the previous \emph{may-dependency} definition, which quantifies overall possible execution traces, the alternative
    definition explicitly relies on two specific witness traces from execution.
    % \\
\highlight{This externalization of the two witness traces also helps in analyzing the dependency quantity through the same 
witness traces as the \emph{may-dependency} relation. In this way, the in-accuracy in formalizing the \emph{adaptivity}
is reduced.}

% \begin{defn}[Variable May-Dependency].
%   \label{def:var_dep}
%   \\
%   A variable ${x}_2^{l_2} \in \lvar(c)$ is in the \emph{variable may-dependency} relation with another
%   variable ${x}_1^{l_1} \in \lvar(c)$ in a program ${c}$, denoted as 
%   %
%   $\vardep({x}_1^{l_1}, {x}_2^{l_2}, {c})$, if and only if.
% \[
%   \begin{array}{l}
% \exists \event_1, \event_2 \in \eventset^{\asn}, \trace \in \mathcal{T} , D \in \dbdom \sthat
% % (\pi_{1}{(\event_1)}, \pi_{2}{(\event_1)}) = ({x}_1, l_1)
% % \land
% % (\pi_{1}{(\event_2)}, \pi_{2}{(\event_2)}) = ({x}_2, l_2)
% \pi_{1}{(\event_1)}^{\pi_{2}{(\event_1)}} = {x}_1^{l_1}
% \land
% \pi_{1}{(\event_2)}^{\pi_{2}{(\event_2)}} = {x}_2^{l_2}% \\ \quad 
% \land 
% \eventdep(\event_1, \event_2, \trace, c, D) 
%   \end{array}
% \]  %
% \end{defn}


\highlight{\paragraph*{Improvements Analysis}
Comparing to previous works,
this new dependency definition is more precise and efficient.
% language and operational semantics design improves the expressiveness, efficiency, and the accuracy to a large extend.
\todo{Add details}
\begin{itemize}
  %   \item \textbf{Improvements on Expressiveness}
  %   \\
  % This language is extended over the standard while language. 
  % In this sense, it supports all the general data analysis.
  \item \textbf{Improvements on Efficiency}
  \\
  Also, it extends the standard while language with.
  \item \textbf{Improvements on Accuracy}
  \end{itemize}
  }

\paragraph*{Dependency Relation through The Two Rounds Example}
\begin{example}[Variable \emph{May-Dependency} Relation in The Two Rounds Data Analysis Example Program]
    In the same $\kw{towRounds(k)}$ example Program,    the analyst asks in total $k+1$ queries to the mechanism in two phases.
    %
    \[         \begin{array}{l}
      \kw{towRounds(k)} \triangleq \\
             \clabel{ \assign{a}{0}}^{0} ;
              \clabel{\assign{j}{k} }^{1} ;\\
              \ewhile ~ \clabel{j > 0}^{2} ~ \edo ~
              \Big(
               \clabel{\assign{x}{\query(\chi[j] \cdot \chi[k])} }^{3}  ;
               \clabel{\assign{j}{j-1}}^{4} ;
              \clabel{\assign{a}{x + a}}^{5}       \Big);\\
              \clabel{\assign{l}{\query(\chi[k]*a)} }^{6}
          \end{array}
          \]    %
    % Queries are of the form $q(e)$ where $e$ is an expression with a special variable $\chi$ representing a possible row. Mainly $e$ represents a function from $X$ to some domain $U$, for example $U$ could be $[-1,1]$ or $[0,1]$. This function characterizes the linear query I are interested in running. As an example, $x \leftarrow q(\chi[2])$ computes an approximation, according to the used mechanism, of the empirical mean of the second attribute, identified by $\chi[2]$. Notice that I don't materialize the mechanism but I assume that it is implicitly run when I execute the query. 
    % \jl{We use $\chi$ to abstract a possible row in the database and }
    % queries are of the form $\query(\qexpr)$, where $\qexpr$ is a special expression 
    In order to analysis the \emph{may-dependency} relation for variable 
    $x^3$ and $a^5$, I first generate the following execution trace with the initial trace
    $\trace_0 = [(k, in, 2, \bullet)]$ as follows,
   %   where $k$ is the 
   %  initial value of input variable $k$ given by user,
   %  we observe the execution trace as
   \\
    $
    \left[\begin{array}{l}
    % \trace_0 \tracecat
     (a, 0, 0, \bullet),
    (j, 1, 2, \bullet),
    (j>0, 2, \etrue, \bullet),
    (x, 3, v_1, \chi[2]*\chi[2]),
    (j, 4, 1, \bullet),
    (a, 5, v_1, \bullet),\\
    (j>0, 2, \etrue, \bullet),
    (x, 3, v_2, \chi[1]*\chi[2]),
    (j, 4, 0, \bullet),
    (a, 5, v_1 + v_2, \bullet),
    (j>0, 4, \efalse, \bullet),\\
    (l, 6, v_3, \chi[2]*( v_1 + v_2))
    \end{array} \right]
    $.
    \\
    We modify the value assigned to $x$ when evaluating the command $ \clabel{\assign{x}{\query(\chi[j] \cdot \chi[k])} }^{3}$
    in the first iteration.
    By manipulating the event in the trace, 
    the event $(x, 3, v_1, \chi[2]*\chi[2])$
    is modified into $(x, 3, v_1', \chi[2]*\chi[2])$ where $v_1 \neq v_1'$.
    Then, through executing the program from the execution point after executing line $3$, we observe another execution trace as follows,
    \\
    $
    \left[\begin{array}{l}
    % \trace_0 \tracecat
     (a, 0, 0, \bullet),
    (j, 1, 2, \bullet),
    (j>0, 2, \etrue, \bullet),
    \highlight{(x, 3, v_1', \chi[2]*\chi[2])},
    (j, 4, 1, \bullet),
    (a, 5, v_1, \bullet),\\
    (j>0, 2, \etrue, \bullet),
    (x, 3, v_2, \chi[1]*\chi[2]),
    (j, 4, 0, \bullet),
    \highlight{(a, 5, v_1' + v_2, \bullet),}
    (j>0, 4, \efalse, \bullet),\\
    (l, 6, v_3, \chi[2]*( v_1' + v_2))
    \end{array} \right]
    $.   
    \\
    In this trace, the event $(a, 5, v_1' + v_2, \bullet),$  is different from $(a, 5, v_1 + v_2, \bullet),$ in the first 
    trace.
    \\
    This change satisfies the Definition~\ref{def:event_dep}, so there exists the variable \emph{may-dependency} relation 
    between variable $x^3$ and $a^5$.
\end{example}