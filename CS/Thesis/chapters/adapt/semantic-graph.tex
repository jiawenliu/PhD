We can now define the \emph{semantics-based dependency graph} of a program $c$. We want this graph to combines quantitative reachability information with dependency information. 

\jl{
For a program $c$, there are some notations used in the following definition.
The labeled variables of $c$,
$\lvar(c) \subseteq \mathcal{LV}$ contains all the variables in $c$'s assignment commands, with the command labels as superscripts. 
The set of query-associated variables (in query request assignments),
$\qvar(c) \subseteq \lvar(c)$ contains all labeled variables in $c$'s query requests. 
The set of initial traces of $c$,
$\mathcal{T}_0(c) \subseteq \mathcal{T}$
contains all possible initial trace of $c$.
Each initial trace,  $\trace_0 \in \mathcal{T}_0(c)$ contains the initial values of all input variables of $c$. 
For instance, the initial trace of $\kw{twoRounds(k)}$ example contains the initial value of the input variable $k$.
}
\begin{defn}[Semantics-based Dependency Graph]
\label{def:trace_graph}
Given a program ${c}$,
its \emph{semantics-based dependency graph} 
$\traceG({c}) = (\traceV({c}), \traceE({c}), \traceW({c}), \traceF({c}))$ is defined as follows,
{\small
\[
\begin{array}{lll}
  \text{Vertices} &
  \traceV({c}) & := \left\{ 
  x^l
  ~ \middle\vert ~ x^l \in \lvar(c)
  \right\}
  \\
  \text{Directed Edges} &
  \traceE({c}) & := 
  \left\{ 
  (x^i, y^j) 
  ~ \middle\vert ~
  x^i, y^j \in \lvar(c) \land \vardep(x^i, y^j, c) 
  \right\}
  \\
  \text{Weights} &
  \traceW({c}) & := 
  \{ 
  (x^l, w) 
  ~ \vert ~ 
  w : \mathcal{T}_0(c) \to \mathbb{N}
  \land
  x^l \in \lvar(c) 
  \\ & &
  \quad \land
  \forall \trace_0 \in \mathcal{T}_0(c), \trace' \in \mathcal{T} \sthat \config{{c}, \trace_0} \to^{*} 
  \config{\eskip, \trace_0\tracecat\trace'} 
  \land w(\trace_0) = \vcounter(\trace', l) \}  
  \\
  \text{Query Annotations} &
  \traceF({c}) & := 
\left\{(x^l, n)  
~ \middle\vert ~
 x^l \in \lvar(c) \land
n = 1 \Leftrightarrow x^l \in \qvar(c) \land n = 0 \Leftrightarrow  x^l \notin \qvar(c)
\right\}
\end{array},
\]
}
%%%%%%%%%%%%%%%%%%%%%%%%%%%%%%%%%%% The Detailed Version in Explaining the Trace Operators %%%%%%%%%%%%%%%%%%%%%%%%%%%%%%%%%%%%%%%%%%%%%%%%%
A semantics-based dependency graph $\traceG({c})= (\traceV({c}), \traceE({c}), \traceW({c}), \traceF({c}))$ 
is \emph{well-formed} if and only if $ \{x^l \ |\ (x^l,w)\in \traceW({c})\} = \traceV({c}) $.
\end{defn}


%%%%%%%%%%%%%%%%%%%%%%%%%%%%%%%%%%% The Explanation of Dependency Graph %%%%%%%%%%%%%%%%%%%%%%%%%%%%%%%%%%%%%%%%%%%%%%%%%
\jl{There are four components in this graph.
\begin{enumerate}
    \item The vertices $\traceV({c})$ of a program $c$ are all its labeled variables, $\lvar(c)$ which are statically collected.
    \item $\traceF(c)$ contains the \emph{query annotation} for 
    every vertex $x^l \in \traceV(c)$. It indicates whether $x^l$ comes from a query request (1) or not (0) by checking if the labeled variable $x^l$ of the vertex is in $\qvar(c)$.
    \item Edges in $\traceE(c)$ are built from the  \emph{variable may-dependency} relation, i.e. $\vardep(x^i, y^j, c)$ in Definition~\ref{def:var_dep} between two labeled variables.
    This is the key definition in order to formalize the intuitive \emph{adaptivity}, and also ingredients the dependency information of the program.
    \item 
  The weight function in $\traceW(c)$ for each vertex, $w : \mathcal{T} \to \mathbb{N}$
maps from a starting trace $\trace_0 \in \mathcal{T}_0(c)$ to a natural number.
A weight function $w \in \traceW(c)$ is a function that for every starting trace $\trace_0 \in \mathcal{T}_0(c)$ 
gives the number of times the assignment of the corresponding vertex $x^l$ is visited. Notice that weight functions are total and with range $\mathbb{N}$. This means that if a program $c$ has some non-terminating behavior, the set $\traceW(c)$ will be empty.
To rule out this situation, we consider as well-formed only graphs which have a weight for every vertex. 
For each vertex $x^l$, it tracks its visiting times (i.e., the evaluation times of the command with the label $l$) when the program $c$ is evaluated from the initial trace $\trace_0$ into $\eskip$, $\config{{c}, \trace_0} \to^{*} \config{\eskip, \trace_0\tracecat\trace'} $.
The visiting times is computed by the counter operator $\vcounter(\trace', l)$
by counting the occurrence of the label $l$ in $\trace'$.
As an instance, in the semantics-based dependency graph of $\kw{twoRounds}$ in Figure~\ref{fig:overview-example}(b), the weight, $w_k$ of the vertex $x^3$ is a function of type $\mathcal{T}_0(\kw{twoRound(k)}) \to \mathbb{N}$.
Given input $\trace_0$, we execute the program under $\trace_0$ as $\config{\kw{twoRound(k)}, \trace_0} \to^{*} \config{\eskip, \trace_0\tracecat\trace'} $. Then $w_k(\trace_0)$ outputs the occurrence time of the label $3$ in $\trace'$.
\end{enumerate}
The main novelty of  the semantics-based dependency graph is the combination of the quantitative and dependency information. 
It can tell both the dependency between queries via the directed edge, and the times they depend on each other via the weight.
}
