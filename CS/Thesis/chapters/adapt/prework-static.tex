In the previous works on estimating the adaptivity, they design a program analysis framework
through constructing a variable-based weighted dependency graph for
estimating the query-based dependence graph in Section~\ref{sec:prework-formalization}.
% Before they analyze and generate this graph, 
% In order to improve the analysis accuracy,
They first re-write the program from the loop language into
a Static Single Assignment (SSA) version of loop language, then estimate the adaptivity than program analysis algorithm.
% This can improve the accuracy in the variable re-assigning cases.
% The following parts summarize the SSA version of the loop language with their program analysis framework,
% followed by the limitations of their analysis method.
\subsection*{SSA-Language}
In order to distinguish different query requests in the case where they are assigned to
%  the issue of re-assignment of query requesting results to 
the same variable, they re-write the program from the loop language into SSA form.
% The SSA labeled command $\ssa{c}$ inherits from the {loop} language, except that the expressions and variables in these commands 
A selection of the (SSA) version of loop language syntax is shown below.
%  now in its SSA version as shown below. 
\[
\begin{array}{llll}
 & \ssa{c} & ::= &   [\assign {\ssa{x}}{ \ssa{\expr}}]^{l} ~|~  [\assign {{\ssa{x}} } {q({\ssa{e_q}})}]^{l}
%
~|~  {{ifvar(\bar{\ssa{x}}, \bar{\ssa{x}}')}}  ~|~ [\eskip]^{l}  ~|~
 \eloop ~ [{\ssa{\aexpr}}]^{l}, {n},  [\bar{\ssa{x}}, \bar{\ssa{x_1}}, \bar{\ssa{x_2}}] ~ \edo ~ {\ssa{c}}  ~|~ \\ &&& \ssa{c};\ssa{c}  ~|~  \eif([\ssa{\bexpr}]^{l}, ([\bar{\ssa{x}}, \bar{\ssa{x_1}}, \bar{\ssa{x_2}}] , [\bar{\ssa{y}}, \bar{\ssa{y_1}}, \bar{\ssa{y_2}}],[\bar{\ssa{z}}, \bar{\ssa{z_1}}, \bar{\ssa{z_2}}] ) , \ssa{c}, \ssa{c}) 	
\end{array}
\]
In this SSA version, if command contains the extra part $([\bar{\ssa{x}}, \bar{\ssa{x_1}}, \bar{\ssa{x_2}}] ,
[\bar{\ssa{y}}, \bar{\ssa{y_1}}, \bar{\ssa{y_2}}],[\bar{\ssa{z}}, \bar{\ssa{z_1}}, \bar{\ssa{z_2}}] )$,
which helps to track the dependency of new assigned variables in both branches($[\bar{\ssa{x}}, \bar{\ssa{x_1}}, \bar{\ssa{x_2}}]$),
then branch $[\bar{\ssa{y}}, \bar{\ssa{y_1}}, \bar{\ssa{y_2}}]$, and else branch $[\bar{\ssa{z}}, \bar{\ssa{z_1}}, \bar{\ssa{z_2}}] $. 
% The $\bar{\ssa{x}}$ is a list of SSA variables,
% in which every element $\ssa{x}$ may depend on the corresponding element(at same location),
% $\ssa{x_1}$ from $\bar{\ssa{x_1}}$ collected in the then branch or the corresponding element $\ssa{x_2}$ from $\bar{\ssa{x_2}}$ collected in the else branch.
% The size of these three lists are required to be the same.
And the loop command also has similar part $ [\bar{\ssa{x}}, \bar{\ssa{x_1}}, \bar{\ssa{x_2}}]$ focusing on the loop body.

Then they transform the operational semantics rules with all the operations defined under the loop language introduced in Section~\ref{sec:prework-language}.
Based on the transformed language, they also redefine the formal adaptivity for the transformed program equivalent to the
loop language-based definition. The complete definition is in Thesis~\cite{weihao22}.
%
\subsection*{Program Analysis Algorithm}
Then they estimate the adaptivity for a program based on rewriting it into SSA form.
This part summarizes
%  the previous analysis algorithm for estimating the adaptivity for a program. 
their analysis algorithm.
It is based on the SSA version of the loop language, and
consists of three auxiliary algorithms:
a variable estimation algorithm $\mathsf{VE}$, a matrix-vector based graph generating algorithm $\mathsf{GG}$ to generate the weighted variable-based dependency graph, and a path-searching algorithm $\mathsf{PS}$ to find the most weighted path in the graph.
%
\begin{enumerate}
    \item \textbf{{Variable Estimation Algorithm}}
\\
%
The $\mathsf{VE}$ specifies the nodes of their variable-based dependency graph. The result of $\mathsf{VE}$ is stored in a global variable list $G$, fed to the next step.
% \\
    Figure~\ref{fig:prework-static_alg1} is a selection of the three key rules from their $\mathsf{VE}$ algorithm. 
    This algorithm adds all the program variables in its SSA form to the global variable list $G$.
    % $\mathsf{VE}$ has the form $\ag{G; w; \ssa{c}}{ G'; w'} $, as shown in . The input of $\mathsf{VE}$ is a list of annotated variables $G$ collected before the program $\ssa{c}$, a loop map $w$ consistent with previous estimation, and an input SSA program $\ssa{c}$. The output of the algorithm is the updated global list $G'$, along with the updated loop maps $w$, for later estimation.  
    \\
    \highlight{
        Their variable estimation for the while loop is low-efficient. As shown in rule \textbf{ag-loop},
        they unfold every iteration of the while loop and create new annotated variables for every iteration.
        This rule causes a major efficiency limitation in their static analysis.
        It also causes a critical expressiveness limitation. By the premise in the rule \textbf{ag-loop},
        the arithmetic expression is required to be a natural number.
        This rule limits the program cannot even
        have a loop with an arithmetic expression like $10 + 4$ in the guard.}
\begin{figure}
{\footnotesize
 \begin{mathpar}
% \inferrule
% {
% }
% { \ag{G ;w; \ssa{[\assign {x}{\expr}]^{l}}}{G ++ [\ssa{x}^{(l,w)}];w}
% % G ;w; \ssa{[\assign {x}{\expr}]^{l}} \to G ++ [x^{(l,w)}];w 
% }
% ~\textbf{ag-asgn}
% \and
\inferrule
{
}
{ \ag{G ;w;  [ \assign{\ssa{x}}{q(\ssa{\expr_q})}]^{l}}{  G ++ [\ssa{x}^{(l,w)}] ; w} 
}~\textbf{ag-query}
%
\and 
%
\inferrule
{
\ag{G; w; \ssa{c_1}}{  G_1;w_1}
\and 
 \ag{G_1;w ; \ssa{c_2}}{  G_2; w_2}
 \\
 {G_3 = G_2 ++ \ssa{[\bar{x}^{(l,w)}]++ \ssa{[\bar{y}^{(l,w)}]}++ \ssa{[\bar{z}^{(l,w)}]} }}
}
{
\ag{G; w;
[\eif(\ssa{\bexpr},[ \bar{\ssa{x}}, \bar{\ssa{x_1}}, \bar{\ssa{x_2}}] ,[ \bar{\ssa{y}}, \bar{\ssa{y_1}}, \bar{\ssa{y_2}}],[ \bar{\ssa{z}}, \bar{\ssa{z_1}}, \bar{\ssa{z_2}}], \ssa{ c_1, c_2)}]^{l} }{ G_3 ;w}
}~\textbf{ag-if}
\and 
\inferrule
{
{G_0 = G \quad w_0 =w }
\and
\forall 0 \leq z < N. 
{ \ag{ G_z ++ \ssa{[\bar{x}^{(l, {w_z}+l)}]} ; (w_z+l); \ssa{c}}{ G_{z+1} ; w_{z+1}}  }
\\
{G_f = G_N ++ \ssa{[\bar{x}^{(l, w_N \setminus l)}]} }
\and
{ \ssa{\aexpr} =  {N}  }
}
{\ag{G; w; [\eloop ~ \ssa{\aexpr}, n, [\bar{\ssa{x}}, \bar{\ssa{x_1}}, \bar{\ssa{x_2}}] ~ \edo ~ \ssa{c}]^{l} }{ G_f; w_N\setminus l }
}~\textbf{ag-loop}
\end{mathpar}
}
 \caption{The key rule of variable estimation algorithm.  }
    \label{fig:prework-static_alg1}
\end{figure}
%
\item \textbf{Graph Generating}
\\
The algorithm $\mathsf{GG}$ generates a matrix-vector-based graph for the program. 
The matrix $M$ records the may-dependency between annotated variables in the global list $G$. It has size $|G| \times |G|$. The vector $V$ has the same size as $G$ and gives a weight to each variable in $G$.
This weight is $1$ when the variable is assigned with a query request and $0$ otherwise. 
% To be precise, the $i$th row, $j$th column of the matrix $M$, written $M[i][j]$, is  $1$ when there may be a dependency from variable $ G[i]$ to $G[j]$. Dually, $M[i][j] =0$ means no dependency. In a similar way, $V[i]=1$ means the variable $G[i]$ is assigned with a query request.
% Figure~\ref{fig:prework-static_alg2} shows some selected rules of this algorithm.
\highlight{The same key rules is selected in Figure~\ref{fig:prework-static_alg2}.
The rule \textbf{ad-loop} is as low-efficient and expressiveness-limited as the rule \textbf{ag-loop} in Figure~\ref{fig:prework-static_alg1} for the same reason.
This will be analyzed in detail in the limitation part.
}
%
\begin{figure}
{\footnotesize
\begin{mathpar}
% \inferrule
% {M = \mathsf{L}(i) * ( \mathsf{R}(\ssa{\expr},i) + \Gamma )
% }
% {
%  \gp{\Gamma;[\assign {\ssa{x}}{\ssa{\expr}} ]^{l}; i }{M; V_{\emptyset}; i+1 }
% % \Gamma \vdash_{M, V_{\emptyset}}^{(i, i+1)} [\assign {\ssa{x}}{\ssa{\expr}} ]^{l}
% }
% ~\textbf{ad-asgn}
% \and
\inferrule
{M = \mathsf{L}(i) * ( \mathsf{R}(\ssa{\expr_q},i) + \Gamma )
\\
V= \mathsf{L}(i)
}
{ 
\gp{\Gamma;[ \assign{\ssa{x}}{q(\ssa{\expr_q})} ]^{l} ; i }{M;V;i+1}
%  \vdash^{(i, i+1)}_{M, V} [ \assign{\ssa{x}}{q(\ssa{\expr})} ]^{l} 
}~\textbf{ad-query}
%
\and 
%
\inferrule
{
{\gp{\Gamma + \mathsf{R}(\ssa{\bexpr}, i_1); \ssa{c_1} ; i_1 }{ M_1;V_1;i_2 }}
% \Gamma + \mathsf{R}(\bexpr, i_1) \vdash^{(i_1, i_2)}_{M_1, V_1} \ssa{c_1} 
% : \Phi \land \bexpr \Rightarrow \Psi
\\
{\gp{\Gamma + \mathsf{R}(\ssa{\bexpr}, i_1);\ssa{c_2} ; i_2 }{ M_2; V_2 ;i_3}}
% \Gamma + \mathsf{R}(\ssa{\bexpr}, i_1) \vdash^{(i_2, i_3)}_{M_2, V_2} \ssa{c_2} 
% : \Phi \land \neg \bexpr \Rightarrow \Psi
\\
% { \forall 0 \leq j < |\bar{x}|. \bar{x}(j) = x_j, \bar{x_1}(j) = x_{1j}, \bar{x_2}(j) = x_{2j}  }
{\gp{\Gamma; [ \bar{\ssa{x}}, \bar{\ssa{x_1}}, \bar{\ssa{x_2}}]; i_3 }{ M_x; V_{\emptyset}; i_3+|\bar{\ssa{x}}| }}
%
\\\\
%
{\gp{\Gamma; [ \bar{\ssa{y}}, \bar{\ssa{y_1}}, \bar{\ssa{y_2}}]; i_3+|\bar{\ssa{x}}| }{ M_y; V_{\emptyset}; i_3+|\bar{\ssa{x}}|+|\bar{\ssa{y}}| }}
%
\\
%
{\gp{\Gamma; [ \bar{\ssa{z}}, \bar{\ssa{z_1}}, \bar{\ssa{z_2}}]; i_3+|\bar{\ssa{x}}|+ |\bar{\ssa{y}}|}{ M_y; V_{\emptyset}; i_3+|\bar{\ssa{x}}|+|\bar{\ssa{y}}| + |\bar{\ssa{z}}| }}
\\
{M = (M_1+M_2)+ M_x+M_y +M_z }
}
{
\gp{\Gamma ; \eif([\ssa{\bexpr}]^{l},[ \bar{\ssa{x}}, \bar{\ssa{x_1}}, \bar{\ssa{x_2}}] ,[ \bar{\ssa{y}}, \bar{\ssa{y_1}}, \bar{\ssa{y_2}}] , [ \bar{\ssa{z}}, \bar{\ssa{z_1}}, \bar{\ssa{z_2}}] , \ssa{ c_1, c_2)} ; i_1}{ M ;V_1 \uplus V_2  ; i_3+|\bar{x}|+|\bar{y}|+|\bar{z}| }
}
~\textbf{ad-if}
\and
% \and 
\inferrule
{
B= |\ssa{\bar{x}}| \and {A = |\ssa{c}|}
% \and
% {\Gamma \vdash^{(i, i+B)}_{M_{10}, V_{10}} [\bar{\ssa{x}}, \bar{\ssa{x_1}}, \bar{\ssa{x_2}}] }
% \and
% {\Gamma \vdash^{(i+B,i+B+A )}_{M_{20}, V_{20}} \ssa{c} 
% }
\\
\forall 0 \leq j < N. 
{\gp{\Gamma;[\bar{\ssa{x}}, \bar{\ssa{x_1}}, \bar{\ssa{x_2}}]; i+ j*(B+A) }{M_{1j};V_{1j}; i+B+j*(B+A) }}
% {\Gamma \vdash^{(i+j*(B+A), i+B+j*(B+A))}_{M_{1j}, V_{1j}}  } [\bar{\ssa{x}}, \bar{\ssa{x_1}}, \bar{\ssa{x_2}}]
\\
{
\gp{\Gamma;\ssa{c} ; i+B+j*(B+A)  }{M_{2j}; V_{2j}; i+B+A+j*(B+A) }
% \Gamma \vdash^{(i+B+j*(B+A),i+B+A+j*(B+A) )}_{M_{2j}, V_{2j}} \ssa{c} 
% : \Phi \land e_n = \lceil{z+1}\rceil \Rightarrow \Psi 
}
\\
{
\gp{\Gamma ; [\bar{\ssa{x}}, \bar{\ssa{x_1}}, \bar{\ssa{x_2}}] ; i+N*(B+A) }{M; V ;i+N*(B+A)+B}
% \Gamma \vdash^{(i+N*(B+A) ,i+N*(B+A)+B )}_{M, V} [\bar{\ssa{x}}, \bar{\ssa{x_1}}, \bar{\ssa{x_2}}]
% : \Psi \Rightarrow \Phi \land e_N = \lceil{z}\rceil 
}
\\
{ \ssa{\aexpr} =  {N}  }
\and
{M' = M+ \sum_{0 \leq j <N}( M_{1j}+M_{2j})  }
\and
{V' = V \uplus \sum_{0 \leq j <N}( V_{1j} \uplus V_{2j})  }
}
{
\gp{\Gamma;\eloop ~ [\ssa{\aexpr}]^{l}, ~0, [\bar{\ssa{x}}, \bar{\ssa{x_1}}, \bar{\ssa{x_2}}] ~ \edo ~ \ssa{c}, i }{ M';V' ;i+N*(B+A)+B }
%  \vdash^{(i,   )}_{M', V'} 
% : \Phi \land \expr_N = \lceil { N} \rceil \Rightarrow \Phi \land \expr_N = \lceil{0}\rceil
}~\textbf{ad-loop}
\end{mathpar}
}
    \caption{The key rules of the graph generating algorithm.}
    \label{fig:prework-static_alg2}
\end{figure}
%
\item \textbf{Longest Weighted Path Search}
\\
They use standard longest path search algorithm to compute the adaptivity bound over the variable-based dependency graph.
\end{enumerate}
\subsection*{Limitations}
\highlight{
\begin{enumerate}
 \item \textbf{Efficiency Limitations}
 \begin{enumerate}
 \item
 In order to address the issue of re-assignment of different queries requesting results to the same variable, they re-write the program from the loop language into SSA form.
 However, this rewriting is low-efficient and unnecessary.
 The re-assignment problem can be resolved efficiently and accurately through many state-of-art static program analysis techniques, such as the
 variable reachable analysis, etc..
 \item Their variable estimation algorithm is low-efficient by their \textbf{ag-loop} rule in Figure~\ref{fig:prework-static_alg1},
 and rule \textbf{ad-loop} in Figure~\ref{fig:prework-static_alg2}.
 These two rules unfold every iteration of the while loop and create new annotated variables for every iteration.
 This operation increased the complexity of the program analysis by exponential factors. 
 \item For the same reason as above, their \textbf{Graph Generation} algorithm is low-efficient as well.
 \end{enumerate}
 %
 \item \textbf{Expressiveness Limitations}
 \begin{enumerate}
 \item Their variable estimation algorithm is unable to analyze the programs with
 non-constant (or non-deterministic) loop iteration numbers.
 This is caused by their \textbf{ag-loop} rule in Figure~\ref{fig:prework-static_alg1}.
 This rule unfolds every iteration of the while loop, and creates new annotated variables for every iteration.
 In order to guarantee the termination of the analysis, they have to limit the loop with the constant number of loop iterations.
 Specifically in rule \textbf{ag-loop} and \textbf{ad-loop},
 % Their variable estimation for the while loop is low-efficient. As shown in rule \textbf{ag-loop},
 % they unfold every iteration of the while loop and create new annotated variables for every iteration.
 % This rule causes a major efficiency limitation of their static analysis.
 % It also causes a critical expressiveness limitation. 
 the premise in these two rules requires
 the arithmetic expression
 % is required 
 to be a natural number. 
 This rule limits the program cannot even
 have a loop with an arithmetic expression like $10 + 4$ in the guard.
 Concretely, a simple example program with a loop iterating $5$ times as follows isn't allowed in their work.
 \[
 {\assign{x}{5}};
 \assign{z}{q(x)};
 \eloop (x ) \edo 
 \{
 \assign{z}{q(x + z)};
 \}\}
 \] 
 \item For the same reason as above, their \textbf{Graph Generation} algorithm is limited as well.
 \end{enumerate}
 \item \textbf{Accuracy Limitations}
 \begin{enumerate}
 \item The estimated adaptivity from this framework is loose.
 It over-approximates in the cases where there isn't semantic dependency between variables even though the variables
 are explicitly used in the query request.
 \item This program analysis framework is naive in the sense that all three steps are standard.
 And the framework is simply a straightforward composition of the three steps.
 \end{enumerate}
\end{enumerate}
}