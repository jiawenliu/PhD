For a program $c$, there are two data \emph{dependency quantities} we are considering.
The first quantity is the reachability times of each labeled variable during the program execution.
The second quantity is the reachability time for every pair of labeled variables with variable \emph{may-dependency} relation.
% \paragraph*{Variable Reachability}
\paragraph{The Dependency Quantity for Labeled Variables}
The reachability time of a labeled variable indicates the evaluation times of the assignment command assigning a value to this variable.  
\begin{defn}[Reachability Time of Labeled Variable]
  \label{def:adapt-var_reachability}
The reachability for every labeled variable overall $c$'s execution traces,
w.r.t. an initial trace $\vtrace \in \mathcal{T}_0(c)$ is defined as follows,
\[
  rb(x^l) \triangleq \forall \vtrace \in \mathcal{T}_0(c), \trace' \in \mathcal{T} \sthat \config{{c}, \trace} \to^{*} \config{\eskip, \trace\tracecat\vtrace'} 
  \implies w(\trace) = \vcounter(\vtrace', l) 
  \]
\end{defn}
%
$(x^l, w) \in \mathcal{LV} \times (\mathcal{T} \to \mathbb{N})$,
with a labeled variable as first component and
its weight $w$ the second component.
Weight $w$ for
% a labeled variable 
$x^l$ is a function $w : \mathcal{T} \to \mathbb{N}$
mapping from a starting trace to a natural number.
When program executes under this starting trace $\trace$,
$\config{{c}, \trace} \to^{*} \config{\eskip, \trace\tracecat\vtrace'} $, it generates an execution trace $\trace'$.
This natural number is the evaluation times of the labeled command corresponding to the vertex, 
computed by the counter operator $w(\trace) = \vcounter(\vtrace', l)$.


In most data analysis programs $c$ we are interested, there are usually some user input variables, such as $k$ in $\kw{twoRounds}$. 
We denote $\mathcal{T}_0(c)$ as the set of initial traces in which all the input variables in $c$ are initialized, it is also reflected in $\traceW({c})$.    
%
\paragraph{Dependency Quantity for the Pair of Labeled Variables}
% \paragraph*{Dependent Variables Reachability}
%
% For a program $c$ I compute the reachability bound for every labeled variable overall $c$'s execution traces,
% w.r.t. an initial trace as follows,
\begin{defn}[Reachability Time of Dependent Variables]
  \label{def:adapt-depvar_reachability}
  The execution-based reachability time for every pair of 
  labeled in the
  \emph{may-dependency} relation w.r.t. an initial trace. Formally as follows,
    \[
    \begin{array}{l}
        rb(x^i, y^j) \triangleq 
%   x^i, y^j \in \lvar(c)
%   \land w \in \mathcal{P}( \mathcal{T}_0(c) \to \mathbb{N})
%   \land 
%   \exists \trace \in \mathcal{T}_0(c), 
%   \trace_1, \trace_2 \in \mathcal{T} \sthat \dep(x^i, y^j,\trace_1, \trace_2, \trace_0, c)
%   \\
%   \land 
\forall \trace_0 \in \mathcal{T}_0(c) \sthat
  w (\trace_0) = \max \left\{ | \sdiff(\trace_1, \trace_2, y)|
  ~\middle\vert~
  \forall \trace_1, \trace_2 \in \mathcal{T} \sthat \dep(x^i, y^j,\trace_1, \trace_2, \trace_0, c) \right\}
\end{array}
\]
\end{defn}
%
For any pair of labeled variable $(x^i, y^j) \in \ldom$, 
$ rb(x^i, y^j)$ is a function $w: \mathcal{T}_0(c) \to \mathbb{N}$,
    where given an initial trace $\trace_0$,
    it is the maximum length of the difference sequence between all pairs of the witness traces $\trace_1, \trace_2$ 
    satisfying the dependency relation.

    \highlight{\paragraph*{Improvements Analysis}
    Previous works do not have any quantity analysis on the dependency relation.
    Comparing to them, this part helps in formalizing adaptivity precisely and efficiently in following senses.
    % It is more scalable to general program, and it provides the program with preciser formal definition for \emph{Adaptivity} than previous definition,
    % specifically as follows.
    % language and operational semantics design improves the expressiveness, efficiency, and the accuracy to a large extend.
    \begin{itemize}
      \item \textbf{Improvements on Accuracy}
      \\
      This quantity analysis can be combined with the
      dependency analysis.
      In this sense, the adaptivity is formalized more precisely with the two information than previous works using
      only the dependency information.
      \item \textbf{Improvements on Efficiency}
      \\
      The reachability times for every variable and every pair of labeled variables does not require
      unfolding the while loop for all possible executions.
      They are designed as functions w.r.t. the input initial traces.
      In this sense, the counting and unfolding is only required when necessary.
      \end{itemize}
      }
\paragraph*{The Dependency Quantity through The Two Rounds Example}
\begin{example}[Variable \emph{May-Dependency} Quantity in The Two Rounds Data Analysis Example Program]
    In the same $\kw{towRounds(k)}$ example Program,    the analyst asks in total $k+1$ queries to the mechanism in two phases.
    %
    \[         \begin{array}{l}
      \kw{towRounds(k)} \triangleq \\
             \clabel{ \assign{a}{0}}^{0} ;
              \clabel{\assign{j}{k} }^{1} ;\\
              \ewhile ~ \clabel{j > 0}^{2} ~ \edo ~
              \Big(
               \clabel{\assign{x}{\query(\chi[j] \cdot \chi[k])} }^{3}  ;
               \clabel{\assign{j}{j-1}}^{4} ;
              \clabel{\assign{a}{x + a}}^{5}       \Big);\\
              \clabel{\assign{l}{\query(\chi[k]*a)} }^{6}
          \end{array}
          \]    %
    % Queries are of the form $q(e)$ where $e$ is an expression with a special variable $\chi$ representing a possible row. Mainly $e$ represents a function from $X$ to some domain $U$, for example $U$ could be $[-1,1]$ or $[0,1]$. This function characterizes the linear query I are interested in running. As an example, $x \leftarrow q(\chi[2])$ computes an approximation, according to the used mechanism, of the empirical mean of the second attribute, identified by $\chi[2]$. Notice that I don't materialize the mechanism but I assume that it is implicitly run when I execute the query. 
    % \jl{We use $\chi$ to abstract a possible row in the database and }
    % queries are of the form $\query(\qexpr)$, where $\qexpr$ is a special expression 
    With the initial trace
    $[(k, in, 2, \bullet)]$ and following execution trace, 
    \\
    $
    \trace_1 \triangleq 
    \left[\begin{array}{l}
    % \trace_0 \tracecat
     (a, 0, 0, \bullet),
    (j, 1, 2, \bullet),
    (j>0, 2, \etrue, \bullet),
    (x, 3, v_1, \chi[2]*\chi[2]),
    (j, 4, 1, \bullet),
    (a, 5, v_1, \bullet),\\
    (j>0, 2, \etrue, \bullet),
    (x, 3, v_2, \chi[1]*\chi[2]),
    (j, 4, 0, \bullet),
    (a, 5, v_1 + v_2, \bullet),
    (j>0, 4, \efalse, \bullet),\\
    (l, 6, v_3, \chi[2]*( v_1 + v_2))
    \end{array} \right]
    $.
    Based on these observations, we analyze the \emph{may-dependency} quantity for every labeled variable,
    and pairs of dependent variables as follows.
\begin{itemize}
    \item \textbf{The Dependency Quantity for Labeled Variables}
    \\
    For the specific two execution traces above,
    the \emph{may-dependency} quantity for every variable
    is computed as follows,
   %   where $k$ is the 
   %  initial value of input variable $k$ given by user,
   %  we observe the execution trace as
   \\
   $rb(a^0) ((k, in, 2, \bullet))  = \vcounter(\trace_1) = 1$
   \\
   $\cdots$
  \\
   $rb(x^3) ((k, in, 2, \bullet))  = \vcounter(\trace_1) = 2$
    \\
    $\cdots$
    \\

    Then, for arbitrary initial trace $\trace_0 \in \mathcal{T}_0(\kw{twoRounds(k)})$,
    the \emph{may-dependency} quantity for every variable under $\trace_0$ is a function
    as follows,
    \\
    $rb(a^0) (\trace_0)  = 1$
    \\
    $\cdots$
    \\
    $rb(x^3) (\trace_0)  = \max\{0, \env(\trace_0) k \} $
    \item \textbf{Dependency Quantity for the Pair of Labeled Variables}
    For the specific two execution traces above,
    the \emph{may-dependency} quantity for every variable
    is computed as follows,
   %   where $k$ is the 
   %  initial value of input variable $k$ given by user,
   %  we observe the execution trace as
   \\
   $rb(a^0) ((k, in, 2, \bullet))  = \vcounter(\trace_1) = 1$
   \\
   $\cdots$
  \\
   $rb(x^3) ((k, in, 2, \bullet))  = \vcounter(\trace_1) = 2$
    \\
    $\cdots$
    \\

    Then, for arbitrary initial trace $\trace_0 \in \mathcal{T}_0(\kw{twoRounds(k)})$,
    the \emph{may-dependency} quantity for every variable under $\trace_0$ is a function
    as follows,
    \\
    $rb(a^0) (\trace_0)  = 1$
    \\
    $\cdots$
    \\
    $rb(x^3) (\trace_0)  = \max\{0, \env(\trace_0) k \} $
\end{itemize}
    % \\
    % We modify the value assigned to $x$ when evaluating the command $ \clabel{\assign{x}{\query(\chi[j] \cdot \chi[k])} }^{3}$
    % in the first iteration.
    % By manipulating the event in the trace, 
    % the event $(x, 3, v_1, \chi[2]*\chi[2])$
    % is modified into $(x, 3, v_1', \chi[2]*\chi[2])$ where $v_1 \neq v_1'$.
    % Then, through executing the program from the execution point after executing line $3$, we observe another execution trace as follows,
    % \\
    % $
    % \left[\begin{array}{l}
    % % \trace_0 \tracecat
    %  (a, 0, 0, \bullet),
    % (j, 1, 2, \bullet),
    % (j>0, 2, \etrue, \bullet),
    % \highlight{(x, 3, v_1', \chi[2]*\chi[2])},
    % (j, 4, 1, \bullet),
    % (a, 5, v_1, \bullet),\\
    % (j>0, 2, \etrue, \bullet),
    % (x, 3, v_2, \chi[1]*\chi[2]),
    % (j, 4, 0, \bullet),
    % \highlight{(a, 5, v_1' + v_2, \bullet),}
    % (j>0, 4, \efalse, \bullet),\\
    % (l, 6, v_3, \chi[2]*( v_1' + v_2))
    % \end{array} \right]
    % $.   
    % \\
    % In this trace, the event $(a, 5, v_1' + v_2, \bullet),$  is different from $(a, 5, v_1 + v_2, \bullet),$ in the first 
    % trace.
    % \\
    % This change satisfies the Definition~\ref{def:event_dep}, so there exists the variable \emph{may-dependency} relation 
    % between variable $x^3$ and $a^5$.
\end{example}