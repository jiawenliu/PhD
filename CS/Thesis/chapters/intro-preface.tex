Program analysis analyzes the behaviors of a computer program.
One of the most significant behaviors with respect to programs is the execution correctness behavior, which determines whether a program executes correctly without getting stuck because of bugs.
 The execution correctness can be proved by showing the functional correctness property of the program with the help of some formal verification methods such as type system and program logic.
 Much attention in the programming language community has focused on the functional correctness property, while not enough effort was put on, a variety of those non-function properties, considering its wide potential applications in modern society. 
 A few decades ago, computer programs were only used for research or business purpose and the expectation of computer programs was just to execute correctly without bugs. 
 However, the expectation has also changed along with the popularity of mobile devices and wide applications of big data. 
 It is not enough for these programs running on mobile phones or those programs handling big data, 
 to just run without bugs. 
 The non-functional execution properties come into play in the new era.

 We think programs on mobile devices are of great significance in today's life.
 To provide people with high-quality
 service through mobile devices,
we choose to study the non-functional properties of programs.
% Skeleton:
Importance of the Program Execution Property in different areas.


% In Machine Learning Area, the Adaptivity Quantity is significant
% ==> Major Work I
% \\
Existing high-quality
service through mobile devices relies heavily on the data analysis results
% machine learning algorithm analysis results over data,
for providing users with personalized accurate services. This brings my attention on the
% The 
first non-functional execution behavior in these data analysis programs, i.e., the quality of the data analysis results.
% in their machine learning algorithms.
% comes from the machine learning area which is popular and widespread applied in our daily mobile life.
% In this area,
% The first execution property 
% the data collected from a large number of mobile users also deserves our attention.
We look at data analysis which analyzes sample data to get some generalized properties of the big data which the sample is drawn from.
The generalization property of a data analysis program over sample data with respect to the population is one of the important non-functional properties of programs.
The generalization error measures the difference of property from the sample data and population, showing the reliability of the data analysis of showing the true properties of the population. 
High generalization error makes the analysis result of these programs not reliable.
% It could be useful to have the program 
% This generalization error 
Fortunately, studies found that some quantitative properties of data analysis programs can help to control the generalization error, especially when the data analysis is adaptive.
These properties are the first non-functional execution properties we are interested in
analyzing.
% \\
% We use the static analysis technique to best exploit the benefit of these non-functional quantitative properties in resource usage and data analysis. However, to fully utilize the aforementioned benefits of the static analysis on these quantitative properties, the appropriate implementation is inevitable. To this end, we also take one step in algorithmizing a refinement type and effect system, which is used to statically estimate the properties of resource usage. In precise, the resource is the evaluation cost of programs in this proposal.

% ==> Major Work II
% In Resource Cost Analysis Area, the Reachability-Bound is significant.
% \\
High-quality
service in people's mobile device life does not just rely on the accuracy of service but also the
 the efficiency of the service. 
This brings my attention to another non-functional execution property, the program's resource cost.
% non-functional properties on resource usage, one of the most useful properties concerning mobile devices. 
Suppose we are playing an online game on our smartphones, what do we care about? We care about the performance of the game, in another word, if it runs fast. We care about whether the game crashed due to being out of memory, and so on. Additionally, resource usage is the key for embedded systems or wearable devices. In this sense, the study of the non-functional properties of resource usage is of great practical value. We observe that the non-functional properties of resource usage are usually quantitative, specifying certain bounds on the target resource to guarantee performance. For instance, an update on a mobile app does not significantly slow down the performance of this update and will not use resources exceeding certain bounds specified before.


% We use combinations of 
We combine the
execution-based and static-based
analysis techniques into the new
analysis frameworks to best exploit the benefit of these non-functional properties in data analysis and resource usage. However, to fully utilize the aforementioned benefits of the new analysis frameworks on these non-functional properties, the appropriate implementation is inevitable. 
To this end, we also take one step in algorithmizing the analysis frameworks
% through
% program analysis framework, 
which is used to formalize and statically estimate these properties.

Last but not least, these non-functional properties are not limited to static analysis. 
We are not only interested in programs implementing the specific algorithm, but also willing to study these lower bound and upper bound of the algorithm itself.
% To this end, we use the formal verification method to study standard algorithms such as sorting and searching in a comparison-based computation model.
