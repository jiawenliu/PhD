\section{Solving the CFL Reachability Problem through $\THESYSTEM$ by Reduction}
\label{ch:cfl_reduction}
\subsection{Introduction and Related Work}
\label{subsec:cfl-backgroung}
Finally, based on the study on traditional way of performing data flow and control analysis,
   I identify the similarity between the traditional way of performing data flow and control analysis, and the 
   adaptivity analysis.  
   Specifically I identify the similarity between 
   solving the feasible path problem in the analysis by reducing to CFL-reachability problems,
   and the way of computing the adaptivity in my static analysis framework.
   Motivated by this observation, 
   % I'm insterested
   % the, There are similarity between
   % solving the data flow problem by reducing to CFL-reachability problem,
   % resource analysis through reducing to CFL-reachability problem, 
   I'm interested in showing that
   CFL-reachability problems can be solved by reducing it into my adaptivity analysis framework.

\subsection{Proposed Methodology}
\label{subsec:cfl-methodology}
Based on the paper\cite{Reps98} where Thomas shows 
three program analysis problems 
which can be solved by reducing to CFL-reachability problem, I will follow the same idea and show the reduction
in following steps.
\begin{itemize}
   \item For the same three program analysis problems, each of them 
   can be solved through  my \emph{adaptivity} analysis framework with 
   a different generalization of $\THESYSTEM$ with higher accuracy.
   \begin{itemize}
      \item Interprocedural Dataflow Analysis.
      \\
      The key problem in the interprocedural data-flow analysis is to identify the feasible path and remove 
      the in-feasible one from the graph.
      This problem can be solved by applying the first two procedure $\THESYSTEM$ and modify the weight
      by $0$ and $1$ representing feasible or  not. In this way, each qualified finite walk in 
      $\THESYSTEM$ represents a feasible path, i.e., the feasible path is just a special case of finite walk 
      with weight $0$ or $1$.
      \item Interprocedural Program Slicing.
      \\
      There are two key problems here and each of them can be solved by $\THESYSTEM$ as follows,
      \\ 
      1. The first key problem to be solved is to identify the data dependency relations between variables. 
      This can be solved by applying the first procedure of $\THESYSTEM$.
      \\
      2.  Similar as the  interprocedural data-flow analysis, the second problem is to identify the feasible path and remove 
      the in-feasible one from the graph is the other key problem. 
      This can be solved in the same way by modifying the second step of by applying the first two procedure $\THESYSTEM$,
      where the weight is reduced to either $0$ and $1$ representing feasible or not.
      \item Shape Analysis.
      \\
      This problem can be solved by generalizing the in the same way as proposed in Section~\ref{sec:generalization}.
      % \item Flow-Insensitive Points-Analysis
   \end{itemize}
   \item Based on summarizing the common property of the three program analysis problems,
   I will show a reduction from CFL-reachability problem 
   by showing that CFL-reachability problem is a special case of 
   computing the adaptivity. 
\end{itemize}
%  system structure as $\THESYSTEM$,
% by modifying the restriction on finite walk, compute different resource cost for program.

