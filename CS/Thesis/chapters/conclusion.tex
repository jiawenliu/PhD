\chapter{Conclusion}
\label{ch:conclusion}

In this chapter, we conclude this dissertation, mainly studying two quantitative properties of programs, the relative cost of two programs, and the adaptivity of adaptive data analysis programs. The relative cost can be reasoned about by relational cost analysis, and we will review the work of {\Arel} and its implementation {\BIAREL}. Adaptivity is another topic, the work {\ADAPTSYSTEM} focuses on it.  We
review its key novelties and contributions discuss several directions on the future researches. 


 
 \section{Dynamic}
 
 Another quantitative property is the adaptivity of programs. We presented {\ADAPTSYSTEM}, 
 a program analysis algorithm that is useful to provide an upper bound on the adaptivity of data analysis under a specific data analysis model. 
 This estimation can help data analysts to control the generalization errors of their analyses by choosing different algorithmic techniques based on adaptivity. 
 Besides, a key contribution of our works is the formalization of the notion of adaptivity for adaptive data analysis. 

 There are some limitations of this work. 
 \begin{enumerate}
     \item One limitation is that our algorithm may over-estimate the adaptivity of a program, 
     as shown in Section~\ref{sec:adapt-example-over}, due to its path-insensitive nature when analyzing if conditions. 
     The future direction is to use the constraint to track the necessary information and help to predict the path when the control flow diverges.
     \item Another one is that our algorithm now can only analyze concrete bound for loops. 
     The direction of future work is to support dynamic or unbounded loops. 
 \end{enumerate}
 
 \section{Static}
