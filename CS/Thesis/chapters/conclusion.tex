In this chapter, I conclude this dissertation, mainly studying the program analysis area, 
with focus on the \emph{Reachability-Bound} problem
and program analysis of \emph{Adaptivity} for data analysis programs.

\section{Reachability-Bound Problem Analysis}
\label{sec:conclude-reachability}

\section{Adaptivity Analysis}
\label{sec:conclude-adapt}
In the execution-based analysis, I formalize the intuitive notion of \emph{adaptivity} as a quantitative 
   property of programs. This analysis is developed in three steps through different methodologies in each step. 
   \\
	a. The dependency relation between every query, through the methodology of semantic data dependency analysis.
   \\
	b. The dependency quantity analysis, through the methodology of execution-based data reachability bound analysis.
   \\
	c. The adaptivity analysis, based on the two analysis results above, give the formal \emph{adaptivity} model 
   for program.
   \\   
   % I will focus on research on how to define the Adaptivity semantically. 
   % (the Trace, Event, the Dependency relation, Dependency depth in terms of the evaluation times and the Adaptivity)
	In the static-based program analysis, I will design a static program analysis for soundly approximating this quantity.
   In this static program analysis, the program will be analysed in the same 3 aspects as the execution-based analysis 
   while through static program analysis techniques, and a sound estimated result will be given in each aspect as follows.
   \\
	a. The data dependency relation analysis through the static data flow analysis technique.
   \\
	b. The dependency quantity analysis through the static program reachability bound analysis techniques.
   \\
	c. The program adaptivity estimation, through newly designed algorithms based on the results estimated above, 
   computing the adaptivity upper bound soundly 
   and accurately.
   \\
I implement my program analysis and show that it can help to analyze the adaptivity of several concrete data analyses with different adaptivity structures.

% Then, through two observations as follows,
% \\
% 1. traditional program's resource cost analysis they failed to consider the case where the program's cost could decrease 
%  implicitly, 
%  \\
%  2. and 
%  % when there isn't a dependency relation between variables.
%  the resource consumption during the program 
%  execution increases and particularly decreases implicitly in the same way as the program's adaptivity, 
%  % Specifically, in line 5 
%  % where the list is re-written and the heap consumption is decreased implicitly. 
%  % This implicit decrease 
%  % of the cost works exactly the same as program's adaptivity decrease.
%  I'm interested in improving the accuracy of program's general resource cost analysis
%  by generalizing my \emph{adaptivity} analysis framework.
%  %  onto the program's resource cost analysis. 
%  % Use this framework,
%  Through the generalized \emph{adaptivity} analysis framework.
%  I will give
%  a more accurate resource cost estimation by taking the program's implicit resource cost into consideration, comparing 
%  to the worst case cost analysis in traditional way.
%  For this work, the analysis framework design is expected to be done with the implementation start off before final defense.


%  Finally, based on the study on traditional way of performing data flow and control analysis,
%  I identify the similarity between the traditional way of performing data flow and control analysis, and the 
%  adaptivity analysis.  
%  Specifically I identify the similarity between 
%  solving the feasible path problem in the analysis by reducing to CFL-reachability problems,
%  and the way of computing the adaptivity in my static analysis framework.
%  Motivated by this observation, 
%  % I'm insterested
%  % the, There are similarity between
%  % solving the data flow problem by reducing to CFL-reachability problem,
%  % resource analysis through reducing to CFL-reachability problem, 
%  I'm interested in showing that
%  CFL-reachability problems can be solved by reducing it into my adaptivity analysis framework. 
%  This work is planed to start off before final defense and develop further sophisticated after.