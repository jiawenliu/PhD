% \paragraph{Solving the CFL Reachability Problem through $\THESYSTEM$ by Reduction}
% \label{ch:cfl_reduction}
\subsection*{Background and Motivation}
\label{sec:cfl-backgroung}
From the background in Section~\ref{sec:generalization},
in the works of data flow and control analysis area,
the traditional way of computing the program resource cost is
% Finally, based on the study on the traditional way of performing data flow and control analysis,
by reducing to the CFL-reachability problems.
%
% I identify the similarity between the traditional way of performing data flow and control analysis and the 
%  adaptivity analysis. 
According to this, 
I identify 
% the similarity between the traditional way of performing data flow and control analysis and the 
that there are some similarities the traditional way of estimating the program resource cost and 
the adaptivity.
%  Specifically, I identify the similarity between 
%  solving the feasible path problem in the analysis by reducing 
Specifically, there are similarities between solving the CFL-reachability problems they reduced to,
%  CFL-reachability problems,
 and the way of computing the adaptivity in 
%  my static analysis framework.
the third step of $\THESYSTEM$.
 Motivated by this observation, 
 % I'm Interested
 % the, There are similarities between
 % solving the data flow problem by reducing to CFL-reachability problem,
 % resource analysis through reducing to CFL-reachability problem, 
 I'm interested in showing that
 CFL-reachability problems can be solved by reducing to my adaptivity analysis framework.
 \subsection*{Methodology Overview}
\label{sec:cfl-methodology}
Based on the paper\cite{Reps98} where Thomas shows 
three program analysis problems 
which can be solved by reducing to CFL-reachability problem, I will follow the same idea and show the reduction
in the following steps.
\begin{itemize}
 \item For the same three program analysis problems, each of them 
 can be solved through my \emph{adaptivity} analysis framework with 
 a different generalization of $\THESYSTEM$ with higher accuracy.
 \begin{enumerate}
    \item \textbf{Interprocedural Dataflow Analysis}.
    \\
%  The key problem in the interprocedural data-flow analysis is to identify the feasible path and remove 
%  the in-feasible one from the graph.
%  This problem can be solved by applying the first two procedures $\THESYSTEM$ and modifying the weight
%  by $0$ and $1$ representing feasible or not. In this way, each qualified finite walk-in 
%  $\THESYSTEM$ represents a feasible path, i.e., the feasible path is just a special case of finite walk 
%  with weight $0$ or $1$.
The key problem in the interprocedural data-flow analysis is to identify the feasible path and remove 
the in-feasible one from the graph.
This problem can be solved by applying the first two procedures $\THESYSTEM$ firstly.
Then in the third step of  $\THESYSTEM$, the dependency graph constructed by $\THESYSTEM$ needs to be modified.
Specifically, the weight in that graph will be changed to $0$ and $1$ 
representing feasible or not. 
In this way, each qualified finite walk-in 
$\THESYSTEM$ represents a feasible path.
In the other words, the feasible path is simply a special case of finite walk 
with weight $0$ or $1$.
\item \textbf{Inter-procedural Program Slicing}.
\\
There are two key problems here and each of them can be solved by $\THESYSTEM$ as follows,
\\ 
1. The first key problem to be solved is to identify the data dependency relations between variables. 
This can be solved by applying the first step of $\THESYSTEM$.
\\
2. Similar to the inter-procedural data-flow analysis, 
the second problem is to identify the feasible path and remove 
the in-feasible one from the graph is the other key problem. 
This can be solved in the same way by modifying the third step and applying the first two procedures $\THESYSTEM$,
where the weight is simplified to either $0$ and $1$ representing feasible or not.
%  \\
%  This problem can be solved by generalizing the in the same way as proposed in Section~\ref{sec:generalization}.
%  % \item Flow-Insensitive Points-Analysis
\item \textbf{Shape Analysis}.
 \\
 This problem can be solved by the generalized full-spectrum analysis framework in Section~\ref{sec:generalization}.
  \end{enumerate}
%  \item Based on summarizing the common property of the three program analysis problems,
%  I will show a reduction in CFL-Reachability problem 
%  by showing that the CFL-reachability problem is a special case of 
%  computing the adaptivity. 
\item Based on summarizing the common property of the three program analysis problems,
I will show a reduction from CFL-Reachability problem to the 
adaptivity analysis problem.
Specifically, I will show that the CFL-reachability problem is a special case of 
computing the adaptivity. 
\end{itemize}
%  system structure as $\THESYSTEM$,
% by modifying the restriction on finite walk, compute different resource cost for program.

