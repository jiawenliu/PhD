%
\begin{lem}[Diff Value Dependency Inversion]
	\label{lem:diffval_inv}
\[
	\begin{array}{l}
	\forall c \in \cdom, \trace_1, \trace_2 \in \mathcal{T}, x^i \in \lvar(c) \st 
	\sdiff(\trace_1, \trace_2, x^i) \neq \emptyset \land |\seq(\trace_1, x^i)| = |\seq(\trace_2, x^i)|
	\\ \quad \implies
	\exists \event_1 = (x, i, \_, \_) \in \trace_1, \event_2 = (x, i, \_, \_) \in \trace_2 \st \diff(\event_1, \event_2)
\end{array}
\]
\end{lem}
Proof Summary, by unfolding the Difference Sequence Definition~\ref{def:diffseq} and Value Sequence Definition~\ref{def:vseq}.
\begin{proof}
	Take arbitrary $c \in \cdom$,
%
let $\trace_1, \trace_2 \in \mathcal{T}, x^i\in \lvar(c)$ 
be the traces and variable satisfying 
$(\diamond) ~ \sdiff(\trace_1, \trace_2, x^i) \neq \emptyset 
\land 
(\star)~|\seq(\trace_1, x^i)| = |\seq(\trace_2, x^i)|$.
\\
% By Difference Sequence in Definition~\ref{def:diffseq}, we know $|\seq(\trace_1, x^i)| \neq 0$ or $|\seq(\trace_1, x^i)| \neq 0$.
% \\
% Then, by $|\seq(\trace_1, x^i)| = |\seq(\trace_2, x^i)|$ we know $|\seq(\trace_1, x^i)| = |\seq(\trace_2, x^i)| \geq 1$.
% \\
Then by Definition~\ref{def:diffseq} and the assumptions $(\diamond)$ and $(\star)$,
we know $|\seq(\trace_1, x^i)| = |\seq(\trace_2, x^i)| \geq 1$.
\\
By $\seq$ in Definition~\ref{def:vseq}, we know there is at least an event $(x, i, \_, \_) \in \trace_1$ and 
$(x, i, \_, \_) \in \trace_2$
in order to have $|\seq(\trace_1, x^i)| = |\seq(\trace_2, x^i)| \geq 1$.
\\
In order to show $\diff(\event_1, \event_2)$, it is sufficient to a contradiction 
by assuming 
\\
$(hc)~ \forall \event_1 = (x, i, \_, \_) \in \trace_1, \event_2 = (x, i, \_, \_) \in \trace_2 \st \neg \diff(\event_1, \event_2)$
\\
By $\seq$ in Definition~\ref{def:vseq} and $(hc)$, we know every value $v_1$ in $\seq(\trace_1, x^i)$ and $v_2$ at the same
index position of $\seq(\trace_2, x^i)$
are equivalent to each other.
\\
Then we know $\sdiff(\trace_1, \trace_2, x^i) \neq \emptyset$. This is contradicted to the hypothesis.
\\
Then we know $(hc)$ is false and this Lemma is proved.
%
\end{proof}

\begin{lem}[Diff Control Dependency Inversion]
	\label{lem:diffctl_inv}
	\[
		\begin{array}{l}
			\forall c \in \cdom, \trace_0 \in \mathcal{T}_0(c), \trace_1, \trace_2 \in \mathcal{T}, x^i, y^j \in \lvar(c) \st 
		\dep(x^i, y^j, \trace_1, \trace_2 , \trace_0, c) \land
		% \sdiff(\trace_1, \trace_2, x^i) \neq \emptyset \land 
		|\seq(\trace_1, y^j)| \neq |\seq(\trace_2, y^j)|
		\\ \quad  \implies
		\exists \event_b = (b, i, v, \bullet) \in \trace_1 \st \neg \event_b \in \trace_2
		\land 
		\forall z \in FV(b), \exists l \in \ldom \st \flowsto(z^l, y^j, c)
	\end{array}
	\]
\end{lem} 
Proof Summary:
\\
Proving by using the Inversion Lemmas~\ref{lem:inv_expr}, \ref{lem:inv_expr_gnl}, 
\ref{lem:inv_event}, and \ref{lem:inv_live}, and the \emph{May-Dependency} definition.
%
\begin{proof}
% Induction on $c \in \cdom$,
% %
% let $\trace_0 \in \mathcal{T}_0(c), \trace_1, \trace_2 \in \mathcal{T}, x^i, y^j \in \lvar(c)$ be the traces and variables satisfying 
% $\dep(x^i, y^j, \trace_1, \trace_2 , \trace_0, c) \land
% % \sdiff(\trace_1, \trace_2, x^i) \neq \emptyset \land 
% |\seq(\trace_1, y^j)| \neq |\seq(\trace_2, y^j)|$.
% \\
% In the cases {$c = \clabel{\assign{x}{\expr}}^l$},
% {$c = \clabel{\assign{x}{\query(\qexpr)}}^l$},
% {$c = \clabel{\eskip}^l$}, 
% and {$c = \clabel{\efun}^l: f(r^{l_r}, x_1, \ldots, x_n) := c$}, 
% we cannot derive $|\seq(\trace_1, y^j)| \neq |\seq(\trace_2, y^j)|$. 
% \\
% Then we have contradiction and the Lemma is vacuously true.
% \\ 
% In the case $c = \ewhile [b]^l \edo c_w$, $c = \eif(\clabel{b}^l, c_t, c_f)$,
% will.
% \\
% \todo{drop the above way}
% \\
Take arbitrary $c \in \cdom$,
%
let $\trace_0 \in \mathcal{T}_0(c), \trace_1, \trace_2 \in \mathcal{T}, x^i, y^j \in \lvar(c)$ be the traces and variables satisfying 
$(\diamond)~\dep(x^i, y^j, \trace_1, \trace_2 , \trace_0, c) 
\land
% \sdiff(\trace_1, \trace_2, x^i) \neq \emptyset \land 
(\star)~|\seq(\trace_1, y^j)| \neq |\seq(\trace_2, y^j)|$.
\\
To show 
$(c1)~\exists \event_b = (b, i, v, \bullet) \in \trace_1 \st \neg \event_b \in \trace_2
\land 
(c2)~\forall z \in FV(b), \exists l \in \ldom \st \flowsto(z^l, y^j, c)$
\\
Splitting the conjunction, there are two sub conclusions need to be proved:
\\
\textbf{Subproof of $\boldsymbol{(c1)~\exists \event_b = (b, i, v, \bullet) \in \trace_1 \st \neg \event_b \in \trace_2}$}:
\\
To show $(c1)$
% $(c1)~\exists \event_b = (b, i, v, \bullet) \in \trace_1 \st \neg \event_b \in \trace_2$, 
it is sufficient to show a contradiction by assuming 
\\
$(hc)~\neg (\exists \event_b = (b, i, v, \bullet) \in \trace_1 \st \neg \event_b \in \trace_2)$.
\\
By $(hc)$, we know there are two cases
\\
(1) $\forall \event \in \trace_1 \st \event \in \eventset^{\asn}$.
\\
(2) $\forall \event_b = (b, i, v, \bullet) \in \trace_1 \st 
\event_b \notin \trace_2$.
%
\subcaseL{(1) $\forall \event \in \trace_1 \st \event \in \eventset^{\asn}$}
By Event Inversion Lemma~\ref{lem:inv_event} on every $\event \in \trace_1$, we know program $c$ has the following form
\\
$c =  \clabel{\eskip}^*;\clabel{\assign{x_1}{\expr_1 / \query(\qexpr_1)}}^{l_1}; \clabel{\eskip}^*; \ldots; \clabel{\eskip}^* $, i.e., $c$ consists of 
only assignment and $\eskip$ commands.
\\
Then by the \emph{May-Dependency} in Definition~\ref{def:var_dep} and 
determinism of evaluation,
we know $\forall \trace_0' \in \mathcal{T}_0(c)$
\\
$
\config{{c}, \vtrace_0} \rightarrow^{*} 
% \config{{c}_1, \vtrace_0' \tracecat [\event_1]}  \rightarrow^{*} 
  \config{\clabel{\eskip}^l, \vtrace_0  \tracecat \trace_1 } 
  \land 
  \config{{c}, \vtrace_0'} \rightarrow^{*} 
\config{\clabel{\eskip}^l, \vtrace_0'  \tracecat \trace_2} 
\implies
|\seq(\trace_1, y^j)| = |\seq(\trace_2, y^j)|$.
\\
This is contradicted to the condition $(\star)~|\seq(\trace_1, y^j)| \neq |\seq(\trace_2, y^j)|$.
% \\%
\subcaseL{(2) $\forall \event_b = (b, i, v, \bullet) \in \trace_1 \st 
\neg \event_b \notin \trace_2$}
% \todo{rewrite}
By Event Inversion Lemma~\ref{lem:inv_event} on every $\event \in \trace_1$, 
\\
Since $\neg \event_b \notin \trace_2$, 
by the
determinism of evaluation, we know the two executions are executing the same program.
\\
In the same way by the \emph{May-Dependency} in Definition~\ref{def:var_dep} and 
we know 
$\forall \trace_0' \in \mathcal{T}_0(c)$
\\
$
\config{{c}, \vtrace_0} \rightarrow^{*} 
% \config{{c}_1, \vtrace_0' \tracecat [\event_1]}  \rightarrow^{*} 
  \config{\clabel{\eskip}^l, \vtrace_0  \tracecat \trace_1 } 
  \land 
  \config{{c}, \vtrace_0'} \rightarrow^{*} 
\config{\clabel{\eskip}^l, \vtrace_0'  \tracecat \trace_2} 
\implies
|\seq(\trace_1, y^j)| = |\seq(\trace_2, y^j)|$.
\\
This is contradicted to the condition $(\star)~|\seq(\trace_1, y^j)| \neq |\seq(\trace_2, y^j)|$.
\\
Then we have $(hc)$ is false, and $(c1)$ is proved.
\\
% the executions,
Let $\event_b = (b, i, v, \bullet) \in \trace_1$ be the testing event such that 
$\neg \event_b \in \trace_2$.
\\
By the $\lvar(c)$ in Definition~\ref{def:lvar}, we know 
$\forall z \in FV(b)$, either $z$ is input variable, we have $z^{\lin} \in \lvar(c)$,
or $z$ is assigned in $c$, we have $\exists r \in \mathbb{N} \st z^r \in \lvar(c)$.
\\
Let $r$ be the label for every $z \in FV(b)$, we prove the second sub-conclusion
$(c2)$ as follows.
\\
\textbf{Subproof of $ \boldsymbol{\forall z \in FV(b), \exists l \in \ldom \st \flowsto(z^l, y^j, c)}$}
% To show $(c2)~\forall z \in FV(b), \exists l \in \ldom \st \flowsto(z^l, y^j, c)$,
% we know from $(c1)$ such that there exists $\event_b = (b, i, v, \bullet) \in \trace_1 \st \neg \event_b \in \trace_2$.
\\
By $(\diamond)$ and $(\star)$,
 we know there exists at least an event $\event_y = (y, j, \_, \_)$ such that
 $ \event_y \in \trace_1$ or $\event_y \in \trace_2$.
 \\
 Without loss of generalization, we assume $\event_y \in \trace_1$ 
%  and another event $\event_y' = (y, j, \_, \_) \in \trace_2$
 (The case of $\event_y \notin \trace_1$ and $\event_y \in \trace_2$ will be proved as a symmetric case of this assumption).
 \\
 Without loss of generalization, let $\event_b$ be the testing event, 
 and $\trace_1^1, \trace_1^2, \trace_1^3, \trace_2^1, \trace_2^2 \in \mathcal{T}$ be traces
 such that 
 \\
 $\trace_1 = \trace_1^1 \tracecat [\event_b] \tracecat  \trace_1^2 \tracecat [\event_y] \tracecat \trace_1^3$,
%  \\
and $\trace_2 = \trace^1_2 \tracecat [\neg\event_b] \tracecat  \trace^2_2$.
% \\
% By the evaluation rules, we know the commands evaluated after $[\event_b]$ and $[\neg\event_b]$ are different.
% % \\
% and $\tlabel(\trace_1^2 \tracecat [\event_y]) \cap \tlabel(\trace^2_2) = \emptyset$.
 \\
Then by Inversion Lemma~\ref{lem:inv_event} on 
$\event_y$, and $\event_b$,
we have the following instance of the first execution in Definition~\ref{def:var_dep},
% Let $\event_{ih} = (b, l_b, n_b, v_b)$, by Eq.~\ref{eq:ctldep_inv1} and {Inversion Lemma~\ref{lem:inv_test}}, we have:
\begin{equation}
	\label{eq:ctldep_exe1}
	\begin{array}{l}   
\config{{c}, \vtrace_0} \rightarrow^{*} 
\config{\eif ([b]^{l_b}, c_t, c_f) / \ewhile [b]^{l_b} \edo c_w;{c}_3', 
\vtrace_0 \tracecat \trace_1^1} 
\\
\qquad 
\rightarrow^{\rname{if-b / while-b}} 
\config{(c_t;c' / c_f;c') /(c_w; \ewhile [b]^{l_b} \edo c_w;c'/[\eskip]; c'), 
\vtrace_0 \tracecat \trace_1^1 \tracecat [\event_b]} 
\\
\qquad  \rightarrow^{*} 
\config{[\assign{y}{\expr_2 / \query(\qexpr_2)}]^{j}; c'', 
\vtrace_0 \tracecat \trace_1^1 [\event_b] \tracecat  \trace_1^2}
\\ \qquad \rightarrow^{\rname{assn/query}}
\config{ c'', 
\vtrace_0 \tracecat \trace_1^1 [\event_b] \tracecat  \trace_1^2 \tracecat [\event_y]}
\rightarrow^{*} 
\config{\eskip, 
\vtrace_0 \tracecat \trace_1^1 [\event_b] \tracecat  \trace_1^2 \tracecat [\event_y] \tracecat \trace_1^3}
% 
\end{array}
\end{equation}
%  %
, where 
% $\trace_1 = \trace_1^1 [\event_b] \tracecat  \trace_1^2 \tracecat [\event_y] \tracecat \trace_1^3$,
% and 
$\eif ([b]^{l_b}, c_t, c_f) / \ewhile [b]^{l_b} \edo c_w$ 
is the conditional command of the assignment commands associated to the 
$\event_b$ by applying Inversion Lemma~\ref{lem:inv_event} on $\event_b$.
\\
The notation $(c_t;c' / c_f;c') /(c_w; \ewhile [b]^{l_b} \edo c_w;c'/[\eskip]; c')$ represents:
\\
In case of $\eif ([b]^{l_b}, c_t, c_f)$, if $\pi_3(\event_b) = \etrue$, we have the evaluation:
$$
\config{\eif ([b]^{l_b}, c_t, c_f) ;{c}', 
\vtrace_1 \tracecat [\event_1] \tracecat \trace} 
\rightarrow^{\rname{if-b}} 
\config{c_t;c' 
\trace_1 \tracecat [\event_1] \tracecat \trace \tracecat [\event_b]} 
$$
%
The same for case of $\eif ([b]^{l_b}, c_t, c_f)$ with $\pi_3(\event_b) = \efalse$,
and case of $\ewhile [b]^{l_b} \edo c_w$ with $\pi_3(\event_b) = \etrue$ and $\pi_3(\event_b) = \efalse$.
%
\\
By applying Inversion Lemma~\ref{lem:inv_event} on $\event_b$, 
and the command label consistency,
we also have the instance of second execution from Definition~\ref{def:var_dep} as follows,
\begin{equation}
\label{eq:ctldep_exe2}
\begin{array}{l}   
	\config{{c}, \vtrace_0'} \rightarrow^{*} 
	\config{\eif ([b]^{l_b}, c_t, c_f) / \ewhile [b]^{l_b} \edo c_w;{c}_3', 
	\vtrace_0' \tracecat \trace^1_2} 
	\\
	\qquad 
	\rightarrow^{\rname{if-b / while-b}} 
	\config{(c_t;c' / c_f;c') /(c_w; \ewhile [b]^{l_b} \edo c_w;c'/[\eskip]; c'), 
	\vtrace_0' \tracecat \trace^1_2 \tracecat [\neg\event_b]} 
	\\
	\qquad  \rightarrow^{*} 
	\config{ \clabel{\eskip}^{l'}
	\vtrace_0' \tracecat \trace^1_2 \tracecat [\neg\event_b] \tracecat  \trace^2_2}
% 
\end{array}
\end{equation}
%
% where $\trace_2 = \trace^1_2 \tracecat [\neg\event_b] \tracecat  \trace^2_2$.
% \\
Then for every $z \in FV(b)$ with label $r$ such that $z^r \in \lvar(c)$,
to show $\flowsto(z^r, y^j, c)$,
by Definition~\ref{def:feasible_flowsto},
it is sufficient to show:
\\
in the case of $\ewhile [b]^{l_b} \edo c_w$, $y^j \in \lvar(c_w)$ ;
\\
in the case of $\eif ([b]^{l_b}, c_t, c_f)$, $y^j \in \lvar(c_t)$ or $y^j \in \lvar(c_f)$ .
% \\
\subcaseL{$\eif ([b]^{l_b}, c_t, c_f) \land \pi_3(\event_b) = \etrue$}
In this case, we have the following execution instances for executions i
n Equation~\ref{eq:ctldep_exe1} and \ref{eq:ctldep_exe1}.
\[
	\begin{array}{l}   
\config{{c}, \vtrace_0} \rightarrow^{*} 
\config{\eif ([b]^{l_b}, c_t, c_f);{c}', 
\vtrace_0 \tracecat \trace_1^1} 
% \\
% \qquad 
\rightarrow^{\rname{if-b}} 
\config{c_t;c', \vtrace_0 \tracecat \trace_1^1 \tracecat [\event_b]} 
% \\
% \qquad  
\rightarrow^{*} 
\config{[\assign{y}{\expr_2 / \query(\qexpr_2)}]^{j}; c'', 
\vtrace_0 \tracecat \trace_1^1 \tracecat[\event_b] \tracecat  \trace_1^2}
\\ \qquad \rightarrow^{\rname{assn/query}}
\config{ c'', 
\vtrace_0 \tracecat \trace_1^1 \tracecat[\event_b] \tracecat  \trace_1^2 \tracecat [\event_y]}
\rightarrow^{*} 
\config{\eskip, 
\vtrace_0 \tracecat \trace_1^1\tracecat [\event_b] \tracecat  \trace_1^2 \tracecat [\event_y] \tracecat \trace_1^3}
% 
\\ 
	\config{{c}, \vtrace_0'} \rightarrow^{*} 
	\config{\eif ([b]^{l_b}, c_t, c_f);{c}', 
	\vtrace_0' \tracecat \trace^1_2} 
	% \\
	% \qquad 
	\rightarrow^{\rname{if-b}} 
	\config{c_f;c', 
	\vtrace_0' \tracecat \trace^1_2 \tracecat [\neg\event_b]} 
	% \\ \qquad 
	 \rightarrow^{*} 
	\config{ \clabel{\eskip}^{l'}
	\vtrace_0' \tracecat \trace^1_2 \tracecat [\neg\event_b] \tracecat  \trace^2_2}
% 
\end{array}
\]
Then, we know $\trace_1^2 \tracecat [\event_y] \tracecat \trace_1^3$ corresponds to evaluation of $c_t;c'$
and 
$\trace^2_2$ corresponds to evaluation of $c_f;c'$ and $\event_y \notin \trace^2_2$.
\\
If $\event_y$ generated from evaluation of $c_t$, we know $y^j \in \lvar(c_t)$ and this case is proved.
\\
If $\event_y$ generated from evaluation of $c'$, since $\event_y \notin \trace^2_2$ and $\trace^2_2$ also contains evaluation of $c'$.
\\
Then there must be another $\eif$ or $\ewhile$ command in $c'$ such that $\event_y$ is generated in the first execution 
but isn't evaluated in the second one.
\\
In that case, there is another $\event_b' \in \trace_1^2$ and $\neg\event_b' \in \trace^2_2$ satisfying the same condition as $\event_b$.
\\
Then we can always choose the $\event_b$ be this $\event_b'$ and choose $\trace_1^1, \trace_1^2, \trace_1^3, \trace_2^1, \trace_2^2 \in \mathcal{T}$ be traces
such that 
\\
$\trace_1 = \trace_1^1 \tracecat [\event_b] \tracecat  \trace_1^2 \tracecat [\event_y] \tracecat \trace_1^3$,
%  \\
and $\trace_2 = \trace^1_2 \tracecat [\neg\event_b] \tracecat  \trace^2_2$.
% \\
% \\
and $\tlabel(\trace_1^2 \tracecat [\event_y]) \cap \tlabel(\trace^2_2) = \emptyset$.
% By 
\\
Then, since $\tlabel(\trace_1^2 \tracecat [\event_y]) \cap \tlabel(\trace^2_2) = \emptyset$, 
we know 
% $\tlabel(\trace_1^2 \tracecat [\event_y])$ doesn't contain any 
% label of the program $c'$.
% \\
% Then we know $\event_y$ isn't generated from evaluating $c'$, i.e., it 
$\event_y$ is generated 
by only executing $c_t$.
\\
Then we know $y^j \in \lvar(c_t)$.
\\
This case is proved.
\\
The \textbf{sub-cases:} 
{$\eif ([b]^{l_b}, c_t, c_f) \land \pi_3(\event_b) = \efalse$},
$\ewhile [b]^{l_b} \edo c_w\land \pi_3(\event_b) = \etrue$, 
and $\ewhile [b]^{l_b} \edo c_w \land \pi_3(\event_b) = \etrue$ 
are proved in the same way.
% By the condition where $|\seq(\trace_1, y^j)| \neq |\seq(\trace_2, y^j)|$, 
% we know 
% \\
%  $|\seq(\trace_1^1 [\event_b] \tracecat  \trace_1^2 \tracecat [\event_y] \tracecat \trace_1^3, y^j)| \neq 
%  |\seq(\trace^1_2 \tracecat [\neg\event_b] \tracecat  \trace^2_2, y^j)|$.
%  \\
%  Then, we know either 
%  $|\seq(\trace_1^2 \tracecat [\event_y] \tracecat \trace_1^3, y^j)| \neq 
%  |\seq(\trace^2_2, y^j)|$.
% \\
% Then, by Event Inversion Lemma~\ref{lem:inv_event} and the program label consistency, we know the
% command $[\assign{y}{\expr_2 / \query(\qexpr_2)}]^{j}$ is evaluated different times in the two executions.
% \\
% Since $\neg\event_b$ will change whether a command is evaluated or not, we know 
% \\
% in the case of $\eif ([b]^{l_b}, c_t, c_f)$, $\event_y$ comes from the evaluation of $c_t$ or $c_f$,
% i.e., $[\assign{y}{\expr_2 / \query(\qexpr_2)}]^{j} \in_c c_t$ or $c_f$, i.e., $y^j \in \lvar(c_t)$ or $y^j \in \lvar(c_f)$;
% \\
% and in the case of $\ewhile [b]^{l_b} \edo c_w$, $\event_y$ comes from the evaluation of $c_w$,
% i.e., $[\assign{y}{\expr_2 / \query(\qexpr_2)}]^{j} \in_c c_w$, and $y^j \in \lvar(c_w)$.
% \\
% In both of the two cases, we know $\forall z \in VAR(\pi_1(\event_b)) $ there is a label $l \in \mathbb{N}$ for this variable,
% and by the $\flowsto$ definition, 
% $\flowsto(z^l, \pi_1(\event_2)^{\pi_2(\event_2)},c)$.
\\
This Lemma is proved.
% By the label consistency, and $\tlabel_{\trace_3} \cap \tlabel_{\trace_3'} = \emptyset$, 
% % i.e., 
% % $ \trace_{3a} \tracecat [\event_2] \tracecat \trace_{3b} \cap \tlabel_{\trace_3'} = \emptyset$
% we know $\trace_3$ and $\trace_3'$ doesn't contain any event of evaluating the commands in $c_3'$.
% Otherwise, $\tlabel_{\trace_3} \cap \tlabel_{\trace_3'} \neq \emptyset$, which is a contradiction.
% \\
% Since $\trace_3= \trace_{3a} \tracecat [\event_2] \tracecat \trace_{3b} $, we know $\event_2$ doesn't comes from evaluating
% of $c_3'$, i.e.,:
% \\
% In the case of $\eif ([b]^{l_b}, c_t, c_f)$, $\event_2$ comes from the evaluation of $c_t$ or $c_f$,
% i.e., $[\assign{{x}_2}{\expr_2 / \query(\qexpr_2)}]^{l_2} \in_c c_t$ or $c_f$;
% \\
% and in the case of $\ewhile [b]^{l_b} \edo c_w$, $\event_2$ comes from the evaluation of $c_w$,
% i.e., $[\assign{{x}_2}{\expr_2 / \query(\qexpr_2)}]^{l_2} \in_c c_w$.
% \\
% In both of the two cases, we know $\forall z \in VAR(\pi_1(\event_b)) $ there is a label $l \in \mathbb{N}$ for this variable,
% and by the $\flowsto$ definition, 
% $\flowsto(z^l, \pi_1(\event_2)^{\pi_2(\event_2)},c)$.
% \\
% This lemma is proved.
\end{proof}
	%
%
\begin{lem}[While Loop Inversion]
	\label{lem:inv_while}
	For every $\trace, \trace' \in \mathcal{T}, c, c_1, c_2 \in \cdom$ 
	if $ \config{c, \trace} \rightarrow^* \config{c_1; c_2, \trace'}$ and 
	$c_1 \in_c c_2$, 
	then there must exist a $\ewhile$ command in $c_2$ and $c_1$ must shows up in the body of that $\ewhile$ command,
	 i.e., $\exists l \in \mathbb{N}, b \in \mathcal{B}, c_w \in \cdom \st 
	(\ewhile [b]^l \edo c_w) \in_c c_2 \land c_1 \in_c c_w$.
	%
	\[
	\begin{array}{l}
	\forall \trace, \trace' \in \mathcal{T}, c, c_1, c_2 \in \cdom \st
		\\ \quad
		\config{c, \trace} \rightarrow^* \config{c_1; c_2, \trace'}
		\implies
		c_1 \in_c c_2
		\implies
		\exists l \in \mathbb{N}, b \in \mathcal{B}, c_w \in \cdom \st 
		(\ewhile [b]^l \edo c_w) \in_c c_2 \land c_1 \in_c c_w
	\end{array}
	\]
	\end{lem}	
	Proof Summary: trivially by induction on $c$ and enumerate all operational semantic rules.
\begin{proof}
	Take arbitrary $\trace \in \mathcal{T}$, by induction on $c$, we have following cases:
		\caseL{$c = [\assign{x}{\expr}]^l$}
		By the evaluation rule $\rname{assn}$, we have
		$
		{
		\config{[\assign{{x}}{\aexpr}]^{l},  \trace } 
		\xrightarrow{} 
		\config{\eskip, \trace \tracecat [({x}, l, v) ]}
		}$.
		\\
		Since there doesn't exist $c_1, c_2 \in \cdom$ satisfying $\eskip = c_1; c_2$, this theorem is vacuously true.
		\caseL{$c = [\assign{x}{\query(\qexpr)}]^l$}
		By the evaluation rule $\rname{query}$, we have
		$
		{
		\config{[\assign{{x}}{\query(\qexpr)}]^{l},  \trace } 
		\xrightarrow{} 
		\config{\eskip, \trace \tracecat [({x}, l, \qval, v) ]}
		}$.
		\\
		Since there doesn't exist $c_1, c_2 \in \cdom$ satisfying $\eskip = c_1; c_2$, this theorem is vacuously true.
		\caseL{$c = \eif([b]^{l}, c_1, c_2)$}
		By the evaluation rule $\rname{query}$ and $\rname{if-f}$, and the label consistency, we know:
		\\
		for all possible $c_{t1}$ and $c_{t2}$ 
		such that $c_t$ has the form $c_t = c_{t1};c_{t2}$;
		\\
		all possible $c_{f1}$ and $c_{f2}$ 
		such that $c_f$ has the form $c_f = c_{f1};c_{f2}$,
		\\
		$c_{t1} \notin c_{t1}$ and $c_{f1} \notin c_{f2}$.
		\\
		Then this theorem is vacuously true.
		\caseL{$c = c_{s1};c_{s2}$}
		By label consistency, we know for every $c_1' \in_c c_{s1}$, $c_1' \notin c_{s2}$.
		\\
		Then by the induction hypothesis on $c_{s1}$ and $c_{s2}$ separately, we have this case proved.
		\caseL{$\ewhile [b]^{l} \edo c$}
		By rule $\rname{while-t}$, we have:
		\[
			\config{{\ewhile [b]^{l} \edo c_w, \trace}}
			\xrightarrow{} 
			\config{{
			c_w; \ewhile [b]^{l} \edo c_w,  \eskip),
			\trace \tracecat [\event]}}
		\]
		%
		If $c_w$ is a sequence command,
		let $c_1 = c_{w1}$ be the any possible command in this sequence, for all possible $c_{w1}$ and $c_{w2}$ 
		such that $c_w$ has the form $c_w = c_{w1};c_{w2}$.
		\\
		Then we have $c_2 = c_{w2};\ewhile [b]^{l} \edo c_w,  \eskip)$ and $c_1 \in_c c_2$.
		\\
		And we also have the existence of $l = l_b, b$ and $c_w$, and $\ewhile [b]^{l} \edo c_w \in_c c_2$ and  $c_1 \in c_w$.
		\\
		If $c_w$ isn't a sequence command, let $c_1 = c_w$, then we have $c_2 = \ewhile [b]^{l} \edo c_w,  \eskip)$ 
		and $c_1 \in_c c_2$.
		\\
		And we also have the existence of $l = l_b, b$ and $c_w$, and $\ewhile [b]^{l} \edo c_w \in_c c_2$ and  $c_1 \in c_w$.
		\\
		This case is proved.
		\\
		By the evaluation rule $\rname{while-f}$, we have
		$
		{
			\config{{\ewhile [b]^{l}, \edo c_w, \trace}}
			\xrightarrow{} 
			\config{{
			[\eskip]^l ,
			\trace \tracecat [((b, l, \efalse))]}}
		}$.
		\\
		Since there doesn't exist $c_1, c_2 \in \cdom$ satisfying $\eskip = c_1; c_2$, this theorem is vacuously true.
	\end{proof}
%
%
\begin{lem}[Only $\eskip$ Command doesn't Produce Event].
	\label{lem:inv_skip}
	For all trace $\trace\in \mathcal{T}$, and $c, c' \in \cdom$,  
	$\config{c, \trace} \rightarrow \config{c', \trace}$ if and only if $c = [\eskip];c'$. 
	\[
		\forall \trace\in \mathcal{T}, c, c' \in \cdom \st
		\config{c, \trace} \rightarrow \config{c', \trace}
		\Leftrightarrow 
		c = [\eskip];c'
	% \footnote{$([\eskip];){}^*$ denotes a sequence command only composed of $[\eskip]$ commands.}
	\]
	\end{lem}
\begin{proof}
	Proved trivially by induction on $c$ and enumerate all operational semantic rules.
\end{proof}
%
