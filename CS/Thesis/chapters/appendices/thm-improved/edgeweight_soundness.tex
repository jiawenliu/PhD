\begin{thm}[Soundness of the Edge Weights Estimation]
    \label{thm:edgeweight_soundness}
  Given a program ${c}$ with its program-based dependency graph 
  $\progG(c) = (\progV, \progE)$,
  % $\traceG = (\traceV, \traceE, \traceW, \traceF)$, 
  we have:
    %
    \[
		\begin{array}{l}
			\forall c \in \cdom 
			% , (v, n) \in \mathcal{VAR} \times \mathbb{N} \times \mathbb{N}
			 \sthat   
			%  \\ \quad
			 \progG({c}) = (\progV, \progE)
			\land 
			\traceG({c}) = (\traceV, \traceE)
			\\ \quad
			\implies
			% \forall v \in \traceV \sthat   
			% v \in \progV \land
			% \traceW(v) \leq \progW(v)
			\forall (v_1, w^p, v_2) \in \traceE,
			(v_1, w^t, v_2) \in \progE, 
			\trace_0 \in \mathcal{T}_0(c), 
			\trace' \in \mathcal{T}, v \in \mathbb{N} \sthat  
			\\ \quad
			\config{{c}, \trace_0} \to^{*} \config{\eskip, \trace\tracecat\vtrace'} 
			\land 
			\config{w^{p}, \trace_0} \earrow v
			\implies
			% \right\} 
			w_{t}(\trace) \leq v
		\end{array}
		\]
  \end{thm}
%
\begin{proof}
  Taking an arbitrary a program ${c}$ with its program-based dependency graph 
  $\progG(c) = (\progV, \progE)$, 
  and an arbitrary pair of labeled variable and weights $(x^l, w) \in \progV$, 
  and arbitrary $\vtrace, \trace' \in \mathbb{T},
  v \in \mathbb{N}$ satisfying
  \\
  % \max \left\{ 
    % \vcounter(\vtrace') l ~ \middle\vert~
  % \forall \vtrace, \trace' \in \mathcal{T} \sthat   
  $\config{{c}, \trace} \to^{*} \config{\eskip, \trace\tracecat\vtrace'} 
  \land 
  \config{\trace, w} \earrow v$
  %  labelled variable $x^l \in \lvar_c$.
  \\
  By Definition of $\progW$ in $\progG(c)$, we know 
  $  w = \absW(l) = \max \{ \absclr(\absevent) | \absevent = (l, \_, \_)\}$.
  \\
  By Lemma~\ref{lem:abscfg_sound}, there exists an abstract event in $\absflow(c)$ of form $(\absevent) = (l, \_, \_)$,
  corresponding to the assignment command associated to labeled variable $x^l$. 
  \\
  Let $(\absevent) = (l, dc, l') \in \absflow(c)$ be this event for some $dc$ and $l'$ such that  $(\absevent) = (l, dc, l') \in \absflow(c)$,
  by the last step of phase 2, we know
  $
  \progW(x^l) 
  \triangleq \absclr(\absevent)
  $.
   Then, it is sufficient to show:
  \[
  %   \max \left\{ \vcounter(\vtrace') l ~ \middle\vert~
  % \forall \vtrace \in \mathcal{T} \sthat   \config{{c}, \trace} \to^{*} \config{\eskip, \trace\tracecat\vtrace'} \right\} 
  % \leq 
  \forall v \in \mathbb{N} \sthat   
  \config{\absclr(\absevent), \trace} \earrow 
  \vcounter(\vtrace', l) \leq v
  \absclr(\absevent)
  \]
  % By line:2 of Algorithm~\ref{alg:add_weights}, there are 2 cases:
  By definition of $\absclr(\absevent)$:
  \[
 \begin{array}{ll}
  \locbound(\absevent) & \locbound(\absevent) \in \constdom \\
  Incr(\locbound(\absevent)) + 
  \sum\{\absclr(\absevent') \times \max(\varinvar(a) + c, 0) | (\absevent', a, c) \in \reset(\locbound(\absevent))\} 
  & \locbound(\absevent) \notin \constdom
\end{array}
\]
  \caseL{$\locbound(\absevent) \in \constdom$}
  \\
  Proved by the soundness of Local bound in Lemma~\ref{lem:local_bound_sound}.
  \caseL{$\locbound(\absevent) \notin \constdom$}
To show:
\[
  \begin{array}{l}
    \max \left\{ \vcounter(\vtrace') l ~ \middle\vert~
\forall \vtrace \in \mathcal{T} \sthat   \config{{c}, \trace} \to^{*} \config{\eskip, \trace\tracecat\vtrace'} \right\} 
\\
\leq 
Incr(\locbound(\absevent)) + 
\sum\{\absclr(\absevent') \times \max(\varinvar(a) + c, 0) | (\absevent', a, c) \in \reset(\locbound(\absevent))\} 
\end{array}
\]
  % \caseL{$l \in prel$}
  % \\
  Taking an arbitrary initial trace
  $\trace_0 \in \mathcal{T}$, 
  executing $c$ with $\trace_0$, let $\trace$ be the trace after evaluation, i.e., $\config{{c}, \trace_0} \to^{*} \config{\eskip,\vtrace}$, it is sufficient to show:
  \[ 
    \begin{array}{l}
      \vcounter(\vtrace') l \leq 
    Incr(\locbound(\absevent)) + 
    \sum\{\absclr(\absevent') \times \max(\varinvar(a) + c, 0) | (\absevent', a, c) \in \reset(\locbound(\absevent))\}
  \end{array}
  \]
%
 By the soundness of the (1) Transition Bound and (2) Variable Bound Invariant 
 in \cite{sinn2017complexity} Theorem 1, 
This case is proved.
\end{proof}
