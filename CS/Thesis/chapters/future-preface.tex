
% \\
\section{Future Works}
\label{sec:intro-cfl}
\paragraph{Program Resource Cost Analysis}
\label{sec:intro-cost}
% Then, motivated by the two following aspects, I'm interested in improving the accuracy of the program's general resource cost analysis
% by generalizing this \emph{adaptivity} analysis framework.
Moving towards the area of general program resource cost analysis,
% Then, motivated by the two following aspects, 
there are two interesting observations as follows.
% I'm interested 
These two observations motivated me in 
% improving the accuracy of the program's general resource cost analysis
improving the accuracy of the program's general resource cost analysis
by generalizing this full-spectrum \emph{adaptivity} analysis.
\begin{itemize}
 \item Firstly, in a traditional program's resource cost analysis,
 There are two categories of program cost analysis, type-system based and data-flow/control-flow analysis based. 
 In the type-system design-based works, they \cite{GustafssonEL05} and \cite{hoffmann_jost_2022}, explicit abstraction or data structure de-allocation in order to save or reduce the cost.
 
 Both of the
 works in these two areas fail to recognize the case where program resource consumption is decreased implicitly.
 \item The resource consumption during the program 
 execution increases and particularly decreases implicitly in the same way as the program's adaptivity. 
 This is explained in detail through an example in Section~\ref*{sec:generalcost-backgroung}.
\end{itemize}
% F
% Then, through two observations,
% that 
% firstly, traditional program's resource cost analysis failed to consider the case where the program's cost could decrease 
% implicitly, and 
% % when there isn't a dependency relation between variables.
% the resource consumption during the program 
% execution increases and particularly decreases implicitly in the same way as the program's adaptivity, 
% % Specifically, in line 5 
% % where the list is re-written and the heap consumption is decreased implicitly. 
% % This implicit decrease 
% % of the cost works exactly the same as the program's adaptivity decrease.
% I'm interested in improving the accuracy of the program's general resource cost analysis
% by generalizing my \emph{adaptivity} analysis framework.
 % onto the program's resource cost analysis. 
 % Use this framework,
 Based on the observations above, in Chapter~\ref{ch:generalization},
%  Based on the observations above, and through the generalized \emph{adaptivity} analysis framework.
%  I will give
%  a more accurate resource cost estimation by taking the program's implicit resource cost into consideration, compared 
%  to the worst-case cost analysis in the traditional way.
 I develop
 an accurate program general resource cost analysis framework through generalizing my full-spectrum \emph{adaptivity} analysis.
 This framework can give more accurate cost bound than traditional worst-case resource cost estimation methods,
 by taking the program's implicit resource cost into consideration.
% F
% Then, through two observations,
%    that 
%    firstly, traditional program's resource cost analysis they failed to consider the case where the program's cost could decrease 
%    implicitly, and 
%    % when there isn't a dependency relation between variables.
%    the resource consumption during the program 
%    execution increases and particularly decreases implicitly in the same way as the program's adaptivity, 
%    % Specifically, in line 5 
%    % where the list is re-written and the heap consumption is decreased implicitly. 
%    % This implicit decrease 
%    % of the cost works exactly the same as program's adaptivity decrease.
%    I'm interested in improving the accuracy of program's general resource cost analysis
%    by generalizing my \emph{adaptivity} analysis framework.
   %  onto the program's resource cost analysis. 
   % Use this framework,
   % Based on the observations above, and through the generalized \emph{adaptivity} analysis framework.
   % I will give
   % a more accurate resource cost estimation by taking the program's implicit resource cost into consideration, comparing 
   % to the worst case cost analysis in traditional way.

%    \section{Towards Solving the CFL-Reachability Problem}
% Finally, based on the study on traditional way of performing data flow and control analysis,
% I identify the similarity between the traditional way of performing data flow and control analysis, and the 
%    adaptivity analysis.  
%    Specifically I identify the similarity between 
%    solving the feasible path problem in the analysis by reducing to CFL-reachability problems,
%    and the way of computing the adaptivity in my static analysis framework.
%    Motivated by this observation, 
%    % I'm insterested
%    % the, There are similarity between
%    % solving the data flow problem by reducing to CFL-reachability problem,
%    % resource analysis through reducing to CFL-reachability problem, 
%    I'm interested in showing that
%    CFL-reachability problems can be solved by reducing it into my adaptivity analysis framework.

\paragraph*{Solving the CFL Reachability Problem}
Still in the area of general program resource cost analysis,
the traditional methodology of performing data flow and control analysis and 
computing the program resource cost is
% Finally, based on the study on the traditional way of performing data flow and control analysis,
to reduce the analysis problem into the CFL-reachability problems.
% Finally, based on the study on the traditional way of performing data flow and control analysis,
According to this, 
I identify x
% the similarity between the traditional way of performing data flow and control analysis and the 
the similarities between the traditional way of estimating the program resource cost and 
the adaptivity.
%  Specifically, I identify the similarity between 
%  solving the feasible path problem in the analysis by reducing 
Specifically, there are similarities between solving the CFL-reachability problems they reduced to,
%  CFL-reachability problems,
 and the way of computing the adaptivity in 
%  my static analysis framework.
the third step of $\THESYSTEM$.
 Motivated by this, 
 % I'm Interested
 % the, There are similarities between
 % solving the data flow problem by reducing to CFL-reachability problem,
 % resource analysis through reducing to CFL-reachability problem, 
 I'm interested in showing that
 CFL-reachability problems can be solved by reducing to my adaptivity analysis framework.
In Chapter~\ref{ch:furthers}, I started this work and proposed plans on developing this work in the furture.
