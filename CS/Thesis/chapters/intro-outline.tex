Two main parts :
\\
the first part focus on solving the Reachability-Bound Problem through static program analysis.
\\
The second part focus on analyzing the adaptivity -- a quantity property -- for programs in data analysis area,
through both the  static program analysis techniques and execution-based analysis techniques.
\begin{enumerate}
    \item \redd{PART \romannum{1} \quad  REACHABILITY BOUND ANALYSIS}

        \paragraph{Chapter~\ref{sec:reachability-intro}} Introduction with sections:
        
        Section~\ref{sec:reachability-background}: {Reachability Bound Problem}
        
        Section~\ref{sec:reachability-motivation}: {Motivation and Overview}
        
        Section~\ref{sec:reachability-outline}: {Chapter Outline}

        \paragraph{Chapter~\ref{sec:reachability-analysis}}: {Path-Sensitive Reachability Bound Analysis}

        Section~{\ref{sec:language}}{{Program Model}}
        1. {Language}
        2. {Trace-Based Operational Semantics}
        3. {{Reachability Bound Formalization}}

        % % % % 
        Section~\ref{sec:reachability-program_refine}: {Program Abstraction and Refinement}
        % \subsection{Constraint Program Refinement}

        Section~\ref{sec:reachability-analysis}: {Path Sensitive Reachability Bound Analysis}
        1.{Outside-In Algorithm}
        2. {Inside-Out Algorithm}

        \paragraph{Chapter~{\ref{sec:reachability-example}}}: {Examples and Experimental Results}

    \item \redd{PART \romannum{2} \quad  PROGRAM ANALYSIS FRAMEWORK FOR ADAPTIVE DATA ANALYSIS}
    shows the work on the program analysis algorithms to study the adaptivity of the adaptive data analysis program.    

    \paragraph{Chapter~\ref{sec:adapt-intro}} gives the introduction of adaptive data analysis in Section~\ref{sec:adapt-background} and the challenges (Section~\ref{sec:adapt-motivation}) we face to obtain the adaptivity to control the generalization error of an adaptive data analysis program, with an outline of this part in Section~\ref{sec:adapt-outline}.
    
    \paragraph{Chapter~\ref{sec:prework}} introduces the previous works on adaptivity analysis.
    It contains a short summary of the language designs in Section~\ref{sec:prework-language} 
    the adaptivity formalization through program analysis over the trace-based operational semantics (Section~\ref{sec:prework-formalization}),
    and previous program analysis algorithm that is used to estimate the adaptivity of the data analysis programs 
    in Section~\ref{sec:prework-static}
    % (Section~\ref{sec:adapt-ve}, Section~\ref{sec:adapt-matrix}). 
    Their algorithm first introduces the SSA version of the loop language
    % (Section~\ref{sec:adapt-syntax-ssa-loop})
     to enable an easier analysis over adaptivity by an easier tracking of dependency relations between variables based of the limitation of direct analysis over the loop language
    %   is covered in Section~\ref{sec:adapt-limit}. The transformation from the loop language to the SSA loop language is presented in Section~\ref{sec:adapt-transformation}.
    Following with a three-step algorithm, which constructs a data control dependency graph, and add weights to the graph. The adaptivity is estimated by the weight of the path with the highest weight.
    % we use to express data analysis programs, and shows the definition of adaptivity from a trace-based operational semantics in Section~\ref{sec:adapt-os}.


    \paragraph*{Chapter~\ref{sec:adapt-analysis}} presents the new adaptivity analysis framework with significant improvement in three sections.
    \\
    Section~\ref{sec:adapt-language} present a more expressive while-like language than previous works. This extended language supports more general adadptive data analysis.
    \\
    Section~\ref{sec:adapt-exe}  develops an execution-based analysis method which can
    % The second challenge is 
    \emph{define} the intuitive \emph{adaptivity} rounds for a given data analysis program formally and accurately.
    % Intuitively, a query $Q$ may depend on another query $P$, if there are two values that $P$ can return which affect
    This execution-based analysis is designed in three steps. The Section~\ref{sec:dynamic-datadep} analyzes the \emph{dependency relation} between every query, 
    through the methodology of semantic data dependency analysis. In Section~\ref{sec:dynamic-reachability}, it analyzes the \emph{dependency quantity} 
    %  analysis, 
    based on the \emph{dependency relation} above.
    This analysis is developed through the methodology of execution-based reachability bound analysis.
    The last step in Section~\ref{sec:dynamic-adapt} is the intuitive \emph{adaptivity} quantity analysis, 
   according to the two analysis results above, specifically \emph{dependency relation} and \emph{dependency quantity}.
    This step 
  %  is developed through 
    gives the formal \emph{adaptivity} definition as the analysis result. 
    \\
    Section~\ref{sec:adapt-static} develops an improved static program adaptivity analysis framework, named {\THESYSTEM}.
    This analysis combines data flow and control flow analysis with reachability bound analysis.
    % Specifically as follows in the same 
    It is developed in 3 aspects similar to the execution-based adaptivity analysis 
    while through static program analysis techniques. 
    The {\THESYSTEM} analyzes the data \emph{dependency relation} through the static data flow analysis technique
    in Section~\ref{sec:static-dep}.
    In Section~\ref{sec:static-quantity}, the \emph{dependency quantity} 
    is estimated by {\THESYSTEM} through the static program reachability bound analysis techniques corresponding to the second step in execution-based adaptivity analysis.
    In the last step, {\THESYSTEM}
    % , specifically estimating 
    estimates the \emph{adaptivity} formalized through execution-based analysis in Section~\ref{sec:static-adapt},
    %  is presented in Section~\ref{sec:static-reachability}.
    % the program adaptivity estimation, 
    % According to the third step of execution-based adaptivity analysis, 
    % {\THESYSTEM} in this step also 
    via constructing a program-based dependence graph for approximating the execution-based dependency graph.
    

    \paragraph{Chapter~\ref{sec:adapt-implementation}} presents five manual examples demonstrating this framework in Section~\ref{sec:adapt-example},
    and the exeprimental results on some real world data analysis algorithms in Section~\ref{sec:adapt-eval}.
    % It includes a variant of two round data analysis algorithm, an adaptive multiple rounds data analysis algorithm and an example showing the over-approximate of our approach (Section~\ref{sec:adapt-example-over}). 
    
    \paragraph{Chapter~\ref{sec:adapt-relatedwork}} discusses the related works from three perspectives:
    Static program analysis (Section~\ref{sec:relatedwork-static}), dynamic program analysis (Section~\ref{sec:relatedwork-exe}) and generalization in adaptive data analysis (Section~\ref{sec:relatedwork-adapt}).  

    \item \redd{PART \romannum{3} \quad  CONCLUSION and Future Works}
    \paragraph*{Chapter~\ref{sec:conclusion}}
    Concludes the Two Studies.
    \\
    \textbf{Reachability-Bound Analysis}
    \\
    \textbf{Adaptivity Analysis}

    \paragraph*{Chapter~\ref{sec:future}}
    Discusses the future directions on studying the two quantitative properties: relative cost and adaptivity.
    \\
    \textbf{Program Non-Monotonic Resource Cost Analysis}
        Moving towards the area of general program resource cost analysis,
        % Then, motivated by the two following aspects, 
        there are two interesting observations as follows.
        % I'm interested 
        These two observations motivated me in 
        % improving the accuracy of the program's general resource cost analysis
        improving the accuracy of the program's general resource cost analysis
        by generalizing this full-spectrum \emph{adaptivity} analysis.
        \begin{itemize}
        \item Firstly, in a traditional program's resource cost analysis,
        There are two categories of program cost analysis, type-system based and data-flow/control-flow analysis based. 
        In the type-system design-based works, they \cite{GustafssonEL05} and \cite{hoffmann_jost_2022}, explicit abstraction or data structure de-allocation in order to save or reduce the cost.
        
        Both of the
        works in these two areas fail to recognize the case where program resource consumption is decreased implicitly.
        \item The resource consumption during the program 
        execution increases and particularly decreases implicitly in the same way as the program's adaptivity. 
        This is explained in detail through an example in Section~\ref*{sec:generalcost-backgroung}.
        \end{itemize}
        Based on the observations above, in Chapter~\ref{ch:generalization},
        %  Based on the observations above, and through the generalized \emph{adaptivity} analysis framework.
        %  I will give
        %  a more accurate resource cost estimation by taking the program's implicit resource cost into consideration, compared 
        %  to the worst-case cost analysis in the traditional way.
        I develop
        an accurate program general resource cost analysis framework through generalizing my full-spectrum \emph{adaptivity} analysis.
        This framework can give more accurate cost bound than traditional worst-case resource cost estimation methods,
        by taking the program's implicit resource cost into consideration.
            \\
            \textbf{Solving the CFL Reachability Problem}
        Still in the area of general program resource cost analysis,
        the traditional methodology of performing data flow and control analysis and 
        computing the program resource cost is
        % Finally, based on the study on the traditional way of performing data flow and control analysis,
        to reduce the analysis problem into the CFL-reachability problems.
        % Finally, based on the study on the traditional way of performing data flow and control analysis,
        According to this, 
        I identify x
        % the similarity between the traditional way of performing data flow and control analysis and the 
        the similarities between the traditional way of estimating the program resource cost and 
        the adaptivity.
        %  Specifically, I identify the similarity between 
        %  solving the feasible path problem in the analysis by reducing 
        Specifically, there are similarities between solving the CFL-reachability problems they reduced to,
        %  CFL-reachability problems,
        and the way of computing the adaptivity in 
        %  my static analysis framework.
        the third step of $\THESYSTEM$.
        Motivated by this, 
        % I'm Interested
        % the, There are similarities between
        % solving the data flow problem by reducing to CFL-reachability problem,
        % resource analysis through reducing to CFL-reachability problem, 
        I'm interested in showing that
        CFL-reachability problems can be solved by reducing to my adaptivity analysis framework.

    \item \redd{PART \romannum{4} \quad  APPENDIX}
\end{enumerate}

\textbf{Previous Published Materials}