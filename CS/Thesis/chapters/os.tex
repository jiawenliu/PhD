The operational semantics is defined based on the event, trace as well as the 
environment, which are introduced firstly as follows.
\paragraph{Event}
An event tracks useful information about each step of the evaluation, as a quadruple. Its first element is either 
an assigned variable (from an assignment command) or a boolean expression (from the guard of if or while command), follows by 
 the label associated to this event, the value evaluated either from the expression assigned to the variable,
or the boolean expression in the guard.
 The last element stores the query information, which a query value whose default is $\bullet$. I declare event projection operator $\pi_i$ which projects the $i$th element from an event.
\[
\begin{array}{llll}
\mbox{Event} 
& \event & ::= & 
    ({x}, l, v, \bullet) ~|~ ({x}, l, v, \qval)  ~~~~~~~~~~~ \mbox{Assignment Event} 
~|~(\bexpr, l, v, \bullet)   
~~~~
\mbox{Testing Event}
% \mbox{Trace} & \trace
% & ::= & [] ~|~ \trace :: \event
\end{array}
\]

The event projection operators $\pi_i$ projects the $i$th element from an event: 
\\
$\pi_i : 
\eventset \to \mathcal{VAR}\cup \mbox{Boolean Expression}  \cup \mathbb{N} \cup \mathcal{VAL} \cup \mathcal{QVAL} $ 
% \wqside{use b for Boolean expression?}

%
In order to distinguish if a query's choice is affected by previous values, 
% \jl{
% we need to be able to identify whether two queries are the equivalent or not, so that when we change the result of one query, whether another query is affected. For the equivalence of queries, } 
we need to be able to identify whether two queries are the equivalent or not, so that when we change the result of one query, 
whether another query is affected. 
To define equivalence of queries,
quiet different from the equality between the evaluation results as the regular assignment results, 
we are neither observing the syntactic equivalence between the two query expressions,
nor two results return from the database. 
Instead, we define the equivalence of query expression by quantifying over all values returned from database on a certain form of query value, formally as follows.

\begin{defn}[Equivalence of Query Expression]
%
\label{def:query_equal}
% \mg{Two} \sout{2} 
Two query expressions $\qexpr_1$, $\qexpr_2$ are equivalent, denoted as $\qexpr_1 =_{q} \qexpr_2$, if and only if
$$
 \begin{array}{l} 
   \forall \trace \in \mathcal{T} \sthat \exists \qval_1, \qval_2 \in \mathcal{QVAL} \sthat
    (\config{\trace,  \qexpr_1} \qarrow \qval_1 \land \config{\trace,  \qexpr_2 } \qarrow \qval_2) 
    \\
    \quad \land (\forall D \in \dbdom, r \in D \sthat 
    \exists v \in \mathcal{VAL} \sthat 
          \config{\trace, \qval_1[r/\chi]} \aarrow v \land \config{\trace,  \qval_2[r/\chi] } \aarrow v)  
  \end{array}.
$$
 where $r \in D$ is a record in the database domain $D$. 
 As usual, we will denote by $\qexpr_1 \neq_{q} \qexpr_2$  the negation of the equivalence.
\end{defn}
%
\begin{defn}[Event Equivalence]
  Two events $\event_1, \event_2 \in \eventset$ are equivalent, 
  % denoted as $\event_1 \eventeq \event_2$ 
  denoted as $\event_1 = \event_2$ 
  if and only if:
  \[
  \pi_1(\event_1) = \pi_1(\event_2) 
  \land  
  \pi_2(\event_1) = \pi_2(\event_2) 
  \land
  \pi_{3}(\event_1) = \pi_{3}(\event_2)
  \land 
  \pi_{4}(\event_1) =_q \pi_{4}(\event_2)
  \]
  %
  As usual, we will denote by $\event_1 \neq \event_2$  the negation of the equivalence.
  % As usual, we will denote by $\event_1 \eventneq \event_2$  the negation of the equivalence.
  % When it is clear from the context, we omit the subscript $\kw{e}$ and use 
  % $\event_1 = \event_2$ (and $\event_1 \neq \event_2$) for event equivalent
\end{defn}


\begin{defn}[Events Different up to Value ($\diff$)]
  Two events $\event_1, \event_2 \in \eventset are $ \emph{Different up to Value}, 
  denoted as $\diff(\event_1, \event_2)$ if and only if:
  \[
    \begin{array}{l}
  \pi_1(\event_1) = \pi_1(\event_2) 
  \land  
  \pi_2(\event_1) = \pi_2(\event_2) \\
  \land  
  \big(
    (\pi_3(\event_1) \neq \pi_3(\event_2)
  \land 
  \pi_{4}(\event_1) = \pi_{4}(\event_2) = \bullet )
  % \qquad \qquad 
  \lor 
  (\pi_4(\event_1) \neq \bullet
  \land 
  \pi_4(\event_2) \neq \bullet
  \land 
  \pi_{4}(\event_1) \neq_q \pi_{4}(\event_2)) 
  \big)
  \end{array}
  \]
  \end{defn}
% %
\paragraph{Trace}
A trace $\trace \in \mathcal{T} $ is a list of events, 
collecting the events generated along the program execution. $\mathcal{T} $ represents the set of traces. There are some useful operators: the trace concatenation operator $\tracecat: \mathcal{T} \to \mathcal{T} \to \mathcal{T}$, combines two traces.
The belongs to operator $\in : \eventset \to \mathcal{T} \to \{\etrue, \efalse \} $ and its opposite $\not\in$
express whether or not an event belongs to a trace.
Another operator $\llabel : \mathcal{T} \to \mathcal{VAR} \to \{\mathbb{N}\}\cup \{\bot\}$,
takes a trace and a variable as input and returns the label of the latest assignment event which assigns value to that variable. 
% I also have the operator $\tlabel : \mathcal{T} \to \ldom$, which gives the set of labels in every event belonging to a trace. 
% The full definitions of these above operators can be found in the appendix.
\[
\begin{array}{llll}
\mbox{Trace} & \trace
& ::= & [] ~|~ \trace :: \event
\end{array}
\]

A trace can be regarded as the program history, which records queries asked by the analyst during the execution of the program. I collect the trace with a trace-based operational semantics based on transitions of the form $ \config{c, \trace} \to \config{c', \trace'} $. It states that a configuration $\config{c, \trace}$, which consists of a command $c$ to be evaluated and a starting trace $\trace$, evaluates to another configuration with the trace updated along with the evaluation of the command $c$ to the normal form of the command $\eskip$.
\begin{defn}[Trace Concatenation, $\tracecat: \mathcal{T} \to \mathcal{T} \to \mathcal{T}$]
  Given two traces $\trace_1, \trace_2 \in \mathcal{T}$, the trace concatenation operator 
  $\tracecat$ is defined as:
  % \[
    % \trace_1 \tracecat \trace_2 \triangleq
    % \left\{
    % \begin{array}{ll} 
    %   \trace_1 \tracecat [] \triangleq \trace_1 & 
    %   \trace_1 \tracecat (\trace_2' :: \event) \triangleq  (\trace_1  \tracecat \trace_2')  :: \event 
    % \end{array}
    % \right.
  % \]
  % \[
    % \trace_1 \tracecat \trace_2 \triangleq
    % \left\{
    % \begin{array}{ll} 
    %   \trace_1 \tracecat [] \triangleq \trace_1 & 
    %   \trace_1 \tracecat (\trace_2' :: \event) \triangleq  (\trace_1  \tracecat \trace_2')  :: \event 
    % \end{array}
    % \right.
  % \]
  \[
    \trace_1 \tracecat \trace_2 \triangleq
    \left\{
    \begin{array}{ll} 
       \trace_1 & \trace_2 = [] \\
       (\trace_1  \tracecat \trace_2')  :: \event & \trace_2 = \trace_2' :: \event
    \end{array}
    \right.
  \]
  \end{defn}
  %
  % \todo{ need to consider the occurrence times }
  % \\
  % \mg{This definition is not well given. You use a different operator to define it t[:e] which you say it is a shorthand. It cannot be a shorthand because it is used in the definition. I think you need to define two operations, either in sequence or mutually recursive.}
  % \jl{I moved this definition from the main paper into the appendix \ref{apdx:flowsto_event_soundness}. Because this operator is only being used in the soundness proof. And I also feel it doesn't worth to spend many lines in the main paper for defining this complex notation.}
  % Subtrace: $[ : ] : \mathcal{{T} \to \eventset \to \eventset \to \mathcal{T}}$ 
  % \wqside{Confusing, I can not understand the subtraction, it takes a trace, and two events, and this operator is used to subtract these two events?}
  % \[
  %   \trace[\event_1 : \event_2] \triangleq
  %   \left\{
  %   \begin{array}{ll} 
  %   \trace'[\event_1: \event_2]             & \trace = \event :: \trace' \land \event \eventneq \event_1 \\
  %   \event_1 :: \trace'[:\event_2]  & \trace = \event :: \trace' \land \event \eventeq \event_1 \\
  %   {[]} & \trace = [] \\
  %   \end{array}
  %   \right.
  % \]
  % For any trace $\trace$ and two events $\event_1, \event_2 \in \eventset$,
  % $\trace[\event_1 : \event_2]$ takes the subtrace of $\trace$ starting with $\event_1$ and ending with $\event_2$ including $\event_1$ and $\event_2$.
  % \\
  % We use $\trace[:\event_2] $ as the shorthand of subtrace starting from head and ending with $\event_2$, and similary for $\trace[\event_1:]$.
  % \[
  %   \trace[:\event] \triangleq
  %   \left\{
  %   \begin{array}{ll} 
  %  \event' :: \trace'[: \event]             & \trace = \event' :: \trace' \land \event' \eventneq \event \\
  %   \event'  & \trace = \event' :: \trace' \land \event' \eventeq \event \\
  %   {[]}  & \trace = [] 
  %   \end{array}
  %   \right.
  % % \]
  % % \[
  %   \quad
  %   \trace[\event: ] \triangleq
  %   \left\{
  %   \begin{array}{ll} 
  %   \trace'[\event: ]     & \trace =  \event' :: \trace' \land \event \eventneq \event' \\
  %   \event' :: \trace'  & \trace = \event' :: \trace' \land \event \eventeq \event' \\
  %   {[ ] } & \trace = []
  %   \end{array}
  %   \right.
  % \]
  % %
  % \mg{why in the next definition you use ( ) while in the previous ones you didn't? They seem like the same cases. And why you use o.w. instead of []?}
  % An event $\event \in \eventset$ belongs to a trace $\trace$, i.e., $\event \eventin \trace$ are defined as follows:
  % %
  % \begin{equation}
  %   \event \eventin \trace  
  %   \triangleq \left\{
  %   \begin{array}{ll} 
  %     \etrue                  & \trace =  (\event' :: \trace') \land (\event \eventeq \event')
  %                               \\
  %     \event \eventin \trace' & \trace =  (\event' :: \trace') \land (\event \eventneq \event') \\ 
  %     \efalse                 & o.w.
  %   \end{array}
  %   \right.
  % \end{equation}
  \begin{defn}(An Event Belongs to A Trace)
    An event $\event \in \eventset$ belongs to a trace $\trace$, i.e., $\event \in \trace$ are defined as follows:
  %
  \begin{equation}
    \event \in \trace  
    \triangleq \left\{
    \begin{array}{ll} 
      \etrue                  & \trace =  \trace' :: \event'
       \land \event = \event'
                                \\
      \event \in \trace' & \trace =  \trace' :: \event'
      \land \event \neq \event' \\ 
      \efalse                 & \trace = []
    \end{array}
    \right.
  \end{equation}
  As usual, we denote by $\event \notin \trace$ that the event $\event$ doesn't belong to the trace $\trace$.
  \end{defn}
  %
  % An event $\event \in \eventset$ belongs to a trace $\trace$ up to value, 
  % i.e., $\event \sigin \trace$ are defined as follows:
  %   %
  % \begin{equation}
  %   \event \sigin \trace  
  %   \triangleq \left\{
  %   \begin{array}{ll} 
  %     \etrue                  & \trace =  (\trace' \tracecate \event')                          \land \pi_1(\event_1) = \pi_1(\event_2) 
  %                               \land  \pi_2(\event_1) = \pi_2(\event_2)  
  %                               % \land \vcounter(\trace \event) = \vcounter()
  %                               \\
  %     \event \sigin \trace'   & \trace =  (\trace' \tracecate \event') 
  %                               \land 
  %                               (\pi_1(\event_1) \neq \pi_1(\event_2) 
  %                               \lor  \pi_2(\event_1) \neq \pi_2(\event_2)) 
  %                               \\ 
  %     \efalse                 & o.w.
  %   \end{array}
  %   \right.
  % \end{equation}
  %
  % % \mg{Why the previous definition used :: and now you switch to ++? Cannot this just be defined using ::? I am trying to anticipate places where a reader might be confused. Also, this definition would be much simpler if we defined event in a more uniform way. Ideally, we want to distinguish three cases, we don't need to distinguish 7 cases.}\\
  % % \jl{My bad, I was really too sticky to the convention.
  % I though the list appending  $::$ can only append element on the left side.}
  % \mg{We introduce a counting operator $\vcounter : \mathcal{T} \to \mathbb{N} \to \mathbb{N}$ whose behavior is defined as follows, \sout{Counter $\vcounter : \mathcal{T} \to \mathbb{N} \to \mathbb{N}$ }}.
  % \wq{The operator counter actually provides the number of times a specific label appears in a trace. Only a number, the position of label is ignored.}
  % \[
  % \begin{array}{lll}
  % \vcounter((x, l, v) :: \trace ) l \triangleq \vcounter(\trace) l + 1
  % &
  % \vcounter((b, l, v):: \trace ) l \triangleq \vcounter(\trace) l + 1
  % &
  % \vcounter((x, l, \qval, v):: \trace ) l \triangleq \vcounter(\trace) l + 1
  % \\
  % \vcounter((x, l', v):: \trace ) l \triangleq \vcounter(\trace ) l
  % &
  % \vcounter((b, l', v):: \trace ) l \triangleq \vcounter(\trace ) l
  % &
  % \vcounter((x, l', \qval, v):: \trace ) l \triangleq \vcounter(\trace ) l
  % \\
  % \vcounter({[]}) l \triangleq 0
  % &&
  % \end{array}
  % \]
  % \[
  % \begin{array}{lll}
  % \vcounter(\trace  \tracecat [(x, l, v)] ) l \triangleq \vcounter(\trace) l + 1
  % &
  % \vcounter(\trace  \tracecat [(b, l, v)] ) l \triangleq \vcounter(\trace) l + 1
  % &
  % \vcounter(\trace  \tracecat [(x, l, \qval, v)] ) l \triangleq \vcounter(\trace) l + 1
  % \\
  % \vcounter(\trace  \tracecat [(x, l', v)] ) l \triangleq \vcounter(\trace ) l
  % &
  % \vcounter(\trace  \tracecat [(b, l', v)] ) l \triangleq \vcounter(\trace ) l
  % &
  % \vcounter(\trace  \tracecat [(x, l', \qval, v)]) l \triangleq \vcounter(\trace ) l
  % \\
  % \vcounter({[]}) l \triangleq 0
  % &&
  % \end{array}
  % \]
  We introduce a counting operator $\vcounter : \mathcal{T} \to \mathbb{N} \to \mathbb{N}$ whose behavior is defined as follows,
  % \[
  % \begin{array}{lll}
  % \vcounter(\trace :: (x, l, v, \bullet) ) l \triangleq \vcounter(\trace) l + 1
  % &
  % \vcounter(\trace  ::(b, l, v, \bullet) ) l \triangleq \vcounter(\trace) l + 1
  % &
  % \vcounter(\trace  :: (x, l, v, \qval) ) l \triangleq \vcounter(\trace) l + 1
  % \\
  % \vcounter(\trace  :: (x, l', v, \bullet) ) l \triangleq \vcounter(\trace ) l, l' \neq l
  % &
  % \vcounter(\trace  :: (b, l', v, \bullet) ) l \triangleq \vcounter(\trace ) l, l' \neq l
  % &
  % \vcounter(\trace  :: (x, l', v, \qval)) l \triangleq \vcounter(\trace ) l, l' \neq l
  % \\
  % \vcounter({[]}) l \triangleq 0
  % &&
  % \end{array}
  % \]
  \[
  \begin{array}{lll}
  \vcounter(\trace :: (x, l, v, \bullet), l ) \triangleq \vcounter(\trace, l) + 1
  &
  \vcounter(\trace  ::(b, l, v, \bullet), l) \triangleq \vcounter(\trace, l) + 1
  &
  \vcounter(\trace  :: (x, l, v, \qval), l) \triangleq \vcounter(\trace, l) + 1
  \\
  \vcounter(\trace  :: (x, l', v, \bullet), l) \triangleq \vcounter(\trace, l), l' \neq l
  &
  \vcounter(\trace  :: (b, l', v, \bullet), l) \triangleq \vcounter(\trace, l), l' \neq l
  &
  \vcounter(\trace  :: (x, l', v, \qval), l) \triangleq \vcounter(\trace, l), l' \neq l
  \\
  \vcounter({[]}, l) \triangleq 0
  &&
  \end{array}
  \]
  
  We introduce an operator $\llabel : \mathcal{T} \to \mathcal{VAR} \to \ldom \cup \{\bot\}$, which 
  takes a trace and a variable and returns the label of the latest assignment event which assigns value to that variable.
  Its behavior is defined as follows,
  % \begin{defn}[Latest Label]
    \[
      % \begin{array}{lll}
    \llabel(\trace  :: (x, l, \_, \_)) x \triangleq l
    ~~~
    \llabel(\trace  :: (y, l, \_, \_)) x \triangleq \llabel(\trace ) x, y \neq x
    % &
    ~~~
    \llabel(\trace :: (b, l, v, \bullet)) x \triangleq \llabel(\trace) x
    % &
    % \\
    % \llabel(\trace  :: (y, l, v, \bullet)) x \triangleq \llabel(\trace ) x
    % &
    % \llabel(\trace :: (y, l, v, \qval)) x \triangleq \llabel(\trace ) x
    % &
    ~~~
    \llabel({[]}) x \triangleq \bot
    % \end{array}
    \]

    The operator $\tlabel : \mathcal{T} \to \mathcal{P}{(\ldom)}$ gives the set of labels in every event belonging to 
    a trace, whoes behavior is defined as follows,
  \[
    % \begin{array}{llll}
  \tlabel{(\trace  :: (\_, l, \_, \_)])} \triangleq \{l\} \cup \tlabel{(\trace )}
  ~~~
  \tlabel({[ ]}) \triangleq \{\}
  % \end{array}
  \]
  
  %
  % \mg{The next lemma is trivial but it still needs a proof sketch.}
  If we observe the operational semantics rules, we can find that no rule will shrink the trace. 
  So we have the Lemma~\ref{lem:tracenondec} with proof in Appendix~\ref{apdx:lemma_sec123}, 
  specifically the trace has the property that its length never decreases during the program execution.
  \begin{lem}
  [Trace Non-Decreasing]
  \label{lem:tracenondec}
  For every program $c \in \cdom$ and traces $\trace, \trace' \in \mathcal{T}$, if 
  $\config{c, \trace} \rightarrow^{*} \config{\eskip, \trace'}$,
  then there exists a trace $\trace'' \in \mathcal{T}$ with $\trace \tracecat \trace'' = \trace'$
  %
  $$
  \forall \trace, \trace' \in \mathcal{T}, c \sthat 
  \config{c, \trace} \rightarrow^{*} \config{\eskip, \trace'} 
  \implies \exists \trace'' \in \mathcal{T} \sthat \trace \tracecat \trace'' = \trace'
  $$
  \end{lem}
  Since the equivalence over two events is defined over the query value equivalence, 
  when there is an event belonging to a trace, 
  if this event is a query assignment event, 
  it is possible that 
  the event showing up in this trace has a different form of query value, 
  but they are equivalent by Definition~\ref{def:query_equal}.
  So we have the following Corollary~\ref{coro:aqintrace} with proof in Appendix~\ref{apdx:lemma_sec123}.
  \begin{coro}
  \label{coro:aqintrace}
  For every event and a trace $\trace \in \mathcal{T}$,
  if $\event \in \trace$, 
  then there exist another event $\event' \in \eventset$ and traces $\trace_1, \trace_2 \in \mathcal{T}$
  such that $\trace_1 \tracecat [\event'] \tracecat \trace_2 = \trace $
  with 
  $\event$ and $\event'$ equivalent but may differ in their query value.
  \[
    \forall \event \in \eventset, \trace \in \mathcal{T} \sthat 
  \event \in \trace \implies \exists \trace_1, \trace_2 \in \mathcal{T}, 
  \event' \in \eventset \sthat (\event \in \event') \land \trace_1 \tracecat [\event'] \tracecat \trace_2 = \trace  
  \]
  \end{coro}
%  
\paragraph{Environment}
An environment is standard, a map from variables to values.
$\env : {\mathcal{T}}  \to \mathcal{VAR} \to \mathcal{VAL} \cup \{\bot\}$, 
which maps a trace and a variable to the latest value assigned to this variable on the trace is defined as follows.
Queries can be uniquely annotated as $\mathcal{AQ}$, and the annotation $(l,w)$ considers the location of the query by line number $l$ and which 
iteration the query is at when it appears in a loop statement, specified by $w$. A trace $t$ is a list of annotated queries accumulated along the execution of the program. 

\[
\begin{array}{lll}
\env(\trace  \traceadd (x, l, v, \bullet)) x \triangleq v
&
\env(\trace \traceadd (y, l, v, \bullet)) x \triangleq \env(\trace) x, y \neq x
&
\env(\trace \traceadd (b, l, v, \bullet)) x \triangleq \env(\trace) x
\\
\env(\trace \traceadd (x, l, v, \qval)) x \triangleq v
&
\env(\trace \traceadd (y, l, v, \qval)) x \triangleq \env(\trace) x, y \neq x
&
\env({[]} ) x \triangleq \bot
\end{array}
\]
 %% trace

 \paragraph*{Operational Semantics}
 The complete operational semantic evaluation rules are presented in Figure~\ref{fig:os},
 with the evaluation rules for expressions in Figure~\ref{fig:expr_eval}.

 The rule $\textbf{assn}$ evaluates a standard assignment $\assign{x}{\expr}$, the expression $\expr$ is first evaluated by our expression evaluation $\config{\trace, \expr} \earrow v $, presented below. And the result $v$ of evaluating $\expr$ is used to construct a new event $\event = (x, l, v,\bullet)$ and attach to the previous trace.  
\begin{figure}
\begin{mathpar}
\boxed{ \config{\trace,\aexpr} \aarrow v \, : \, \mbox{Trace  $\times$ Arithmetic Expr $\Rightarrow$ Arithmetic Value} }
\\
% \text{\mg{Missing. Without these rules it is difficult to understand why we need a trace to evaluate expressions.}}
% \\
\inferrule{ 
  \empty
}{
 \config{\trace,  n} 
 \aarrow n
}
\and
\inferrule{ 
  \env(\trace) x = v
}{
 \config{\trace,  x} 
 \aarrow v
}
\and
\inferrule{ 
  \config{\trace, \aexpr_1} \aarrow v_1
  \and 
  \config{\trace, \aexpr_2} \aarrow v_2
  \and 
   v_1 \oplus_a v_2 = v
}{
 \config{\trace,  \aexpr_1 \oplus_a \aexpr_2} 
 \aarrow v
}
\and
\inferrule{ 
  \config{\trace, \aexpr} \aarrow v'
  \and 
  \elog v' = v
}{
 \config{\trace,  \elog \aexpr} 
 \aarrow v
}
\and
\inferrule{ 
  \config{\trace, \aexpr} \aarrow v'
  \and 
  \esign v' = v
}{
 \config{\trace,  \esign \aexpr} 
 \aarrow v
}
\\
\boxed{ \config{\trace, \bexpr} \barrow v \, : \, \mbox{Trace $\times$ Boolean Expr $\Rightarrow$ Boolean Value} }
\\
\inferrule{ 
  \empty
}{
 \config{\trace,  \efalse} 
 \barrow \efalse
}
\and 
\inferrule{ 
  \empty
}{
 \config{\trace,  \etrue} 
 \barrow \etrue
}
\and 
\inferrule{ 
  \config{\trace, \bexpr} \barrow v'
  \and 
  \neg v' = v
}{
 \config{\trace,  \neg \bexpr} 
 \barrow v
}
\and 
\inferrule{ 
  \config{\trace, \bexpr_1} \barrow v_1
  \and 
  \config{\trace, \bexpr_2} \barrow v_2
  \and 
   v_1 \oplus_b v_2 = v
}{
 \config{\trace,  \bexpr_1 \oplus_b \bexpr_2} 
 \barrow v
}
\and 
\inferrule{ 
  \config{\trace, \aexpr_1} \aarrow v_1
  \and 
  \config{\trace, \aexpr_2} \aarrow v_2
  \and 
   v_1 \sim v_2 = v
}{
 \config{\trace,  \aexpr_1 \sim \aexpr_2} 
 \barrow v
}
\\
\boxed{ \config{\trace, \expr} \earrow v \, : \, \mbox{Trace $\times$ Expression $\Rightarrow$ Value} }
\\
\inferrule{ 
  \config{\trace, \aexpr} \aarrow v
}{
 \config{\trace,  \aexpr} 
 \earrow v
}
\and
\inferrule{ 
  \config{\trace, \bexpr} \barrow v
}{
 \config{\trace,  \bexpr} 
 \earrow v
}
\and
\inferrule{ 
  \config{\trace, \expr_1} \earrow v_1
  \cdots
  \config{\trace, \expr_n} \earrow v_n
}{
 \config{\trace,  [\expr_1, \cdots, \expr_n]} 
 \earrow [v_1, \cdots, v_n]
}
\and
\inferrule{ 
  \empty
}{
 \config{\trace,  v} 
 \earrow v
}
\\
\boxed{ \config{\trace, \qexpr} \qarrow \qval \, : \, \mbox{Trace  $\times$ Query Expr $\Rightarrow$ Query Value} }
\\
\inferrule{ 
  \config{\trace, \aexpr} \aarrow n
}{
 \config{\trace,  \aexpr} 
 \qarrow n
}
\and
\inferrule{ 
  \config{\trace, \qexpr_1} \qarrow \qval_1
  \and
  \config{\trace, \qexpr_2} \qarrow \qval_2
}{
 \config{\trace,  \qexpr_1 \oplus_a \qexpr_2} 
 \qarrow \qval_1 \oplus_a \qval_2
}
\and
\inferrule{ 
  \config{\trace, \aexpr} \aarrow n
}{
 \config{\trace, \chi[\aexpr]} \qarrow \chi[n]
}
\and
\inferrule{ 
  \empty
}{
 \config{\trace,  \qval} 
 \qarrow \qval
}
 \end{mathpar}
 \caption{Trace-based Operational Semantics Evaluation Rules for Expression.}
 \label{fig:expr_eval}
\end{figure}

The expression evaluation rules rely on the evaluation of 
arithmetic expressions $\config{\trace,\aexpr} \aarrow v $ 
and boolean expressions $\config{\trace, \bexpr} \barrow v $.


 The step trace-based operational semantics has the form of $ \config{c, \trace} \xrightarrow{} { \config{c', \trace'}}$. 
It reads the configuration $\config{c, \trace}$ consisting of a labeled command $c$ and a pre-trace $\trace$, 
and evaluates it to another configuration, 
in which $c$ is evaluated to $c'$ and trace $\trace$ is updated to $\trace'$. 
\begin{figure}
\begin{mathpar}
\boxed{
\mbox{Command $\times$ Trace}
\xrightarrow{}
\mbox{Command $\times$ Trace}
}
\and
\boxed{\config{{c, \trace}}
\xrightarrow{} 
\config{{c',  \trace'}}
}
\\
\inferrule
{
\empty
}
{
\config{\clabel{\eskip}^l,  \trace } 
\xrightarrow{} 
\config{\clabel{\eskip}^l, \trace}
}
~\textbf{skip}
%
\and
%
\inferrule
{
\event = ({x}, l, v, \bullet)
}
{
\config{[\assign{{x}}{\aexpr}]^{l},  \trace } 
\xrightarrow{} 
\config{\clabel{\eskip}^l, \trace \traceadd \event}
}
~\textbf{assn}
%
\and
%
{
\inferrule
{
 \trace, \qexpr \qarrow \qval
 \and 
\query(\qval) = v
\and 
\event = ({x}, l, v, \qval)
}
{
\config{{[\assign{x}{\query(\qexpr)}]^l, \trace}}
\xrightarrow{} 
\config{{\clabel{\eskip}^l,  \trace \traceadd \event} }
}
~\textbf{query}
}
%
\and
%
\inferrule
{
 \trace, b \barrow \etrue
 \and 
 \event = (b, l, \etrue, \bullet)
}
{
\config{{\ewhile [b]^{l} \edo c, \trace}}
\xrightarrow{} 
\config{{
c; \ewhile [b]^{l} \edo c),
\trace \traceadd \event}}
}
~\textbf{while-t}
%
%
\and
%
\inferrule
{
 \trace, b \barrow \efalse
 \and 
 \event = (b, l, \efalse, \bullet)
}
{
\config{{\ewhile [b]^{l}, \edo c, \trace}}
\xrightarrow{} 
\config{{
  \clabel{\eskip}^l,
\trace \traceadd \event}}
}
~\textbf{while-f}
\and
\inferrule
{
\config{{c_1, \trace}}
\xrightarrow{}
\config{{c_1',  \trace'}}
}
{
\config{{c_1; c_2, \trace}} 
\xrightarrow{} 
\config{{c_1'; c_2, \trace'}}
}
~\textbf{seq1}
%
\and
%
\inferrule
{
  \config{{c_2, \trace}}
  \xrightarrow{}
  \config{{c_2',  \trace'}}
}
{
\config{{\clabel{\eskip}^l; c_2, \trace}} \xrightarrow{} \config{{ c_2', \trace'}}
}
~\textbf{seq2}
%
\and
%
%
\inferrule
{
   \trace, b \barrow \etrue
 \and 
 \event = (b, l, \etrue, \bullet)
}
{
 \config{{
\eif([b]^{l}, c_1, c_2), 
\trace}}
\xrightarrow{} 
\config{{c_1, \trace \traceadd \event}}
}
~\textbf{if-t}
%
\and
%
\inferrule
{
 \trace, b \barrow \efalse
 \and 
 \event = (b, l, \efalse, \bullet)
}
{
\config{{\eif([b]^{l}, c_1, c_2), \trace}}
\xrightarrow{} 
\config{{c_2, \trace \traceadd \event}}
}
~\textbf{if-f}
\end{mathpar}
% \end{subfigure}
    \caption{Trace-based Operational Semantics for Language.}
    \label{fig:os}
\end{figure}

Distinguished from the standard assignment evaluation, 
the rule $\textbf{query}$ 
evaluates a query requesting command $\clabel{\assign{x}{\query(\qexpr)}}^l$ in two steps.
The query expression $\qexpr$ is first evaluated into a query value $\qval$ by following the rules below.
Then, by sending this query request $\query(\qval)$  to a hidden mechanism, this query is evaluated to a result value returned from it, $v = \query(\qval)$.
% by sending this query request $\query(\qval)$  to it.
Also, the generated event stores both the query value $\alpha$ here, and the result value of the query request.

The rules for if and while both have two versions, 
when the boolean expressions in the guards are evaluated to true and false, respectively. 
In these rules, the evaluation of the guard generates a testing event and the trace is updated as well by appending this event.


% \todo{Make sure the operational semantics is big step, and correct assn rules.}
% \jl{
