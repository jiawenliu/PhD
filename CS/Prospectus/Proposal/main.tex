\documentclass[a4paper,11pt]{article}
\usepackage[table]{xcolor}
%Packages
% \usepackage{lipsum}
\usepackage[T1]{fontenc}
\usepackage{fourier} 
\usepackage[english]{babel} 
\usepackage{amsmath,amsfonts} 
\usepackage{amsthm} 
\usepackage{color}   %May be necessary if you want to color links
\usepackage{hyperref}
\usepackage{lscape}
\usepackage{geometry}
\usepackage{amsmath}
\usepackage{algorithm}
\usepackage{algorithmic}
\usepackage{amssymb}
\usepackage{amsfonts}
\usepackage{times}
\usepackage{bm}
\usepackage{ stmaryrd }
\SetSymbolFont{stmry}{bold}{U}{stmry}{m}{n}
\usepackage{ amssymb }
\usepackage{ textcomp }
\usepackage[normalem]{ulem}
% For derivation rules
\usepackage{mathpartir}
\usepackage{color}
\usepackage{a4wide}
\usepackage{caption}
\usepackage{subcaption}
\usepackage{mathpartir}
\usepackage{amsmath,amsfonts}
\usepackage{ amssymb }
\usepackage{color}
\usepackage{algorithm}
\usepackage{algorithmic}
\usepackage{microtype}
\usepackage{eucal}
\usepackage{url}
\usepackage{tikz}
\usepackage{xspace}
\usepackage{array}
\usepackage{listings}
\usepackage{import}

\usetikzlibrary{shapes.geometric}
\usetikzlibrary{arrows.meta,arrows}
\usetikzlibrary{decorations.text}
% % % % 

%%%% Extra packages
\usepackage[T1]{fontenc}
\usepackage[latin9]{inputenc}
\usepackage{amsmath}
\usepackage{amssymb}
\usepackage{cancel}

%%%%%%%%%%%%%%%%%%%%%%%%%%%%%% User specified LaTeX commands.
% \usepackage{typesetting/latex8}
% \usepackage{times}
% \usepackage{color}
% \usepackage{epsfig}
% \usepackage{graphicx}
% \usepackage{graphics}
% \usepackage{amsmath, nicefrac}
% \usepackage{amssymb, amsthm}
% \usepackage{wrapfig}
% \usepackage{algorithm, algorithmic}
% \usepackage{setspace}
% \usepackage{caption}
% \usepackage{float}
% \usepackage{afterpage}
% % \usepackage{typesetting/abstract}
% \usepackage{tabularx}
% \usepackage{booktabs}
% \usepackage{calc}
% \usepackage{multirow}
% \usepackage{longtable}
% \usepackage{footnote}
% \usepackage{threeparttable}
% \usepackage{colortbl}
% % \usepackage{tweaklist}
% \usepackage{fancyhdr}
% \usepackage[retainorgcmds]{IEEEtrantools}
% \usepackage{floatflt}
% \usepackage{xspace}

% \usepackage{endnotes}
% \usepackage{paralist}
% % \usepackage{typesetting/shortcuts}
% \usepackage{tabulary}
% \usepackage{mdwlist}
% \usepackage{listings}
% \usepackage{balance}
% \usepackage{url}
% \usepackage{parskip}
% \usepackage{textcomp}
% \usepackage{subcaption}

% \usepackage{epstopdf}
% \usepackage{fancyvrb}

% \makeatletter

%%%%%%%%%%%%%%%%%%%%%%%%%%%%%%%%%%%%%%%%%%%%%%%%%%%%%%%%%%%%%%%%%%%%%%%%%%%%%%%%%%%%%%%%%%%%%%%%%%%%%%%%%%%%%%%%%%%%%%%%%%%%%%%%%%%%%%%%%
%%%%%%%%%%%%%%%%%%%%%%%%%%%%%%%%%%%%%%%%%%%%%%%%%%%%% COMMANDS FOR GENERAL PAPER WRITING %%%%%%%%%%%%%%%%%%%%%%%%%%%%%%%%%%%%%%%%%%%%%%%%%%%%
%%%%%%%%%%%%%%%%%%%%%%%%%%%%%%%%%%%%%%%%%%%%%%%%%%%%%%%%%%%%%%%%%%%%%%%%%%%%%%%%%%%%%%%%%%%%%%%%%%%%%%%%%%%%%%%%%%%%%%%%%%%%%%%%%%%%%%%%%

%%%%%%%%%%%%%%% Extra Ldefs:

% \renewcommand{\textfraction}{0.1}
% \renewcommand{\topfraction}{0.95}
% \renewcommand{\bottomfraction}{0.95}
% \renewcommand\floatpagefraction{0.9}
% \setcounter{totalnumber}{50} \setcounter{topnumber}{50} \setcounter{bottomnumber}{50}
% \renewcommand{\floatsep}{10pt}
% \renewcommand{\intextsep}{10pt}
% \setlength{\textfloatsep}{10pt}

% \renewcommand{\headrulewidth}{0pt} \renewcommand{\footrulewidth}{0pt}

% \newcommand{\eqnlinespace}{\\[5pt]}
% \newcommand{\eqnlinespacelarge}{\\[10pt]}
% \newcommand{\captionlinespace}{\\[0.05in]}

% \renewcommand{\baselinestretch}{1.6}

% %\renewcommand{\textwidth}{5.95in}
% \setlength{\textwidth}{6.875in}

% \renewcommand{\oddsidemargin}{0.5in}
\newcommand{\todo}[1]{{\color{red}\textbf{[[ #1 ]]}}}


%%%%%%%%%%%%%%%%%%%%%%%%%%%% Theorem, Definition and Proof
\newtheorem{lem}{Lemma}[section]
\newtheorem{thm}{Theorem}[section]
\newtheorem{defn}{Definition}
\newtheorem{coro}{Corollary}[thm]

\newtheorem{example}{Example}[section]
\newcommand{\ADAPTSYSTEM}{ADAPT}

\newcommand{\highlight}[1]{\textcolor[rgb]{.0,0.0,1.0}{ #1}}

% \newcommand{\todo}[1]{{\color{red}\textbf{[[ #1 ]]}}}
\newcommand{\todomath}[1]{{\scriptstyle \color{red}\mathbf{[[ #1 ]]}}}
\newcommand{\completeness}[1]{{\color{blue}\textbf{[[ #1 ]]}}}
\newcommand{\caseL}[1]{\item \textbf{case: #1}\newline}
\newcommand{\subcaseL}[1]{\item \textbf{sub-case: #1}\newline}
\newcommand{\subsubcaseL}[1]{\item \textbf{subsub-case: #1}\newline}
\newcommand{\subsubsubcaseL}[1]{\item \textbf{subsubsub-case: \boldmath{#1}}\newline}

\newcommand{\blue}[1]{{\tiny \color{blue}{ #1 }}}


\let\originalleft\left
\let\originalright\right
\renewcommand{\left}{\mathopen{}\mathclose\bgroup\originalleft}
\renewcommand{\right}{\aftergroup\egroup\originalright}
\newcommand{\ts}[1]{ \llparenthesis {#1} \rrparenthesis }

\theoremstyle{definition}

\newtheorem{case}{Case}
\newtheorem{subcase}{Case}
\numberwithin{subcase}{case}
\newtheorem{subsubcase}{Case}
\numberwithin{subsubcase}{subcase}

\newtheorem{subsubsubcase}{Case}
\numberwithin{subsubsubcase}{subsubcase}

\newcommand{\dist}{P}
\newcommand{\mech}{M}
\newcommand{\univ}{\mathcal{X}}
\newcommand{\anyl}{A}
% \newcommand{\query}{f}
% \newcommand{\qlen}{k}
\newcommand{\qrounds}{r}
\newcommand{\answer}{a}
\newcommand{\sample}{X}

\newcommand{\sthat}{~.~}

%%%%COLORS
\definecolor{periwinkle}{rgb}{0.8, 0.8, 1.0}
\definecolor{powderblue}{rgb}{0.69, 0.88, 0.9}
\definecolor{sandstorm}{rgb}{0.93, 0.84, 0.25}
\definecolor{trueblue}{rgb}{0.0, 0.45, 0.81}

\newlength\Origarrayrulewidth
% horizontal rule equivalent to \cline but with 2pt width
\newcommand{\Cline}[1]{%
 \noalign{\global\setlength\Origarrayrulewidth{\arrayrulewidth}}%
 \noalign{\global\setlength\arrayrulewidth{2pt}}\cline{#1}%
 \noalign{\global\setlength\arrayrulewidth{\Origarrayrulewidth}}%
}

% draw a vertical rule of width 2pt on both sides of a cell
\newcommand\Thickvrule[1]{%
  \multicolumn{1}{!{\vrule width 2pt}c!{\vrule width 2pt}}{#1}%
}

% draw a vertical rule of width 2pt on the left side of a cell
\newcommand\Thickvrulel[1]{%
  \multicolumn{1}{!{\vrule width 2pt}c|}{#1}%
}

% draw a vertical rule of width 2pt on the right side of a cell
\newcommand\Thickvruler[1]{%
  \multicolumn{1}{|c!{\vrule width 2pt}}{#1}%
}

\newenvironment{subproof}[1][\proofname]{%
  \renewcommand{\qedsymbol}{$\blacksquare$}%
  \begin{proof}[#1]%
}{%
  \end{proof}%
}
%%%%%%%%%%%%%%%%%%%%%%%%%%%%%%% Fonts Definition %%%%%%%%%%%%%%%%%%%%%%%%%%%
\newcommand{\omitthis}[1]{}

% Misc.
\newcommand{\etal}{\textit{et al.}}
% \wd
\newcommand{\bump}{\hspace{3.5pt}}
% Text fonts
\newcommand{\tbf}[1]{\textbf{#1}}

% Math fonts
\newcommand{\mbb}[1]{\mathbb{#1}}
\newcommand{\mbf}[1]{\mathbf{#1}}
\newcommand{\mrm}[1]{\mathrm{#1}}
\newcommand{\mtt}[1]{\mathtt{#1}}
\newcommand{\mcal}[1]{\mathcal{#1}}
\newcommand{\mfrak}[1]{\mathfrak{#1}}
\newcommand{\msf}[1]{\mathsf{#1}}
\newcommand{\mscr}[1]{\mathscr{#1}}

\newcommand{\diam}{{\color{red}\diamond}}
\newcommand{\dagg}{{\color{blue}\dagger}}
\let\oldstar\star
\renewcommand{\star}{\oldstar}

\newcommand{\im}[1]{\ensuremath{#1}}

\newcommand{\kw}[1]{\im{\mathtt{#1}}}
\newcommand{\set}[1]{\im{\{{#1}\}}}

\newcommand{\mmax}{\ensuremath{\mathsf{max}}}

\lstnewenvironment{ocaml}[2][]%
  {\lstset{language=ocaml,style=ocaml-pretty,captionpos=t,abovecaptionskip=-\medskipamount,caption={#2},#1}}
  %
  {}

\makeatletter
\newcommand{\mysmallishfont}{\@setfontsize\mysmallishfont{8.7pt}{9.7pt}}
\makeatother

\makeatletter
\newcommand{\myecfont}{\@setfontsize\myecfont{9.7pt}{10.7pt}}
\makeatother

\makeatletter
\newcommand{\myecsmfont}{\@setfontsize\myecfont{8.7pt}{9.7pt}}
\makeatother

\lstdefinelanguage{ocaml}{
  style=ocaml-default,
  keywordsprefix={'},
  morekeywords=[1]{},
  morekeywords=[2]{type,op,axiom,lemma,module,pred,const,declare},
  morekeywords=[3]{var,proc},
  morekeywords=[4]{while,if,then,else,elif,return,proof,qed,realize,rec, match},
}

\lstdefinestyle{ocaml-default}{
  escapechar=\#,
  upquote=true,
  columns=fullflexible,
  captionpos=b,
  frame=tb,
  xleftmargin=0pt,
  xrightmargin=0pt,
  rangebeginprefix={(**\ begin\ },
  rangeendprefix={(**\ end\ },
  rangesuffix={\ *)},
  includerangemarker=false,
  basicstyle=\mysmallishfont\sffamily,
  identifierstyle={},
  keywordstyle=[1]{\itshape},
  keywordstyle=[2]{\bfseries},
  keywordstyle=[3]{\bfseries},
  keywordstyle=[4]{\bfseries},
  keywordstyle=[5]{\bfseries},
  keywordstyle=[6]{\bfseries},
  keywordstyle=[7]{},
  keywordstyle=[8]{\bfseries},
  keywordstyle=[9]{\bfseries},
  literate={phi}{{$\!\phi\,$}}1
           {phi1}{{$\!\phi_1$}}1
           {phi2}{{$\!\phi_2$}}1
           {phi3}{{$\!\phi_3$}}1
           {phin}{{$\!\phi_n$}}1
}

\lstdefinestyle{ocaml-pretty}{
    basicstyle=\mysmallishfont\sffamily,
    literate={:=}{{$\mathrel{\gets}\;$}}1
              {<=}{{$\mathrel{\leq}\;$}}1
              {>=}{{$\mathrel{\geq}\;$}}1
              {<>}{{$\mathrel{\neq}\;$}}1
              {=\$}{{$\stackrel{\$}{\gets}\;$}}1
              {->}{{$\rightarrow\;$}}1
              {<-}{{$\leftarrow\;$}}1
              {<->}{{$\leftrightarrow\;$}}1
              {<=>}{{$\Leftrightarrow\;$}}1
              {=>}{{$\Rightarrow\;$}}1
              {==>}{{$\Longrightarrow\;$}}1
              {\/\\}{{$\wedge\;$}}1
              {\\\/}{{$\vee\;$}}1
              {\^}{{\textasciicircum}}1
              {procx}{{proc}}1
}


%%%%%%%%%%%%%%%%%%%%%%%%%%%%%%%%%%%%%%%%%%%%%%%%%%%%%%%%%%%%%%%%%%%%%%%%%%%%%%%%%%%%%%%%%%%%%%%%%%%%%%%%%%%%%%%%%%%%%%%%%%%%%%%%%%%%%%%%%%%%%%%%%%%%%%%%%%
%%%%%%%%%%%%%%%%%%%%%%%%%%%%%%%%%%%%%%%%%%%%%%%%%%%%%%%%%%%% Query While Language %%%%%%%%%%%%%%%%%%%%%%%%%%%%%%%%%%%%%%%%%%%%%%%%%%%%%%%%%%%%%%%%
%%%%%%%%%%%%%%%%%%%%%%%%%%%%%%%%%%%%%%%%%%%%%%%%%%%%%%%%%%%%%%%%%%%%%%%%%%%%%%%%%%%%%%%%%%%%%%%%%%%%%%%%%%%%%%%%%%%%%%%%%%%%%%%%%%%%%%%%%%%%%%%%%%%%%%%%%%
% Language
\newcommand{\command}{c}
%Label
\newcommand{\lin}{\kw{in}}
\newcommand{\lex}{\kw{ex}}
% expression
\newcommand{\expr}{e}
\newcommand{\aexpr}{a}
\newcommand{\bexpr}{b}
\newcommand{\sexpr}{\ssa{\expr} }
\newcommand{\qexpr}{\psi}
\newcommand{\qval}{\alpha}
\newcommand{\query}{{\tt query}}
\newcommand{\eif}{\;\kw{if}\;}
\newcommand{\ethen}{\kw{\;then\;}}
\newcommand{\eelse}{\kw{\;else\;}} 
\newcommand{\eapp}{\;}
\newcommand{\eprojl}{\kw{fst}}
\newcommand{\eprojr}{\kw{snd}}
\newcommand{\eifvar}{\kw{ifvar}}
\newcommand{\ewhile}{\;\kw{while}\;}
\newcommand{\bop}{\;*\;}
\newcommand{\uop}{\;\circ\;}
\newcommand{\eskip}{\kw{skip}}
\newcommand{\edo}{\;\kw{do}\;}
% More unary expression operators:
\newcommand{\esign}{~\kw{sign}~}
\newcommand{\elog}{~\kw{log}~}

%%%%%%%%%% Extended
\newcommand{\efun}{~\kw{fun}~}
\newcommand{\ecall}{~\kw{call}~}


% Domains
\newcommand{\qdom}{\mathcal{QD}}
\newcommand{\memdom}{\mathcal{M}}
\newcommand{\dbdom}{\mathcal{DB}}
\newcommand{\cdom}{\mathcal{C}}
\newcommand{\ldom}{\mathcal{L}}

\newcommand{\emap}{~\kw{map}~}
\newcommand{\efilter}{~\kw{filter}~}

%configuration
\newcommand{\config}[1]{\langle #1 \rangle}
\newcommand{\ematch}{\kw{match}}
\newcommand{\clabel}[1]{\left[ #1 \right]}

\newcommand{\etrue}{\kw{true}}
\newcommand{\efalse}{\kw{false}}
\newcommand{\econst}{c}
\newcommand{\eop}{\delta}
\newcommand{\efix}{\mathop{\kw{fix}}}
\newcommand{\elet}{\mathop{\kw{let}}}
\newcommand{\ein}{\mathop{ \kw{in}} }
\newcommand{\eas}{\mathop{ \kw{as}} }
\newcommand{\enil}{\kw{nil}}
\newcommand{\econs}{\mathop{\kw{cons}}}
\newcommand{\term}{t}
\newcommand{\return}{\kw{return}}
\newcommand{\bernoulli}{\kw{bernoulli}}
\newcommand{\uniform}{\kw{uniform}}
\newcommand{\app}[2]{\mathrel{ {#1} \, {#2} }}


% Operational Semantics
\newcommand{\env}{\rho}
\newcommand{\rname}[1]{\textsf{\small{#1}}}
\newcommand{\aarrow}{\Downarrow_a}
\newcommand{\barrow}{\Downarrow_b}
\newcommand{\earrow}{\Downarrow_e}
\newcommand{\qarrow}{\Downarrow_q}
\newcommand{\cmd}{c}
\newcommand{\node}{N}
\newcommand{\assign}[2]{ \mathrel{ #1  \leftarrow #2 } }


%%%%%%%%%%%%%%%%%%%%%%%%%%%%%%%%%%%%%%%%%%%%%%%%%%%%%%%%%%%%%%%%%%%%%%%% Trace and Events %%%%%%%%%%%%%%%%%%%%%%%%%%%%%%%%%%%%%%%%
%%%%%%%%%%%%%%%%%%%%%%%%%%%%%%%%%%%%%%%%%%%%%%%%%%%%%%%%%%%%%%%%%%%%%% Trace 
%%%%%%%% annotated query
\newcommand{\aq}{\kw{aq}}
\newcommand{\qtrace}{\kw{qt}}
%annotated variables
\newcommand{\av}{\kw{av}}
\newcommand{\vtrace}{\kw{\tau}}
\newcommand{\ostrace}{{\kw{\tau}}}
\newcommand{\posttrace}{{\kw{\tau}}}

\newcommand{\trace}{\kw{\tau}}

% \newcommand{\vcounter}{\kw{\zeta}}
\newcommand{\vcounter}{\kw{cnt}}

\newcommand{\postevent}{{\kw{\epsilon}}}

% \newcommand{\event}{\kw{\epsilon}}
% \newcommand{\eventset}{\mathcal{E}}
% \newcommand{\eventin}{\in_{\kw{e}}}
% \newcommand{\eventeq}{=_{\kw{e}}}
% \newcommand{\eventneq}{\neq_{\kw{e}}}
% \newcommand{\eventgeq}{\geq_{\kw{e}}}
% \newcommand{\eventlt}{<_{\kw{e}}}
% \newcommand{\eventleq}{\leq_{\kw{e}}}
% \newcommand{\eventdep}{\mathsf{DEP_{\kw{e}}}}
% \newcommand{\asn}{\kw{{asn}}}
% \newcommand{\test}{\kw{{test}}}
% \newcommand{\ctl}{\kw{{ctl}}}
\newcommand{\event}{\kw{\epsilon}}
\newcommand{\eventset}{\mathcal{E}}
\newcommand{\eventin}{\in_{\kw{e}}}
\newcommand{\eventeq}{=_{\kw{e}}}
\newcommand{\eventneq}{\neq_{\kw{e}}}
\newcommand{\eventgeq}{\geq_{\kw{e}}}
\newcommand{\eventlt}{<_{\kw{e}}}
\newcommand{\eventleq}{\leq_{\kw{e}}}
\newcommand{\eventdep}{\mathsf{DEP_{\kw{e}}}}
\newcommand{\asn}{\kw{{asn}}}
\newcommand{\test}{\kw{{test}}}
\newcommand{\ctl}{\kw{{ctl}}}

\newcommand{\sig}{\kw{sig}}
\newcommand{\sigeq}{=_{\sig}}
\newcommand{\signeq}{\neq_{\sig}}
\newcommand{\notsigin}{\notin_{\sig}}
\newcommand{\sigin}{\in_{\sig}}
\newcommand{\sigdiff}{\kw{Diff}_{\sig}}
\newcommand{\action}{\kw{act}}
\newcommand{\diff}{\kw{Diff}}
\newcommand{\seq}{\kw{seq}}
\newcommand{\sdiff}{\kw{Diff}_{\seq}}

\newcommand{\tracecat}{{\scriptscriptstyle ++}}
\newcommand{\traceadd}{{\small ::}}

\newcommand{\ism}{\kw{ism}}
\newcommand{\ismdiff}{\kw{Diff}_{\sig}}
\newcommand{\ismeq}{=_{\ism}}
\newcommand{\ismneq}{\neq_{\ism}}
\newcommand{\notismin}{\notin_{\ism}}
\newcommand{\ismin}{\in_{\ism}}

%operations on the trace and Annotated Query
\newcommand{\projl}[1]{\kw{\pi_{l}(#1)}}
\newcommand{\projr}[1]{\kw{\pi_{r}(#1)}}

% operations on annotated query, i.e., aq
\newcommand{\aqin}{\in_{\kw{aq}}}
\newcommand{\aqeq}{=_{\kw{aq}}}
\newcommand{\aqneq}{\neq_{\kw{aq}}}
\newcommand{\aqgeq}{\geq_{\kw{aq}}}

% operations on annotated variables, i.e., av
\newcommand{\avin}{\in_{\kw{av}}}
\newcommand{\aveq}{=_{\kw{av}}}
\newcommand{\avneq}{\neq_{\kw{av}}}
\newcommand{\avgeq}{\geq_{\kw{av}}}
\newcommand{\avlt}{<_{\kw{av}}}

% adaptivity
\newcommand{\adap}{\kw{adap}}
\newcommand{\ddep}[1]{\kw{depth}_{#1}}
\newcommand{\nat}{\mathbb{N}}
\newcommand{\natb}{\nat_{\bot}}
\newcommand{\natbi}{\natb^\infty}
\newcommand{\nnatA}{Z}
\newcommand{\nnatB}{m}
\newcommand{\nnatbA}{s}
\newcommand{\nnatbB}{t}
\newcommand{\nnatbiA}{q}
\newcommand{\nnatbiB}{r}


%%%%%%%%%%%%%%%%%%%%%%%%%%%%%%%%%%%%%%%%%%%%%%%%%%%%%%%%%%%%%%%%%%%%%%%%%%%%%%%%%%%%%%%%%%%%%%%%%%%%%%%%%%%%%%%%%%%%%%%%%%%%%%%%%%%%%%%%%%%%%%%%%%%%%%%%%%
%%%%%%%%%%%%%%%%%%%%%%%%%%%%%%%%%%%%%%%%%%%%%%%%%%%%%%%%%%%% Dynamic Program Analysis %%%%%%%%%%%%%%%%%%%%%%%%%%%%%%%%%%%%%%%%%%%%%%%%%%%%%%%%%%%%%%%%
%%%%%%%%%%%%%%%%%%%%%%%%%%%%%%%%%%%%%%%%%%%%%%%%%%%%%%%%%%%%%%%%%%%%%%%%%%%%%%%%%%%%%%%%%%%%%%%%%%%%%%%%%%%%%%%%%%%%%%%%%%%%%%%%%%%%%%%%%%%%%%%%%%%%%%%%%%

%%%%%%%%%%%%%%%%%%%%%%%%%%%%%%%%% Execution Based Dependency, Graph and Adaptivity 
\newcommand{\paths}{\mathcal{PATH}}
\newcommand{\walks}{\mathcal{WK}}
\newcommand{\progwalks}{\mathcal{WK}^{\kw{prog}}}

\newcommand{\len}{\kw{len}}
\newcommand{\lvar}{\mathbb{LV}}
\newcommand{\avar}{\mathbb{AV}}
\newcommand{\qvar}{\mathbb{QV}}
\newcommand{\qdep}{\mathsf{DEP_{q}}}
\newcommand{\vardep}{\mathsf{DEP_{var}}}
\newcommand{\avdep}{\mathsf{DEP_{\av}}}
\newcommand{\finitewalk}{\kw{fw}}
\newcommand{\pfinitewalk}{\kw{fwp}}
\newcommand{\dep}{\mathsf{DEP}}

\newcommand{\llabel}{\iota}
\newcommand{\entry}{\kw{entry}}
\newcommand{\tlabel}{\mathbb{TL}}

\newcommand{\traceG}{\kw{{G_{trace}}}}
\newcommand{\traceV}{\kw{{V_{trace}}}}
\newcommand{\traceE}{\kw{{E_{trace}}}}
\newcommand{\traceF}{\kw{{Q_{trace}}}}
\newcommand{\traceW}{\kw{{W_{trace}}}}

%%%%%%%%%%%%%%%%%%%%%%%%%%%%%%%%%%%%%%%%%%%%%%%%%%%%%%%%%%%%%%%%%%%%%%%%%%%%%%%%%%%%%%%%%%%%%%%%%%%%%%%%%%%%%%%%%%%%%%%%%%%%%%%%%%%%%%%%%%%%%%%%%%%%%%%%%%
%%%%%%%%%%%%%%%%%%%%%%%%%%%%%%%%%%%%%%%%%%%%%%%%%%%%%%%%%%%% Static Program Analysis %%%%%%%%%%%%%%%%%%%%%%%%%%%%%%%%%%%%%%%%%%%%%%%%%%%%%%%%%%%%%%%%
%%%%%%%%%%%%%%%%%%%%%%%%%%%%%%%%%%%%%%%%%%%%%%%%%%%%%%%%%%%%%%%%%%%%%%%%%%%%%%%%%%%%%%%%%%%%%%%%%%%%%%%%%%%%%%%%%%%%%%%%%%%%%%%%%%%%%%%%%%%%%%%%%%%%%%%%%%

%Static Adaptivity Definition:
\newcommand{\flowsto}{\kw{flowsTo}}
\newcommand{\live}{\kw{RD}}

%Analysis Algorithms and Graphs
\newcommand{\weight}{\mathsf{W}}
\newcommand{\green}[1]{{ \color{green} #1 } }

\newcommand{\func}[2]{\mathsf{AD}(#1) \to (#2)}
\newcommand{\varCol}{\bf{VetxCol}}
\newcommand{\graphGen}{\bf{FlowGen}}
\newcommand{\progG}{\kw{{G_{prog}}}}
\newcommand{\progV}{\kw{{V_{prog}}}}
\newcommand{\progE}{\kw{{E_{prog}}}}
\newcommand{\progF}{\kw{{Q_{prog}}}}
\newcommand{\progW}{\kw{{W_{prog}}}}
\newcommand{\progA}{A_{\kw{prog}}}

\newcommand{\midG}{\kw{{G_{mid}}}}
\newcommand{\midV}{\kw{{V_{mid}}}}
\newcommand{\midE}{\kw{{E_{mid}}}}
\newcommand{\midF}{\kw{{Q_{mid}}}}



\newcommand{\sccgraph}{\kw{G^{SCC}}}
\newcommand{\sccG}{\kw{{G_{scc}}}}
\newcommand{\sccV}{\kw{{V_{scc}}}}
\newcommand{\sccE}{\kw{{E_{scc}}}}
\newcommand{\sccF}{\kw{{Q_{scc}}}}
\newcommand{\sccW}{\kw{{W_{scc}}}}


\newcommand{\visit}{\kw{visit}}

\newcommand{\flag}{\kw{F}}
\newcommand{\Mtrix}{\kw{M}}
\newcommand{\rMtrix}{\kw{RM}}
\newcommand{\lMtrix}{\kw{LM}}
\newcommand{\vertxs}{\kw{V}}
\newcommand{\qvertxs}{\kw{QV}}
\newcommand{\qflag}{\kw{Q}}
\newcommand{\edges}{\kw{E}}
\newcommand{\weights}{\kw{W}}
\newcommand{\qlen}{\len^{\tt q}}
\newcommand{\pwalks}{\mathcal{WK}_{\kw{p}}}

\newcommand{\rb}{\mathsf{RechBound}}
\newcommand{\pathsearch}{\mathsf{AdaptSearch}}

%program abstraction
\newcommand{\abst}[1]{\kw{abs}{#1}}
\newcommand{\absexpr}{\abst{\kw{expr}}}
\newcommand{\absevent}{\stackrel{\scriptscriptstyle{\alpha}}{\event{}}}
\newcommand{\absfinal}{\abst{\kw{final}}}
\newcommand{\absinit}{\abst{\kw{init}}}
\newcommand{\absflow}{\abst{\kw{trace}}}
\newcommand{\absG}{\abst{\kw{G}}}
\newcommand{\absV}{\abst{\kw{V}}}
\newcommand{\absE}{\abst{\kw{E}}}
\newcommand{\absF}{\abst{\kw{F}}}
\newcommand{\absW}{\abst{\kw{W}}}
\newcommand{\locbound}{\kw{locb}}
\newcommand{\absdom}{\mathcal{ADOM}}
\newcommand{\inpvar}{\mathcal{VAR}_{\kw{in}}}
\newcommand{\grdvar}{\mathcal{VAR}_{\kw{guard}}}


\newcommand{\absclr}{{\kw{Tclosure}}}
\newcommand{\varinvar}{{\kw{Vinvar}}}
\newcommand{\init}{\kw{init}}
\newcommand{\constdom}{\mathcal{SMBCST}}
\newcommand{\dcdom}{\mathcal{DC}}
\newcommand{\reset}{\kw{re}}
\newcommand{\resetchain}{\kw{rechain}}
\newcommand{\inc}{\kw{inc}}
\newcommand{\dec}{\kw{dec}}




%%%%%%%%%%%%%%%%%%%%%%%%%%%%%%%%%%%%%%%%%%%%%%%%%%%%%%%%%%%%%%%%%%%%%%%%%%%%%%%%%%%%%%%%%%%%%%%%%%%%%%%%%%%%%%%%%%%%%%%%%%%%%%%%%%%%%%%%%%%%%%%%%%%%%%%%%%
%%%%%%%%%%%%%%%%%%%%%%%%%%%%%%%%%%%%%%%%%%%%%%%%%%%%%%%%%%%%%%%%%%%%%%% author comments in draft mode %%%%%%%%%%%%%%%%%%%%%%%%%%%%%%%%%%%%%%%%%%%%%%%%%%%%%%%%%%%%%%%%%%%%%%
%%%%%%%%%%%%%%%%%%%%%%%%%%%%%%%%%%%%%%%%%%%%%%%%%%%%%%%%%%%%%%%%%%%%%%%%%%%%%%%%%%%%%%%%%%%%%%%%%%%%%%%%%%%%%%%%%%%%%%%%%%%%%%%%%%%%%%%%%%%%%%%%%%%%%%%%%%

\newif\ifdraft
%\draftfalse
\drafttrue

\ifdraft
% Jiawen
\newcommand{\jl}[1]{\textcolor[rgb]{.00,0.00,1.00}{[JL: #1]}}
\newcommand{\jlside}[1]{\marginpar{\tiny \sf \textcolor[rgb]{.00,0.80,0.00}{[jl: #1]}}}
% Deepak
\newcommand{\dg}[1]{\textcolor[rgb]{.00,0.80,0.00}{[DG: #1]}}
\newcommand{\dgside}[1]{\marginpar{\tiny \sf \textcolor[rgb]{.00,0.80,0.00}{[DG: #1]}}}
% Marco
\newcommand{\mg}[1]{\textcolor[rgb]{.80,0.00,0.00}{[MG: #1]}}
\newcommand{\mgside}[1]{\marginpar{\tiny \sf \textcolor[rgb]{.80,0.00,0.00}{[MG: #1]}}}
% Weihao
\newcommand{\wq}[1]{\textcolor[rgb]{.00,0.80,0.00}{[WQ: #1]}}
\newcommand{\wqside}[1]{\marginpar{\tiny \sf \textcolor[rgb]{.00,0.80,0.00}{[WQ: #1]}}}
\else
\newcommand{\mg}[1]{}
\newcommand{\mgside}[1]{}
\newcommand{\wq}[1]{}
\newcommand{\wqside}[1]{}
\newcommand{\rname}[1]{$\textbf{#1}$}
\fi

\newcommand{\THESYSTEM}{\textsf{AdaptFun}}

%%%%%%%%%%%%%%%%%%%%%%%%%%%%%%%%%%%%%%%%%%%%
\makeatletter

\makeatother

\usepackage{babel}
\begin{document}

\title{{\em Dissertation Prospectus}
\\  A Full-Spectrum Program Analysis for Adaptivity of Adaptive Data Analysis, 
\\ with Generalization on Program Resource Cost}

\author{Jiawen Liu\\ Department of Computer Science, Boston University}
\maketitle
\begin{abstract}
Data analyses are usually designed to identify some property of the population from which the data are drawn, 
generalizing beyond the specific data sample. For this reason, data analyses are often designed in a way that guarantees that they produce a low generalization error.
 That is, they are designed so that the result of a data analysis run on sample 
 data does not differ too much from the result one would achieve by running the analysis over the entire population. 
 
 An adaptive data analysis can be seen as a process composed of multiple queries interrogating some data, where the choice of which query to run next may rely on the results of previous queries. 
 The generalization error of individual query/analysis can be controlled by using an array of well-established statistical techniques.
 However, when queries are arbitrarily composed, the different errors can propagate through the chain of different queries and bring high generalization errors. 
 To address this issue, data analysts are designing several techniques that not only guarantee bounds on the generalization errors of single queries, but also guarantee bounds on the generalization error of the composed analyses. 
 The choice of which of these techniques to use, 
 often depends on the chain of queries that an adaptive data analysis can generate.
 Specifically, the total number of queries and the depth of the chain of queries is of great significance 
 to guarantee the generalization error, 
 when the composed data analyses are adaptive. 
 So in order to give a precise guarantee of generalization error
 for the program,
 I'm interested in analyzing the depth of the chain of queries in a program, i.e., the program's \emph{adaptivity} property.
 % Gap
 % Unfortunately, this depth which relies on the program(implementation) itself is costly in human efforts, and how to statically obtain this information is not well studied to support data analysts.

 In this proposal, I firstly focus on formalizing and analyzing the intuitive \emph{adaptivity} property for 
 the adaptive data analysis program
 and present 
 my full-spectrum \emph{adaptivity} analysis framework.
 Next, based on the implementation and experimental results of my \emph{adaptivity} analysis framework, 
 I propose three significant 
 further features can be improved in this framework.
 % and plan to finish the improvement 
 % before the final defense.
Then according to the connection between the \emph{adaptivity} and program's resource cost,
I propose 
 % I propose extensions of this analysis with improved techniques, 
 % and 
an accurate full-spectrum program resource cost analysis via
the generalization of my \emph{adaptivity} analysis framework.
% plan to finish the design and implementation in the thesis.
In the end, 
I propose an interesting further work on solving the 
CFL-Reachability problem by reducing it into my \emph{adaptivity} analysis framework, 
based on observing the similarities between them.
 % onto general program's resource cost analysis,
 % .
  
\end{abstract}
\tableofcontents{}

\section{Introduction to Adaptivity Analysis}
\label{sec:introduction}
% % %%%% Benefit of reasoning about 
% \subsection{Motivation of Reasoning about Adaptivity}
% % 
% In Section~\ref{sec:intro-background},
% I introduce the background and limitation of the 
% adaptive data analysis, 
% and the motivation of reasoning about the \emph{adaptivity} quantity property 
% for adaptive data analysis.
% % in Section~\ref{sec:intro-motivation}.
% % analyzing 
% In order to analyze this quantity property for the adaptive data analysis, there are 3 challenges
% % problems encountered.
% introduced.
% % I introduce these three problems
% % and the full-spectrum analysis methodologies developed according to these problems 
% Targeting to the three challenges, I introduce the methodologies 
% of the program analysis framework for the adaptive data analysis's adaptivity property
% accordingly, in Section~\ref{sec:intro-adapt}.
% Concretely, 
% % the full-spectrum 
% this analysis framework is developed through the language formalization,
% the execution-based analysis, and the static-based program analysis.
% %
% Based on the implementation and experimental results on this analysis framework, 
% I propose three significant 
% further features can be improved 
% % in the analysis methodologies 
% in Section~\ref{sec:intro-improve}, 
% and plan to finish these improvements 
% before the final defense.
% %
% % Next, based on the implementation and experimental results, I proposed two significant 
% % further features can be improved for my analysis framework, and plan to finish the improvement 
% % before the final defense.
% %
% Then, in Section~\ref{sec:intro-cost}, through two observations, 
% I introduce the motivations and methodologies
% for 
% % the accurate 
% analyzing program's \emph{non-monotonic} quantitative property accurately.
% with implicit cost decreases.
% \\
% 1. traditional program's resource cost analysis failed to consider the case where the program's cost could decrease 
% implicitly, 
% \\
% 2. and 
% % when there isn't a dependency relation between variables.
% the resource consumption during the program 
% execution increases and particularly decreases implicitly in the same way as the program's adaptivity, 
% % Specifically, in line 5 
% % where the list is re-written and the heap consumption is decreased implicitly. 
% % This implicit decrease 
% % of the cost works the same as the program's adaptivity decreases.
% I am interested in improving the accuracy of the program's general resource cost analysis
% by 
% % onto the program's resource cost analysis. 
% % Use this framework,
% Through the generalized \emph{adaptivity} analysis framework.
% I will give
% a more accurate resource cost estimation by taking the program's implicit resource cost into consideration, comparing 
% to the worst case cost analysis in a traditional way.
% \paragraph{Background and Motivation\todo{Rewrite into Program Analysis of the Quantitative Property}}
% \label{sec:intro-background}
Program analysis analyzes the behaviors of a computer program\wq{computer programs with respect to some properties, such as correctness, by analyzing the execution correctness behavior  } .
One of the most significant behaviors with respect to programs is the execution correctness behavior, which determines whether a program executes correctly without getting stuck because of bugs.
 The execution correctness can be proved \wq{prove a behaviour?} by showing the functional correctness property of the \wq{the?} program with the help of some formal verification methods such as type system\wq{s?} and program logic.
 Much attention in the programming language community has focused on the functional correctness property, while not enough effort was put on, a variety of those non-function properties, considering its wide potential applications in modern society. \wq{i think you can remove the next sentence.}
 A few decades ago, computer programs were only used for research or business purpose and the expectation of computer programs was just to execute correctly without bugs. 
 
 \highlight{
However, the expectation has also changed along with the popularity of mobile devices and wide applications of big data. 
To provide people with high-quality
 service through mobile devices,
 it is not enough for these programs running on mobile phones or those programs handling big data, 
 to just run without bugs. 
 These programs are also expected to run efficiently, produce accurate outputs, do not leak the data, etc.,
 when running on mobile device and handling the big data.
%  \\
 In this sense, the non-functional properties
 come into play in the new era.\wq{I think you can remove this paragraph.}
 }
%  We think programs on mobile devices are of great significance in today's life.
% %  To provide people with high-quality
% %  service through mobile devices,
% we choose to study the non-functional properties of programs,
% specifically the reachability quantitative property in this proposal.
% Skeleton:
% Importance of the Program Execution Property in different areas.


% In Machine Learning Area, the Adaptivity Quantity is significant
% ==> Major Work I
% \\
\highlight{
For a large amount of mobile devices applications
such as video or music player, game, shopping, etc.,
a high-quality
service \wq{which service?}
aims to provide the users with accurate personalized services.\wq{Previous sentence is not easy to follow.}
% machine learning algorithm analysis results over data,
Accurate personalized services relies heavily on the\wq{the?} data analysis results.
% These analysis results are used to provide the users with accurately personalized services.
This brings my attention on\wq{(to?)} the
% The 
first non-functional behavior of the\wq{remove the?} data analysis programs, \wqside{the first behaviour of programs, it sounds weird by using the first? then, what is the second?}
i.e., the accuracy quality of the\wq{do you specify certain result, if so, use the} data analysis results.
% in their machine learning algorithms.
% comes from the machine learning area which is popular and widespread applied in our daily mobile life.
% In this area,
% The first execution property 
% the data collected from a large number of mobile users also deserves our attention.
We look at \wq{the scenario?} data analysis which analyzes sample data to get some generalized results
% with respect to the population
on the population data
where the sample is drawn.
Then, these generalized results are used to provide
users who come from the unknown population with personalized services.
Some users occasionally receive inaccurate services,
because they are not in the sample data but still provided with the services by using the generalized analysis results.
In these situations, the generalization error comes into play. 
It measures the difference of the analysis results between the sample data and population data, 
% The generalization error measures the difference of property from the sample data and population, 
showing the reliability of the data analysis and reflecting the quality of the service based on it.
% of showing the true properties of the population. 
High generalization error makes the analysis results
%  of these programs 
not reliable.
This generalization error of a data analysis program over sample data with respect to the population
is one of the important program's reachability quantitative properties \wq{of programs?}.
% It could be useful to have the program 
% This generalization error 
Fortunately, studies found that some quantitative properties of data analysis programs can help to control the generalization error, especially when the data analysis is adaptive.
These properties are the first reachability quantitative properties we are interested in
analyzing \wq{in the dissertation}.
% \\
}
% We use the static analysis technique to best exploit the benefit of these non-functional quantitative properties in resource usage and data analysis. However, to fully utilize the aforementioned benefits of the static analysis on these quantitative properties, the appropriate implementation is inevitable. To this end, we also take one step in algorithmizing a refinement type and effect system, which is used to statically estimate the properties of resource usage. In precise, the resource is the evaluation cost of programs in this proposal.

% ==> Major Work II
% In Resource Cost Analysis Area, the Reachability-Bound is significant.
% \\
\highlight{
A high-quality
service in people's mobile device life\wq{mobile device life sounds ...} does not
just mean the accuracy of service but also
the performance, the\wq{remove the?} privacy, etc., of the service. 
% non-functional properties on resource usage, one of the most useful properties concerning mobile devices. 
% Suppose we are playing an online game on our smartphones, what do we care about? 
% We care about the performance of the game, in another word, if it runs fast.
% We care about whether the game crashed due to being out of memory, and so on.
Suppose we are playing an online game on our smartphones, what do we care about? 
In games such as racing, 
we care about the performance of the game, in another word, if it runs fast, if it crashes due to being out of memory, and so on.
% We care about whether the game crashed due to being out of memory, and so on.
In some other strategy games, we also care about whether my strategy is leaked to other party\wq{parties}, i.e., the privacy quality of the game.
% \\
% Additionally, resource usage is the key for embedded systems or wearable devices.
% In this sense, the study of the non-functional properties of resource usage is of great practical value.
% We observe that the non-functional properties of resource usage or data leakage are usually
% related to two aspects, reachability and quantity.
% \\
We observe that performance of the\wq{remove} program\wq{s} is usually related to resource cost, which is
% to how fast the program runs, which largely 
mainly determined by
whether some pieces of code are executed and how times these codes\wq{"Code" in the sense of "computer code" is a mass noun, and so has no plural. It would be "lines of code". In other senses (such as "code" meaning an encrypted system of communication or an encrypted message), it is a countable noun and can be pluralised as normal. } are executed.
From the same perspective, we observe that whether the data is leaked is also related to whether certain pieces of the program code are executed,
and how many times these codes are executed.
% \\
This brings my attention to another two non-functional properties, the reachability and execution times of the certain program codes.
The two properties combined is a reachability quantitative property because it has both the reachable and quantitative aspects.\wq{previous sentence is not clear. }
% \todo{the program's resource cost and the data leakage}.
% \\
% Providing information 
Specifying certain bounds
on this reachability quantitative property
can help to control the usage of some resources and the data\wq{the data? which data?}.
%  and improve the program reliability,
% or specify certain bounds on the target resource to 
In this sense, it can help to guarantee both the performance and safety properties for the high-quality services\wq{service is plural?} on mobile device\wq{s}.
% For instance, an update on a mobile app does not significantly slow down the performance
% of this update and will not use resources exceeding certain bounds specified before.
}

% We use combinations of 
% We combine the
% execution-based and static-based
\highlight{We design new program
analysis frameworks\wq{you have two frameworks?}
%  into the new
% analysis frameworks 
to best exploit the benefit of these reachability quantitative properties in data analysis and resource usage.
However, to fully utilize the aforementioned benefits of the new analysis frameworks on these reachability quantitative properties, 
the appropriate implementation is inevitable. 
To this end, we also take one step in algorithmizing and implementing the program analysis frameworks,
% through
% program analysis framework, 
which is used to automatically estimate these properties.\wq{estimate a property? maybe estimate is not a right word here}
}

Last but not least, these reachability quantitative properties are not limited to static analysis.
We are not only interested in programs implementing the\wq{remove the?} specific algorithm\wq{s?}, but also willing to study these lower bound and upper bound of the algorithm\wq{s} itself\wq{themselves}.
% To this end, we use the formal verification method to study standard algorithms such as sorting and searching in a comparison-based computation model.

% \todo{
% In program quantitative property analysis area, the methodologies are mainly based on
% two different kinds of analysis techniques, type-system-based and the data-flow/control-flow analysis based.
% % There are mainly two categories of methodology in the static program resource cost analysis areas, 
% % through type-system based and data-flow/control-flow analysis based. 
% % They can be summarized as follows, but to the best of my knowledge,
% % all these works in the two categories fail to recognize the case where program resource consumption is decreased implicitly.
% \paragraph*{Type-System Based}
% Existing
% static program analysis based type-system is mainly through 
% effect systems, 
% % control-flow analysis, and data-flow analysis~\cite{ryder1988incremental}. 
% % The idea of statically estimating a sound upper bound for the adaptivity from the semantics is indirectly inspired by prior 
% Previous work on cost analysis via effect systems~\cite{cciccek2017relational,radivcek2017monadic,qu2019relational} statically estimating a sound upper bound for program's cost accumulatingly.
% % The idea of defining adaptivity using data flow is inspired by the work of graded 
% Hoare logic~\cite{gaboardi2021graded}, and amortized type system~\cite{hoffmann_jost_2022}.
% %
% In these systems, the cost is accumulating through the type of induction. 
% The only way to save the cost into the potential
% type, as in~\cite{GustafssonEL05} and \cite{hoffmann_jost_2022}, 
% is through explicit abstraction or data structure de-allocation.
% That is to say, they cannot deal with the case where the cost (for example the adaptivity) decreases when there isn't a dependency relation between variables.
% \paragraph*{Data-flow/Control-flow Analysis Based}
% Existing static program analysis works via the control flow or data flow analysis 
% in program resource cost analysis 
% mainly falls into two areas, the program complexity analysis, and worst case execution time analysis. 
% They are focusing on analyzing the cost of the entire program. 
% The techniques are based on
% type system~\cite{CicekBG0H17, RajaniG0021}, Hoare logic~\cite{CarbonneauxHS15}, abstract interpretation~\cite{GustafssonEL05, HumenbergerJK18},
% invariant generation through cost equations or ranking functions~\cite{BrockschmidtEFFG16,AlbertAGP08,AliasDFG10,Flores-MontoyaH14}
% or a combination of program abstraction and invariant inferring~\cite{GulwaniZ10, SinnZV17, GulwaniJK09}.
% In general, these techniques give the approximated upper bound of the program's total running time or resource cost.
% However, they failed to consider the case where the program's cost could decrease when there isn't a dependency relation between variables.
% \\
% While some work in paper [][][]in the context of memory usage for specific models of garbage collection [5,8,12],
% they don't give a generic framework to estimate the non-cumulative quantitative property.
% \\
% The work in paper "Non-cumulative Resource Analysis" gave a generic resource analysis framework for a today’s imperative language enriched with instructions to acquire and release resources. 
% However, they failed in the path-sensitive case. Their method is also imprecise in the sense that they over-approximate the
% set of acquire instructions globally for the local execution location.
% }


% \subsection{Proposal Outline}
% \label{sec:intro-outline}
% This proposal 
% \paragraph{Automated Program Analysis Framework for Adaptive Data Analysis (In Improvements)}
% \label{sec:intro-adapt}

% \paragraph{Path-Sensitive Reachability-Bound Analysis (In Preparation)}
% \label{sec:intro-reachability}

% \paragraph{Towards Accurate Program Non-Monotonic Quantitative Property Analysis (In Preparation)}
% \label{sec:intro-cost}
% Moving towards the area of program's quantitative property analysis,
% % Then, motivated by the two following aspects, 
% there are two interesting observations as follows.
% % I am interested 
% These two observations motivated me in 
% % improving the accuracy of the program's general resource cost analysis
% improving the accuracy of the program's general resource cost analysis
% by generalizing this \emph{adaptivity} analysis framework.
% \begin{itemize}
% \item 
% % In the traditional program's resource cost and quantitative property analysis,
% There are two research areas in the traditional program's resource cost and quantitative property analysis.
% % of program cost analysis, 
% One area is type-system based and the other is data-flow/control-flow analysis based. 
% In the type-system design-based areas (\cite{GustafssonEL05}, \cite{hoffmann_jost_2022}), 
% the analysis technique requires explicit abstraction or data structure de-allocation in order to save or reduce the cost.
% The
% works in both of these two areas fail to recognize the case where program resource consumption or quantitative properties 
% are decreased implicitly or increased \emph{non-monotonically}.
% \item This kind of resource consumption or quantitative properties during the program 
% execution increase and particularly decrease implicitly in the same way as the program's adaptivity. 
% This is explained in detail through an example in Section~\ref*{sec:generalization}.
% \end{itemize}
% Based on the observations above, 
% I plan to develop
% an accurate program \emph{non-monotonic} quantitative property analysis framework through generalizing 
% my \emph{adaptivity} analysis framework.
% This framework can give more accurate cost bound than traditional worst-case resource cost estimation methods,
% by taking the program's implicit resource cost decreasing into consideration.
% compared 
% to the worst-case cost analysis in the traditional way.

\paragraph*{Proposal Structure}
To sum up, this proposal covers the following topic in each part.
\begin{enumerate}
 \item \redd{PART I}: A program analysis framework for analyzing the adaptivity for the program that implements an adaptive data analysis.
 \item \redd{PART II} A path-sensitive reachability-bound analysis algorithm for computing the program's accurate reachability-bound.
% \item A while-like language extended with query request feature, named {\tt Query While} Language, 
% used to implement 
% the adaptive data analysis in Section~\ref{sec:language}.
% \item A formal adaptivity model through execution-based adaptivity analysis in Section~\ref{sec:dynamic}.
% \item A static program analysis algorithm named {\THESYSTEM} in Section~\ref{sec:static}.
% \item Three proposed further features to be improved for the adaptivity analysis framework,
% % based on the 
% % % formal adaptivity model and {\THESYSTEM}, 
% % full-spectrum 
% in Section~\ref{sec:furthers}.
% % presented in Section~\ref{sec:language},~\ref{sec:dynamic} and~\ref{sec:static},
% This proposed work is planned to be done before the final defense.
% \item A proposed automated program non-monotonic quantitative quantity analysis framework in \redd{PART III}.
% generalized from {\THESYSTEM} in Section~\ref{sec:generalization}. 
% The analysis framework design is expected to be done with the implementation start off before the final defense.
% \item A proposed plan for solving the CFL-reachability problem via reduction into the {\THESYSTEM} framework in Section~\ref{sec:cfl_reduction},
% expected to start before final defense and developing further after.
\end{enumerate}
\cleardoublepage



\section{The Query Language for Adaptive Data Analysis Program}
\label{sec:language}
In this chapter, we formally introduce the language we will focus on for writing data analyses.  
This is a simple loop language with some primitives for calling queries. 
After defining the syntax of the language and showing an example, we will define its trace-based operational semantics. 
This is the main technical ingredient we will use to define the program's adaptivity.

\subsection{Syntax of Query While Language}
\label{sec:language-syntax}
In this section, we formally introduce the language we will focus on for writing data analyses.  This is a standard while language with some primitives for calling queries. After defining the syntax of the language and showing an example, we will define its trace-based operational semantics. This is the main technical ingredient we will use to define the program's adaptivity.
% We will conclude this section by discussing the limitation of this language with respect to static analysis for adaptivity.

The syntax is shown as follows,
\[
\begin{array}{llll}
\mbox{Arithmetic Operators} 
& \oplus_a & ::= & + ~|~ - ~|~ \times 
%
~|~ \div ~|~ \max ~|~ \min\\  
% \mbox{Unary Operators} 
% & \oplus_a & ::= & + ~|~ - ~|~ \times 
% %
% ~|~ \div \\  
\mbox{Boolean Operators} 
& \oplus_b & ::= & \lor ~|~ \land
\\
%
\mbox{Relational Operators} 
& \sim & ::= & < ~|~ \leq ~|~ == 
\\  
%
\mbox{Label} 
& l & \in & \mathbb{N} \cup \{\lin, \lex\} 
\\ 
%
\mbox{Arithmetic Expression} 
& \aexpr & ::= & 
n ~|~ {x} ~|~ \aexpr \oplus_a \aexpr  
% \\
% &  &  & 
 ~|~ \elog \aexpr  ~|~ \esign \aexpr
\\
%
\mbox{Boolean Expression} & \bexpr & ::= & 
%
\etrue ~|~ \efalse  ~|~ \neg \bexpr
 ~|~ \bexpr \oplus_b \bexpr
%
~|~ \aexpr \sim \aexpr 
\\
%
\mbox{Expression} & \expr & ::= & v ~|~ \aexpr ~|~ \bexpr ~|~ [\expr, \dots, \expr]
\\  
%
\mbox{Value} 
& v & ::= & { n ~|~ \etrue ~|~ \efalse ~|~ [] ~|~ [v, \dots, v]}  
\\
%
\highlight{\mbox{Query Expression}  }
& {\qexpr} & ::= 
& \highlight{ \qval ~|~ \aexpr ~|~ \qexpr \oplus_a \qexpr ~|~ \chi[\aexpr]}
\\
%
\highlight{\mbox{Query Value} }& \qval & ::= 
& \highlight{n ~|~ \chi[n] ~|~ \qval \oplus_a  \qval ~|~ n \oplus_a  \chi[n]
    ~|~ \chi[n] \oplus_a  n}
    \\
% &&& \text{\mg{I don’t think this is what I want. Isn’t $\chi[n+1]$ a query value?}}\\
% &&& \text{\mg{What about $\chi[i] + \chi[i] + \chi[i]$? They are not in the grammar}}
% \\
% &&& \text{\jl{ $\chi[i] + \chi[i] + \chi[i]$ and $\chi[n+1]$ are both expressions, they will be evaluated to a value 
% }}
% \\%
\mbox{Labeled Command} 
& {c} & ::= &   [\assign {{x}}{ {\expr}}]^{l} ~|~  \highlight{[\assign {{x} } {{\query(\qexpr)}}]^{l}}
~|~ {\ewhile [ \bexpr ]^{l} \edo {c} }
\\
&&&
~|~ {c};{c}  
~|~ \eif([\bexpr]{}^l , {c}, {c}) 
~|~ [\eskip]^l
\\ 
\mbox{Event} 
& \event & ::= & 
    ({x}, l, v, \bullet) ~|~ ({x}, l, v, \qval)  ~~~~~~~~~~~ \mbox{Assignment Event} \\
&&& ~|~(\bexpr, l, v, \bullet)   ~~~~~~~~~~~~~~~~~~~~~~~~~~~~~~~~~~ \mbox{Testing Event}
\\
\mbox{Trace} & \trace
& ::= & [] ~|~ \trace :: \event
\\
\end{array}
\]
% \[
% \begin{array}{llll}
% \mbox{Arithmetic Operators} 
% & \oplus_a & ::= & + ~|~ - ~|~ \times 
% %
% ~|~ \div ~|~ \max ~|~ \min\\  
% % ~|~ \div \\  
% \mbox{Boolean Operators} 
% & \oplus_b & ::= & \lor ~|~ \land
% \\
% %
% \mbox{Relational Operators} 
% & \sim & ::= & < ~|~ \leq ~|~ == 
% \\  
% %
% \mbox{Arithmetic Expression} 
% & \aexpr & ::= & 
% n ~|~ {x} ~|~ \aexpr \oplus_a \aexpr  
%  ~|~ \elog \aexpr  ~|~ \esign \aexpr
% \\
% %
% \mbox{Boolean Expression} & \bexpr & ::= & 
% %
% \etrue ~|~ \efalse  ~|~ \neg \bexpr
%  ~|~ \bexpr \oplus_b \bexpr
% %
% ~|~ \aexpr \sim \aexpr 
% \\
% %
% \mbox{Expression} & \expr & ::= & v ~|~ \aexpr ~|~ \bexpr ~|~ [\expr, \dots, \expr] ~|~ \highlight{\fname}
% \\  
% %
% \mbox{Value} 
% & v & ::= & { n ~|~ \etrue ~|~ \efalse ~|~ [] ~|~ [v, \dots, v]}  
% \\ 
% &&&
% \highlight
% {
% ~|~ (x_0, x_1, \ldots, x_n) := c
% }
% \\
% %
% \highlight{\mbox{Query Expression}} 
% & {\qexpr} & ::= 
% & \highlight{ \qval ~|~ \aexpr ~|~ \qexpr \oplus_a \qexpr ~|~ \chi[\aexpr]} 
% \\
% %
% \mbox{Query Value} & \qval & ::= 
% & \highlight{n ~|~ \chi[n] ~|~ \qval \oplus_a  \qval ~|~ n \oplus_a  \chi[n]
%     ~|~ \chi[n] \oplus_a  n}
% \\
% % \\%
% \mbox{Label} 
% & l & ::= & (n \in \mathbb{N} \cup \{\lin, \lex\}) ~|~ (l, n)
% \\ 
% %
% \mbox{Labeled Command} 
% & {c} & ::= &  
% \clabel{\assign{x}{\expr}}^l 
% ~|~ \clabel{\assign{x}{\query(\qexpr)}}^l
% ~|~  \clabel{\eskip}^l
% ~|~ \ewhile \clabel{\bexpr}^{l} \edo {c}
% ~|~ \eif(\clabel{\bexpr}^{l} , {c}, {c}) 
% \\ 
% &&&
% \highlight
% {
% ~|~ \clabel{\efun}^l: \fname (x_0, x_1, \ldots, x_n) := c
% ~|~ \clabel{\assign{x}{\ecall(x, e_1, \ldots, e_n)}}^l
% }
% ~|~ {c};{c}  
% \\ 
% % \\
% \mbox{Event} 
% & \event & ::= & 
%     ({x}, l, v, \bullet) ~|~ ({x}, l, v, \qval) ~|~ (\fname, l, v, \qval)  ~~~~~~~~~~~ \mbox{Assignment Event} \\
% &&& ~|~(\bexpr, l, v, \bullet)   ~~~~~~~~~~~~~~~~~~~~~~~~~~~~~~~~~~ \mbox{Testing Event}
% \\
% % &&& \text{\mg{I think it would be better to use quadruples for events, where the}}\\
% % &&& \text{\mg{first element is either a variable or a boolean expression and }}\\
% % &&& \text{\mg{the last is either a query value or some default value $\bullet$}}\\
% %
% % \mbox{Trace} & \trace
% % & ::= & \cdot | \trace \cdot \event | \trace \tracecat \trace 
% % \\
% %
% % \mbox{Trace} & \trace
% % & ::= & [] ~|~ \event:: \trace ~|~ \trace \tracecat \trace  \\
% \mbox{Trace} & \trace
% & ::= & [] ~|~ \trace :: \event\\
% % &&& \text{\mg{I don't understand why you need both :: and ++ as constructors.}}\\
% % &&& \text{\jl{Because append is to the left but we are adding element to the left in the OS}}\\
% % &&& \text{\jl{I was too sticky to the convention, it is a good idea to append to the left and just use $::$}}
% % %
% % \mbox{Event Signature} & \sig
% % & ::= & (x, l, n) | (x, l, n, \query) | (b, l, n)
% % \\
% % %
% \end{array}
% \]
For clarity, the following notations are used to represent the set of corresponding terms:
\[
\begin{array}{lll}
\mathcal{VAR} & : & \mbox{Set of Variables}  
\\ 
%
\mathcal{VAL} & : & \mbox{Set of Values} 
\\ 
%
\mathcal{QVAL} & : & \mbox{Set of Query Values} 
\\ 
%
\cdom & : & \mbox{Set of Commands} 
\\ 
%
\mathcal{LV} & : & \mbox{Set of Labeled Variables}
\\
%
\eventset  & : & \mbox{Set of Events}  
\\
%
\eventset^{\asn}  & : & \mbox{Set of Assignment Events}  
\\
%
\eventset^{\test}  & : & \mbox{Set of Testing Events}  
\\
%
\ldom  & : & \mbox{Set of Labels}  
\\
%%
\mathcal{VAL}  & : & \mbox{Set of Labeled Variables}  
\\
%%
\dbdom  & : & \mbox{{Set of Databases}} 
\\
%
{\mathcal{T}} & : & \mbox{Set of Traces}
\\
%
% \qdom = {[-1,1]} & : & \mbox{{Domain of Query Results}}\\
\qdom & : & \mbox{{Domain of Query Results}}\\
% &&\text{\mg{I don't think you need to hard code [-1,1] here}}\\
\end{array}
\]
\paragraph*{Standard Expression}
The expressions are either the standard one or the extended one.
A standard expression is
% can be 
either a standard arithmetic expression or a boolean expression, or a list of expressions.
An arithmetic expression can be a constant $n$ denoting integer, a variable $x$ from some countable set $\mathcal{VAR}$, binary operation $\oplus_a$ such as addition, product, subtraction, etc, over arithmetic expressions, and also log and sign operation. 
%
A boolean expression can be either {\tt true} or {\tt false}, basic boolean connectives such as logical negation, logical and and or denoted by $\oplus_b$, and basic comparison $sym$ between arithmetic expressions, e.g., $\leq,=,<,$ etc.
Additionally, I also introduce list in expression.
Our language supports primitives for queries, 
where a specific query is specified by a query expression $\qexpr$. 
A query expression contains the necessary information for a query request, for example, 
$\chi[\aexpr]$ represents the values at a certain index $\aexpr$ in a row $\chi$ of the database. 
Query expressions combine access to the database with other expressions, 
for example, $\chi[3] + 5$ represents a query which asks the value from the column 3 of each database raw $\chi$, adds 5 to each of these values, 
and then computes the average of these values.
\paragraph*{Query Expression}
The key extension is
%  language supports 
the primitive for queries, where a specific query is specified by a query expression $\qexpr$. 
A query expression contains the necessary information for a query request, 
for example, $\chi[\qexpr]$ represents the values at a certain index $\qexpr$ in a row $\chi$ of the database. 
When this expression is encapsulated by the symbol $\query$,
 $ \query(\chi[\qexpr]) $ computes the average value at certain index over each row of the database as follows,
 \[
  \query(\chi[\qexpr]) = \frac{1}{n}\sum\limits_{i = 0}^{n}\chi_i[\qexpr]
  \]
Query expressions combine access to the database with other expressions, 
for example, 
$\chi[3] + 5$ represents a query that asks the value from column 3 of each database raw $\chi$, 
adds 5 to each of these values, and then computes the average of these values as follows, where $n$ is 
data base $\chi$'s number of raw.
%
\[
  \query(\chi[3] + 5) = \frac{1}{n}\sum\limits_{i = 0}^{n}\chi_i[3] + 5
  \]

% the expression also includes the special variable $\chi$ representing a row of the database, and access to values at a certain index in $\chi$, as $\chi[\aexpr]$. Additionally, list over expressions is supported and $[]$ stands for the empty list. The access to elements in the list can be achieved through $x[\aexpr]$ when variable $x$ is referred to a list. The value $v$ now contains the natural number $n$, the boolean primitives $\etrue$ and $\efalse$, the special row $\chi$ and access to it $\chi[v]$, the empty list $[]$ and non-empty list $[v, \dots, v]$.
% 
% Another extension is the inter-procedure call and function definition.
% In the function define command $\clabel{\efun}^l: \fname (x_0, x_1, \ldots, x_n) := c$,
% the function body $c$ is assigned to the function of name $\fname$, $x_1, \ldots, x_n$ is the function
% arguments and the first element $x_0$ in the arguments is the return variable.
% We only support the first-order function definition and function call. 

% %
\paragraph*{Labeled Command}
 A labeled command $c$ is just a command with a label --- I assume that labels are unique, so that they can help to identify uniquely every subexpression. 
%  I have $\eskip$, assignment $\assign{x}{\expr}$, the composition of two commands $c;c$, an if statement $\eif(\bexpr, c, c)$, a while statement  $\ewhile \bexpr \edo {c} $.
 The main novelty of the syntax is the query request command $\assign{x}{\query(\qexpr)}$. 
 For instance, if a data analyst wants to ask a simple linear query which returns the first element of the row, 
 they can simply use the command $ \assign{x}{\query(\chi[1])}$ in their data analysis program.
%  \wq{Shall I distinguish command and labeled command, they are now both $c$. }
%  \jl{I'm not sure, I don't want to programmer to add the label when writing the program. The label is just added by us for analysis. but I'm worried it is too complicate if use two notations for command and labeled command }
%
% \[
% \begin{array}{llll}
% \mbox{Label} 
% & l & \in & \mathbb{N} \cup \{in, ex\} \\
% \mbox{Labeled Commands} 
% & {c} & ::= &   [\assign {{x}}{ {\expr}}]^{l} ~|~  [\assign {{x} } {{\query(\qexpr)}}]^{l}
% ~|~ {\ewhile [ \bexpr ]^{l} \edo {c} }
%  \\
%  &&&
% ~|~ {c};{c}  
% ~|~ \eif([\bexpr]{}^l , {c}, {c}) 
% ~|~ [\eskip]^l 
% \end{array}
% \]
\paragraph*{Labeled Variables}
The labeled variables and assigned variables are set of variables annotated by a label. 
We use  
%$\mathcal{LVAR} = \mathcal{VAR} \times \mathcal{L} $ 
$\mathcal{LV}$ represents the universe of all the labeled variables and 
$\avar_c \in \mathcal{P}(\mathcal{VAR} \times \mathbb{N}) \subset \mathcal{LV}$ and 
$\lvar_c \in \mathcal{P}(\mathcal{VAR} \times \mathcal{L}) \subseteq \mathcal{LV}$,
represents the the set of assigned variables and labeled variables for a labeled command $c$,
defined in Definition~\ref{def:lvar} and \ref{def:avar}.
%
% \\
$FV: \expr \to \mathcal{P}(\mathcal{VAR})$, computes the set of free variables in an expression. To be precise,
$FV(\aexpr)$, $FV(\bexpr)$ and $FV(\qexpr)$ represent the set of free variables in arithmetic
expression $\aexpr$, boolean expression $\bexpr$ and query expression $\qexpr$ respectively.
Labeled variables in $c$ is the set of assigned variables and all the free variables
showing up in $c$ with a default label $in$. 
The free variables
showing up in $c$, which aren't defined before be used, are actually the input variables of this program.
%
%
\begin{defn}[Assigned Variables (
% $\avar_{c} \subseteq \mathcal{VAR} \times \mathbb{N}$ or 
$\avar : \cdom \to \mathcal{P}(\mathcal{VAR} \times \mathbb{N})$)]
% labelled Variables 
% (
% % $\lvar_{c} \subseteq \mathcal{VAR} \times \mathbb{N}$ or 
% $\lvar : \cdom \to \mathcal{P}(\mathcal{VAR} \times \mathcal{L})$
\label{def:avar}
$$ \avar_{c} \triangleq
  \left\{
  \begin{array}{ll}
      \{{x}^l\}                   
      & {c} = [{\assign x e}]^{l} 
      \\
      \{{x}^l\}                   
      & {c} = [{\assign x \query(\qexpr)}]^{l} 
      \\
      \avar_{{c_1}} \cup \avar_{{c_2}}  
      & {c} = {c_1};{c_2}
      \\
      \avar_{{c}} \cup \avar_{{c_2}} 
      & {c} =\eif([\bexpr]^{l}, c_1, c_2) 
      \\
      \avar_{{c}'}
      & {c}   = \ewhile ([\bexpr]^{l}, {c}')
\end{array}
\right.
$$
\end{defn}
%
%
\begin{defn}[labelled Variables 
(
% $\lvar_{c} \subseteq \mathcal{VAR} \times \mathbb{N}$ or 
$\lvar : \cdom \to \mathcal{P}(\mathcal{LV})$]
\label{def:lvar}
$$
  \lvar_{c} \triangleq
  \left\{
  \begin{array}{ll}
      \{{x}^l\} \cup FV(\expr)^{in}                  
      & {c} = [{\assign x e}]^{l} 
      \\
      \{{x}^l\}   \cup FV(\qexpr)^{in}                
      & {c} = [{\assign x \query(\qexpr)}]^{l} 
      \\
      \lvar_{{c_1}} \cup \lvar_{{c_2}}  
      & {c} = {c_1};{c_2}
      \\
      \lvar_{{c}} \cup \lvar_{{c_2}} \cup FV(\bexpr)^{in}
      & {c} =\eif([\bexpr]^{l}, c_1, c_2) 
      \\
      \lvar_{{c}'} \cup FV(\bexpr)^{in}
      & {c}   = \ewhile ([\bexpr]^{l}, {c}')
\end{array}
\right.
$$
\end{defn}
%
%
%
% is a subset of the program's assigned variables, where every variable in this set is assigned by a query in the program.
% \mg{The set of query variables of a program is the set of variables set to the result of a query in the program.}\\
% In the same way, in order to 
\paragraph*{Query Variables}
Distinctively, a key definition for the extension of the query primitives 
is the set of query variables for a program $c$.
This definition is the key point to track the query requests in the Following full-spectrum adaptivity analysis.
% track the I also defined the set of query variables for a program $c$.
It is defined as the set of variables,
which are assigned by the result of a query request in the program formally in Definition~\ref{def:qvar}.
% \mg{In the next definition, why do you call it a vector? It seems that you define it as a set.}\\
% \jl{fixed}\\
%
% \begin{defn}[Query Variables ($\qvar_{c} \subseteq \mathcal{VAR} \times \mathbb{N}$)].
  % \\
\begin{defn}[Query Variables ($\qvar: \cdom \to \mathcal{P}(\mathcal{LV})$)] 
  \label{def:qvar}
Given a program $c$, its query variables 
% \mg{it seems you are missing the $_c$ subscript. Also, this is a minor point but I don't think it is a good idea to use a subscript, cannot you just use $\qvar(c)$.}
$\qvar(c)$ is the set of variables set to the result of a query in the program.
% \jl{fixed}
It is defined as follows:
{
$$
  % \qvar_{{c}} \triangleq
  \qvar(c) \triangleq
  \left\{
  \begin{array}{ll}
      % \{\}                  
      % & {c} = [{\assign x e}]^{(l, w)} 
      % \\
      % \{{x}^l\}                  
      % & {c} = [{\assign x \query(\qexpr)}]^{(l, w)} 
      % \\
      % \qvar_{{c_1}} \cup \qvar_{{c_2}}  
      % & {c} = {c_1};{c_2}
      % \\
      % \qvar_{{c_1}} \cup \qvar_{{c_2}} 
      % & {c} =\eif([\bexpr]^{l}, c_1, c_2) 
      % \\
      % \qvar_{{c}'}
      % & {c}   = \ewhile ([\bexpr]^{l}, {c}')
      \{\}                  
      & {c} = [{\assign x \expr}]^{l} 
      \\
      \{{x}^l\}                  
      & {c} = [{\assign x \query(\qexpr)}]^{l} 
      \\
      \qvar(c_1) \cup \qvar(c_2)  
      & {c} = {c_1};{c_2}
      \\
      \qvar(c_1) \cup \qvar(c_2) 
      & {c} =\eif([\bexpr]^{l}, c_1, c_2) 
      \\
      \qvar(c')
      & {c}   = \ewhile ([\bexpr]^{l}, {c}')
\end{array}
\right.
$$
}
\end{defn}
%
It is easy to see that a program $c$'s query variables is a subset of 
its labeled variables, $\qvar(c) \subseteq \lvar(c)$.
%
% \mg{In this definition as well as in others, I have the impression that you assume that the labelled variables are unique in the program. For example, it would not make sense to assign a query to the same labelled variable over and over. If this is the case, I need to make this very explicit in the paper.}
% \jl{TODO}
%
Every labeled variable in a program is unique, formally as follows with proof in Appendix~\ref{apdx:lemma_sec123}.
\begin{lem}[Uniqueness of the Labeled Variables]
  \label{lem:lvar_unique}
  For every program $c \in \cdom$ and every two labeled variables such that
  $x^i, y^j \in \lvar(c)$, then $x^i \neq y^j$.
  \[
    \forall c \in \cdom, x^i, y^j \in \mathcal{L} \sthat x^i, y^j \in \lvar(c)\implies x^i \neq y^j.
    \]
\end{lem}

\highlight{\paragraph*{Improvements through Examples}
It is expressive in two following aspects.
\begin{itemize}
  \item \textbf{Improvements from Standard While Language}
  \\
  It also extends the standard while language with query requests. 
  The general data analysis program with query requests on data  are supported in this {\tt Query While} language.
  The program can access the database through a special  interface $\chi$ encapsulated by the identifier $\query$ (for example the program below) in the new language.
  \[
    {\assign{x}{20}};
    \assign{y}{\query(\chi[2])};
    \ewhile (x < 100) \edo 
    \{
      \assign{x}{x + 1};
      \assign{y}{\query(\chi[x]*\chi[n])};
      \}\}
    \] 
%
    \item \textbf{Improvements from Previous Works}
  \\
This {\tt Query While} language is also more expressive than the language designed in previous works.
The previous language only supports the data analysis with constant number of loop iterations.
Comparing to it, in the new language design,
the general data analysis program with non-deterministic loop iterations
(for example the program below as shown in Section~\ref{sec:prework-language})
is supported.
\[
  {\assign{x}{20}};
  \assign{y}{40};
  \ewhile (x < y) \edo 
  \{
    \assign{x}{x + 1};
    \assign{y}{y - 2};
    \}\}
  \] 
Previous work does not support data analysis program with user inputs, which is supported in the new language as well.
\end{itemize}
}

\subsection{Trace-based Operational Semantics}
\label{sec:language-os}
The operational semantics is defined based on the event and trace, which are introduced firstly as follows.
% \\
\paragraph*{Event}
An event tracks useful information about each step of the evaluation, as a quadruple. Its first element is either 
an assigned variable (from an assignment command) or a boolean expression (from the guard of if or while command), follows by 
 the label associated with this event, the value evaluated either from the expression assigned to the variable,
or the boolean expression in the guard.
 The last element stores the query information, which is a query value whose default is $\bullet$. I declare event projection operator $\pi_i$ which projects the $i$th element from an event.
\[
\begin{array}{llll}
\mbox{Event} 
& \event & ::= & 
 ({x}, l, v, \bullet) ~|~ ({x}, l, v, \qval) ~~~~~~~~~~~ \mbox{Assignment Event} 
~|~(\bexpr, l, v, \bullet) 
~~~~
\mbox{Testing Event}
% \mbox{Trace} & \trace
% & ::= & [] ~|~ \trace :: \event
\end{array}
\]
% \input{event}
% To distinguish if a query's choice is affected by previous values, 
% % \jl{we need to be able to identify whether two queries are equivalent or not so that when we change the result of one query, another query is affected. For the equivalence of queries, } 
% we need to be able to identify whether two queries are equivalent or not, so that when we change the result of one query, whether or not another query is affected. 
% To define equivalence of queries,
% quite different from the equality between the evaluation results as the regular assignment results, 
% we are neither observing the syntactic equivalence between the two query expressions,
% nor two results return from the database. 
% Instead, we define the equivalence of query expression by quantifying over all values returned from the database on a certain form of query value, formally as follows.
% \begin{defn}[Equivalence of Query Expression]
% %
% \label{def:query_equal}
% % \mg{Two} \sout{2} 
% Two query expressions $\qexpr_1$, $\qexpr_2$ are equivalent, denoted as $\qexpr_1 =_{q} \qexpr_2$, if and only if
% % $$
% % \begin{array}{l} 
% % \exists \qval_1, \qval_2 \in \mathcal{QVAL} \st \forall \trace \in \mathcal{T} \st
% % (\config{\trace, \qexpr_1} \qarrow \qval_1 \land \config{\trace, \qexpr_2 } \qarrow \qval_2) 
% % \\
% % \quad \land (\forall D \in \dbdom, r \in D \st 
% % \exists v \in \mathcal{VAL} \st 
% % \config{\trace, \qval_1[r/\chi]} \aarrow v \land \config{\trace, \qval_2[r/\chi] } \aarrow v) 
% % \end{array}.
% % $$
% $$
% \begin{array}{l} 
% \forall \trace \in \mathcal{T} \st \exists \qval_1, \qval_2 \in \mathcal{QVAL} \st
% (\config{\trace, \qexpr_1} \qarrow \qval_1 \land \config{\trace, \qexpr_2 } \qarrow \qval_2) 
% \\
% \quad \land (\forall D \in \dbdom, r \in D \st 
% \exists v \in \mathcal{VAL} \st 
% \config{\trace, \qval_1[r/\chi]} \aarrow v \land \config{\trace, \qval_2[r/\chi] } \aarrow v) 
% \end{array}.
% $$
% % \mg{$$
% % \begin{array}{l} 
% % \forall \trace \in \mathcal{T} \st \exists \qval_1, \qval_2 \in \mathcal{QVAL} \st
% % (\config{\trace, \qexpr_1} \qarrow \qval_1 \land \config{\trace, \qexpr_2 } \qarrow \qval_2) 
% % \\
% % \quad \land (\forall D \in \dbdom, r \in D \st 
% % \exists v \in \mathcal{VAL} \st 
% % \config{\trace, \qval_1[r/\chi]} \aarrow v \land \config{\trace, \qval_2[r/\chi] } \aarrow v) 
% % \end{array}.
% % $$
% % }
% %
% where $r \in D$ is a record in the database domain $D$. 
% I denote by $\qexpr_1 \neq_{q} \qexpr_2$ the negation of the equivalence relation.
% % \\ 
% % where $r \in D$ is a record in the database domain $D$,
% % \mg{is $FV(\qexpr)$ being defined here? If yes, I suggest putting it in a different place, rather than in the middle of another definition.} 
% % $FV(\qexpr)$ is the set of free variables in the query expression $\qexpr$.
% % \sout{$\qexpr_1 \neq_{q}^{\trace} \qexpr_2$ is defined vice versa.}
% % \mg{As usual, we will denote by $\qexpr_1 \neq_{q}^{\trace} \qexpr_2$ the negation of the equivalence.}
% %
% \end{defn}
%
% \mg{In the next definition you don’t need the subscript e, it is clear that it is an equivalence of events by the fact that the elements on the two sides of = are events. That is also true for query expressions. Also, I am confused by this definition. What happens for two query events?}
% \\
% \jl{The last component of the event is equal based on Query equivalence, $\pi_{4}(\event_1) =_q \pi_{4}(\event_2)$.
% In the previous version, the query expression is in the third component and I defined $v \neq \qexpr$ for all $v$ that isn't a query value.}
% \begin{defn}[Event Equivalence $\eventeq$]
% Two events $\event_1, \event_2 \in \eventset$ \mg{are equivalent, \sout{is in \emph{Equivalence} relation,}} denoted as $\event_1 \eventeq \event_2$ if and only if:
% \[
% \pi_1(\event_1) = \pi_1(\event_2) 
% \land 
% \pi_2(\event_1) = \pi_2(\event_2) 
% \land
% \pi_{3}(\event_1) = \pi_{3}(\event_2)
% \land 
% \pi_{4}(\event_1) =_q \pi_{4}(\event_2)
% \]
% %
% % \sout{The $\event_1 \eventneq \event_2$ is defined as vice versa.}
% % \mg{As usual, we will denote by $\event_1 \eventneq \event_2$ the negation of the equivalence.}
% \end{defn}
% \wq{Now we can compare two events by defining the event equivalence and difference relation.}
% Now we can compare two events by defining the event equivalence and difference relation based on the query equivalence.
% \begin{defn}[Event Equivalence]
% \label{def:event_eq}
% Two events $\event_1, \event_2 \in \eventset$ are equivalent, 
% % denoted as $\event_1 \eventeq \event_2$ 
% denoted as $\event_1 = \event_2$ 
% if and only if:
% \[
% \pi_1(\event_1) = \pi_1(\event_2) 
% \land 
% \pi_2(\event_1) = \pi_2(\event_2) 
% \land
% \pi_{3}(\event_1) = \pi_{3}(\event_2)
% \land 
% \pi_{4}(\event_1) =_q \pi_{4}(\event_2)
% \]
% %
% As usual, we will denote by $\event_1 \neq \event_2$ the negation of the equivalence.
% % As usual, we will denote by $\event_1 \eventneq \event_2$ the negation of the equivalence.
% % When it is clear from the context, we omit the subscript $\kw{e}$ and use 
% % $\event_1 = \event_2$ (and $\event_1 \neq \event_2$) for event equivalent
% \end{defn}
% %
% %
% % \begin{defn}[Signature Equivalence of Events $\sigeq$]
% % Two events $\event_1, \event_2 \in \eventset$ is in \emph{signature equivalence} relation, denoted as $\event_1 \sigeq \event_2$ if and only if:
% % \[
% % \forall i \in \{1, 2, 3\} \st \pi_{\sig}(\event_1) = \pi_{\sig}(\event_2) 
% % \]
% % The $\event_1 \signeq \event_2$ is defined as vice versa.
% % \end{defn}
% %
% % \begin{defn}[Events Different up to Value ($\diff$)]
% % Two events $\event_1, \event_2 \in \eventset$ \mg{are \sout{is}} \emph{Different up to Value}, 
% % denoted as $\diff(\event_1, \event_2)$ if and only if:
% % \[
% % \pi_1(\event_1) = \pi_1(\event_2) 
% % \land 
% % \pi_2(\event_1) = \pi_2(\event_2) 
% % \land 
% % \pi_3(\event_1) \neq_q \pi_3(\event_2)
% % \]
% % \end{defn}
% \begin{defn}[Events Different up to Value ($\diff$)]
% Two events $\event_1, \event_2 \in \eventset$ are \emph{Different up to Value}, 
% denoted as $\diff(\event_1, \event_2)$ if and only if:
% \[
% \begin{array}{l}
% \pi_1(\event_1) = \pi_1(\event_2) 
% \land 
% \pi_2(\event_1) = \pi_2(\event_2) \\
% \land 
% \big(
% (\pi_3(\event_1) \neq \pi_3(\event_2)
% \land 
% \pi_{4}(\event_1) = \pi_{4}(\event_2) = \bullet )
% % \qquad \qquad 
% \lor 
% (\pi_4(\event_1) \neq \bullet
% \land 
% \pi_4(\event_2) \neq \bullet
% \land 
% \pi_{4}(\event_1) \neq_q \pi_{4}(\event_2)) 
% \big)
% \end{array}
% \]
% \end{defn}
% %
% %
\paragraph*{Trace}
A trace $\trace \in \mathcal{T} $ is a list of events, 
collecting the events generated along the program execution. $\mathcal{T} $ represents the set of traces. There are some useful operators: the trace concatenation operator $\tracecat: \mathcal{T} \to \mathcal{T} \to \mathcal{T}$, combines two traces.
The belongs to operator $\in : \eventset \to \mathcal{T} \to \{\etrue, \efalse \} $ and its opposite $\not\in$
express whether or not an event belongs to a trace.
Another operator $\llabel : \mathcal{T} \to \mathcal{VAR} \to \{\mathbb{N}\}\cup \{\bot\}$,
takes a trace and a variable as input and returns the label of the latest assignment event which assigns value to that variable. 
% I also have the operator $\tlabel : \mathcal{T} \to \ldom$, which gives the set of labels in every event belonging to a trace. 
% The full definitions of these above operators can be found in the appendix.
% \[
% \begin{array}{llll}
% \mbox{Trace} & \trace
% & ::= & [] ~|~ \trace :: \event
% \end{array}
% \]
%
A trace can be regarded as the program history, which records queries asked by the analyst during the execution of the program. I collect the trace with a trace-based operational semantics based on transitions of the form $ \config{c, \trace} \to \config{c', \trace'} $. It states that a configuration $\config{c, \trace}$, which consists of a command $c$ to be evaluated and a starting trace $\trace$, evaluates to another configuration with the trace updated along with the evaluation of the command $c$ to the normal form of the command $\eskip$.
% \jl{I introduce some operations here: the trace concatenation $\tracecat: \mathcal{T} \to \mathcal{T} \to \mathcal{T}$, which combines two traces; they belong to operator $\in$ so that an event $\event \in \eventset$ belongs to a trace $\trace$ is notated by $\event \in \trace$. 
% As usual, we denote by $\event \notin \trace$ that the event $\event$ doesn't belong to the trace $\trace$. 
% Another operator $\llabel : \mathcal{T} \to \mathcal{VAR} \to \{\mathbb{N}\}\cup \{\bot\}$,
% takes a trace and a variable and returns the label of the latest assignment event which assigns value to that variable. I also have the operator $\tlabel : \mathcal{T} \to \mathcal{P}{(\mathbb{N})}$, which gives the set of labels in every event belonging to a trace. The full definitions of these above operators can be found in the appendix.
% }
% \wq{It seems trace concatenation and event belonging to a trace do not deserve so much space here:-)}
%\jl{I agree}

% \\
% I also introduce a counting operator $\vcounter : \mathcal{T} \to \mathbb{N} \to \mathbb{N}$, 
% % \wq{which counts the occurrence of a variable in the trace,} 
% which counts the occurrence of a labeled variable in the trace,
% whose behavior is defined as follows,
% % \[
% % \begin{array}{lll}
% % \vcounter(\trace :: (x, l, v, \bullet) ) l \triangleq \vcounter(\trace) l + 1
% % &
% % \vcounter(\trace ::(b, l, v, \bullet) ) l \triangleq \vcounter(\trace) l + 1
% % &
% % \vcounter(\trace :: (x, l, v, \qval) ) l \triangleq \vcounter(\trace) l + 1
% % \\
% % \vcounter(\trace :: (x, l', v, \bullet) ) l \triangleq \vcounter(\trace ) l, l' \neq l
% % &
% % \vcounter(\trace :: (b, l', v, \bullet) ) l \triangleq \vcounter(\trace ) l, l' \neq l
% % &
% % \vcounter(\trace :: (x, l', v, \qval)) l \triangleq \vcounter(\trace ) l, l' \neq l
% % \\
% % \vcounter({[]}) l \triangleq 0
% % &&
% % \end{array}
% % \]
% \[
% \begin{array}{lll}
% \vcounter(\trace :: (x, l, v, \bullet), l ) \triangleq \vcounter(\trace, l) + 1
% &
% \vcounter(\trace ::(b, l, v, \bullet), l) \triangleq \vcounter(\trace, l) + 1
% &
% \vcounter(\trace :: (x, l, v, \qval), l) \triangleq \vcounter(\trace, l) + 1
% \\
% \vcounter(\trace :: (x, l', v, \bullet), l) \triangleq \vcounter(\trace, l), l' \neq l
% &
% \vcounter(\trace :: (b, l', v, \bullet), l) \triangleq \vcounter(\trace, l), l' \neq l
% &
% \vcounter(\trace :: (x, l', v, \qval), l) \triangleq \vcounter(\trace, l), l' \neq l
% \\
% \vcounter({[]}, l) \triangleq 0
% &&
% \end{array}
% \]
% \input{trace}
%%% trace, queries
% A memory is standard, a map from variables to values. Queries can be uniquely annotated as $\mathcal{AQ}$, and the annotation $(l,w)$ considers the location of the query by line number $l$ and which iteration the query is at when it appears in a loop statement, specified by $w$. A trace $t$ is a list of annotated queries accumulated along the execution of the program. 



\paragraph*{Environment}
The function $\env : {\mathcal{T}} \to \mathcal{VAR} \to \mathcal{VAL} \cup \{\bot\}$, which maps a trace and a variable to the latest value assigned to this variable on the trace is defined as follows.
% \wq{Question: Seem $\env$ is a function that looks up in the input trace and returns you the latest value of the variable. I have a question, in the two-round example, I see $env(\tau)(k)$ while $k$ is not defined(it is input), so in our two-round example in Overview, the value is stored in the second event is $\bot$? Also, another important, $\env$ relies on the input trace, so it will not appear in the trace, or config, is it precise?}
% \jl{yes, it is precise. 
% I have initial trace and everything belonging is defined over all possible initial traces.
%in the two-round example, there is an initial trace where the value of k is defined there. It is worth explaining this here.
% }
\[
\begin{array}{lll}
\env(\trace \traceadd (x, l, v, \bullet)) x \triangleq v
&
\env(\trace \traceadd (y, l, v, \bullet)) x \triangleq \env(\trace) x, y \neq x
&
\env(\trace \traceadd (b, l, v, \bullet)) x \triangleq \env(\trace) x
\\
\env(\trace \traceadd (x, l, v, \qval)) x \triangleq v
&
\env(\trace \traceadd (y, l, v, \qval)) x \triangleq \env(\trace) x, y \neq x
&
\env({[]} ) x \triangleq \bot
\end{array}
\]
 %% trace

%
% figure, evaluation rules.
% {\footnotesize
% \begin{figure}
% \begin{mathpar}
% \boxed{ \config{m, c, t,w} \xrightarrow{} \config{m', c', t', w'} \; }
% \and
% %
% {\inferrule
% {
% \valr_N > 0
% }
% {
% \config{m, \eloop ~ [\valr_N]^{l} ~ \edo ~ c , t, w }
% \xrightarrow{} \config{m, c ; \eloop ~ [(\valr_N-1)]^{l} ~ \edo ~ c , t, (w + l) }
% }
% ~\textbf{low-loop}
% }
% %
% \and
% %
% \inferrule
% {
% }
% {
% \config{m, [\eskip]^{l} ; c_2, t,w} \xrightarrow{} \config{m, c_2, t,w}
% }
% ~\textbf{low-seq2}
% %
% \quad
% %
% {
% \inferrule
% {
% \valr_N = 0
% }
% {
% \config{m, \eloop ~ [\valr_N]^{l} ~ \edo ~ c , t, w }
% \xrightarrow{} \config{m, [\eskip]^{l} , t, (w \setminus l) }
% }
% ~\textbf{low-loop-exit}
% }
% \and
% %
% \inferrule
% {
% }
% {
% \config{m, \eif([\efalse]^{l}, c_1, c_2), t,w} 
% \xrightarrow{} \config{m, c_2, t,w}
% }
% ~\textbf{low-if-f}
% %
% ~~
% % { Memory \times Com \times Trace \times WhileMap \Rightarrow^{} Memory \times Com \times Trace \times WhileMap}
% \inferrule
% {
% \config{m,\expr} \to \expr'
% }
% {
% \config{m, [\assign{x}{q(\expr)}]^l, t, w} \xrightarrow{} \config{m, [\assign{x}{q(\expr')}]^l, t, w}
% }
% ~\textbf{low-query-e}
% %
% \and
% %
% %
% \inferrule
% {
% \config{m, c_1, t,w} \xrightarrow{} \config{m', c_1', t',w'}
% }
% {
% \config{m, c_1; c_2, t,w} \xrightarrow{} \config{m', c_1'; c_2, t',w'}
% }
% ~\textbf{low-seq1}
% ~~
% \inferrule
% {
% q(v) = v_q
% }
% {
% \config{m, [\assign{x}{q(v)}]^l, t, w} \xrightarrow{} \config{m[ v_q/ x], \eskip, t \mathrel{++} [q(v)^{(l,w )}],w }
% }
% ~\textbf{low-query-v}
% %
% % \inferrule
% % {
% % }
% % {
% % \config{m, [\assign x v]^{l}, t,w} \xrightarrow{} \config{m[v/x], [\eskip]^{l}, t,w}
% % }
% % ~\textbf{low-assn}
% %
% %
% %
% \and
% %
% \inferrule
% {
% \config{ m, \bexpr} \barrow \bexpr'
% }
% {
% \config{m, \eif([\bexpr]^{l}, c_1, c_2), t,w} 
% \xrightarrow{} \config{m, \eif([\bexpr']^{l}, c_1, c_2), t,w}
% }
% ~\textbf{low-if}
% %
% ~~~~
% %
% \inferrule
% {
% }
% {
% \config{m, \eif([\etrue]^{l}, c_1, c_2),t,w} 
% \xrightarrow{} \config{m, c_1, t,w}
% }
% ~\textbf{low-if-t}
% %
% % %
% %
% \end{mathpar}
% \vspace{-0.3cm}
% \caption{Trace-based operational semantics}
% \label{fig:evaluation}
% \vspace{-0.5cm}
% \end{figure}
% }
%
% explanation of rules

%
\begin{figure}
 \begin{mathpar}
 \boxed{
 \mbox{Command $\times$ Trace}
 \xrightarrow{}
 \mbox{Command $\times$ Trace}
 }
 \and
 \boxed{\config{{c, \trace}}
 \xrightarrow{} 
 \config{{c', \trace'}}
 }
 \\
 % \inferrule
 % {
 % \empty
 % }
 % {
 % \config{\clabel{\eskip}^l, \trace } 
 % \xrightarrow{} 
 % \config{\clabel{\eskip}^l, \trace}
 % }
 % ~\textbf{skip}
 %
 % \and
 %
 \inferrule
 {
 \config{\trace, \expr} \earrow v 
 \and
 \event = ({x}, l, v, \bullet)
 }
 {
 \config{[\assign{{x}}{\expr}]^{l}, \trace } 
 \xrightarrow{} 
 \config{\clabel{\eskip}^l, \trace \traceadd \event}
 }
 ~\textbf{assn}
 %
 \and
 %
 \highlight{
 \inferrule
 {
\config{ \trace, \qexpr} \qarrow \qval
 \and 
 \query(\qval) = v
 \and 
 \event = ({x}, l, v, \qval)
 }
 {
 \config{{[\assign{x}{\query(\qexpr)}]^l, \trace}}
 \xrightarrow{} 
 \config{{\clabel{\eskip}^l, \trace \traceadd \event} }
 }
 ~\textbf{query}
 }
 %
 \and
 %
 \inferrule
 {
\config{ \trace, b} \barrow \etrue
 \and 
 \event = (b, l, \etrue, \bullet)
 }
 {
 \config{{\ewhile [b]^{l} \edo c, \trace}}
 \xrightarrow{} 
 \config{{
 c; \ewhile [b]^{l} \edo c),
 \trace \traceadd \event}}
 }
 ~\textbf{while-t}
 %
 %
 \quad
 %
 \inferrule
 {
 \config{\trace, b} \barrow \efalse
 \and 
 \event = (b, l, \efalse, \bullet)
 }
 {
 \config{{\ewhile [b]^{l}, \edo c, \trace}}
 \xrightarrow{} 
 \config{{
 \clabel{\eskip}^l,
 \trace \traceadd \event}}
 }
 ~\textbf{while-f}
 %
 %
 \and
 %
 %
 \inferrule
 {
 \config{{c_1, \trace}}
 \xrightarrow{}
 \config{{\clabel{\eskip}^l, \trace'}}
 \and 
 \config{{\clabel{\eskip}^l; c_2, \trace'}} \xrightarrow{} \config{{ \clabel{\eskip}^l, \trace''}}
 }
 {
 \config{{c_1; c_2, \trace}} 
 \xrightarrow{} 
 \config{{\clabel{\eskip}^l, \trace''}}
 }
 ~\textbf{seq}
 %
 % \and
 % %
 % \inferrule
 % {
 % \config{{c_2, \trace}}
 % \xrightarrow{}
 % \config{{c_2', \trace'}}
 % }
 % {
 % \config{{\clabel{\eskip}^l; c_2, \trace}} \xrightarrow{} \config{{ c_2', \trace'}}
 % }
 % ~\textbf{seq2}
 %
 \quad
 %
 %
 \inferrule
 {
 \trace, b \barrow \etrue
 \and 
 \event = (b, l, \etrue, \bullet)
 }
 {
 \config{{
 \eif([b]^{l}, c_1, c_2), 
 \trace}}
 \xrightarrow{} 
 \config{{c_1, \trace \traceadd \event}}
 }
 ~\textbf{if-t}
 %
 % \and
 % %
 % \inferrule
 % {
 % \trace, b \barrow \efalse
 % \and 
 % \event = (b, l, \efalse, \bullet)
 % }
 % {
 % \config{{\eif([b]^{l}, c_1, c_2), \trace}}
 % \xrightarrow{} 
 % \config{{c_2, \trace \traceadd \event}}
 % }
 % ~\textbf{if-f}
 \end{mathpar}
 % \end{subfigure}
 \vspace{-0.5cm}
 \caption{Trace-based Operational Semantics for Language.}
 \label{fig:os}
 \end{figure}
 %

% {The big step trace-based operational semantics has the form of $ \config{c, \trace} \xrightarrow{} { \config{c', \trace'}}$. It reads that the configuration $(c, \trace)$ with labeled command $c$ and trace $\trace$, will be evaluated to another configuration, in which $c$ is evaluated to $c'$ and the trace is updated during the evaluation, to $\trace'$. 
% }
% The step trace-based operational semantics has the form of $ \config{c, \trace} \xrightarrow{} { \config{c', \trace'}}$. 
% It reads the configuration $\config{c, \trace}$ consisting of a labeled command $c$ and a pre-trace $\trace$, 
% and evaluates it to another configuration, 
% in which $c$ is evaluated to $c'$ and trace $\trace$ is updated to $\trace'$. 
% is updated during the evaluation,
\paragraph*{Operational Semantics}
I give a selection of rules of the trace-based operational semantics in Figure~\ref{fig:os}. 

% \todo{Make sure the operational semantics is a big step and correct assn rules.}
% \jl{
The rule $\textbf{assn}$ evaluates a standard assignment $\assign{x}{\expr}$, the expression $\expr$ is first evaluated by our expression evaluation $\config{\trace, \expr} \earrow v $, presented below. And the result $v$ of evaluating $\expr$ is used to construct a new event $\event = (x, l, v,\bullet)$ and attach it to the previous trace. 
\begin{mathpar}
% \boxed{ \config{\trace, \expr} \earrow v \, : \, \mbox{Trace $\times$ Expression $\Rightarrow$ Value} }
% \\
\inferrule{ 
 \config{\trace, \aexpr} \aarrow v
}{
 \config{\trace, \aexpr} 
 \earrow v
}
\and
\inferrule{ 
 \config{\trace, \bexpr} \barrow v
}{
 \config{\trace, \bexpr} 
 \earrow v
}
\and
\inferrule{ 
 \config{\trace, \expr_1} \earrow v_1
 \cdots
 \config{\trace, \expr_n} \earrow v_n
}{
 \config{\trace, [\expr_1, \cdots, \expr_n]} 
 \earrow [v_1, \cdots, v_n]
}
\and
\inferrule{ 
 \empty
}{
 \config{\trace, v} 
 \earrow v
}
\end{mathpar}
The expression evaluation rules also rely on the evaluation of arithmetic expressions $\config{\trace,\aexpr} \aarrow v $ and boolean expressions $\config{\trace, \bexpr} \barrow v $. The full rules can be found in the appendix.
% \begin{mathpar}
% \boxed{ \config{\trace,\aexpr} \aarrow v \, : \, \mbox{Trace $\times$ Arithmetic Expr $\Rightarrow$ Arithmetic Value} }
% % \text{\mg{Missing. Without these rules, it is difficult to understand why we need a trace to evaluate expressions.}}
% \\
% \inferrule{ 
% \empty
% }{
% \config{\trace, n} 
% \aarrow n
% }
% \and
% \inferrule{ 
% \env(\trace) x = v
% }{
% \config{\trace, x} 
% \aarrow v
% }
% \and
% \inferrule{ 
% \config{\trace, \aexpr_1} \aarrow v_1
% \and 
% \config{\trace, \aexpr_2} \aarrow v_2
% \and 
% v_1 \oplus_a v_2 = v
% }{
% \config{\trace, \aexpr_1 \oplus_a \aexpr_2} 
% \aarrow v
% }
% % \and
% % \inferrule{ 
% % \config{\trace, \aexpr} \aarrow v'
% % \and 
% % \elog v' = v
% % }{
% % \config{\trace, \elog \aexpr} 
% % \aarrow v
% % }
% % \and
% % \inferrule{ 
% % \config{\trace, \aexpr} \aarrow v'
% % \and 
% % \esign v' = v
% % }{
% % \config{\trace, \esign \aexpr} 
% % \aarrow v
% % }
% \\
% \boxed{ \config{\trace, \bexpr} \barrow v \, : \, \mbox{Trace $\times$ Boolean Expr $\Rightarrow$ Boolean Value} }
% % \text{\mg{Missing. Without these rules, it is difficult to understand why we need a trace to evaluate expressions.}}
% \\
% % \inferrule{ 
% % \empty
% % }{
% % \config{\trace, \efalse} 
% % \barrow \efalse
% % }
% % \and 
% % \inferrule{ 
% % \empty
% % }{
% % \config{\trace, \etrue} 
% % \barrow \etrue
% % }
% % \and 
% \inferrule{ 
% \config{\trace, \bexpr} \barrow v'
% \\ 
% \neg v' = v
% }{
% \config{\trace, \neg \bexpr} 
% \barrow v
% }
% \and 
% \inferrule{ 
% \config{\trace, \bexpr_1} \barrow v_1
% \\ 
% \config{\trace, \bexpr_2} \barrow v_2
% \\ 
% v_1 \oplus_b v_2 = v
% }{
% \config{\trace, \bexpr_1 \oplus_b \bexpr_2} 
% \barrow v
% }
% \and 
% \inferrule{ 
% \config{\trace, \aexpr_1} \aarrow v_1
% \\ 
% \config{\trace, \aexpr_2} \aarrow v_2
% \\ 
% v_1 \sim v_2 = v
% }{
% \config{\trace, \aexpr_1 \sim \aexpr_2} 
% \barrow v
% }
% \end{mathpar}
% % }


Distinguished from the standard assignment evaluation, 
the rule $\textbf{query}$ 
evaluates a query requesting command $\clabel{\assign{x}{\query(\qexpr)}}^l$ in two steps.
The query expression $\qexpr$ is first evaluated into a query value $\qval$ by following the rules below.
Then, by sending this query request $\query(\qval)$ to a hidden mechanism, this query is evaluated to a result value returned from it, $v = \query(\qval)$.
% by sending this query request $\query(\qval)$ to it.
Also, the generated event stores both the query value $\alpha$ here, and the result value of the query request.

\begin{mathpar}
% \boxed{ \config{\trace, \qexpr} \qarrow \qval \, : \, \mbox{Trace $\times$ Query Expr $\Rightarrow$ Query Value} }
% \\
\inferrule{ 
 \config{\trace, \aexpr} \aarrow n
}{
 \config{\trace, \aexpr} 
 \qarrow n
}
\and
\inferrule{ 
 \config{\trace, \qexpr_1} \qarrow \qval_1
 \and
 \config{\trace, \qexpr_2} \qarrow \qval_2
}{
 \config{\trace, \qexpr_1 \oplus_a \qexpr_2} 
 \qarrow \qval_1 \oplus_a \qval_2
}
\and
\inferrule{ 
 \config{\trace, \aexpr} \aarrow n
}{
 \config{\trace, \chi[\aexpr]} \qarrow \chi[n]
}
\and
\inferrule{ 
 \empty
}{
 \config{\trace, \qval} 
 \qarrow \qval
}
 \end{mathpar}
% }
% \wq{The rules for if hand while both have two versions, when the guard evaluates to true and false, respectively. In these rules, the evaluation of the guard also generates a testing event and our trace is updated as well. }
The rules for if and while both have two versions 
when the boolean expressions in the guards are evaluated to true and false, respectively. 
In these rules, the evaluation of the guard generates a testing event and the trace is updated as well by appending this event.
% The rule $\textbf{query}$ evaluates the argument of a query request to a normal form and obtains the answer $v_q$ of the query $\query(v)$ from the mechanism. 
% Then the trace is expanded by appending the query expression $\query(v)$ with the current annotation $(l,w)$. 

% The rule for assignment is standard and the trace remains unchanged. The sequence rule keeps tracking the modification of the trace, and the evaluation rule for if conditional 

% \jl{If we observe the operational semantics rules, we can find that no rule will shrink the trace.} 
% If we observe the operational semantics rules, we can find that no rule will shrink the trace. It is proved in the appendix.
% So we have the Lemma~\ref{lem:tracenondec}, specifically, the trace has the property that its length never decreases during the program execution.

% \begin{lem}
% [Trace Non-Decreasing]
% \label{lem:tracenondec}
% For every program $c \in \cdom$ and traces $\trace, \trace' \in \mathcal{T}$, if 
% $\config{c, \trace} \rightarrow^{*} \config{\eskip, \trace'}$,
% then there exists a trace $\trace'' \in \mathcal{T}$ with $\trace \tracecat \trace'' = \trace'$
% %
% $$
% \forall \trace, \trace' \in \mathcal{T}, c \st
% \config{c, \trace} \rightarrow^{*} \config{\eskip, \trace'} 
% \implies \exists \trace'' \in \mathcal{T} \st \trace \tracecat \trace'' = \trace'
% $$
% \end{lem}
% %
% % \mg{This corollary needs some explanation. In particular, we should stress that $\event$ and $\event'$ may differ in the query value.}
% % Since the equivalence over two events is defined over the query value equivalence, 
% % when there is an event 
% % belonging to a trace, 
% % it is possible that the event showing up in this trace has a different form of query value, but they are equivalent by Definition~\ref{def:query_equal}.
% Since the equivalence over two events is defined over the query value equivalence, 
% when there is an event belonging to a trace, 
% if this event is a query assignment event, 
% it is possible that 
% the event showing up in this trace has a different form of query value, 
% but they are equivalent by Definition~\ref{def:query_equal}.
% So we have the following Corollary~\ref{coro:aqintrace} with proof in Appendix.
% % ~\ref{apdx:lemma_sec123}.
% % \todo{we should stress that $\event$ and $\event'$ may differ in the query value.}
% \begin{coro}
% \label{coro:aqintrace}
% For every event and a trace $\trace \in \mathcal{T}$,
% if $\event \in \trace$, 
% then there exist another event $\event' \in \eventset$ and traces $\trace_1, \trace_2 \in \mathcal{T}$
% such that $\trace_1 \tracecat [\event'] \tracecat \trace_2 = \trace $
% with 
% $\event$ and $\event'$ equivalent but may differ in their query value.
% \[
% \forall \event \in \eventset, \trace \in \mathcal{T} \st
% \event \in \trace \implies \exists \trace_1, \trace_2 \in \mathcal{T}, 
% \event' \in \eventset \st (\event = \event') \land \trace_1 \tracecat [\event'] \tracecat \trace_2 = \trace 
% \]
% \end{coro}


% In this chapter, 
I formally introduce the language I will focus on for writing data analyses.  
This is a standard while language with some primitives for calling queries. 
After defining the syntax of the language and showing an example, 
I will define its trace-based operational semantics. 
This is the main technical ingredient I will use to define the program's adaptivity.
\section{Syntax of {\tt Query While} Language}
\label{sec:language-syntax}
The syntax is shown as follows,
\[
\begin{array}{llll}
\mbox{Arithmetic Operators} 
& \oplus_a & ::= & + ~|~ - ~|~ \times 
%
~|~ \div ~|~ \max ~|~ \min\\  
% \mbox{Unary Operators} 
% & \oplus_a & ::= & + ~|~ - ~|~ \times 
% %
% ~|~ \div \\  
\mbox{Boolean Operators} 
& \oplus_b & ::= & \lor ~|~ \land
\\
%
\mbox{Relational Operators} 
& \sim & ::= & < ~|~ \leq ~|~ == 
\\  
%
\mbox{Label} 
& l & \in & \mathbb{N} \cup \{\lin, \lex\} 
\\ 
%
\mbox{Arithmetic Expression} 
& \aexpr & ::= & 
n ~|~ {x} ~|~ \aexpr \oplus_a \aexpr  
% \\
% &  &  & 
 ~|~ \elog \aexpr  ~|~ \esign \aexpr
\\
%
\mbox{Boolean Expression} & \bexpr & ::= & 
%
\etrue ~|~ \efalse  ~|~ \neg \bexpr
 ~|~ \bexpr \oplus_b \bexpr
%
~|~ \aexpr \sim \aexpr 
\\
%
\mbox{Expression} & \expr & ::= & v ~|~ \aexpr ~|~ \bexpr ~|~ [\expr, \dots, \expr]
\\  
%
\mbox{Value} 
& v & ::= & { n ~|~ \etrue ~|~ \efalse ~|~ [] ~|~ [v, \dots, v]}  
\\
%
\highlight{\mbox{Query Expression}  }
& {\qexpr} & ::= 
& \highlight{ \qval ~|~ \aexpr ~|~ \qexpr \oplus_a \qexpr ~|~ \chi[\aexpr]}
\\
%
\highlight{\mbox{Query Value} }& \qval & ::= 
& \highlight{n ~|~ \chi[n] ~|~ \qval \oplus_a  \qval ~|~ n \oplus_a  \chi[n]
    ~|~ \chi[n] \oplus_a  n}
    \\
% &&& \text{\mg{I don’t think this is what I want. Isn’t $\chi[n+1]$ a query value?}}\\
% &&& \text{\mg{What about $\chi[i] + \chi[i] + \chi[i]$? They are not in the grammar}}
% \\
% &&& \text{\jl{ $\chi[i] + \chi[i] + \chi[i]$ and $\chi[n+1]$ are both expressions, they will be evaluated to a value 
% }}
% \\%
\mbox{Labeled Command} 
& {c} & ::= &   [\assign {{x}}{ {\expr}}]^{l} ~|~  \highlight{[\assign {{x} } {{\query(\qexpr)}}]^{l}}
~|~ {\ewhile [ \bexpr ]^{l} \edo {c} }
\\
&&&
~|~ {c};{c}  
~|~ \eif([\bexpr]{}^l , {c}, {c}) 
~|~ [\eskip]^l
\\ 
\mbox{Event} 
& \event & ::= & 
    ({x}, l, v, \bullet) ~|~ ({x}, l, v, \qval)  ~~~~~~~~~~~ \mbox{Assignment Event} \\
&&& ~|~(\bexpr, l, v, \bullet)   ~~~~~~~~~~~~~~~~~~~~~~~~~~~~~~~~~~ \mbox{Testing Event}
\\
\mbox{Trace} & \trace
& ::= & [] ~|~ \trace :: \event
\\
\end{array}
\]
% \[
% \begin{array}{llll}
% \mbox{Arithmetic Operators} 
% & \oplus_a & ::= & + ~|~ - ~|~ \times 
% %
% ~|~ \div ~|~ \max ~|~ \min\\  
% % ~|~ \div \\  
% \mbox{Boolean Operators} 
% & \oplus_b & ::= & \lor ~|~ \land
% \\
% %
% \mbox{Relational Operators} 
% & \sim & ::= & < ~|~ \leq ~|~ == 
% \\  
% %
% \mbox{Arithmetic Expression} 
% & \aexpr & ::= & 
% n ~|~ {x} ~|~ \aexpr \oplus_a \aexpr  
%  ~|~ \elog \aexpr  ~|~ \esign \aexpr
% \\
% %
% \mbox{Boolean Expression} & \bexpr & ::= & 
% %
% \etrue ~|~ \efalse  ~|~ \neg \bexpr
%  ~|~ \bexpr \oplus_b \bexpr
% %
% ~|~ \aexpr \sim \aexpr 
% \\
% %
% \mbox{Expression} & \expr & ::= & v ~|~ \aexpr ~|~ \bexpr ~|~ [\expr, \dots, \expr] ~|~ \highlight{\fname}
% \\  
% %
% \mbox{Value} 
% & v & ::= & { n ~|~ \etrue ~|~ \efalse ~|~ [] ~|~ [v, \dots, v]}  
% \\ 
% &&&
% \highlight
% {
% ~|~ (x_0, x_1, \ldots, x_n) := c
% }
% \\
% %
% \highlight{\mbox{Query Expression}} 
% & {\qexpr} & ::= 
% & \highlight{ \qval ~|~ \aexpr ~|~ \qexpr \oplus_a \qexpr ~|~ \chi[\aexpr]} 
% \\
% %
% \mbox{Query Value} & \qval & ::= 
% & \highlight{n ~|~ \chi[n] ~|~ \qval \oplus_a  \qval ~|~ n \oplus_a  \chi[n]
%     ~|~ \chi[n] \oplus_a  n}
% \\
% % \\%
% \mbox{Label} 
% & l & ::= & (n \in \mathbb{N} \cup \{\lin, \lex\}) ~|~ (l, n)
% \\ 
% %
% \mbox{Labeled Command} 
% & {c} & ::= &  
% \clabel{\assign{x}{\expr}}^l 
% ~|~ \clabel{\assign{x}{\query(\qexpr)}}^l
% ~|~  \clabel{\eskip}^l
% ~|~ \ewhile \clabel{\bexpr}^{l} \edo {c}
% ~|~ \eif(\clabel{\bexpr}^{l} , {c}, {c}) 
% \\ 
% &&&
% \highlight
% {
% ~|~ \clabel{\efun}^l: \fname (x_0, x_1, \ldots, x_n) := c
% ~|~ \clabel{\assign{x}{\ecall(x, e_1, \ldots, e_n)}}^l
% }
% ~|~ {c};{c}  
% \\ 
% % \\
% \mbox{Event} 
% & \event & ::= & 
%     ({x}, l, v, \bullet) ~|~ ({x}, l, v, \qval) ~|~ (\fname, l, v, \qval)  ~~~~~~~~~~~ \mbox{Assignment Event} \\
% &&& ~|~(\bexpr, l, v, \bullet)   ~~~~~~~~~~~~~~~~~~~~~~~~~~~~~~~~~~ \mbox{Testing Event}
% \\
% % &&& \text{\mg{I think it would be better to use quadruples for events, where the}}\\
% % &&& \text{\mg{first element is either a variable or a boolean expression and }}\\
% % &&& \text{\mg{the last is either a query value or some default value $\bullet$}}\\
% %
% % \mbox{Trace} & \trace
% % & ::= & \cdot | \trace \cdot \event | \trace \tracecat \trace 
% % \\
% %
% % \mbox{Trace} & \trace
% % & ::= & [] ~|~ \event:: \trace ~|~ \trace \tracecat \trace  \\
% \mbox{Trace} & \trace
% & ::= & [] ~|~ \trace :: \event\\
% % &&& \text{\mg{I don't understand why you need both :: and ++ as constructors.}}\\
% % &&& \text{\jl{Because append is to the left but we are adding element to the left in the OS}}\\
% % &&& \text{\jl{I was too sticky to the convention, it is a good idea to append to the left and just use $::$}}
% % %
% % \mbox{Event Signature} & \sig
% % & ::= & (x, l, n) | (x, l, n, \query) | (b, l, n)
% % \\
% % %
% \end{array}
% \]
For clarity, the following notations are used to represent the set of corresponding terms:
\[
\begin{array}{lll}
\mathcal{VAR} & : & \mbox{Set of Variables}  
\\ 
%
\mathcal{VAL} & : & \mbox{Set of Values} 
\\ 
%
\mathcal{QVAL} & : & \mbox{Set of Query Values} 
\\ 
%
\cdom & : & \mbox{Set of Commands} 
\\ 
%
\mathcal{LV} & : & \mbox{Set of Labeled Variables}
\\
%
\eventset  & : & \mbox{Set of Events}  
\\
%
\eventset^{\asn}  & : & \mbox{Set of Assignment Events}  
\\
%
\eventset^{\test}  & : & \mbox{Set of Testing Events}  
\\
%
\ldom  & : & \mbox{Set of Labels}  
\\
%%
\mathcal{VAL}  & : & \mbox{Set of Labeled Variables}  
\\
%%
\dbdom  & : & \mbox{{Set of Databases}} 
\\
%
{\mathcal{T}} & : & \mbox{Set of Traces}
\\
%
% \qdom = {[-1,1]} & : & \mbox{{Domain of Query Results}}\\
\qdom & : & \mbox{{Domain of Query Results}}\\
% &&\text{\mg{I don't think you need to hard code [-1,1] here}}\\
\end{array}
\]
\paragraph*{Standard Expression}
The expressions are either the standard one or the extended one.
A standard expression is
% can be 
either a standard arithmetic expression or a boolean expression, or a list of expressions.
An arithmetic expression can be a constant $n$ denoting integer, a variable $x$ from some countable set $\mathcal{VAR}$, binary operation $\oplus_a$ such as addition, product, subtraction, etc, over arithmetic expressions, and also log and sign operation. 
%
A boolean expression can be either {\tt true} or {\tt false}, basic boolean connectives such as logical negation, logical and and or denoted by $\oplus_b$, and basic comparison $sym$ between arithmetic expressions, e.g., $\leq,=,<,$ etc.
Additionally, I also introduce list in expression.
Our language supports primitives for queries, 
where a specific query is specified by a query expression $\qexpr$. 
A query expression contains the necessary information for a query request, for example, 
$\chi[\aexpr]$ represents the values at a certain index $\aexpr$ in a row $\chi$ of the database. 
Query expressions combine access to the database with other expressions, 
for example, $\chi[3] + 5$ represents a query which asks the value from the column 3 of each database raw $\chi$, adds 5 to each of these values, 
and then computes the average of these values.
\paragraph*{Query Expression}
The key extension is
%  language supports 
the primitive for queries, where a specific query is specified by a query expression $\qexpr$. 
A query expression contains the necessary information for a query request, 
for example, $\chi[\qexpr]$ represents the values at a certain index $\qexpr$ in a row $\chi$ of the database. 
When this expression is encapsulated by the symbol $\query$,
 $ \query(\chi[\qexpr]) $ computes the average value at certain index over each row of the database as follows,
 \[
  \query(\chi[\qexpr]) = \frac{1}{n}\sum\limits_{i = 0}^{n}\chi_i[\qexpr]
  \]
Query expressions combine access to the database with other expressions, 
for example, 
$\chi[3] + 5$ represents a query that asks the value from column 3 of each database raw $\chi$, 
adds 5 to each of these values, and then computes the average of these values as follows, where $n$ is 
data base $\chi$'s number of raw.
%
\[
  \query(\chi[3] + 5) = \frac{1}{n}\sum\limits_{i = 0}^{n}\chi_i[3] + 5
  \]

% the expression also includes the special variable $\chi$ representing a row of the database, and access to values at a certain index in $\chi$, as $\chi[\aexpr]$. Additionally, list over expressions is supported and $[]$ stands for the empty list. The access to elements in the list can be achieved through $x[\aexpr]$ when variable $x$ is referred to a list. The value $v$ now contains the natural number $n$, the boolean primitives $\etrue$ and $\efalse$, the special row $\chi$ and access to it $\chi[v]$, the empty list $[]$ and non-empty list $[v, \dots, v]$.
% 
% Another extension is the inter-procedure call and function definition.
% In the function define command $\clabel{\efun}^l: \fname (x_0, x_1, \ldots, x_n) := c$,
% the function body $c$ is assigned to the function of name $\fname$, $x_1, \ldots, x_n$ is the function
% arguments and the first element $x_0$ in the arguments is the return variable.
% We only support the first-order function definition and function call. 

% %
\paragraph*{Labeled Command}
 A labeled command $c$ is just a command with a label --- I assume that labels are unique, so that they can help to identify uniquely every subexpression. 
%  I have $\eskip$, assignment $\assign{x}{\expr}$, the composition of two commands $c;c$, an if statement $\eif(\bexpr, c, c)$, a while statement  $\ewhile \bexpr \edo {c} $.
 The main novelty of the syntax is the query request command $\assign{x}{\query(\qexpr)}$. 
 For instance, if a data analyst wants to ask a simple linear query which returns the first element of the row, 
 they can simply use the command $ \assign{x}{\query(\chi[1])}$ in their data analysis program.
%  \wq{Shall I distinguish command and labeled command, they are now both $c$. }
%  \jl{I'm not sure, I don't want to programmer to add the label when writing the program. The label is just added by us for analysis. but I'm worried it is too complicate if use two notations for command and labeled command }
%
% \[
% \begin{array}{llll}
% \mbox{Label} 
% & l & \in & \mathbb{N} \cup \{in, ex\} \\
% \mbox{Labeled Commands} 
% & {c} & ::= &   [\assign {{x}}{ {\expr}}]^{l} ~|~  [\assign {{x} } {{\query(\qexpr)}}]^{l}
% ~|~ {\ewhile [ \bexpr ]^{l} \edo {c} }
%  \\
%  &&&
% ~|~ {c};{c}  
% ~|~ \eif([\bexpr]{}^l , {c}, {c}) 
% ~|~ [\eskip]^l 
% \end{array}
% \]
\paragraph*{Labeled Variables}
The labeled variables and assigned variables are set of variables annotated by a label. 
We use  
%$\mathcal{LVAR} = \mathcal{VAR} \times \mathcal{L} $ 
$\mathcal{LV}$ represents the universe of all the labeled variables and 
$\avar_c \in \mathcal{P}(\mathcal{VAR} \times \mathbb{N}) \subset \mathcal{LV}$ and 
$\lvar_c \in \mathcal{P}(\mathcal{VAR} \times \mathcal{L}) \subseteq \mathcal{LV}$,
represents the the set of assigned variables and labeled variables for a labeled command $c$,
defined in Definition~\ref{def:lvar} and \ref{def:avar}.
%
% \\
$FV: \expr \to \mathcal{P}(\mathcal{VAR})$, computes the set of free variables in an expression. To be precise,
$FV(\aexpr)$, $FV(\bexpr)$ and $FV(\qexpr)$ represent the set of free variables in arithmetic
expression $\aexpr$, boolean expression $\bexpr$ and query expression $\qexpr$ respectively.
Labeled variables in $c$ is the set of assigned variables and all the free variables
showing up in $c$ with a default label $in$. 
The free variables
showing up in $c$, which aren't defined before be used, are actually the input variables of this program.
%
%
\begin{defn}[Assigned Variables (
% $\avar_{c} \subseteq \mathcal{VAR} \times \mathbb{N}$ or 
$\avar : \cdom \to \mathcal{P}(\mathcal{VAR} \times \mathbb{N})$)]
% labelled Variables 
% (
% % $\lvar_{c} \subseteq \mathcal{VAR} \times \mathbb{N}$ or 
% $\lvar : \cdom \to \mathcal{P}(\mathcal{VAR} \times \mathcal{L})$
\label{def:avar}
$$ \avar_{c} \triangleq
  \left\{
  \begin{array}{ll}
      \{{x}^l\}                   
      & {c} = [{\assign x e}]^{l} 
      \\
      \{{x}^l\}                   
      & {c} = [{\assign x \query(\qexpr)}]^{l} 
      \\
      \avar_{{c_1}} \cup \avar_{{c_2}}  
      & {c} = {c_1};{c_2}
      \\
      \avar_{{c}} \cup \avar_{{c_2}} 
      & {c} =\eif([\bexpr]^{l}, c_1, c_2) 
      \\
      \avar_{{c}'}
      & {c}   = \ewhile ([\bexpr]^{l}, {c}')
\end{array}
\right.
$$
\end{defn}
%
%
\begin{defn}[labelled Variables 
(
% $\lvar_{c} \subseteq \mathcal{VAR} \times \mathbb{N}$ or 
$\lvar : \cdom \to \mathcal{P}(\mathcal{LV})$]
\label{def:lvar}
$$
  \lvar_{c} \triangleq
  \left\{
  \begin{array}{ll}
      \{{x}^l\} \cup FV(\expr)^{in}                  
      & {c} = [{\assign x e}]^{l} 
      \\
      \{{x}^l\}   \cup FV(\qexpr)^{in}                
      & {c} = [{\assign x \query(\qexpr)}]^{l} 
      \\
      \lvar_{{c_1}} \cup \lvar_{{c_2}}  
      & {c} = {c_1};{c_2}
      \\
      \lvar_{{c}} \cup \lvar_{{c_2}} \cup FV(\bexpr)^{in}
      & {c} =\eif([\bexpr]^{l}, c_1, c_2) 
      \\
      \lvar_{{c}'} \cup FV(\bexpr)^{in}
      & {c}   = \ewhile ([\bexpr]^{l}, {c}')
\end{array}
\right.
$$
\end{defn}
%
%
%
% is a subset of the program's assigned variables, where every variable in this set is assigned by a query in the program.
% \mg{The set of query variables of a program is the set of variables set to the result of a query in the program.}\\
% In the same way, in order to 
\paragraph*{Query Variables}
Distinctively, a key definition for the extension of the query primitives 
is the set of query variables for a program $c$.
This definition is the key point to track the query requests in the Following full-spectrum adaptivity analysis.
% track the I also defined the set of query variables for a program $c$.
It is defined as the set of variables,
which are assigned by the result of a query request in the program formally in Definition~\ref{def:qvar}.
% \mg{In the next definition, why do you call it a vector? It seems that you define it as a set.}\\
% \jl{fixed}\\
%
% \begin{defn}[Query Variables ($\qvar_{c} \subseteq \mathcal{VAR} \times \mathbb{N}$)].
  % \\
\begin{defn}[Query Variables ($\qvar: \cdom \to \mathcal{P}(\mathcal{LV})$)] 
  \label{def:qvar}
Given a program $c$, its query variables 
% \mg{it seems you are missing the $_c$ subscript. Also, this is a minor point but I don't think it is a good idea to use a subscript, cannot you just use $\qvar(c)$.}
$\qvar(c)$ is the set of variables set to the result of a query in the program.
% \jl{fixed}
It is defined as follows:
{
$$
  % \qvar_{{c}} \triangleq
  \qvar(c) \triangleq
  \left\{
  \begin{array}{ll}
      % \{\}                  
      % & {c} = [{\assign x e}]^{(l, w)} 
      % \\
      % \{{x}^l\}                  
      % & {c} = [{\assign x \query(\qexpr)}]^{(l, w)} 
      % \\
      % \qvar_{{c_1}} \cup \qvar_{{c_2}}  
      % & {c} = {c_1};{c_2}
      % \\
      % \qvar_{{c_1}} \cup \qvar_{{c_2}} 
      % & {c} =\eif([\bexpr]^{l}, c_1, c_2) 
      % \\
      % \qvar_{{c}'}
      % & {c}   = \ewhile ([\bexpr]^{l}, {c}')
      \{\}                  
      & {c} = [{\assign x \expr}]^{l} 
      \\
      \{{x}^l\}                  
      & {c} = [{\assign x \query(\qexpr)}]^{l} 
      \\
      \qvar(c_1) \cup \qvar(c_2)  
      & {c} = {c_1};{c_2}
      \\
      \qvar(c_1) \cup \qvar(c_2) 
      & {c} =\eif([\bexpr]^{l}, c_1, c_2) 
      \\
      \qvar(c')
      & {c}   = \ewhile ([\bexpr]^{l}, {c}')
\end{array}
\right.
$$
}
\end{defn}
%
It is easy to see that a program $c$'s query variables is a subset of 
its labeled variables, $\qvar(c) \subseteq \lvar(c)$.
%
% \mg{In this definition as well as in others, I have the impression that you assume that the labelled variables are unique in the program. For example, it would not make sense to assign a query to the same labelled variable over and over. If this is the case, I need to make this very explicit in the paper.}
% \jl{TODO}
%
Every labeled variable in a program is unique, formally as follows with proof in Appendix~\ref{apdx:lemma_sec123}.
\begin{lem}[Uniqueness of the Labeled Variables]
  \label{lem:lvar_unique}
  For every program $c \in \cdom$ and every two labeled variables such that
  $x^i, y^j \in \lvar(c)$, then $x^i \neq y^j$.
  \[
    \forall c \in \cdom, x^i, y^j \in \mathcal{L} \sthat x^i, y^j \in \lvar(c)\implies x^i \neq y^j.
    \]
\end{lem}

\highlight{\paragraph*{Improvements through Examples}
It is expressive in two following aspects.
\begin{itemize}
  \item \textbf{Improvements from Standard While Language}
  \\
  It also extends the standard while language with query requests. 
  The general data analysis program with query requests on data  are supported in this {\tt Query While} language.
  The program can access the database through a special  interface $\chi$ encapsulated by the identifier $\query$ (for example the program below) in the new language.
  \[
    {\assign{x}{20}};
    \assign{y}{\query(\chi[2])};
    \ewhile (x < 100) \edo 
    \{
      \assign{x}{x + 1};
      \assign{y}{\query(\chi[x]*\chi[n])};
      \}\}
    \] 
%
    \item \textbf{Improvements from Previous Works}
  \\
This {\tt Query While} language is also more expressive than the language designed in previous works.
The previous language only supports the data analysis with constant number of loop iterations.
Comparing to it, in the new language design,
the general data analysis program with non-deterministic loop iterations
(for example the program below as shown in Section~\ref{sec:prework-language})
is supported.
\[
  {\assign{x}{20}};
  \assign{y}{40};
  \ewhile (x < y) \edo 
  \{
    \assign{x}{x + 1};
    \assign{y}{y - 2};
    \}\}
  \] 
Previous work does not support data analysis program with user inputs, which is supported in the new language as well.
\end{itemize}
}
\section{Trace-based Operational Semantics}
\label{sec:language-os}
The operational semantics is defined based on the event and trace, which are introduced firstly as follows.
% \\
\paragraph*{Event}
An event tracks useful information about each step of the evaluation, as a quadruple. Its first element is either 
an assigned variable (from an assignment command) or a boolean expression (from the guard of if or while command), follows by 
 the label associated with this event, the value evaluated either from the expression assigned to the variable,
or the boolean expression in the guard.
 The last element stores the query information, which is a query value whose default is $\bullet$. I declare event projection operator $\pi_i$ which projects the $i$th element from an event.
\[
\begin{array}{llll}
\mbox{Event} 
& \event & ::= & 
 ({x}, l, v, \bullet) ~|~ ({x}, l, v, \qval) ~~~~~~~~~~~ \mbox{Assignment Event} 
~|~(\bexpr, l, v, \bullet) 
~~~~
\mbox{Testing Event}
% \mbox{Trace} & \trace
% & ::= & [] ~|~ \trace :: \event
\end{array}
\]
% \input{event}
% To distinguish if a query's choice is affected by previous values, 
% % \jl{we need to be able to identify whether two queries are equivalent or not so that when we change the result of one query, another query is affected. For the equivalence of queries, } 
% we need to be able to identify whether two queries are equivalent or not, so that when we change the result of one query, whether or not another query is affected. 
% To define equivalence of queries,
% quite different from the equality between the evaluation results as the regular assignment results, 
% we are neither observing the syntactic equivalence between the two query expressions,
% nor two results return from the database. 
% Instead, we define the equivalence of query expression by quantifying over all values returned from the database on a certain form of query value, formally as follows.
% \begin{defn}[Equivalence of Query Expression]
% %
% \label{def:query_equal}
% % \mg{Two} \sout{2} 
% Two query expressions $\qexpr_1$, $\qexpr_2$ are equivalent, denoted as $\qexpr_1 =_{q} \qexpr_2$, if and only if
% % $$
% % \begin{array}{l} 
% % \exists \qval_1, \qval_2 \in \mathcal{QVAL} \st \forall \trace \in \mathcal{T} \st
% % (\config{\trace, \qexpr_1} \qarrow \qval_1 \land \config{\trace, \qexpr_2 } \qarrow \qval_2) 
% % \\
% % \quad \land (\forall D \in \dbdom, r \in D \st 
% % \exists v \in \mathcal{VAL} \st 
% % \config{\trace, \qval_1[r/\chi]} \aarrow v \land \config{\trace, \qval_2[r/\chi] } \aarrow v) 
% % \end{array}.
% % $$
% $$
% \begin{array}{l} 
% \forall \trace \in \mathcal{T} \st \exists \qval_1, \qval_2 \in \mathcal{QVAL} \st
% (\config{\trace, \qexpr_1} \qarrow \qval_1 \land \config{\trace, \qexpr_2 } \qarrow \qval_2) 
% \\
% \quad \land (\forall D \in \dbdom, r \in D \st 
% \exists v \in \mathcal{VAL} \st 
% \config{\trace, \qval_1[r/\chi]} \aarrow v \land \config{\trace, \qval_2[r/\chi] } \aarrow v) 
% \end{array}.
% $$
% % \mg{$$
% % \begin{array}{l} 
% % \forall \trace \in \mathcal{T} \st \exists \qval_1, \qval_2 \in \mathcal{QVAL} \st
% % (\config{\trace, \qexpr_1} \qarrow \qval_1 \land \config{\trace, \qexpr_2 } \qarrow \qval_2) 
% % \\
% % \quad \land (\forall D \in \dbdom, r \in D \st 
% % \exists v \in \mathcal{VAL} \st 
% % \config{\trace, \qval_1[r/\chi]} \aarrow v \land \config{\trace, \qval_2[r/\chi] } \aarrow v) 
% % \end{array}.
% % $$
% % }
% %
% where $r \in D$ is a record in the database domain $D$. 
% I denote by $\qexpr_1 \neq_{q} \qexpr_2$ the negation of the equivalence relation.
% % \\ 
% % where $r \in D$ is a record in the database domain $D$,
% % \mg{is $FV(\qexpr)$ being defined here? If yes, I suggest putting it in a different place, rather than in the middle of another definition.} 
% % $FV(\qexpr)$ is the set of free variables in the query expression $\qexpr$.
% % \sout{$\qexpr_1 \neq_{q}^{\trace} \qexpr_2$ is defined vice versa.}
% % \mg{As usual, we will denote by $\qexpr_1 \neq_{q}^{\trace} \qexpr_2$ the negation of the equivalence.}
% %
% \end{defn}
%
% \mg{In the next definition you don’t need the subscript e, it is clear that it is an equivalence of events by the fact that the elements on the two sides of = are events. That is also true for query expressions. Also, I am confused by this definition. What happens for two query events?}
% \\
% \jl{The last component of the event is equal based on Query equivalence, $\pi_{4}(\event_1) =_q \pi_{4}(\event_2)$.
% In the previous version, the query expression is in the third component and I defined $v \neq \qexpr$ for all $v$ that isn't a query value.}
% \begin{defn}[Event Equivalence $\eventeq$]
% Two events $\event_1, \event_2 \in \eventset$ \mg{are equivalent, \sout{is in \emph{Equivalence} relation,}} denoted as $\event_1 \eventeq \event_2$ if and only if:
% \[
% \pi_1(\event_1) = \pi_1(\event_2) 
% \land 
% \pi_2(\event_1) = \pi_2(\event_2) 
% \land
% \pi_{3}(\event_1) = \pi_{3}(\event_2)
% \land 
% \pi_{4}(\event_1) =_q \pi_{4}(\event_2)
% \]
% %
% % \sout{The $\event_1 \eventneq \event_2$ is defined as vice versa.}
% % \mg{As usual, we will denote by $\event_1 \eventneq \event_2$ the negation of the equivalence.}
% \end{defn}
% \wq{Now we can compare two events by defining the event equivalence and difference relation.}
% Now we can compare two events by defining the event equivalence and difference relation based on the query equivalence.
% \begin{defn}[Event Equivalence]
% \label{def:event_eq}
% Two events $\event_1, \event_2 \in \eventset$ are equivalent, 
% % denoted as $\event_1 \eventeq \event_2$ 
% denoted as $\event_1 = \event_2$ 
% if and only if:
% \[
% \pi_1(\event_1) = \pi_1(\event_2) 
% \land 
% \pi_2(\event_1) = \pi_2(\event_2) 
% \land
% \pi_{3}(\event_1) = \pi_{3}(\event_2)
% \land 
% \pi_{4}(\event_1) =_q \pi_{4}(\event_2)
% \]
% %
% As usual, we will denote by $\event_1 \neq \event_2$ the negation of the equivalence.
% % As usual, we will denote by $\event_1 \eventneq \event_2$ the negation of the equivalence.
% % When it is clear from the context, we omit the subscript $\kw{e}$ and use 
% % $\event_1 = \event_2$ (and $\event_1 \neq \event_2$) for event equivalent
% \end{defn}
% %
% %
% % \begin{defn}[Signature Equivalence of Events $\sigeq$]
% % Two events $\event_1, \event_2 \in \eventset$ is in \emph{signature equivalence} relation, denoted as $\event_1 \sigeq \event_2$ if and only if:
% % \[
% % \forall i \in \{1, 2, 3\} \st \pi_{\sig}(\event_1) = \pi_{\sig}(\event_2) 
% % \]
% % The $\event_1 \signeq \event_2$ is defined as vice versa.
% % \end{defn}
% %
% % \begin{defn}[Events Different up to Value ($\diff$)]
% % Two events $\event_1, \event_2 \in \eventset$ \mg{are \sout{is}} \emph{Different up to Value}, 
% % denoted as $\diff(\event_1, \event_2)$ if and only if:
% % \[
% % \pi_1(\event_1) = \pi_1(\event_2) 
% % \land 
% % \pi_2(\event_1) = \pi_2(\event_2) 
% % \land 
% % \pi_3(\event_1) \neq_q \pi_3(\event_2)
% % \]
% % \end{defn}
% \begin{defn}[Events Different up to Value ($\diff$)]
% Two events $\event_1, \event_2 \in \eventset$ are \emph{Different up to Value}, 
% denoted as $\diff(\event_1, \event_2)$ if and only if:
% \[
% \begin{array}{l}
% \pi_1(\event_1) = \pi_1(\event_2) 
% \land 
% \pi_2(\event_1) = \pi_2(\event_2) \\
% \land 
% \big(
% (\pi_3(\event_1) \neq \pi_3(\event_2)
% \land 
% \pi_{4}(\event_1) = \pi_{4}(\event_2) = \bullet )
% % \qquad \qquad 
% \lor 
% (\pi_4(\event_1) \neq \bullet
% \land 
% \pi_4(\event_2) \neq \bullet
% \land 
% \pi_{4}(\event_1) \neq_q \pi_{4}(\event_2)) 
% \big)
% \end{array}
% \]
% \end{defn}
% %
% %
\paragraph*{Trace}
A trace $\trace \in \mathcal{T} $ is a list of events, 
collecting the events generated along the program execution. $\mathcal{T} $ represents the set of traces. There are some useful operators: the trace concatenation operator $\tracecat: \mathcal{T} \to \mathcal{T} \to \mathcal{T}$, combines two traces.
The belongs to operator $\in : \eventset \to \mathcal{T} \to \{\etrue, \efalse \} $ and its opposite $\not\in$
express whether or not an event belongs to a trace.
Another operator $\llabel : \mathcal{T} \to \mathcal{VAR} \to \{\mathbb{N}\}\cup \{\bot\}$,
takes a trace and a variable as input and returns the label of the latest assignment event which assigns value to that variable. 
% I also have the operator $\tlabel : \mathcal{T} \to \ldom$, which gives the set of labels in every event belonging to a trace. 
% The full definitions of these above operators can be found in the appendix.
% \[
% \begin{array}{llll}
% \mbox{Trace} & \trace
% & ::= & [] ~|~ \trace :: \event
% \end{array}
% \]
%
A trace can be regarded as the program history, which records queries asked by the analyst during the execution of the program. I collect the trace with a trace-based operational semantics based on transitions of the form $ \config{c, \trace} \to \config{c', \trace'} $. It states that a configuration $\config{c, \trace}$, which consists of a command $c$ to be evaluated and a starting trace $\trace$, evaluates to another configuration with the trace updated along with the evaluation of the command $c$ to the normal form of the command $\eskip$.
% \jl{I introduce some operations here: the trace concatenation $\tracecat: \mathcal{T} \to \mathcal{T} \to \mathcal{T}$, which combines two traces; they belong to operator $\in$ so that an event $\event \in \eventset$ belongs to a trace $\trace$ is notated by $\event \in \trace$. 
% As usual, we denote by $\event \notin \trace$ that the event $\event$ doesn't belong to the trace $\trace$. 
% Another operator $\llabel : \mathcal{T} \to \mathcal{VAR} \to \{\mathbb{N}\}\cup \{\bot\}$,
% takes a trace and a variable and returns the label of the latest assignment event which assigns value to that variable. I also have the operator $\tlabel : \mathcal{T} \to \mathcal{P}{(\mathbb{N})}$, which gives the set of labels in every event belonging to a trace. The full definitions of these above operators can be found in the appendix.
% }
% \wq{It seems trace concatenation and event belonging to a trace do not deserve so much space here:-)}
%\jl{I agree}

% \\
% I also introduce a counting operator $\vcounter : \mathcal{T} \to \mathbb{N} \to \mathbb{N}$, 
% % \wq{which counts the occurrence of a variable in the trace,} 
% which counts the occurrence of a labeled variable in the trace,
% whose behavior is defined as follows,
% % \[
% % \begin{array}{lll}
% % \vcounter(\trace :: (x, l, v, \bullet) ) l \triangleq \vcounter(\trace) l + 1
% % &
% % \vcounter(\trace ::(b, l, v, \bullet) ) l \triangleq \vcounter(\trace) l + 1
% % &
% % \vcounter(\trace :: (x, l, v, \qval) ) l \triangleq \vcounter(\trace) l + 1
% % \\
% % \vcounter(\trace :: (x, l', v, \bullet) ) l \triangleq \vcounter(\trace ) l, l' \neq l
% % &
% % \vcounter(\trace :: (b, l', v, \bullet) ) l \triangleq \vcounter(\trace ) l, l' \neq l
% % &
% % \vcounter(\trace :: (x, l', v, \qval)) l \triangleq \vcounter(\trace ) l, l' \neq l
% % \\
% % \vcounter({[]}) l \triangleq 0
% % &&
% % \end{array}
% % \]
% \[
% \begin{array}{lll}
% \vcounter(\trace :: (x, l, v, \bullet), l ) \triangleq \vcounter(\trace, l) + 1
% &
% \vcounter(\trace ::(b, l, v, \bullet), l) \triangleq \vcounter(\trace, l) + 1
% &
% \vcounter(\trace :: (x, l, v, \qval), l) \triangleq \vcounter(\trace, l) + 1
% \\
% \vcounter(\trace :: (x, l', v, \bullet), l) \triangleq \vcounter(\trace, l), l' \neq l
% &
% \vcounter(\trace :: (b, l', v, \bullet), l) \triangleq \vcounter(\trace, l), l' \neq l
% &
% \vcounter(\trace :: (x, l', v, \qval), l) \triangleq \vcounter(\trace, l), l' \neq l
% \\
% \vcounter({[]}, l) \triangleq 0
% &&
% \end{array}
% \]
% \input{trace}
%%% trace, queries
% A memory is standard, a map from variables to values. Queries can be uniquely annotated as $\mathcal{AQ}$, and the annotation $(l,w)$ considers the location of the query by line number $l$ and which iteration the query is at when it appears in a loop statement, specified by $w$. A trace $t$ is a list of annotated queries accumulated along the execution of the program. 



\paragraph*{Environment}
The function $\env : {\mathcal{T}} \to \mathcal{VAR} \to \mathcal{VAL} \cup \{\bot\}$, which maps a trace and a variable to the latest value assigned to this variable on the trace is defined as follows.
% \wq{Question: Seem $\env$ is a function that looks up in the input trace and returns you the latest value of the variable. I have a question, in the two-round example, I see $env(\tau)(k)$ while $k$ is not defined(it is input), so in our two-round example in Overview, the value is stored in the second event is $\bot$? Also, another important, $\env$ relies on the input trace, so it will not appear in the trace, or config, is it precise?}
% \jl{yes, it is precise. 
% I have initial trace and everything belonging is defined over all possible initial traces.
%in the two-round example, there is an initial trace where the value of k is defined there. It is worth explaining this here.
% }
\[
\begin{array}{lll}
\env(\trace \traceadd (x, l, v, \bullet)) x \triangleq v
&
\env(\trace \traceadd (y, l, v, \bullet)) x \triangleq \env(\trace) x, y \neq x
&
\env(\trace \traceadd (b, l, v, \bullet)) x \triangleq \env(\trace) x
\\
\env(\trace \traceadd (x, l, v, \qval)) x \triangleq v
&
\env(\trace \traceadd (y, l, v, \qval)) x \triangleq \env(\trace) x, y \neq x
&
\env({[]} ) x \triangleq \bot
\end{array}
\]
 %% trace

%
% figure, evaluation rules.
% {\footnotesize
% \begin{figure}
% \begin{mathpar}
% \boxed{ \config{m, c, t,w} \xrightarrow{} \config{m', c', t', w'} \; }
% \and
% %
% {\inferrule
% {
% \valr_N > 0
% }
% {
% \config{m, \eloop ~ [\valr_N]^{l} ~ \edo ~ c , t, w }
% \xrightarrow{} \config{m, c ; \eloop ~ [(\valr_N-1)]^{l} ~ \edo ~ c , t, (w + l) }
% }
% ~\textbf{low-loop}
% }
% %
% \and
% %
% \inferrule
% {
% }
% {
% \config{m, [\eskip]^{l} ; c_2, t,w} \xrightarrow{} \config{m, c_2, t,w}
% }
% ~\textbf{low-seq2}
% %
% \quad
% %
% {
% \inferrule
% {
% \valr_N = 0
% }
% {
% \config{m, \eloop ~ [\valr_N]^{l} ~ \edo ~ c , t, w }
% \xrightarrow{} \config{m, [\eskip]^{l} , t, (w \setminus l) }
% }
% ~\textbf{low-loop-exit}
% }
% \and
% %
% \inferrule
% {
% }
% {
% \config{m, \eif([\efalse]^{l}, c_1, c_2), t,w} 
% \xrightarrow{} \config{m, c_2, t,w}
% }
% ~\textbf{low-if-f}
% %
% ~~
% % { Memory \times Com \times Trace \times WhileMap \Rightarrow^{} Memory \times Com \times Trace \times WhileMap}
% \inferrule
% {
% \config{m,\expr} \to \expr'
% }
% {
% \config{m, [\assign{x}{q(\expr)}]^l, t, w} \xrightarrow{} \config{m, [\assign{x}{q(\expr')}]^l, t, w}
% }
% ~\textbf{low-query-e}
% %
% \and
% %
% %
% \inferrule
% {
% \config{m, c_1, t,w} \xrightarrow{} \config{m', c_1', t',w'}
% }
% {
% \config{m, c_1; c_2, t,w} \xrightarrow{} \config{m', c_1'; c_2, t',w'}
% }
% ~\textbf{low-seq1}
% ~~
% \inferrule
% {
% q(v) = v_q
% }
% {
% \config{m, [\assign{x}{q(v)}]^l, t, w} \xrightarrow{} \config{m[ v_q/ x], \eskip, t \mathrel{++} [q(v)^{(l,w )}],w }
% }
% ~\textbf{low-query-v}
% %
% % \inferrule
% % {
% % }
% % {
% % \config{m, [\assign x v]^{l}, t,w} \xrightarrow{} \config{m[v/x], [\eskip]^{l}, t,w}
% % }
% % ~\textbf{low-assn}
% %
% %
% %
% \and
% %
% \inferrule
% {
% \config{ m, \bexpr} \barrow \bexpr'
% }
% {
% \config{m, \eif([\bexpr]^{l}, c_1, c_2), t,w} 
% \xrightarrow{} \config{m, \eif([\bexpr']^{l}, c_1, c_2), t,w}
% }
% ~\textbf{low-if}
% %
% ~~~~
% %
% \inferrule
% {
% }
% {
% \config{m, \eif([\etrue]^{l}, c_1, c_2),t,w} 
% \xrightarrow{} \config{m, c_1, t,w}
% }
% ~\textbf{low-if-t}
% %
% % %
% %
% \end{mathpar}
% \vspace{-0.3cm}
% \caption{Trace-based operational semantics}
% \label{fig:evaluation}
% \vspace{-0.5cm}
% \end{figure}
% }
%
% explanation of rules

%
\begin{figure}
 \begin{mathpar}
 \boxed{
 \mbox{Command $\times$ Trace}
 \xrightarrow{}
 \mbox{Command $\times$ Trace}
 }
 \and
 \boxed{\config{{c, \trace}}
 \xrightarrow{} 
 \config{{c', \trace'}}
 }
 \\
 % \inferrule
 % {
 % \empty
 % }
 % {
 % \config{\clabel{\eskip}^l, \trace } 
 % \xrightarrow{} 
 % \config{\clabel{\eskip}^l, \trace}
 % }
 % ~\textbf{skip}
 %
 % \and
 %
 \inferrule
 {
 \config{\trace, \expr} \earrow v 
 \and
 \event = ({x}, l, v, \bullet)
 }
 {
 \config{[\assign{{x}}{\expr}]^{l}, \trace } 
 \xrightarrow{} 
 \config{\clabel{\eskip}^l, \trace \traceadd \event}
 }
 ~\textbf{assn}
 %
 \and
 %
 \highlight{
 \inferrule
 {
\config{ \trace, \qexpr} \qarrow \qval
 \and 
 \query(\qval) = v
 \and 
 \event = ({x}, l, v, \qval)
 }
 {
 \config{{[\assign{x}{\query(\qexpr)}]^l, \trace}}
 \xrightarrow{} 
 \config{{\clabel{\eskip}^l, \trace \traceadd \event} }
 }
 ~\textbf{query}
 }
 %
 \and
 %
 \inferrule
 {
\config{ \trace, b} \barrow \etrue
 \and 
 \event = (b, l, \etrue, \bullet)
 }
 {
 \config{{\ewhile [b]^{l} \edo c, \trace}}
 \xrightarrow{} 
 \config{{
 c; \ewhile [b]^{l} \edo c),
 \trace \traceadd \event}}
 }
 ~\textbf{while-t}
 %
 %
 \quad
 %
 \inferrule
 {
 \config{\trace, b} \barrow \efalse
 \and 
 \event = (b, l, \efalse, \bullet)
 }
 {
 \config{{\ewhile [b]^{l}, \edo c, \trace}}
 \xrightarrow{} 
 \config{{
 \clabel{\eskip}^l,
 \trace \traceadd \event}}
 }
 ~\textbf{while-f}
 %
 %
 \and
 %
 %
 \inferrule
 {
 \config{{c_1, \trace}}
 \xrightarrow{}
 \config{{\clabel{\eskip}^l, \trace'}}
 \and 
 \config{{\clabel{\eskip}^l; c_2, \trace'}} \xrightarrow{} \config{{ \clabel{\eskip}^l, \trace''}}
 }
 {
 \config{{c_1; c_2, \trace}} 
 \xrightarrow{} 
 \config{{\clabel{\eskip}^l, \trace''}}
 }
 ~\textbf{seq}
 %
 % \and
 % %
 % \inferrule
 % {
 % \config{{c_2, \trace}}
 % \xrightarrow{}
 % \config{{c_2', \trace'}}
 % }
 % {
 % \config{{\clabel{\eskip}^l; c_2, \trace}} \xrightarrow{} \config{{ c_2', \trace'}}
 % }
 % ~\textbf{seq2}
 %
 \quad
 %
 %
 \inferrule
 {
 \trace, b \barrow \etrue
 \and 
 \event = (b, l, \etrue, \bullet)
 }
 {
 \config{{
 \eif([b]^{l}, c_1, c_2), 
 \trace}}
 \xrightarrow{} 
 \config{{c_1, \trace \traceadd \event}}
 }
 ~\textbf{if-t}
 %
 % \and
 % %
 % \inferrule
 % {
 % \trace, b \barrow \efalse
 % \and 
 % \event = (b, l, \efalse, \bullet)
 % }
 % {
 % \config{{\eif([b]^{l}, c_1, c_2), \trace}}
 % \xrightarrow{} 
 % \config{{c_2, \trace \traceadd \event}}
 % }
 % ~\textbf{if-f}
 \end{mathpar}
 % \end{subfigure}
 \vspace{-0.5cm}
 \caption{Trace-based Operational Semantics for Language.}
 \label{fig:os}
 \end{figure}
 %

% {The big step trace-based operational semantics has the form of $ \config{c, \trace} \xrightarrow{} { \config{c', \trace'}}$. It reads that the configuration $(c, \trace)$ with labeled command $c$ and trace $\trace$, will be evaluated to another configuration, in which $c$ is evaluated to $c'$ and the trace is updated during the evaluation, to $\trace'$. 
% }
% The step trace-based operational semantics has the form of $ \config{c, \trace} \xrightarrow{} { \config{c', \trace'}}$. 
% It reads the configuration $\config{c, \trace}$ consisting of a labeled command $c$ and a pre-trace $\trace$, 
% and evaluates it to another configuration, 
% in which $c$ is evaluated to $c'$ and trace $\trace$ is updated to $\trace'$. 
% is updated during the evaluation,
\paragraph*{Operational Semantics}
I give a selection of rules of the trace-based operational semantics in Figure~\ref{fig:os}. 

% \todo{Make sure the operational semantics is a big step and correct assn rules.}
% \jl{
The rule $\textbf{assn}$ evaluates a standard assignment $\assign{x}{\expr}$, the expression $\expr$ is first evaluated by our expression evaluation $\config{\trace, \expr} \earrow v $, presented below. And the result $v$ of evaluating $\expr$ is used to construct a new event $\event = (x, l, v,\bullet)$ and attach it to the previous trace. 
\begin{mathpar}
% \boxed{ \config{\trace, \expr} \earrow v \, : \, \mbox{Trace $\times$ Expression $\Rightarrow$ Value} }
% \\
\inferrule{ 
 \config{\trace, \aexpr} \aarrow v
}{
 \config{\trace, \aexpr} 
 \earrow v
}
\and
\inferrule{ 
 \config{\trace, \bexpr} \barrow v
}{
 \config{\trace, \bexpr} 
 \earrow v
}
\and
\inferrule{ 
 \config{\trace, \expr_1} \earrow v_1
 \cdots
 \config{\trace, \expr_n} \earrow v_n
}{
 \config{\trace, [\expr_1, \cdots, \expr_n]} 
 \earrow [v_1, \cdots, v_n]
}
\and
\inferrule{ 
 \empty
}{
 \config{\trace, v} 
 \earrow v
}
\end{mathpar}
The expression evaluation rules also rely on the evaluation of arithmetic expressions $\config{\trace,\aexpr} \aarrow v $ and boolean expressions $\config{\trace, \bexpr} \barrow v $. The full rules can be found in the appendix.
% \begin{mathpar}
% \boxed{ \config{\trace,\aexpr} \aarrow v \, : \, \mbox{Trace $\times$ Arithmetic Expr $\Rightarrow$ Arithmetic Value} }
% % \text{\mg{Missing. Without these rules, it is difficult to understand why we need a trace to evaluate expressions.}}
% \\
% \inferrule{ 
% \empty
% }{
% \config{\trace, n} 
% \aarrow n
% }
% \and
% \inferrule{ 
% \env(\trace) x = v
% }{
% \config{\trace, x} 
% \aarrow v
% }
% \and
% \inferrule{ 
% \config{\trace, \aexpr_1} \aarrow v_1
% \and 
% \config{\trace, \aexpr_2} \aarrow v_2
% \and 
% v_1 \oplus_a v_2 = v
% }{
% \config{\trace, \aexpr_1 \oplus_a \aexpr_2} 
% \aarrow v
% }
% % \and
% % \inferrule{ 
% % \config{\trace, \aexpr} \aarrow v'
% % \and 
% % \elog v' = v
% % }{
% % \config{\trace, \elog \aexpr} 
% % \aarrow v
% % }
% % \and
% % \inferrule{ 
% % \config{\trace, \aexpr} \aarrow v'
% % \and 
% % \esign v' = v
% % }{
% % \config{\trace, \esign \aexpr} 
% % \aarrow v
% % }
% \\
% \boxed{ \config{\trace, \bexpr} \barrow v \, : \, \mbox{Trace $\times$ Boolean Expr $\Rightarrow$ Boolean Value} }
% % \text{\mg{Missing. Without these rules, it is difficult to understand why we need a trace to evaluate expressions.}}
% \\
% % \inferrule{ 
% % \empty
% % }{
% % \config{\trace, \efalse} 
% % \barrow \efalse
% % }
% % \and 
% % \inferrule{ 
% % \empty
% % }{
% % \config{\trace, \etrue} 
% % \barrow \etrue
% % }
% % \and 
% \inferrule{ 
% \config{\trace, \bexpr} \barrow v'
% \\ 
% \neg v' = v
% }{
% \config{\trace, \neg \bexpr} 
% \barrow v
% }
% \and 
% \inferrule{ 
% \config{\trace, \bexpr_1} \barrow v_1
% \\ 
% \config{\trace, \bexpr_2} \barrow v_2
% \\ 
% v_1 \oplus_b v_2 = v
% }{
% \config{\trace, \bexpr_1 \oplus_b \bexpr_2} 
% \barrow v
% }
% \and 
% \inferrule{ 
% \config{\trace, \aexpr_1} \aarrow v_1
% \\ 
% \config{\trace, \aexpr_2} \aarrow v_2
% \\ 
% v_1 \sim v_2 = v
% }{
% \config{\trace, \aexpr_1 \sim \aexpr_2} 
% \barrow v
% }
% \end{mathpar}
% % }


Distinguished from the standard assignment evaluation, 
the rule $\textbf{query}$ 
evaluates a query requesting command $\clabel{\assign{x}{\query(\qexpr)}}^l$ in two steps.
The query expression $\qexpr$ is first evaluated into a query value $\qval$ by following the rules below.
Then, by sending this query request $\query(\qval)$ to a hidden mechanism, this query is evaluated to a result value returned from it, $v = \query(\qval)$.
% by sending this query request $\query(\qval)$ to it.
Also, the generated event stores both the query value $\alpha$ here, and the result value of the query request.

\begin{mathpar}
% \boxed{ \config{\trace, \qexpr} \qarrow \qval \, : \, \mbox{Trace $\times$ Query Expr $\Rightarrow$ Query Value} }
% \\
\inferrule{ 
 \config{\trace, \aexpr} \aarrow n
}{
 \config{\trace, \aexpr} 
 \qarrow n
}
\and
\inferrule{ 
 \config{\trace, \qexpr_1} \qarrow \qval_1
 \and
 \config{\trace, \qexpr_2} \qarrow \qval_2
}{
 \config{\trace, \qexpr_1 \oplus_a \qexpr_2} 
 \qarrow \qval_1 \oplus_a \qval_2
}
\and
\inferrule{ 
 \config{\trace, \aexpr} \aarrow n
}{
 \config{\trace, \chi[\aexpr]} \qarrow \chi[n]
}
\and
\inferrule{ 
 \empty
}{
 \config{\trace, \qval} 
 \qarrow \qval
}
 \end{mathpar}
% }
% \wq{The rules for if hand while both have two versions, when the guard evaluates to true and false, respectively. In these rules, the evaluation of the guard also generates a testing event and our trace is updated as well. }
The rules for if and while both have two versions 
when the boolean expressions in the guards are evaluated to true and false, respectively. 
In these rules, the evaluation of the guard generates a testing event and the trace is updated as well by appending this event.
% The rule $\textbf{query}$ evaluates the argument of a query request to a normal form and obtains the answer $v_q$ of the query $\query(v)$ from the mechanism. 
% Then the trace is expanded by appending the query expression $\query(v)$ with the current annotation $(l,w)$. 

% The rule for assignment is standard and the trace remains unchanged. The sequence rule keeps tracking the modification of the trace, and the evaluation rule for if conditional 

% \jl{If we observe the operational semantics rules, we can find that no rule will shrink the trace.} 
% If we observe the operational semantics rules, we can find that no rule will shrink the trace. It is proved in the appendix.
% So we have the Lemma~\ref{lem:tracenondec}, specifically, the trace has the property that its length never decreases during the program execution.

% \begin{lem}
% [Trace Non-Decreasing]
% \label{lem:tracenondec}
% For every program $c \in \cdom$ and traces $\trace, \trace' \in \mathcal{T}$, if 
% $\config{c, \trace} \rightarrow^{*} \config{\eskip, \trace'}$,
% then there exists a trace $\trace'' \in \mathcal{T}$ with $\trace \tracecat \trace'' = \trace'$
% %
% $$
% \forall \trace, \trace' \in \mathcal{T}, c \st
% \config{c, \trace} \rightarrow^{*} \config{\eskip, \trace'} 
% \implies \exists \trace'' \in \mathcal{T} \st \trace \tracecat \trace'' = \trace'
% $$
% \end{lem}
% %
% % \mg{This corollary needs some explanation. In particular, we should stress that $\event$ and $\event'$ may differ in the query value.}
% % Since the equivalence over two events is defined over the query value equivalence, 
% % when there is an event 
% % belonging to a trace, 
% % it is possible that the event showing up in this trace has a different form of query value, but they are equivalent by Definition~\ref{def:query_equal}.
% Since the equivalence over two events is defined over the query value equivalence, 
% when there is an event belonging to a trace, 
% if this event is a query assignment event, 
% it is possible that 
% the event showing up in this trace has a different form of query value, 
% but they are equivalent by Definition~\ref{def:query_equal}.
% So we have the following Corollary~\ref{coro:aqintrace} with proof in Appendix.
% % ~\ref{apdx:lemma_sec123}.
% % \todo{we should stress that $\event$ and $\event'$ may differ in the query value.}
% \begin{coro}
% \label{coro:aqintrace}
% For every event and a trace $\trace \in \mathcal{T}$,
% if $\event \in \trace$, 
% then there exist another event $\event' \in \eventset$ and traces $\trace_1, \trace_2 \in \mathcal{T}$
% such that $\trace_1 \tracecat [\event'] \tracecat \trace_2 = \trace $
% with 
% $\event$ and $\event'$ equivalent but may differ in their query value.
% \[
% \forall \event \in \eventset, \trace \in \mathcal{T} \st
% \event \in \trace \implies \exists \trace_1, \trace_2 \in \mathcal{T}, 
% \event' \in \eventset \st (\event = \event') \land \trace_1 \tracecat [\event'] \tracecat \trace_2 = \trace 
% \]
% \end{coro}



\cleardoublepage

\section{The Execution-Based Program Analysis for Adaptivity}
\label{sec:dynamic}

Based on the language and the trace-based operational semantics in Section~\ref{sec:language},
I formalize the intuitive through an execution based program analysis in this section.

\subsection{Introduction and Related Work}
\label{subsec:dynamic-intro}

I first introduce some related works as background of the execution-based analysis, 
then structure of this execution-based analysis.  
% \\
% The construction of this graph requires me to think about the dependency relation between two queries using what we have at hand - 
% the trace generated in Section~\ref{sec:language}. 
 \paragraph*{Related Work}
 {
My framework constructs a execution-based dependency graph based on the execution traces of a program. I define semantic dependence on this graph by considering (intraprocedural) data and control dependency~\cite{bilardi1996framework,cytron1991efficiently,pollock1989incremental}.    
One related work  
\cite{austin1992dynamic} presents a methodology to construct a dynamic dependency graph (DDG) based on the dynamic execution of a program in an imperative language, where edges represent dependency between instructions. Data dependency, control dependency, storage dependency, and resource dependency between instructions are all considered. My execution-based dependency graph only needs data dependency and control dependency between variable assignment results. 
% Critical path length analysis on DDGs is useful for understanding the scope for parallelization, while we use the length of the longest path to define adaptivity.  
%
DDGs have been used in many other domains. \cite{nagar2018automated} use DDGs to find serializability violations. \cite{hammer2006dynamic} use similar \emph{program dependency graphs} \cite{ferrante1987program} for dynamic program slicing.
\cite{mastroeni2008data} propose ways of constructing different kinds of program slices, by choose different program dependency. 
% For example, in either syntactic or semantics sense.
% This abstract dependency is based on properties rather than exact data.
% Aims to give finer and smaller program slice. 
They actually use a combination of  
static and dynamic dependency graphs but in a manner that is different from how we use the two. Their slicing uses both static and dynamic dependency graphs, while we use the dynamic dependency graph as the basis of a definition, which is then soundly approximated by an analysis based on the static dependency graph.}

{My execution-based data dependency relation definition over variables 
is inspired by the method in \cite{Cousot19a}, where the dependency relation is also identified by looking into the differences on two execution traces. 
However, Cousot excludes timing channels~\cite{SabelfeldM03} and empty observation, which are also not considered as a form of dependency in traditional dependency analysis \cite{DenningD77}.
% In the cases of empty observation and timing channels, the second query is executed 
% in one trace and isn't in another trace by modifying value of first query. 
% Then, the second query is indeed depend on the first query and there exists an
% adaptivity round between the two queries. 
My definition includes timing channels and empty observation by observing both the disappearance and value variation.
}
\paragraph*{Analysis Strcuture}
In order to formalize a quantitative property w.r.t. the dependency relation in program, I
use a three-step analysis methodology developed, 
 as follows,
\\
 a. The dependency relation between every query, through the methodology of semantic data dependency analysis.
\\
 b. The dependency quantity analysis, through the methodology of execution-based data reachability bound analysis. Then 
\\
 c. The adaptivity analysis, based on the two analysis results above, 
 I construct an execution-based dependency graph combining the dependency relation and the dependency quantity
    and give the formal \emph{adaptivity} definition 
    for program.
    This analysis is the first part of the analysis in Figure~\ref{fig:structure}.

\subsection{Methodology}
\label{subsec:dynamic-methodology}

\subsubsection{Data Dependency Analysis}
\label{subsubsec:dynamic-datadep}
\paragraph*{Challenge}
In the data analysis model our programming framework supports, 
%  an \emph{analyst} asks a sequence of queries to the mechanism, and receives the answers to these queries from the mechanism. In this model, the adaptivity we are interested in is the length of the longest sequence of such adaptively chosen queries, among all the queries the data analyst asks. 
  we define that a query is adaptively chosen when it is affected by answers of previous queries. The next thing is to decide how do we define whether one query is "affected" by previous answers, with the limited information we have? As a reminder, 
 when the analyst asks a query, the only known information will be the answers to previous queries and the current execution trace of the program.


There are two possible situations that a query will be "affected",  
either when the query expression directly uses the results of previous queries (data dependency), or when the control flow of the program with respect to a query (whether to ask this query or not) depends on the results of previous queries (control flow dependency).
% As a first step, we give a definition of when one query may depend on a previous query, which is supposed to consider both control dependency and data dependency. We first look at two possible candidates:
% \begin{enumerate}
%     \item One query may depend on a previous query if and only if a change of the answer to the previous query may also change the result of the query.
%     \item One query may depend on a previous query if and only if a change of the answer to the previous query may also change the appearance of the query.
% \end{enumerate}


Since the the results of previous queries can be stored or used in variables
which aren't associated to the query request,
it is necessary to track the dependency between queries, through all the program's variables,  
and then we can distinguish variables which are assigned with query requests.
 We give a definition of when one variable \emph{may-depend} on a previous variable with two candidates.
{
\begin{enumerate}
    \item One variable may depend on a previous variable if and only if a change of the value assigned to the previous variable may also change the value assigned to the variable.
    \item One variable may depend on a previous variable if and only if a change of the value assigned to the previous variable may also change the appearance of the assignment command to this variable 
    % in\wq{during?} 
    during execution.
\end{enumerate}
}
%   The first candidate works well by witnessing the result of one query according to the change of the answer of another query. We can easily find that the two queries have nothing to do with each other in a simple example   

{   
% The first situations works well by witnessing the result assigned to variable 
% according to the change of the value assigned to another query. 
% We can easily find that the two queries have nothing to do with each other in a simple example 
% In the first one, by defining the dependency as
The first definition is defined as
% witnessing 
% the query expressions equivalence (or the value equality for non-query assignment )
the witness of a variation on the value assigned to the same variable through two executions,
% assigned to the same variable through two executions, 
according to the change of the value assigned to another variable in pre-trace.
% the situation of data-dependency works well. \wq{long sentence, make it short?}
In particular for query requests, the variation we observe is on the query value instead of on the query requesting results.
% We can find that two queries 
% % have nothing to do with each other in this simple example 
% % depends on each other\wq{not each other, one direction.} 
% satisfy this definition
In 
%this 
the simple program $c_1 =\assign{x}{\query(\chi[2])} ;\assign{y}{\query(\chi[3] + x)}$.
 %
 From our perspective, $\query(\chi[1])$ is different from $\query(\chi[2]))$. Informally, we think $\query(\chi[3] + x)$ may depend on the query $\query(\chi[2]))$, because equipped function of the former $\chi[3] + x$ may depend on the data stored in x assigned with the result of $\query(\chi[2]))$, according to this definition. }
%
% in this example: $c_1 = \assign{x}{\query(0)}; \assign{z}{\query(\chi[x])}$.
% This candidate definition works well 
Nevertheless, the first definition fails to catch control dependency because it just monitors the changes to a query, but misses the appearance of the query when the answers of its previous queries change. 
For instance, it fails to handle $
      c_2 = \assign{x}{\query(\chi[1])} ; \eif( x > 2 , \assign{y}{\query(\chi[2])}, \eskip )
   $, but the second definition can. However, it only considers the control dependency and misses the data dependency. This reminds us to define a \emph{may-dependency} relation between labeled variables by combining the two definitions to capture the two situations.
%
%
%
%
\paragraph{Dependency}
 To define the may dependency relation on two labeled variables, we rely on the limited information at hand - the trace generated by the operational semantics. In this end, we first define the \emph{may-dependency} between events, and use it as a foundation of the variable may-dependency relation.
\begin{defn}[Events Different up to Value ($\diff$)]
  Two events $\event_1, \event_2 \in \eventset$ are  \emph{Different up to Value}, 
  denoted as $\diff(\event_1, \event_2)$ if and only if:
  \[
    \begin{array}{l}
  \pi_1(\event_1) = \pi_1(\event_2) 
  \land  
  \pi_2(\event_1) = \pi_2(\event_2) \\
  \land  
  \big(
    (\pi_3(\event_1) \neq \pi_3(\event_2)
  \land 
  \pi_{4}(\event_1) = \pi_{4}(\event_2) = \bullet )
  % \qquad \qquad 
  \lor 
  (\pi_4(\event_1) \neq \bullet
  \land 
  \pi_4(\event_2) \neq \bullet
  \land 
  \pi_{4}(\event_1) \neq_q \pi_{4}(\event_2)) 
  \big)
  \end{array}
  \]
  \end{defn}
 %
 We compare two events by defining $\diff(\event_1, \event_2)$. We use $\qexpr_1 =_{q} \qexpr_2$ and $\qexpr_1 \neq_{q} \qexpr_2$ to notate query expression equivalence and in-equivalence, distinct from standard equality. A program $c$'s
%  , its 
 labeled variables 
%  and assigned variables are subsets of 
is a subset of
the labeled variables $\mathcal{LV}$, denoted by $\lvar(c) \in \mathcal{P}(\mathcal{VAR} \times \mathcal{L}) \subseteq \mathcal{LV}$.
% annotated by a label. 
% We use  
%$\mathcal{LVAR} = \mathcal{VAR} \times \mathcal{L} $ 
% $\mathcal{LV}$ represents the universe of all the labeled variables and 
% $\avar(c) \in \mathcal{P}(\mathcal{VAR} \times \mathbb{N}) \subset \mathcal{LV}$ and 
% $\lvar(c) \in \mathcal{P}(\mathcal{VAR} \times \mathcal{L}) \subseteq \mathcal{LV}$ for them. 
We also define the set of query variables for a program $c$, $\qvar: \cdom \to 
\mathcal{P}(\mathcal{LV})$.

A program $c$'s query variables is a subset of 
its labeled variables, $\qvar(c) \subseteq \lvar(c)$. We have the operator $\tlabel : \mathcal{T} \to \ldom$, which gives the set of labels in every event belonging to the trace.
Then we introduce a counting operator $\vcounter : \mathcal{T} \to \mathbb{N} \to \mathbb{N}$, 
% \wq{which counts the occurrence of of a variable in the trace,} 
which counts the occurrence of a labeled variable in the trace,
whose behavior is defined as follows,
\[
\begin{array}{ll}
\vcounter(\trace :: (\_, l, \_, \_), l ) \triangleq \vcounter(\trace, l) + 1
&
\vcounter(\trace  ::(b, l, v, \bullet), l) \triangleq \vcounter(\trace, l) + 1
\\
\vcounter(\trace  :: (x, l, v, \qval), l) \triangleq \vcounter(\trace, l) + 1
&
% \vcounter(\trace :: (\_, l', \_, \_), l ) \triangleq \vcounter(\trace, l), l' \neq l 
% &
\vcounter(\trace  :: (x, l', v, \bullet), l) \triangleq \vcounter(\trace, l), l' \neq l
\\
\vcounter(\trace  :: (b, l', v, \bullet), l) \triangleq \vcounter(\trace, l), l' \neq l
&
\vcounter(\trace  :: (x, l', v, \qval), l) \triangleq \vcounter(\trace, l), l' \neq l
\\
\vcounter({[]}, l) \triangleq 0
\end{array}
\]
The full definitions of these above operators can be found in the appendix.
\begin{defn}[Event May-Dependency].
\label{def:event_dep}
\\ 
  An event $\event_2$ is in the \emph{event may-dependency} relation with an assignment
  event $\event_1 \in \eventset^{\asn}$ in a program ${c}$
  with a hidden database $D$ and a trace $\trace \in \mathcal{T}$ denoted as 
  %
  $\eventdep(\event_1, \event_2, [\event_1 ] \tracecat \trace \tracecat [\event_2], c, D)$, iff
  %
  \[
    \begin{array}{l}
  \exists \vtrace_0,
  \vtrace_1, \vtrace' \in \mathcal{T},\event_1' \in \eventset^{\asn}, {c}_1, {c}_2  \in \cdom  \sthat
  \diff(\event_1, \event_1') \land 
      \\ \quad
      (
        \exists  \event_2' \in \eventset \sthat 
    \left(
    \begin{array}{ll}   
   & \config{{c}, \vtrace_0} \rightarrow^{*} 
  \config{{c}_1, \vtrace_1 \tracecat [\event_1]}  \rightarrow^{*} 
    \config{{c}_2,  \vtrace_1 \tracecat [\event_1] \tracecat \vtrace \tracecat [\event_2] } 
    % 
   \\ 
   \bigwedge &
    \config{{c}_1, \vtrace_1 \tracecat [\event_1']}  \rightarrow^{*} 
    \config{{c}_2,  \vtrace_1 \tracecat[ \event_1'] \tracecat \vtrace' \tracecat [\event_2'] } 
  \\
  \bigwedge & 
  \diff(\event_2,\event_2' ) \land 
  \vcounter(\vtrace, \pi_2(\event_2))
  = 
  \vcounter(\vtrace', \pi_2(\event_2'))\\
  \end{array}
  \right)
  \\ \quad
  \lor 
  \exists \vtrace_3, \vtrace_3'  \in \mathcal{T}, \event_b \in \eventset^{\test} \sthat 
  \\ \quad
  \left(
  \begin{array}{ll}   
    & \config{{c}, \vtrace_0} \rightarrow^{*} 
      \config{{c}_1, \vtrace_1 \tracecat [\event_1]}  \rightarrow^{*} 
      \config{c_2,  \vtrace_1 \tracecat [\event_1] \tracecat \trace \tracecat [\event_b] \tracecat  \trace_3} 
    \\ 
    \bigwedge &
    \config{{c}_1, \vtrace_1 \tracecat [\event_1']}  \rightarrow^{*} 
    \config{c_2,  \vtrace_1 \tracecat [\event_1'] \tracecat \trace' \tracecat [(\neg \event_b)] \tracecat \trace_3'} 
    \\
    \bigwedge &  \tlabel_{\trace_3} \cap \tlabel_{\trace_3'} = \emptyset
     \land \vcounter(\trace', \pi_2(\event_b)) = \vcounter(\trace, \pi_2(\event_b)) 
    %   \land \event_2 \eventin \trace_3
    % \land \event_2 \not\eventin \trace_3'
    \land \event_2 \in \trace_3
    \land \event_2 \not\in \trace_3'
  \end{array}
  \right)
  )
\end{array}
   \]
% , where ${\tt label}(\event_2) = \pi_2(\event_2)$.
  %  
%
\end{defn}
% \todo{add explnanation}
% \jl{
Our event \emph{may-dependency} relation of 
two events $\event_1 \in \eventset^{\asn}$ and $\event_2 \in \eventset$, 
for a program $c$ and hidden database $D$ is w.r.t to
a trace $[\event_1 ] \tracecat \trace \tracecat [\event_2]$.
The $\event_1 \in \eventset^{\asn}$ is an assignment event because only a change on an assignment event will affect the execution trace, according to our operational semantics.
In order to observe the changes of $\event_2$ under the modification of $\event_1$, this trace 
$[\event_1 ] \tracecat \trace \tracecat [\event_2]$
starts with $\event_1$ and ends with $\event_2$.
% }
{The \emph{may-dependency} relation considers both the value dependency and value control dependency as discussed in Section~\ref{sec:design_choice}. The relation can be divided into two parts naturally in Definition~\ref{def:event_dep} (line $2-4$, $5-8$ respectively, starting from line $1$). The idea of the event $\event_1$ may depend on $\event_2$ can be briefly described:
we have one execution of the program as reference (See line $2$ and $6$, for the two kinds of dependency). 
When the value assigned to the 
% first variable 
first variable in $\event_1$ is modified, the reference trace $\trace_1 \tracecat [\event_1]$ is modified correspondingly to $\trace_1 \tracecat [\event_1']$.
We use $\diff(\event_1, \event_1')$ at line $1$ to express this modification, which guarantees that $\event_1$ and $\event_1'$ only differ in their assigned values and are equal on variable name and label. We perform a second run of the program by continuing the execution of the same program from the same execution point, 
but with the modified trace $\trace_1 \tracecat [\event_1']$ (See line $3$, $7$). 
The expected may dependency will be caught by observing two different possible changes (See line $4, 8$ respectively) when comparing the second execution with the reference one (similar definitions as in \cite{Cousot19a}). 

% \wq{
% In the first situation, we are witnessing 
In the first part (line $2-4$ of Definition~\ref{def:event_dep}), we witness
% that the value assigned to the second variable in $\event_2$
the appearance of $\event_2'$ in the second execution, and
% a variation in $\event_2$, which changes into $\event_2'$.
a variation between $\event_2$ and $\event_2'$ on their values.
% changes in $\event_2'$.
% \jl{
We have special requirement $\diff(\event_2, \event_2')$, which guarantees that they
have the same variable name and label but only differ 
% % in their assigned value. 
in their evaluated values.
% assigned to the same variable. 
In particularly for queries, if $\event_2$ and $\event_2'$ are 
% query assignment events, then 
generated from query requesting, then $\diff(\event_2, \event_2')$ guarantees that
they differ in their query values rather than the 
% query requesting value. 
query requesting results. 
Additionally, in order to handle multiple occurrences of the same event through iterations of the while loop,
 where  $\event_2$ and $\event_2'$ could be 
in different while loops,
we restrict the same occurrence of $\event_2$'s label in $\trace$ from the first execution with  the occurrence of $\event_2'$'s label in $\trace'$ from the second execution,
through $\vcounter(\vtrace, \pi_2(\event_2))
= 
\vcounter(\vtrace', \pi_2(\event_2'))$ at line $4$.
% }
% }

% \wq{
In the second part (line $5-8$ of Definition~\ref{def:event_dep}), we 
% are witnessing 
witness
the disappearance of $\event_2$ through observing the change of a testing event $\event_b$.
% In order to change the appearance of 
% % and event, the command that generating $\event_2$ must not be executed in 
% 5yhan event, 
To witness
the disappearance, the command that generates $\event_2$ must not be executed in 
the second execution. 
The only way to control whether a command will be executed, is through the change of a guard's 
evaluation result in an if or while command, which generates a testing event $\event_b$ in the first place.
So we observe when
$\event_b$ changes into $\neg \event_b$ in the second execution firstly, 
whether it follows with the disappearance of $\event_2$ in the second trace. We restrict the occurrence of $\event_b$'s label in the two traces being the same
}
% s to the occurrence times of $\event_2'$'s label in the second trace,
through $\vcounter(\trace', \pi_2(\event_b)) = \vcounter(\trace, \pi_2(\event_b))$ to handle the while loop.
% changes in $\event_2'$, have the same variable and label and only differ in their assigned value. 
Again, for queries, we observe the disappearance based on the query value equivalence.
% if $\event_2$ and $\event_2'$ are query assignment events, then 
% they differ in their query value rather than the assigned value. 
% }
%
% \mg{I don't understand this explanation. What are the ``assignment commands associated to the two labelled variables''}
% \jl{revised but need more think}
% Explanation: 

{Considering 
% a program's all possible executions 
all events generated during a program's executions
under an initial trace,
% among all events generated during these executions
% and the variables and labels of these events are 
% corresponding to the two labeled variables,
% evaluations of the assignment commands associated to the two labelled variables respectively, 
as long as there is one pair of events satisfying the \emph{event may-dependency} relation in Definition~\ref{def:event_dep}, 
 we say the two 
related
variables satisfy the \emph{variable may-dependency} relation, in Definition~\ref{def:var_dep}.
}

\begin{defn}[Variable May-Dependency].
  \label{def:var_dep}
  \\
  A variable ${x}_2^{l_2} \in \lvar(c)$ is in the \emph{variable may-dependency} relation with another
  variable ${x}_1^{l_1} \in \lvar(c)$ in a program ${c}$, denoted as 
  %
  $\vardep({x}_1^{l_1}, {x}_2^{l_2}, {c})$, if and only if.
\[
  \begin{array}{l}
\exists \event_1, \event_2 \in \eventset^{\asn}, \trace \in \mathcal{T} , D \in \dbdom \sthat
% (\pi_{1}{(\event_1)}, \pi_{2}{(\event_1)}) = ({x}_1, l_1)
% \land
% (\pi_{1}{(\event_2)}, \pi_{2}{(\event_2)}) = ({x}_2, l_2)
\pi_{1}{(\event_1)}^{\pi_{2}{(\event_1)}} = {x}_1^{l_1}
\land
\pi_{1}{(\event_2)}^{\pi_{2}{(\event_2)}} = {x}_2^{l_2}% \\ \quad 
\land 
\eventdep(\event_1, \event_2, \trace, c, D) 
  \end{array}
\]  %
\end{defn}
\subsubsection{Data Dependency Quantity Analysis}
\label{subsubsec:dynamic-reachability}
For a program $c$, there are two data \emph{dependency quantities} we are considering.
The first quantity is the reachability times of each labeled variable during the program execution.
The second quantity is the reachability time for every pair of labeled variables with variable \emph{may-dependency} relation.
% \paragraph*{Variable Reachability}
\paragraph{The Dependency Quantity for Labeled Variables}
The reachability time of a labeled variable indicates the evaluation times of the assignment command assigning a value to this variable.  
\begin{defn}[Reachability Time of Labeled Variable]
  \label{def:adapt-var_reachability}
The reachability for every labeled variable overall $c$'s execution traces,
w.r.t. an initial trace $\vtrace \in \mathcal{T}_0(c)$ is defined as follows,
\[
  rb(x^l) \triangleq \forall \vtrace \in \mathcal{T}_0(c), \trace' \in \mathcal{T} \sthat \config{{c}, \trace} \to^{*} \config{\eskip, \trace\tracecat\vtrace'} 
  \implies w(\trace) = \vcounter(\vtrace', l) 
  \]
\end{defn}
%
$(x^l, w) \in \mathcal{LV} \times (\mathcal{T} \to \mathbb{N})$,
with a labeled variable as first component and
its weight $w$ the second component.
Weight $w$ for
% a labeled variable 
$x^l$ is a function $w : \mathcal{T} \to \mathbb{N}$
mapping from a starting trace to a natural number.
When program executes under this starting trace $\trace$,
$\config{{c}, \trace} \to^{*} \config{\eskip, \trace\tracecat\vtrace'} $, it generates an execution trace $\trace'$.
This natural number is the evaluation times of the labeled command corresponding to the vertex, 
computed by the counter operator $w(\trace) = \vcounter(\vtrace', l)$.


In most data analysis programs $c$ we are interested, there are usually some user input variables, such as $k$ in $\kw{twoRounds}$. 
We denote $\mathcal{T}_0(c)$ as the set of initial traces in which all the input variables in $c$ are initialized, it is also reflected in $\traceW({c})$.    
%
\paragraph{Dependency Quantity for the Pair of Labeled Variables}
% \paragraph*{Dependent Variables Reachability}
%
% For a program $c$ I compute the reachability bound for every labeled variable overall $c$'s execution traces,
% w.r.t. an initial trace as follows,
\begin{defn}[Reachability Time of Dependent Variables]
  \label{def:adapt-depvar_reachability}
  The execution-based reachability time for every pair of 
  labeled in the
  \emph{may-dependency} relation w.r.t. an initial trace. Formally as follows,
    \[
    \begin{array}{l}
        rb(x^i, y^j) \triangleq 
%   x^i, y^j \in \lvar(c)
%   \land w \in \mathcal{P}( \mathcal{T}_0(c) \to \mathbb{N})
%   \land 
%   \exists \trace \in \mathcal{T}_0(c), 
%   \trace_1, \trace_2 \in \mathcal{T} \sthat \dep(x^i, y^j,\trace_1, \trace_2, \trace_0, c)
%   \\
%   \land 
\forall \trace_0 \in \mathcal{T}_0(c) \sthat
  w (\trace_0) = \max \left\{ | \sdiff(\trace_1, \trace_2, y)|
  ~\middle\vert~
  \forall \trace_1, \trace_2 \in \mathcal{T} \sthat \dep(x^i, y^j,\trace_1, \trace_2, \trace_0, c) \right\}
\end{array}
\]
\end{defn}
%
For any pair of labeled variable $(x^i, y^j) \in \ldom$, 
$ rb(x^i, y^j)$ is a function $w: \mathcal{T}_0(c) \to \mathbb{N}$,
    where given an initial trace $\trace_0$,
    it is the maximum length of the difference sequence between all pairs of the witness traces $\trace_1, \trace_2$ 
    satisfying the dependency relation.

    \highlight{\paragraph*{Improvements Analysis}
    Previous works do not have any quantity analysis on the dependency relation.
    Comparing to them, this part is stronger in following senses.
    % It is more scalable to general program, and it provides the program with preciser formal definition for \emph{Adaptivity} than previous definition,
    % specifically as follows.
    % language and operational semantics design improves the expressiveness, efficiency, and the accuracy to a large extend.
    \todo{Add details}
    \begin{itemize}
      \item \textbf{Improvements on Efficiency}
      \\
      It is also efficient.
      \item \textbf{Improvements on Accuracy}
      This quantity analysis can help to improve the precision of the adaptivity formalization.
      \end{itemize}
      }

\paragraph*{The Dependency Quantity through The Two Rounds Example}
\begin{example}[Variable \emph{May-Dependency} Quantity in The Two Rounds Data Analysis Example Program]
    In the same $\kw{towRounds(k)}$ example Program,    the analyst asks in total $k+1$ queries to the mechanism in two phases.
    %
    \[         \begin{array}{l}
      \kw{towRounds(k)} \triangleq \\
             \clabel{ \assign{a}{0}}^{0} ;
              \clabel{\assign{j}{k} }^{1} ;\\
              \ewhile ~ \clabel{j > 0}^{2} ~ \edo ~
              \Big(
               \clabel{\assign{x}{\query(\chi[j] \cdot \chi[k])} }^{3}  ;
               \clabel{\assign{j}{j-1}}^{4} ;
              \clabel{\assign{a}{x + a}}^{5}       \Big);\\
              \clabel{\assign{l}{\query(\chi[k]*a)} }^{6}
          \end{array}
          \]    %
    % Queries are of the form $q(e)$ where $e$ is an expression with a special variable $\chi$ representing a possible row. Mainly $e$ represents a function from $X$ to some domain $U$, for example $U$ could be $[-1,1]$ or $[0,1]$. This function characterizes the linear query I are interested in running. As an example, $x \leftarrow q(\chi[2])$ computes an approximation, according to the used mechanism, of the empirical mean of the second attribute, identified by $\chi[2]$. Notice that I don't materialize the mechanism but I assume that it is implicitly run when I execute the query. 
    % \jl{We use $\chi$ to abstract a possible row in the database and }
    % queries are of the form $\query(\qexpr)$, where $\qexpr$ is a special expression 
    With the initial trace
    $[(k, in, 2, \bullet)]$ and following execution trace, 
    \\
    $
    \trace_1 \triangleq 
    \left[\begin{array}{l}
    % \trace_0 \tracecat
     (a, 0, 0, \bullet),
    (j, 1, 2, \bullet),
    (j>0, 2, \etrue, \bullet),
    (x, 3, v_1, \chi[2]*\chi[2]),
    (j, 4, 1, \bullet),
    (a, 5, v_1, \bullet),\\
    (j>0, 2, \etrue, \bullet),
    (x, 3, v_2, \chi[1]*\chi[2]),
    (j, 4, 0, \bullet),
    (a, 5, v_1 + v_2, \bullet),
    (j>0, 4, \efalse, \bullet),\\
    (l, 6, v_3, \chi[2]*( v_1 + v_2))
    \end{array} \right]
    $.
    Based on these observations, we analyze the \emph{may-dependency} quantity for every labeled variable,
    and pairs of dependent variables as follows.
\begin{itemize}
    \item \textbf{The Dependency Quantity for Labeled Variables}
    \\
    For the specific two execution traces above,
    the \emph{may-dependency} quantity for every variable
    is computed as follows,
   %   where $k$ is the 
   %  initial value of input variable $k$ given by user,
   %  we observe the execution trace as
   \\
   $rb(a^0) ((k, in, 2, \bullet))  = \vcounter(\trace_1) = 1$
   \\
   $\cdots$
  \\
   $rb(x^3) ((k, in, 2, \bullet))  = \vcounter(\trace_1) = 2$
    \\
    $\cdots$
    \\

    Then, for arbitrary initial trace $\trace_0 \in \mathcal{T}_0(\kw{twoRounds(k)})$,
    the \emph{may-dependency} quantity for every variable under $\trace_0$ is a function
    as follows,
    \\
    $rb(a^0) (\trace_0)  = 1$
    \\
    $\cdots$
    \\
    $rb(x^3) (\trace_0)  = \max\{0, \env(\trace_0) k \} $
    \item \textbf{Dependency Quantity for the Pair of Labeled Variables}
    For the specific two execution traces above,
    the \emph{may-dependency} quantity for every variable
    is computed as follows,
   %   where $k$ is the 
   %  initial value of input variable $k$ given by user,
   %  we observe the execution trace as
   \\
   $rb(a^0) ((k, in, 2, \bullet))  = \vcounter(\trace_1) = 1$
   \\
   $\cdots$
  \\
   $rb(x^3) ((k, in, 2, \bullet))  = \vcounter(\trace_1) = 2$
    \\
    $\cdots$
    \\

    Then, for arbitrary initial trace $\trace_0 \in \mathcal{T}_0(\kw{twoRounds(k)})$,
    the \emph{may-dependency} quantity for every variable under $\trace_0$ is a function
    as follows,
    \\
    $rb(a^0) (\trace_0)  = 1$
    \\
    $\cdots$
    \\
    $rb(x^3) (\trace_0)  = \max\{0, \env(\trace_0) k \} $
\end{itemize}
    % \\
    % We modify the value assigned to $x$ when evaluating the command $ \clabel{\assign{x}{\query(\chi[j] \cdot \chi[k])} }^{3}$
    % in the first iteration.
    % By manipulating the event in the trace, 
    % the event $(x, 3, v_1, \chi[2]*\chi[2])$
    % is modified into $(x, 3, v_1', \chi[2]*\chi[2])$ where $v_1 \neq v_1'$.
    % Then, through executing the program from the execution point after executing line $3$, we observe another execution trace as follows,
    % \\
    % $
    % \left[\begin{array}{l}
    % % \trace_0 \tracecat
    %  (a, 0, 0, \bullet),
    % (j, 1, 2, \bullet),
    % (j>0, 2, \etrue, \bullet),
    % \highlight{(x, 3, v_1', \chi[2]*\chi[2])},
    % (j, 4, 1, \bullet),
    % (a, 5, v_1, \bullet),\\
    % (j>0, 2, \etrue, \bullet),
    % (x, 3, v_2, \chi[1]*\chi[2]),
    % (j, 4, 0, \bullet),
    % \highlight{(a, 5, v_1' + v_2, \bullet),}
    % (j>0, 4, \efalse, \bullet),\\
    % (l, 6, v_3, \chi[2]*( v_1' + v_2))
    % \end{array} \right]
    % $.   
    % \\
    % In this trace, the event $(a, 5, v_1' + v_2, \bullet),$  is different from $(a, 5, v_1 + v_2, \bullet),$ in the first 
    % trace.
    % \\
    % This change satisfies the Definition~\ref{def:event_dep}, so there exists the variable \emph{may-dependency} relation 
    % between variable $x^3$ and $a^5$.
\end{example}%
\subsubsection{Execution-Based Adaptivity Analysis}
\label{subsubsec:dynamic-adapt}

Based on the variable \emph{may-dependency} relation in Section~\ref{subsec:dynamic-datadep} and 
the dependency quantity analysis in Section~\ref{subsec:dynamic-reachability}.
% gives us the edges, 
I firstly define the execution-based dependency graph, then formalize the \emph{adaptivity} in this section.
% \wq{Just a few sentences here, some overview of this subsection. See 4.2 for instance.}
\paragraph{Execution Based Dependency Graph}
\label{para:execution-base-graph-def}
Based on the variable \emph{may-dependency} relation,
% gives us the edges, 
we define the execution-based dependency graph.
% \wq{Just a few sentences here, some overview of this subsection. See 4.2 for instance.}
\begin{defn}[Execution Based Dependency Graph]
\label{def:trace_graph}
Given a program ${c}$,
its \emph{execution-based dependency graph} 
$\traceG({c}) = (\traceV({c}), \traceE({c}), \traceW({c}), \traceF({c}))$ is defined as follows,
{
  \small
\[
\begin{array}{rlcl}
  \text{Vertices} &
  \traceV({c}) & := & \left\{ 
  x^l \in \mathcal{LV}
  ~ \middle\vert ~ x^l \in \lvar(c)
  \right\}
  \\
  \text{Directed Edges} &
  \traceE({c}) & := & 
  \left\{ 
  (x^i, y^j) 
%   \in \mathcal{LV} \times \mathcal{LV}
  ~ \middle\vert ~
  x^i, y^j \in \lvar(c) \land \vardep(x^i, y^j, c) 
  % \text{\mg{$\land$ instead of ,}}
  \right\}
  \\
  \text{Weights} &
  \traceW({c}) & := & 
%   \left
  \{ 
  (x^l, w) 
  % \in \mathcal{LV} \times \mathbb{N}
  ~ \vert ~ 
  w : \mathcal{T} \to \mathbb{N}
  \land
  x^l \in \lvar(c) 
  \\ & & &
  \land
  % n = \max \left\{ 
    % ~ \middle\vert~
  \forall \vtrace \in \mathcal{T}_0(c), \trace' \in \mathcal{T} \sthat \config{{c}, \trace} \to^{*} \config{\eskip, \trace\tracecat\vtrace'} 
  \implies w(\trace) = \vcounter(\vtrace', l) 
  %  \right\}
%   \right
\}
  \\
  % \text{Query Label} &
  \text{Query Annotation} &
  \traceF({c}) & := & 
\left\{(x^l, n)  
% \in  \mathcal{LV}\times \{0, 1\} 
~ \middle\vert ~
 x^l \in \lvar(c) \land
n = 1 \Leftrightarrow x^l \in \qvar(c) \land n = 0 \Leftrightarrow  x^l \notin \qvar(c)
\right\}
\end{array}.
\]
}
\end{defn}
%
There are four components of the execution-based dependency graph. 
The vertices $\traceV(c)$ is the set of program $c$'s labeled variables $\lvar(c)$,
which are statically collected.
The query annotation is 
a set of pairs $\traceF(c) \in \mathcal{P}(\mathcal{LV} \times \{0, 1\} )$ 
mapping each $x^l \in \traceV(c)$ to $0$ or $1$, 
indicating whether this labeled variable is in program $c$'s query variable set $\qvar(c)$.
{
The weights is a set of pairs, $(x^l, w) \in \mathcal{LV} \times (\mathcal{T} \to \mathbb{N})$,
with a labeled variable as first component and
its weight $w$ the second component.
Weight $w$ for
% a labeled variable 
$x^l$ is a function $w : \mathcal{T} \to \mathbb{N}$
mapping from a starting trace to a natural number.
When program executes under this starting trace $\trace$,
$\config{{c}, \trace} \to^{*} \config{\eskip, \trace\tracecat\vtrace'} $, it generates an execution trace $\trace'$.
This natural number is the evaluation times of the labeled command corresponding to the vertex, 
computed by the counter operator $w(\trace) = \vcounter(\vtrace', l)$.
We can see in the execution-based dependency graph of $\kw{twoRounds}$ in Figure~\ref{fig:overview-example}(b), the weight of vertices in the while loop is  $\env(\trace) k$, which depends on the value of the user input $k$ specified in the starting trace $\tau$.
The directed edges $\traceE({c})$ is also a set of pairs with two labeled variables $ (x^i, y^j) \in \mathcal{LV} \times \mathcal{LV}$, from $x^i$ pointing to $y^j$ in the graph.
The edges are constructed directly from our variable may-dependency relation. 
For any two vertices $x^{i}$ and $y^{j}$ in $\traceV(c)$, if they satisfy the variable may-dependency relation $\vardep(x^i, y^j, c)$, there is a direct edge between the two vertices in our execution-based dependency graph for program $c$.
} 
In most data analysis programs $c$ we are interested, there are usually some user input variables, such as $k$ in $\kw{twoRounds}$. 
We denote $\mathcal{T}_0(c)$ as the set of initial traces in which all the input variables in $c$ are initialized, it is also reflected in $\traceW({c})$.    

\paragraph{Trace-based Adaptivity}

% \wq{
% Given 
% a program $c$'s execution-based dependency graph 
% % $G_{trace}(c)(\trace) = (\vertxs, \edges, \weights, \qflag)$,
% $\traceG({c}) = (\traceV({c}), \traceE({c}), \traceW({c}), \traceF({c}))$
% we define adaptivity 
% with respect to $\trace$ by the finite walk in the graph, which has the most query requests along the walk.
% }

Given 
a program $c$'s execution-based dependency graph 
% $G_{trace}(c)(\trace) = (\vertxs, \edges, \weights, \qflag)$,
$\traceG({c})$,
we define adaptivity 
with respect to an initial trace $\trace_0 \in \mathcal{T}_0(c)$ by the finite walk in the graph, which has the most query requests along the walk.
We show the definition of a finite walk as follows.
%
% The query length of a walk $k$ is the number of vertices which correspond to query variables in the vertices sequence of this walk. 
% Instead of counting all 
% the vertices in $k$'s vertices sequence, i

\begin{defn}[Finite Walk (k)].
  \label{def:finitewalk}
  \\
%   Given a program $c$'s execution-based dependency graph $\traceG({c})(\trace)$, 
%   a \emph{finite walk} $fw$ in $\traceG({c})(\trace)$ is a sequence of edges $(e_1 \ldots e_{n - 1})$ 
%   for which there is a sequence of vertices $(v_1, \ldots, v_{n})$ such that:
%   \begin{itemize}
%       \item $e_i = (v_{i},v_{i + 1})$ for every $1 \leq i < n$.
%       \item every vertex $v \in \traceV({c}) $ appears in $(v_1, \ldots, v_{n})$ at most 
%       \wq{$\traceW({c})(\trace)$} times.  
%   \end{itemize}
%   %
%   The length of $fw$ is the number of vertices in its vertex sequence, i.e., $\len(k) = n$.
  Given the execution-based dependency graph $\traceG({c}) = (\traceV({c}), \traceE({c}), \traceW({c}), \traceF({c}))$ of a program $c$,
  a \emph{finite walk} $k$ in $\traceG({c})$ is a 
  function $k: \mathcal{T} \to $ sequence of edges.
  For a initial trace $\trace_0 \in \mathcal{T}_0(c)$, 
  $k(\trace_0)$ is a sequence of edges $(e_1 \ldots e_{n - 1})$ 
  for which there is a sequence of vertices 
  $(v_1, \ldots, v_{n})$ such that:
  \begin{itemize}
      \item $e_i = (v_{i},v_{i + 1}) \in \traceE(c)$ for every $1 \leq i < n$.
      \item every $v_i \in \traceV(c)$
      and $(v_i, w_i) \in \traceW(c)$, 
       $v_i$ appears in $(v_1, \ldots, v_{n})$ at most 
    %   \wq{$\traceW({c})(\trace)$} 
    $w(\trace_0)$
      times.  
  \end{itemize}
  %
  The length of $k(\trace_0)$ is the number of vertices in its vertices sequence, i.e., $\len(k)(\trace_0) = n$.
 \end{defn}

We use $\walks(\traceG(c))$ to denote 
% \mg{``the set'', not ``a set''}a set containing all finite walks $k$ in $G$;
the set containing all finite walks $k$ in $\traceG(c)$;
and $k_{v_1 \to v_2} \in \walks(\traceG(c))$ with $v_1, v_2 \in \traceV(c)$ denotes the walk from vertex $v_1$ to $v_2$ . 
\\
We are interested in queries, so we need to recover the 
variables corresponding to queries from the walk. We define the query length of a walk, 
instead of counting all 
the vertices in $k$'s vertices sequence, we just count the number of vertices which correspond to query variables in this sequence.
%
% \mg{I don't understand this definition. Is wrt a single query?if yes, who is chosing the query? Or is it any query?}
% \jl{It is for any query, as long as the vertex is a query variable, in another worlds, this length just counting the number of query variables in the walk, instead of counting all 
% the vertices.}
% \todo{Make the definition clear}
\begin{defn}[Query Length of the Finite Walk($\qlen$)].
\label{def:qlen}
\\
% Given 
% % labelled weighted graph $G = (\vertxs, \edges, \weights, \qflag)$, 
% a program $c$'s execution-based dependency graph $\traceG(c)(\trace)$
%  and a \emph{finite walk} $k$ in $\traceG(c)(\trace)$ with its vertex sequence $(v_1, \ldots, v_{n})$, 
% %  the length of $k$ w.r.t query is defined as:
% The query length of $k$ is the number of vertices which correspond to query variables in $(v_1, \ldots, v_{n})$ as follows, 
% \[
%   \qlen(k) = \len\big( v \mid v \in (v_1, \ldots, v_{n}) \land \qflag(v) = 1 \big)
% \]
% , where $\big(v \mid v \in (v_1, \ldots, v_{n}) \land \qflag(v) = 1 \big)$ is a subsequence of $(v_1, \ldots, v_{n})$.
Given 
% labelled weighted graph $G = (\vertxs, \edges, \weights, \qflag)$, 
the execution-based dependency graph $\traceG({c}) = (\traceV({c}), \traceE({c}), \traceW({c}), \traceF({c}))$ of a program $c$,
 and a \emph{finite walk} 
%  $k$ in $\traceG(c)(\trace)$
 $k \in \walks(\traceG(c))$. 
%  with its vertex sequence $(v_1, \ldots, v_{n})$, 
%  the length of $k$ w.r.t query is defined as:
The query length of $k$ is a function $\qlen(k): \mathcal{T} \to \mathbb{N}$, such that with an initial trace  $\trace_0 \in \mathcal{T}_0(c)$, $\qlen(k)(\trace_0)$ is
the number of vertices which correspond to query variables in the vertices sequence of the walk $k(\trace_0)$
$(v_1, \ldots, v_{n})$ as follows, 
\[
  \qlen(k)(\trace_0) = |\big( v \mid v \in (v_1, \ldots, v_{n}) \land \qflag(v) = 1 \big)|.
\]
\end{defn}
The definition of adaptivity is then presented in Def~\ref{def:trace_adapt} below.

\begin{defn}
  [Adaptivity of a Program].
  \label{def:trace_adapt}
  \\
  Given a program ${c}$, 
  its adaptivity $A(c)$ is function 
  $A(c) : \mathcal{T} \to \mathbb{N}$ such that for an
  % with respect to a starting trace $\trace$ 
  initial trace $\trace_0 \in \mathcal{T}_0(c)$, 
  % is defined as follows:
  %
 $$
  A(c)(\trace_0) = \max \big 
  \{ \qlen(k)(\trace_0) \mid k \in \walks(\traceG(c)) \big \} $$
  \end{defn}%
%
\subsection{Adaptivity through An Example}
\label{subsec:dynamic-examples}
\begin{example}[twoRounds]
    In this example program $\kw{towRounds(k)}$, the analyst asks in total $k+1$ queries to the mechanism in two phases.
    In the first phase, the analyst asks $k$ queries and stores the answers that are provided by the mechanism. 
    In the second phase, the analyst constructs a new query based on the results of the previous $k$ queries and sends this query to the mechanism. More specifically, we assume that, in this example, the domain $\dbdom$ 
    contains at least $k$ numeric attributes, which we index just by natural numbers. 
    The queries inside the while loop correspond to the first phase and compute an approximation of 
    the product of the empirical mean of the first $k$ attributes. 
    The query outside the loop corresponds to the second phase and computes an approximation of the empirical mean where each record is weighted by the sum of the empirical mean of the first $k$ attributes.
    %
    % Queries are of the form $q(e)$ where $e$ is an expression with a special variable $\chi$ representing a possible row. Mainly $e$ represents a function from $X$ to some domain $U$, for example $U$ could be $[-1,1]$ or $[0,1]$. This function characterizes the linear query we are interested in running. As an example, $x \leftarrow q(\chi[2])$ computes an approximation, according to the used mechanism, of the empirical mean of the second attribute, identified by $\chi[2]$. Notice that we don't materialize the mechanism but we assume that it is implicitly run when we execute the query. 
    % \jl{We use $\chi$ to abstract a possible row in the database and }
    % queries are of the form $\query(\qexpr)$, where $\qexpr$ is a special expression 
    %
    {Since statistical query computes the empirical mean of a function on rows, we use $\chi$ to abstract a possible row in the database and }
    queries are of the form $\query(\qexpr)$, where $\qexpr$ is a special expression 
    (as in our syntax in Section~\ref{sec:language})
    {
    % from $X$ to some domain $U$, 
    % for example $U$ 
      We use $U$ to denote the co-domain of queries, and it could be $[-1,1]$, $[0,1]$ or $[-R,+R]$, for some $R$ we consider.
      This function characterizes the linear query we are interested in running. 
      As an example, $x \leftarrow \query(\chi[j] \cdot \chi[k])$ computes an approximation, according to the used mechanism, of the empirical mean of the product of the $j^{th}$ attribute and $k^{th}$ attribute, identified by $\chi[j] \cdot \chi[k]$. Notice that we don't materialize the mechanism but we assume that it is implicitly run when we execute the query. } 

      The graph in Figure~\ref{fig:twoRounds_example}(b). This graph is built by considering all the possible execution traces of the program in   Figure~\ref{fig:twoRounds_example}(a).
      Each vertex in this graph has a superscript representing its weight, and a subscript $1$ or $0$ telling if the vertex corresponds to a query or not. We will call this subscript a query annotation. 
      For example the vertex $l^{6}:{}^{w_1}_1$, 
      % the superscript $1$ represents the weight $1$, and the subscript for the query annotation.
      has weight $w_1$, a constant function which returns $1$ for every starting state, since 
      this query at line $6$ is at most executed once regardless of the initial trace.
      The query annotation of this vertex is $1$, which  indicates that 
      $\clabel{\assign{l}{\query(\chi[k] * a)}}^6$ is a query request.
      % The assignment in the while loop, such as node $x^{3}$, 
      Another vertex, $x^{3}:{}^{w_k}_1$, appears in the while loop. 
      It has as weight a function $w_k$ that for every initial state returns the value that $k$ has in this state, since this is also the number the while loop will be iterated. 
      The node $j^{4}:{}^{w_k}_0$ has as a subscript $0$ representing a non-query assignment.
      
      
      Since the edges between two vertices represent the fact that one program variable may depend on the other,
      % the queries that are executed and the edges between two nodes represent the fact that one query may depend on the other. 
      we can define the program adaptivity with respect to a initial trace by means of a walk traversing the graph, visiting each vertex no more than its weight with respect to the initial trace, and visiting as many query nodes as possible.
      % In the walk that passes the most times of query nodes, the total visiting times of this walk on 
      % these query nodes is defined as adaptivity.
      %
      So, looking again at our example, we can see that
      % if the input variable $k$ is less than $1$ in an initial trace $\trace_1$, then it is easy to see the weight of vertex $x^3$, $w_3(\trace_1) < 1$ and we can only find a walk with one vertex $l^{6}$, according to  the definition of finite walk in Definition~\ref{def:finitewalk}. So the adaptivity for $\trace_1$, as the number of query vertices along the walk, is $1$. It is easy to understand because when $k <1$, the while loop will not be executed and only one query is asked in total. However, in reality, people want the adaptivity of this example when $k \geq 1$. With this initial trace, it is easy to see that 
      in the walk along the dotted arrows,  $l^{6} \to a^5 \to x^3 $, there are $2$ vertices with query annotation $1$ and that this number is maximal, i.e. we cannot find another walk having more than $2$ vertices with query annotation $1$, under the assumption that $k \geq 1$. So the adaptivity of the program in Figure~\ref{fig:twoRounds_example}(a)  is $2$,
      % longest walk in the graph in Figure~\ref{fig:twoRounds_example}(b), which we mark with a red dashed arrow, is $2$, 
      as expected.
{\small
\begin{figure}
\centering
\begin{subfigure}{.2\textwidth}
\begin{centering}
$
    \begin{array}{l}
    \kw{towRounds(k)} \triangleq \\
           \clabel{ \assign{a}{0}}^{0} ;
            \clabel{\assign{j}{k} }^{1} ; \\
            \ewhile ~ \clabel{j > 0}^{2} ~ \edo ~ \\
            \Big(
             \clabel{\assign{x}{\query(\chi[j] \cdot \chi[k])} }^{3}  ; \\
             \clabel{\assign{j}{j-1}}^{4} ;\\
            \clabel{\assign{a}{x + a}}^{5}       \Big);\\
            \clabel{\assign{l}{\query(\chi[k]*a)} }^{6}\\
        \end{array}
$
\caption{}
\end{centering}
\end{subfigure}
\begin{subfigure}{.75\textwidth}
%}
\qquad
\begin{centering}
 \begin{tikzpicture}[scale=\textwidth/18cm,samples=200]
\draw[] (0, 10) circle (0pt) node
{{ $a^0: {}^{w_1}_{0}$}};
\draw[] (0, 7) circle (0pt) node
{\textbf{$x^3: {}^{w_k}_{1}$}};
\draw[] (0, 4) circle (0pt) node
{{ $a^5: {}^{w_k}_{0}$}};
\draw[] (0, 1) circle (0pt) node
{{ $l^6: {}^{w_1}_{1}$}};
% Counter Variables
\draw[] (5, 9) circle (0pt) node {\textbf{$j^1: {}^{w_1}_{0}$}};
\draw[] (5, 6) circle (0pt) node {{ $j^4: {}^{w_k}_{0}$}};
%
% Value Dependency Edges:
\draw[ ultra thick, -latex, densely dotted,] (0, 1.5)  -- 
% The Weight for this edge
node [left] {\highlight{$\trace_0 \to 1 $}}(0, 3.5) ;
\draw[ ultra thick, -latex, densely dotted,] (0, 4.5)  -- 
node [left] {\highlight{$\trace_0 \to \env(\trace_0) k $}}(0, 6.5) ;
\draw[ thick, -latex] (0, 4.5)  to  [out=-230,in=230]  
node [left] {\highlight{$\trace_0 \to \env(\trace_0) k $}}(0, 9.5) ;
\draw[ thick, -Straight Barb] (1.5, 3.5) arc (120:-200:1);
    % The Weight for this edge
    \draw[](3, 3) node [] {\highlight{$\trace_0 \to \env(\trace_0) k  $}};
\draw[ thick, -Straight Barb] (6.5, 6.5) arc (150:-150:1);
    % The Weight for this edge
    \draw[](9, 6) node [] {\highlight{$\trace_0 \to \env(\trace_0) k  $}};
\draw[ thick, -latex] (5, 6.5)  -- 
% The Weight for this edge
node [right] {\highlight{$\trace_0 \to \env(\trace_0) k $}} (5, 8.5) ;
% Control Dependency
\draw[ thick,-latex] (1.5, 7)  -- (4, 9) ;
\draw[ thick,-latex] (1.5, 4)  -- 
% The Weight for this edge
node [] {\highlight{$\trace_0 \to \env(\trace_0) k $}} (4, 9) ;
\draw[ thick,-latex] (1.5, 7)  -- (4, 6) ;
\draw[ thick,-latex] (1.5, 4)  -- (4, 6) ;
\end{tikzpicture}
\caption{}
\end{centering}
\end{subfigure}
 \caption{(a) The program $\kw{towRounds(k)}$, an example 
%  of a program 
with two rounds of adaptivity (b) The corresponding execution-based dependency graph.}
\label{fig:twoRounds_example}
\end{figure}
}
\end{example}
%

% In terms of techniques, our work relies on ideas from both static analysis and dynamic analysis. 
We discuss closely related work in both areas.


%
% \subsection{Implementation}
% \label{subsec:dynamic-implementation}
%

\section{The Program Static Analysis for Adaptivity}
\label{sec:static}
\chapter{The Program Static Analysis for Adaptivity}
\label{ch:adapt-algo}


\section{Introduction}
\label{sec:static-intro}

\section{Static Data Dependency Analysis}
\label{sec:static-datadep}

\section{Static Reachability Bound Analysis}
\label{sec:static-reachability}

\section{Static Adaptivity Analysis}
\label{sec:static-adapt}

\section{Examples}
\label{sec:static-examples}
%
\section{Implementation}
\label{sec:static-implementation}

\section{Related Work}
\label{sec:static-relatedwork}
In terms of techniques, our work relies on ideas from both static analysis and dynamic analysis. 
We discuss closely related work in both areas.


\cleardoublepage


\section{Further Features}
\label{sec:furthers}
\subsection{Accurate Execution-Based Dependency Depth Analysis}
\label{subsec:furthers-dep-depth}
%
The program's adaptivity in the formal model through the execution-based analysis,
% which we define over the program's execution-based dependency graph from the dynamic 
% analysis 
in Definition~\ref{def:trace_adapt}, isn't precise enough w.r.t. the intuitive adaptivity rounds.
It comes across an over-approximation 
% on the program's
%  intuitive adaptivity rounds.
% It is 
resulted from difference between its Dependency Depth analysis and the \emph{variable may-dependency} definition.
It occurs when the weight is computed over the traces different from the traces used in 
witness the \emph{variable may-dependency} relation.
As shown in the Example~\ref{ex:overdefined_adapt}.
\begin{example}[Over-Defined Adaptivtiy Example]
    \label{ex:overdefined_adapt}
    The program's adaptivity in our formal model,
    % which we define over the program's execution-based dependency graph from the dynamic 
    % analysis 
    in Definition~\ref{def:trace_adapt} also
     comes across an over-approximation on the program's
     intuitive adaptivity rounds.
    It is resulted from difference between its weight calculation and the \emph{variable may-dependency} definition.
    It occurs when the weight is computed over the traces different from the traces used in 
    witness the \emph{variable may-dependency} relation.
    % control flow can be decided in a particular way in front of conditional branches, while the static analysis fails to witness. 
    
    % We use one example to show the over-approximated definition, 
    As the program in Figure~\ref{fig:overdefn_example}(a),
    % This example is the variant of the multiple rounds strategy, 
    % we call it a multiple rounds odd iteration algorithm.
    % This example is still 
    which is a variant of the multiple rounds strategy, 
    % we call it a multiple rounds single iteration algorithm, 
    named $\kw{multipleRoundSingle(k)}$ with input $k$.
    % as the input variable.
    In this algorithm, 
    at line 7 of every iteration, 
    a query $\query(\chi[y] + p)$ based on previous query results stored in $p$ and $y$ is asked by the analyst like in the multiple rounds strategy. 
    The difference is that only the query answers from the one single iterations ($j = k - 2 $) are 
    % used to $b$. 
    used in this query $\query(\chi[y] + p)$.
    Because the execution trace updates 
    %   $b$ using the query answers at odd iterations, so the answers from even iterations do not affect the queries at odd iterations. From the query-based dependency graph in Figure~\ref{fig:overappr_example}(b), we can see that there is no edge from queries at odd iterations (such as $q_1,q_3,q_5$) to queries at even iteration(such as $q_2,q_4$). The longest path is dashed with a length $3$.  However, {\THESYSTEM} fails to realize that odd iteration will always execute then branch and even iteration means else branch, so its dependency graph considers both branches for every iteration. In this sense, the dependency graph by {\THESYSTEM} is similar to the one in the multiple rounds strategy. We show the estimated graph in Figure~\ref{fig:overappr_example}(c). The estimated upper bound is then, $5$, instead of $3$. 
    $p$ using the constant $0$ for all the iterations where ($j \neq k - 2$) at line $10$ after the 
    query request at line $7$.
    In this way, all the query answers stored in $p$ will not be accessed in next query request at line $7$ in the iterations 
    where  ($j \neq k - 2$).
    Only query answer at one single iteration where ($j = k - 2 $) will be used in next query request
    $\query(\chi[y] + p)$ at line $7$.
    So the adaptivity for this example is $2$. 
    % so the answers from odd iterations do not affect the queries at even iterations. 
    % However, from the execution-based dependency graph in Figure~\ref{fig:overappr_example}(b), 
    However, our adaptivity model fails to realize that there is only dependency relation 
    between $p^7$ and $p^7$ in one single iteration, 
    not the others. 
    % there is no edge from queries at odd iterations (such as $q_1,q_3,q_5$) to queries at even iteration(such as $q_2,q_4$). The longest path is dashed with a length $3$.  
    As shown in the execution-based dependency graph in Figure~\ref{fig:overdefn_example}(b), 
    there is an edge from $p^7$ to itself representing the existence of \emph{Variable May-Dependency} from $p^7$ on itself,
    and the visiting times of labeled variable $p^7$ is 
    $w_k(\trace_0)$ with a initial trace $\trace_0$. 
    % will always execute then branch and even iteration means else branch, so 
    % % its dependency 
    % it considers both branches for every iteration. 
    % In this sense, the weight estimated for $y^6$ and $w^6$ are both 
    % $k$.
    As a result, the walk with the longest query length 
    is
    $p^7  \to \cdots \to p^7 \to y^4  \to z^1 $ with the vertex $p^7$ visited $w_k(\trace_0)$,
    as the dotted arrows. 
    The adaptivity 
    % the Program-Based Dependency graph from {\THESYSTEM} by finding 
    based on
    this walk
    % walk with the longest query length 
    is $2 + w(\trace_0)$, instead of $2$. 
    % %
    % T% estimated from the Program-Based Dependency graph from by finding the walk with the longest query length 
    % is $1 + 2 * k$, instead of $1 + K$.
    Though the $\THESYSTEM$ is able to give us $2 + k$,  as an accurate bound w.r.t this definition.
    %  we show the estimated graph in Figure~\ref{fig:overappr_example}(c). 
    
        {\small
    \begin{figure}
     \centering
    %}
    \quad
    \begin{subfigure}{.35\textwidth}
    \begin{centering}
    $
        \begin{array}{l}
            \kw{multipleRoundsSingle(k)}\\
               \clabel{ \assign{j}{0}}^{0} ; 
                \clabel{\assign{z}{\query(0)} }^{1} ;             
                \clabel{\assign{p}{0} }^{2} ; \\
                \eif(\clabel{ k = 0}^{3}, 
                \clabel{ \assign{y}{\query(z)}}^{4}, \clabel{\eskip}^5);\\
                \ewhile ~ \clabel{j \neq k}^{6} ~ \edo ~ \\
                \Big(
                 \clabel{\assign{p}{\query(\chi[y]+p)} }^{7}  ; 
                 \clabel{\assign{j}{j + 1}}^{8}\\
              \eif(\clabel{ j \neq k - 2}^{9}, 
              \clabel{ \assign{p}{0}}^{10} ,\clabel{\eskip}^{10})
         \Big);\\
            \end{array}
    $
    \caption{}
    \end{centering}
    \end{subfigure}
    \begin{subfigure}{.6\textwidth}
        \begin{centering}
        \begin{tikzpicture}[scale=\textwidth/28cm,samples=150]
    % Variables Initialization
    \draw[] (-5, 1) circle (0pt) node{{ $z^1: {}^{w_1}_{1}$}};
    \draw[] (-5, 7) circle (0pt) node{{$p^2: {}^{w_1}_{0}$}};
    \draw[] (-5, 4) circle (0pt) node{{ $y^4: {}^{w_1}_{1}$}};
    % Variables Inside the Loop
     \draw[] (0, 6) circle (0pt) node{{ $p^7: {}^{w_k}_{1}$}};
     \draw[] (0, 2) circle (0pt) node{{ $p^{10}: {}^{w_k}_{0}$}};
     % Counter Variables
     \draw[] (5, 6) circle (0pt) node {{$j^0: {}^{w_1}_{0}$}};
     \draw[] (5, 2) circle (0pt) node {{ $j^8: {}^{w_k}_{0}$}};
     %
     % Value Dependency Edges:
     \draw[ thick, -Straight Barb] (1.4, 1.6) arc (120:-200:1);
     \draw[ ultra thick, -Straight Barb, densely dotted,] (0.8, 7) arc (220:-100:1);
     \draw[ thick, -latex] (-1.5, 6)  to  [out=-130,in=130]  (-1.5, 2);
     % Value Dependency Edges on Initial Values:
     \draw[ ultra thick, -latex, densely dotted,] (-5, 3.5)  -- (-5, 1.5) ;
     \draw[ thick, -latex,] (-1.5, 6)  -- (-4, 7) ;
     \draw[  ultra thick, -latex, densely dotted,] (-1.5, 6)  -- (-4, 4.7) ;
     %
     % Value Dependency For Control Variables:
     \draw[ thick, -Straight Barb] (6.5, 2.5) arc (150:-150:1);
    %  \draw[ ultra thick, -latex, densely dotted,] (-0.5, 1.5)  to  [out=-250,in=250]  (-0.5, 7);
     % Control Dependency
     \draw[ thick, -latex] (5, 2.5)  -- (5, 5.5) ;
     \draw[ thick,-latex] (1.5, 6)  -- (3.5, 6) ;
     \draw[ thick,-latex] (1.5, 1.8)  -- (3.5, 6) ;
     \draw[ thick,-latex] (1.5, 6)  -- (3.5, 2) ;
    %  \draw[ thick,-latex] (1.5, 4)  -- (4, 6) ;
     \draw[ thick,-latex] (1.5, 1.8)  -- (3.5, 2) ; 
    \end{tikzpicture}
     \caption{}
        \end{centering}
        \end{subfigure}
    % \end{wrapfigure}
    % \end{equation*}
    % \vspace{-0.4cm}
     \caption{(a) The multi rounds single example
     (b) The execution-based dependency graph.}
    \label{fig:overdefn_example}
    % \vspace{-0.5cm}
    \end{figure}
        }
    \end{example}
%%
\subsubsection{Proposed Methodology}
\label{subsubsec:furthers-dep-depth}
% In terms of techniques, our work relies on ideas from both static analysis and dynamic analysis. 
We discuss closely related work in both areas.


1. In the first stage of the execution-based analysis, 
I will give an alternative variable \emph{may-dependency} definition 
by referring to the analysis methodology in \cite{Cousot19a}.
\\
Specifically, I will define the variables dependency relation over two witness traces and an initial trace. Comparing to 
the existing \emph{may-dependency} definition, which quantifies overall possible execution traces, the alternative
definition explicitly relies on two specific witness traces from execution.
This externalization helps in analyzing the dependency quantity through the same 
witness traces as the \emph{may-dependency} relation. In this way, the over-approximation as illustrated above
can be reduced.
\\
2. In the second stage of the execution-based analysis, 
based on the new \emph{may-dependency} definition,
I will compute the weight of every edge constructed from 
\emph{may-dependency} relation in the execution-based dependency graph, w.r.t. to the witness traces.
\\
3. Then in the third stage, I will formalize the \emph{adaptivity} still as the 
length of the longest finite walk. Differently from the previous one, I restrict 
the occurrence of every edge in a finite walk no more than its weight as well.
\\
Through the three steps above, I give a more accurate formalization of the intuitive \emph{adaptivity}.
%
\subsection{Static Path Sensitive Reachability Bound Analysis}
\label{subsec:furthers-reachability}
In static program analysis framework $\THESYSTEM$, specifically on the dependency quantity, 
I adopt the reachability bound analysis technique to estimate this dependency quantity.
% In existing static reachability bound analysis, 
However, it isn't precise enough w.r.t. the execution-based reachability bound on every program command.
It comes across an over-approximation on the estimation due to its path-insensitive nature. 
It occurs when the control flow can be decided in a particular way in front of conditional branches, 
while the static analysis fails to witness. 
As shown in Example~\ref{ex:overapproximate}.
\begin{example}
[Multiple Rounds Odds Algorithm]
\label{ex:overapproximate}
The $\THESYSTEM$ comes across an over-approximation on the estimation due to its path-insensitive nature. 
It occurs when the control flow can be decided in a particular way in front of conditional branches, while the static analysis fails to witness. 

We show the over-approximation, in Figure~\ref{fig:overappr_example}(a),
we call it a multiple rounds odd iteration algorithm. In this algorithm, at line 5 of every iteration, 
a query $\query(\chi[x])$ based on previous query results stored in $x$ is asked by the analyst like in the multiple rounds strategy. The difference is that only the query answers from the even iterations ($i =0, 2, \cdots $) are 
% used to $b$. 
used in the query 
in line 7, $\query(\chi[\ln(y)])$.
  Because the execution trace only updates 
%   $b$ using the query answers at odd iterations, so the answers from even iterations do not affect the queries at odd iterations. From the query-based dependency graph in Figure~\ref{fig:overappr_example}(b), we can see that there is no edge from queries at odd iterations (such as $q_1,q_3,q_5$) to queries at even iteration(such as $q_2,q_4$). The longest path is dashed with a length $3$.  However, {\THESYSTEM} fails to realize that odd iteration will always execute then branch and even iteration means else branch, so its dependency graph considers both branches for every iteration. In this sense, the dependency graph by {\THESYSTEM} is similar to the one in the multiple rounds strategy. We show the estimated graph in Figure~\ref{fig:overappr_example}(c). The estimated upper bound is then, $5$, instead of $3$. 
$x$ using the query answers in even iterations, so the answers from odd iterations do not affect the queries in even iterations. 
From the execution-based dependency graph in Figure~\ref{fig:overappr_example}(b), 
we can see that the weight for the vertex $y^5$ is 
$w_k/2$. a function which takes any initial trace $\trace_0$, return the value of $k/2$ evaluated in $\trace_0$.  
However, {\THESYSTEM} fails to realize that odd iteration will always execute the then branch and even iteration means else branch, so 
% its dependency 
it considers both branches for every iteration. 
In this sense, the weight estimated for $y^5$ and $p^6$ are both 
$k$ as in Figure~\ref{fig:overappr_example}(c).
As a result, {\THESYSTEM}  estimates the longest walk from Figure~\ref{fig:overappr_example}(c),
$y^5  \to x^7  \to y^5  \to \cdots \to x^7  $ with each vertex visited $k$ times,
as the dotted arrows. 
And the adaptivity computed 
% estimated from the program-based dependency graph graph from by finding the walk with the longest query length 
is $1 + 2 * k$, instead of $1 + k$. 
% We omitted the estimated graph, which is identical to the graph in Figure~\ref{fig:overappr_example}(b). 
%

{ \small
\begin{figure}
\centering
    \begin{subfigure}{0.33\textwidth}
\centering
\small{
    \[
    %
    \begin{array}{l}
        \kw{multipleRoundsOdd}(k) \triangleq \\
        \clabel{ \assign{j}{k}}^{0} ; 
        \clabel{ \assign{x}{\query(\chi[0])} }^{1} ; \\
            \ewhile ~ \clabel{j > 0}^{2} ~ \edo ~ 
            \Big(
             \clabel{\assign{j}{j-1}}^{3} ;\\
             \eif(\clabel{j \% 2 == 0}^{4}, \\
             \clabel{\assign{y}{\chi[x]}}^{5}, 
             \clabel{\assign{p}{\chi[x]}}^{6});\\                            
             \clabel{\assign{x}{\query(\chi(\ln(y)))} }^{7} \Big)
        \end{array}
    \]
}
 \caption{}
    \end{subfigure}
%
\begin{subfigure}{.31\textwidth}
    \begin{centering}
    \begin{tikzpicture}[scale=\textwidth/11cm,samples=200]
% Variables Initialization
\draw[] (5, 1) circle (0pt) node{{ $x^1: {}^{w_1}_{1}$}};
% Variables Inside the Loop
 \draw[] (0, 7) circle (0pt) node{{ $y^5: {}^{w_k/2}_{1}$}};
 \draw[] (0, 4) circle (0pt) node{{ $p^6: {}^{w_k/2}_{1}$}};
 \draw[] (0, 1) circle (0pt) node{{ $x^7: {}^{w_k}_{1}$}};
 % Counter Variables
 \draw[] (5, 7) circle (0pt) node {{$j^0: {}^{w_1}_{0}$}};
 \draw[] (5, 4) circle (0pt) node {{ $j^3: {}^{w_k}_{0}$}};
 %
 % Value Dependency Edges:
 \draw[ thick, -latex,]  (0, 3.5) -- (0, 1.5) ;
%  \draw[ thick, -Straight Barb] (1, 4.2) arc (220:-100:1);
 \draw[ thick, -Straight Barb] (6.5, 4.5) arc (150:-150:1);
 \draw[ thick, -latex] (5, 4.5)  -- (5, 6.5) ;
%  \draw[ thick, -Straight Barb] (1., 1.5) arc (120:-200:1);
 % Value Dependency Edges on Initial Values:
 \draw[ thick, -latex,] (1.5, 1)  -- (4, 1) ;
 %
 \draw[ ultra thick, -latex, densely dotted,] (-0.6, 1.5)  to  [out=-220,in=220]  (-0.5, 6.5);
\draw[ ultra thick, -latex, densely dotted,]  (0.5, 6.5) to  [out=-30,in=30] (0.6, 1.6) ;
%  \draw[ ultra thick, -latex, densely dotted,]  (0.5, 10)  to  [out=-50,in=50] (0.5, 4);
 % Control Dependency
 \draw[ thick,-latex] (1.5, 7)  -- (4, 6) ;
 \draw[ thick,-latex] (1.5, 4)  -- (4, 6) ;
 \draw[ thick,-latex] (1.5, 1)  -- (4, 6) ;
%  \draw[ thick,-latex] (1.5, 10)  -- (4, 6) ;
 \end{tikzpicture}
 \caption{}
    \end{centering}
    \end{subfigure}
    \begin{subfigure}{.31\textwidth}
        \begin{centering}
        \begin{tikzpicture}[scale=\textwidth/11cm,samples=200]
    % Variables Initialization
    \draw[] (5, 1) circle (0pt) node{{ $x^1: {}^1_{1}$}};
    % Variables Inside the Loop
     \draw[] (0, 7) circle (0pt) node{{ $y^5: {}^{k}_{1}$}};
     \draw[] (0, 4) circle (0pt) node{{ $\mathbf{p^6: {}^{k}_{1}}$}};
     \draw[] (0, 1) circle (0pt) node{{ $\mathbf{x^7: {}^{k}_{1}}$}};
     % Counter Variables
     \draw[] (5, 7) circle (0pt) node {{$j^0: {}^{1}_{0}$}};
     \draw[] (5, 4) circle (0pt) node {{ $j^3: {}^{k}_{0}$}};
     %
% Value Dependency Edges:
 \draw[ thick, -latex,]  (0, 3.5) -- (0, 1.5) ;
%  \draw[ thick, -Straight Barb] (1, 4.2) arc (220:-100:1);
 \draw[ thick, -Straight Barb] (6.5, 4.5) arc (150:-150:1);
 \draw[ thick, -latex] (5, 4.5)  -- (5, 6.5) ;
%  \draw[ thick, -Straight Barb] (1., 1.5) arc (120:-200:1);
 % Value Dependency Edges on Initial Values:
 \draw[ thick, -latex,] (1.5, 1)  -- (4, 1) ;
 %
 \draw[ ultra thick, -latex, densely dotted,] (-0.6, 1.5)  to  [out=-220,in=220]  (-0.5, 6.5);
\draw[ ultra thick, -latex, densely dotted,]  (0.5, 6.5) to  [out=-30,in=30] (0.6, 1.6) ;
%  \draw[ ultra thick, -latex, densely dotted,]  (0.5, 10)  to  [out=-50,in=50] (0.5, 4);
 % Control Dependency
 \draw[ thick,-latex] (1.5, 7)  -- (4, 6) ;
 \draw[ thick,-latex] (1.5, 4)  -- (4, 6) ;
 \draw[ thick,-latex] (1.5, 1)  -- (4, 6) ;
%  \draw[ thick,-latex] (1.5, 10)  -- (4, 6) ;
     \end{tikzpicture}
     \caption{}
        \end{centering}
        \end{subfigure}
        \vspace{-0.4cm}
\caption{(a) The multiple rounds odd example 
(b) The execution-based dependency graph
(c) The program-based dependency graph graph from $\THESYSTEM$.}
    \label{fig:overappr_example}
    % \vspace{-0.5cm}
\end{figure}
}
%
\end{example}

% as follo
\subsubsection{Proposed Methodology}
\label{subsubsec:furthers-reachability}
Given the imprecision comes from the second stage of the static program analysis,
I will insist on the same $\THESYSTEM$ framework,
and design new algorithm for this stage.
In this stage, 
methodology on reachability bound analysis isn't path sensitive. 
I will design a path sensitive reachability bound analysis algorithm computing the 
reachability bounds for every labeled command taking the different paths inside while loop into consideration.
% Comparing to just compute the reachability bound for the while loop command, new methodology improves the accuracy of the 
% reachability bound for every labeled command.
\\
Then, I will use this improved analysis result to estimate the dependency quantity.
%
% \subsection{Static Adaptivity Computation towards Completeness}
% \label{subsec:furthers-adaptcomplete}
% The Algorithm is conditional completeness as proved in appendix, but Algorithm~\ref{alg:adaptscc} isn't.
% In the algorithm design at line: in Algorithm~\ref{alg:adaptscc}, an over-approximation happens here. 

% As  following motivating example shows.
% \subsubsection{Proposed Methodology}
% \label{subsubsec:furthers-adaptcomplete-methodology}
% %
% 1. looking into more over-approximated example and summarize the common properties of these examples.
% \\
% 2. Modify the Algorithm~\ref{alg:adaptscc}, targeting the line: 12 of the algorithm. 
% The goal is to reduce the over-approximation in computing adaptivity statically.


% 
Based on the language and the trace-based operational semantics in Section~\ref{sec:language},
I formalize the intuitive through an execution based program analysis in this section.

\subsection{Introduction and Related Work}
\label{subsec:dynamic-intro}

I first introduce some related works as background of the execution-based analysis, 
then structure of this execution-based analysis.  
% \\
% The construction of this graph requires me to think about the dependency relation between two queries using what we have at hand - 
% the trace generated in Section~\ref{sec:language}. 
 \paragraph*{Related Work}
 {
My framework constructs a execution-based dependency graph based on the execution traces of a program. I define semantic dependence on this graph by considering (intraprocedural) data and control dependency~\cite{bilardi1996framework,cytron1991efficiently,pollock1989incremental}.    
One related work  
\cite{austin1992dynamic} presents a methodology to construct a dynamic dependency graph (DDG) based on the dynamic execution of a program in an imperative language, where edges represent dependency between instructions. Data dependency, control dependency, storage dependency, and resource dependency between instructions are all considered. My execution-based dependency graph only needs data dependency and control dependency between variable assignment results. 
% Critical path length analysis on DDGs is useful for understanding the scope for parallelization, while we use the length of the longest path to define adaptivity.  
%
DDGs have been used in many other domains. \cite{nagar2018automated} use DDGs to find serializability violations. \cite{hammer2006dynamic} use similar \emph{program dependency graphs} \cite{ferrante1987program} for dynamic program slicing.
\cite{mastroeni2008data} propose ways of constructing different kinds of program slices, by choose different program dependency. 
% For example, in either syntactic or semantics sense.
% This abstract dependency is based on properties rather than exact data.
% Aims to give finer and smaller program slice. 
They actually use a combination of  
static and dynamic dependency graphs but in a manner that is different from how we use the two. Their slicing uses both static and dynamic dependency graphs, while we use the dynamic dependency graph as the basis of a definition, which is then soundly approximated by an analysis based on the static dependency graph.}

{My execution-based data dependency relation definition over variables 
is inspired by the method in \cite{Cousot19a}, where the dependency relation is also identified by looking into the differences on two execution traces. 
However, Cousot excludes timing channels~\cite{SabelfeldM03} and empty observation, which are also not considered as a form of dependency in traditional dependency analysis \cite{DenningD77}.
% In the cases of empty observation and timing channels, the second query is executed 
% in one trace and isn't in another trace by modifying value of first query. 
% Then, the second query is indeed depend on the first query and there exists an
% adaptivity round between the two queries. 
My definition includes timing channels and empty observation by observing both the disappearance and value variation.
}
\paragraph*{Analysis Strcuture}
In order to formalize a quantitative property w.r.t. the dependency relation in program, I
use a three-step analysis methodology developed, 
 as follows,
\\
 a. The dependency relation between every query, through the methodology of semantic data dependency analysis.
\\
 b. The dependency quantity analysis, through the methodology of execution-based data reachability bound analysis. Then 
\\
 c. The adaptivity analysis, based on the two analysis results above, 
 I construct an execution-based dependency graph combining the dependency relation and the dependency quantity
    and give the formal \emph{adaptivity} definition 
    for program.
    This analysis is the first part of the analysis in Figure~\ref{fig:structure}.

\subsection{Methodology}
\label{subsec:dynamic-methodology}

\subsubsection{Data Dependency Analysis}
\label{subsubsec:dynamic-datadep}
\paragraph*{Challenge}
In the data analysis model our programming framework supports, 
%  an \emph{analyst} asks a sequence of queries to the mechanism, and receives the answers to these queries from the mechanism. In this model, the adaptivity we are interested in is the length of the longest sequence of such adaptively chosen queries, among all the queries the data analyst asks. 
  we define that a query is adaptively chosen when it is affected by answers of previous queries. The next thing is to decide how do we define whether one query is "affected" by previous answers, with the limited information we have? As a reminder, 
 when the analyst asks a query, the only known information will be the answers to previous queries and the current execution trace of the program.


There are two possible situations that a query will be "affected",  
either when the query expression directly uses the results of previous queries (data dependency), or when the control flow of the program with respect to a query (whether to ask this query or not) depends on the results of previous queries (control flow dependency).
% As a first step, we give a definition of when one query may depend on a previous query, which is supposed to consider both control dependency and data dependency. We first look at two possible candidates:
% \begin{enumerate}
%     \item One query may depend on a previous query if and only if a change of the answer to the previous query may also change the result of the query.
%     \item One query may depend on a previous query if and only if a change of the answer to the previous query may also change the appearance of the query.
% \end{enumerate}


Since the the results of previous queries can be stored or used in variables
which aren't associated to the query request,
it is necessary to track the dependency between queries, through all the program's variables,  
and then we can distinguish variables which are assigned with query requests.
 We give a definition of when one variable \emph{may-depend} on a previous variable with two candidates.
{
\begin{enumerate}
    \item One variable may depend on a previous variable if and only if a change of the value assigned to the previous variable may also change the value assigned to the variable.
    \item One variable may depend on a previous variable if and only if a change of the value assigned to the previous variable may also change the appearance of the assignment command to this variable 
    % in\wq{during?} 
    during execution.
\end{enumerate}
}
%   The first candidate works well by witnessing the result of one query according to the change of the answer of another query. We can easily find that the two queries have nothing to do with each other in a simple example   

{   
% The first situations works well by witnessing the result assigned to variable 
% according to the change of the value assigned to another query. 
% We can easily find that the two queries have nothing to do with each other in a simple example 
% In the first one, by defining the dependency as
The first definition is defined as
% witnessing 
% the query expressions equivalence (or the value equality for non-query assignment )
the witness of a variation on the value assigned to the same variable through two executions,
% assigned to the same variable through two executions, 
according to the change of the value assigned to another variable in pre-trace.
% the situation of data-dependency works well. \wq{long sentence, make it short?}
In particular for query requests, the variation we observe is on the query value instead of on the query requesting results.
% We can find that two queries 
% % have nothing to do with each other in this simple example 
% % depends on each other\wq{not each other, one direction.} 
% satisfy this definition
In 
%this 
the simple program $c_1 =\assign{x}{\query(\chi[2])} ;\assign{y}{\query(\chi[3] + x)}$.
 %
 From our perspective, $\query(\chi[1])$ is different from $\query(\chi[2]))$. Informally, we think $\query(\chi[3] + x)$ may depend on the query $\query(\chi[2]))$, because equipped function of the former $\chi[3] + x$ may depend on the data stored in x assigned with the result of $\query(\chi[2]))$, according to this definition. }
%
% in this example: $c_1 = \assign{x}{\query(0)}; \assign{z}{\query(\chi[x])}$.
% This candidate definition works well 
Nevertheless, the first definition fails to catch control dependency because it just monitors the changes to a query, but misses the appearance of the query when the answers of its previous queries change. 
For instance, it fails to handle $
      c_2 = \assign{x}{\query(\chi[1])} ; \eif( x > 2 , \assign{y}{\query(\chi[2])}, \eskip )
   $, but the second definition can. However, it only considers the control dependency and misses the data dependency. This reminds us to define a \emph{may-dependency} relation between labeled variables by combining the two definitions to capture the two situations.
%
%
%
%
\paragraph{Dependency}
 To define the may dependency relation on two labeled variables, we rely on the limited information at hand - the trace generated by the operational semantics. In this end, we first define the \emph{may-dependency} between events, and use it as a foundation of the variable may-dependency relation.
\begin{defn}[Events Different up to Value ($\diff$)]
  Two events $\event_1, \event_2 \in \eventset$ are  \emph{Different up to Value}, 
  denoted as $\diff(\event_1, \event_2)$ if and only if:
  \[
    \begin{array}{l}
  \pi_1(\event_1) = \pi_1(\event_2) 
  \land  
  \pi_2(\event_1) = \pi_2(\event_2) \\
  \land  
  \big(
    (\pi_3(\event_1) \neq \pi_3(\event_2)
  \land 
  \pi_{4}(\event_1) = \pi_{4}(\event_2) = \bullet )
  % \qquad \qquad 
  \lor 
  (\pi_4(\event_1) \neq \bullet
  \land 
  \pi_4(\event_2) \neq \bullet
  \land 
  \pi_{4}(\event_1) \neq_q \pi_{4}(\event_2)) 
  \big)
  \end{array}
  \]
  \end{defn}
 %
 We compare two events by defining $\diff(\event_1, \event_2)$. We use $\qexpr_1 =_{q} \qexpr_2$ and $\qexpr_1 \neq_{q} \qexpr_2$ to notate query expression equivalence and in-equivalence, distinct from standard equality. A program $c$'s
%  , its 
 labeled variables 
%  and assigned variables are subsets of 
is a subset of
the labeled variables $\mathcal{LV}$, denoted by $\lvar(c) \in \mathcal{P}(\mathcal{VAR} \times \mathcal{L}) \subseteq \mathcal{LV}$.
% annotated by a label. 
% We use  
%$\mathcal{LVAR} = \mathcal{VAR} \times \mathcal{L} $ 
% $\mathcal{LV}$ represents the universe of all the labeled variables and 
% $\avar(c) \in \mathcal{P}(\mathcal{VAR} \times \mathbb{N}) \subset \mathcal{LV}$ and 
% $\lvar(c) \in \mathcal{P}(\mathcal{VAR} \times \mathcal{L}) \subseteq \mathcal{LV}$ for them. 
We also define the set of query variables for a program $c$, $\qvar: \cdom \to 
\mathcal{P}(\mathcal{LV})$.

A program $c$'s query variables is a subset of 
its labeled variables, $\qvar(c) \subseteq \lvar(c)$. We have the operator $\tlabel : \mathcal{T} \to \ldom$, which gives the set of labels in every event belonging to the trace.
Then we introduce a counting operator $\vcounter : \mathcal{T} \to \mathbb{N} \to \mathbb{N}$, 
% \wq{which counts the occurrence of of a variable in the trace,} 
which counts the occurrence of a labeled variable in the trace,
whose behavior is defined as follows,
\[
\begin{array}{ll}
\vcounter(\trace :: (\_, l, \_, \_), l ) \triangleq \vcounter(\trace, l) + 1
&
\vcounter(\trace  ::(b, l, v, \bullet), l) \triangleq \vcounter(\trace, l) + 1
\\
\vcounter(\trace  :: (x, l, v, \qval), l) \triangleq \vcounter(\trace, l) + 1
&
% \vcounter(\trace :: (\_, l', \_, \_), l ) \triangleq \vcounter(\trace, l), l' \neq l 
% &
\vcounter(\trace  :: (x, l', v, \bullet), l) \triangleq \vcounter(\trace, l), l' \neq l
\\
\vcounter(\trace  :: (b, l', v, \bullet), l) \triangleq \vcounter(\trace, l), l' \neq l
&
\vcounter(\trace  :: (x, l', v, \qval), l) \triangleq \vcounter(\trace, l), l' \neq l
\\
\vcounter({[]}, l) \triangleq 0
\end{array}
\]
The full definitions of these above operators can be found in the appendix.
\begin{defn}[Event May-Dependency].
\label{def:event_dep}
\\ 
  An event $\event_2$ is in the \emph{event may-dependency} relation with an assignment
  event $\event_1 \in \eventset^{\asn}$ in a program ${c}$
  with a hidden database $D$ and a trace $\trace \in \mathcal{T}$ denoted as 
  %
  $\eventdep(\event_1, \event_2, [\event_1 ] \tracecat \trace \tracecat [\event_2], c, D)$, iff
  %
  \[
    \begin{array}{l}
  \exists \vtrace_0,
  \vtrace_1, \vtrace' \in \mathcal{T},\event_1' \in \eventset^{\asn}, {c}_1, {c}_2  \in \cdom  \sthat
  \diff(\event_1, \event_1') \land 
      \\ \quad
      (
        \exists  \event_2' \in \eventset \sthat 
    \left(
    \begin{array}{ll}   
   & \config{{c}, \vtrace_0} \rightarrow^{*} 
  \config{{c}_1, \vtrace_1 \tracecat [\event_1]}  \rightarrow^{*} 
    \config{{c}_2,  \vtrace_1 \tracecat [\event_1] \tracecat \vtrace \tracecat [\event_2] } 
    % 
   \\ 
   \bigwedge &
    \config{{c}_1, \vtrace_1 \tracecat [\event_1']}  \rightarrow^{*} 
    \config{{c}_2,  \vtrace_1 \tracecat[ \event_1'] \tracecat \vtrace' \tracecat [\event_2'] } 
  \\
  \bigwedge & 
  \diff(\event_2,\event_2' ) \land 
  \vcounter(\vtrace, \pi_2(\event_2))
  = 
  \vcounter(\vtrace', \pi_2(\event_2'))\\
  \end{array}
  \right)
  \\ \quad
  \lor 
  \exists \vtrace_3, \vtrace_3'  \in \mathcal{T}, \event_b \in \eventset^{\test} \sthat 
  \\ \quad
  \left(
  \begin{array}{ll}   
    & \config{{c}, \vtrace_0} \rightarrow^{*} 
      \config{{c}_1, \vtrace_1 \tracecat [\event_1]}  \rightarrow^{*} 
      \config{c_2,  \vtrace_1 \tracecat [\event_1] \tracecat \trace \tracecat [\event_b] \tracecat  \trace_3} 
    \\ 
    \bigwedge &
    \config{{c}_1, \vtrace_1 \tracecat [\event_1']}  \rightarrow^{*} 
    \config{c_2,  \vtrace_1 \tracecat [\event_1'] \tracecat \trace' \tracecat [(\neg \event_b)] \tracecat \trace_3'} 
    \\
    \bigwedge &  \tlabel_{\trace_3} \cap \tlabel_{\trace_3'} = \emptyset
     \land \vcounter(\trace', \pi_2(\event_b)) = \vcounter(\trace, \pi_2(\event_b)) 
    %   \land \event_2 \eventin \trace_3
    % \land \event_2 \not\eventin \trace_3'
    \land \event_2 \in \trace_3
    \land \event_2 \not\in \trace_3'
  \end{array}
  \right)
  )
\end{array}
   \]
% , where ${\tt label}(\event_2) = \pi_2(\event_2)$.
  %  
%
\end{defn}
% \todo{add explnanation}
% \jl{
Our event \emph{may-dependency} relation of 
two events $\event_1 \in \eventset^{\asn}$ and $\event_2 \in \eventset$, 
for a program $c$ and hidden database $D$ is w.r.t to
a trace $[\event_1 ] \tracecat \trace \tracecat [\event_2]$.
The $\event_1 \in \eventset^{\asn}$ is an assignment event because only a change on an assignment event will affect the execution trace, according to our operational semantics.
In order to observe the changes of $\event_2$ under the modification of $\event_1$, this trace 
$[\event_1 ] \tracecat \trace \tracecat [\event_2]$
starts with $\event_1$ and ends with $\event_2$.
% }
{The \emph{may-dependency} relation considers both the value dependency and value control dependency as discussed in Section~\ref{sec:design_choice}. The relation can be divided into two parts naturally in Definition~\ref{def:event_dep} (line $2-4$, $5-8$ respectively, starting from line $1$). The idea of the event $\event_1$ may depend on $\event_2$ can be briefly described:
we have one execution of the program as reference (See line $2$ and $6$, for the two kinds of dependency). 
When the value assigned to the 
% first variable 
first variable in $\event_1$ is modified, the reference trace $\trace_1 \tracecat [\event_1]$ is modified correspondingly to $\trace_1 \tracecat [\event_1']$.
We use $\diff(\event_1, \event_1')$ at line $1$ to express this modification, which guarantees that $\event_1$ and $\event_1'$ only differ in their assigned values and are equal on variable name and label. We perform a second run of the program by continuing the execution of the same program from the same execution point, 
but with the modified trace $\trace_1 \tracecat [\event_1']$ (See line $3$, $7$). 
The expected may dependency will be caught by observing two different possible changes (See line $4, 8$ respectively) when comparing the second execution with the reference one (similar definitions as in \cite{Cousot19a}). 

% \wq{
% In the first situation, we are witnessing 
In the first part (line $2-4$ of Definition~\ref{def:event_dep}), we witness
% that the value assigned to the second variable in $\event_2$
the appearance of $\event_2'$ in the second execution, and
% a variation in $\event_2$, which changes into $\event_2'$.
a variation between $\event_2$ and $\event_2'$ on their values.
% changes in $\event_2'$.
% \jl{
We have special requirement $\diff(\event_2, \event_2')$, which guarantees that they
have the same variable name and label but only differ 
% % in their assigned value. 
in their evaluated values.
% assigned to the same variable. 
In particularly for queries, if $\event_2$ and $\event_2'$ are 
% query assignment events, then 
generated from query requesting, then $\diff(\event_2, \event_2')$ guarantees that
they differ in their query values rather than the 
% query requesting value. 
query requesting results. 
Additionally, in order to handle multiple occurrences of the same event through iterations of the while loop,
 where  $\event_2$ and $\event_2'$ could be 
in different while loops,
we restrict the same occurrence of $\event_2$'s label in $\trace$ from the first execution with  the occurrence of $\event_2'$'s label in $\trace'$ from the second execution,
through $\vcounter(\vtrace, \pi_2(\event_2))
= 
\vcounter(\vtrace', \pi_2(\event_2'))$ at line $4$.
% }
% }

% \wq{
In the second part (line $5-8$ of Definition~\ref{def:event_dep}), we 
% are witnessing 
witness
the disappearance of $\event_2$ through observing the change of a testing event $\event_b$.
% In order to change the appearance of 
% % and event, the command that generating $\event_2$ must not be executed in 
% 5yhan event, 
To witness
the disappearance, the command that generates $\event_2$ must not be executed in 
the second execution. 
The only way to control whether a command will be executed, is through the change of a guard's 
evaluation result in an if or while command, which generates a testing event $\event_b$ in the first place.
So we observe when
$\event_b$ changes into $\neg \event_b$ in the second execution firstly, 
whether it follows with the disappearance of $\event_2$ in the second trace. We restrict the occurrence of $\event_b$'s label in the two traces being the same
}
% s to the occurrence times of $\event_2'$'s label in the second trace,
through $\vcounter(\trace', \pi_2(\event_b)) = \vcounter(\trace, \pi_2(\event_b))$ to handle the while loop.
% changes in $\event_2'$, have the same variable and label and only differ in their assigned value. 
Again, for queries, we observe the disappearance based on the query value equivalence.
% if $\event_2$ and $\event_2'$ are query assignment events, then 
% they differ in their query value rather than the assigned value. 
% }
%
% \mg{I don't understand this explanation. What are the ``assignment commands associated to the two labelled variables''}
% \jl{revised but need more think}
% Explanation: 

{Considering 
% a program's all possible executions 
all events generated during a program's executions
under an initial trace,
% among all events generated during these executions
% and the variables and labels of these events are 
% corresponding to the two labeled variables,
% evaluations of the assignment commands associated to the two labelled variables respectively, 
as long as there is one pair of events satisfying the \emph{event may-dependency} relation in Definition~\ref{def:event_dep}, 
 we say the two 
related
variables satisfy the \emph{variable may-dependency} relation, in Definition~\ref{def:var_dep}.
}

\begin{defn}[Variable May-Dependency].
  \label{def:var_dep}
  \\
  A variable ${x}_2^{l_2} \in \lvar(c)$ is in the \emph{variable may-dependency} relation with another
  variable ${x}_1^{l_1} \in \lvar(c)$ in a program ${c}$, denoted as 
  %
  $\vardep({x}_1^{l_1}, {x}_2^{l_2}, {c})$, if and only if.
\[
  \begin{array}{l}
\exists \event_1, \event_2 \in \eventset^{\asn}, \trace \in \mathcal{T} , D \in \dbdom \sthat
% (\pi_{1}{(\event_1)}, \pi_{2}{(\event_1)}) = ({x}_1, l_1)
% \land
% (\pi_{1}{(\event_2)}, \pi_{2}{(\event_2)}) = ({x}_2, l_2)
\pi_{1}{(\event_1)}^{\pi_{2}{(\event_1)}} = {x}_1^{l_1}
\land
\pi_{1}{(\event_2)}^{\pi_{2}{(\event_2)}} = {x}_2^{l_2}% \\ \quad 
\land 
\eventdep(\event_1, \event_2, \trace, c, D) 
  \end{array}
\]  %
\end{defn}
\subsubsection{Data Dependency Quantity Analysis}
\label{subsubsec:dynamic-reachability}
For a program $c$, there are two data \emph{dependency quantities} we are considering.
The first quantity is the reachability times of each labeled variable during the program execution.
The second quantity is the reachability time for every pair of labeled variables with variable \emph{may-dependency} relation.
% \paragraph*{Variable Reachability}
\paragraph{The Dependency Quantity for Labeled Variables}
The reachability time of a labeled variable indicates the evaluation times of the assignment command assigning a value to this variable.  
\begin{defn}[Reachability Time of Labeled Variable]
  \label{def:adapt-var_reachability}
The reachability for every labeled variable overall $c$'s execution traces,
w.r.t. an initial trace $\vtrace \in \mathcal{T}_0(c)$ is defined as follows,
\[
  rb(x^l) \triangleq \forall \vtrace \in \mathcal{T}_0(c), \trace' \in \mathcal{T} \sthat \config{{c}, \trace} \to^{*} \config{\eskip, \trace\tracecat\vtrace'} 
  \implies w(\trace) = \vcounter(\vtrace', l) 
  \]
\end{defn}
%
$(x^l, w) \in \mathcal{LV} \times (\mathcal{T} \to \mathbb{N})$,
with a labeled variable as first component and
its weight $w$ the second component.
Weight $w$ for
% a labeled variable 
$x^l$ is a function $w : \mathcal{T} \to \mathbb{N}$
mapping from a starting trace to a natural number.
When program executes under this starting trace $\trace$,
$\config{{c}, \trace} \to^{*} \config{\eskip, \trace\tracecat\vtrace'} $, it generates an execution trace $\trace'$.
This natural number is the evaluation times of the labeled command corresponding to the vertex, 
computed by the counter operator $w(\trace) = \vcounter(\vtrace', l)$.


In most data analysis programs $c$ we are interested, there are usually some user input variables, such as $k$ in $\kw{twoRounds}$. 
We denote $\mathcal{T}_0(c)$ as the set of initial traces in which all the input variables in $c$ are initialized, it is also reflected in $\traceW({c})$.    
%
\paragraph{Dependency Quantity for the Pair of Labeled Variables}
% \paragraph*{Dependent Variables Reachability}
%
% For a program $c$ I compute the reachability bound for every labeled variable overall $c$'s execution traces,
% w.r.t. an initial trace as follows,
\begin{defn}[Reachability Time of Dependent Variables]
  \label{def:adapt-depvar_reachability}
  The execution-based reachability time for every pair of 
  labeled in the
  \emph{may-dependency} relation w.r.t. an initial trace. Formally as follows,
    \[
    \begin{array}{l}
        rb(x^i, y^j) \triangleq 
%   x^i, y^j \in \lvar(c)
%   \land w \in \mathcal{P}( \mathcal{T}_0(c) \to \mathbb{N})
%   \land 
%   \exists \trace \in \mathcal{T}_0(c), 
%   \trace_1, \trace_2 \in \mathcal{T} \sthat \dep(x^i, y^j,\trace_1, \trace_2, \trace_0, c)
%   \\
%   \land 
\forall \trace_0 \in \mathcal{T}_0(c) \sthat
  w (\trace_0) = \max \left\{ | \sdiff(\trace_1, \trace_2, y)|
  ~\middle\vert~
  \forall \trace_1, \trace_2 \in \mathcal{T} \sthat \dep(x^i, y^j,\trace_1, \trace_2, \trace_0, c) \right\}
\end{array}
\]
\end{defn}
%
For any pair of labeled variable $(x^i, y^j) \in \ldom$, 
$ rb(x^i, y^j)$ is a function $w: \mathcal{T}_0(c) \to \mathbb{N}$,
    where given an initial trace $\trace_0$,
    it is the maximum length of the difference sequence between all pairs of the witness traces $\trace_1, \trace_2$ 
    satisfying the dependency relation.

    \highlight{\paragraph*{Improvements Analysis}
    Previous works do not have any quantity analysis on the dependency relation.
    Comparing to them, this part is stronger in following senses.
    % It is more scalable to general program, and it provides the program with preciser formal definition for \emph{Adaptivity} than previous definition,
    % specifically as follows.
    % language and operational semantics design improves the expressiveness, efficiency, and the accuracy to a large extend.
    \todo{Add details}
    \begin{itemize}
      \item \textbf{Improvements on Efficiency}
      \\
      It is also efficient.
      \item \textbf{Improvements on Accuracy}
      This quantity analysis can help to improve the precision of the adaptivity formalization.
      \end{itemize}
      }

\paragraph*{The Dependency Quantity through The Two Rounds Example}
\begin{example}[Variable \emph{May-Dependency} Quantity in The Two Rounds Data Analysis Example Program]
    In the same $\kw{towRounds(k)}$ example Program,    the analyst asks in total $k+1$ queries to the mechanism in two phases.
    %
    \[         \begin{array}{l}
      \kw{towRounds(k)} \triangleq \\
             \clabel{ \assign{a}{0}}^{0} ;
              \clabel{\assign{j}{k} }^{1} ;\\
              \ewhile ~ \clabel{j > 0}^{2} ~ \edo ~
              \Big(
               \clabel{\assign{x}{\query(\chi[j] \cdot \chi[k])} }^{3}  ;
               \clabel{\assign{j}{j-1}}^{4} ;
              \clabel{\assign{a}{x + a}}^{5}       \Big);\\
              \clabel{\assign{l}{\query(\chi[k]*a)} }^{6}
          \end{array}
          \]    %
    % Queries are of the form $q(e)$ where $e$ is an expression with a special variable $\chi$ representing a possible row. Mainly $e$ represents a function from $X$ to some domain $U$, for example $U$ could be $[-1,1]$ or $[0,1]$. This function characterizes the linear query I are interested in running. As an example, $x \leftarrow q(\chi[2])$ computes an approximation, according to the used mechanism, of the empirical mean of the second attribute, identified by $\chi[2]$. Notice that I don't materialize the mechanism but I assume that it is implicitly run when I execute the query. 
    % \jl{We use $\chi$ to abstract a possible row in the database and }
    % queries are of the form $\query(\qexpr)$, where $\qexpr$ is a special expression 
    With the initial trace
    $[(k, in, 2, \bullet)]$ and following execution trace, 
    \\
    $
    \trace_1 \triangleq 
    \left[\begin{array}{l}
    % \trace_0 \tracecat
     (a, 0, 0, \bullet),
    (j, 1, 2, \bullet),
    (j>0, 2, \etrue, \bullet),
    (x, 3, v_1, \chi[2]*\chi[2]),
    (j, 4, 1, \bullet),
    (a, 5, v_1, \bullet),\\
    (j>0, 2, \etrue, \bullet),
    (x, 3, v_2, \chi[1]*\chi[2]),
    (j, 4, 0, \bullet),
    (a, 5, v_1 + v_2, \bullet),
    (j>0, 4, \efalse, \bullet),\\
    (l, 6, v_3, \chi[2]*( v_1 + v_2))
    \end{array} \right]
    $.
    Based on these observations, we analyze the \emph{may-dependency} quantity for every labeled variable,
    and pairs of dependent variables as follows.
\begin{itemize}
    \item \textbf{The Dependency Quantity for Labeled Variables}
    \\
    For the specific two execution traces above,
    the \emph{may-dependency} quantity for every variable
    is computed as follows,
   %   where $k$ is the 
   %  initial value of input variable $k$ given by user,
   %  we observe the execution trace as
   \\
   $rb(a^0) ((k, in, 2, \bullet))  = \vcounter(\trace_1) = 1$
   \\
   $\cdots$
  \\
   $rb(x^3) ((k, in, 2, \bullet))  = \vcounter(\trace_1) = 2$
    \\
    $\cdots$
    \\

    Then, for arbitrary initial trace $\trace_0 \in \mathcal{T}_0(\kw{twoRounds(k)})$,
    the \emph{may-dependency} quantity for every variable under $\trace_0$ is a function
    as follows,
    \\
    $rb(a^0) (\trace_0)  = 1$
    \\
    $\cdots$
    \\
    $rb(x^3) (\trace_0)  = \max\{0, \env(\trace_0) k \} $
    \item \textbf{Dependency Quantity for the Pair of Labeled Variables}
    For the specific two execution traces above,
    the \emph{may-dependency} quantity for every variable
    is computed as follows,
   %   where $k$ is the 
   %  initial value of input variable $k$ given by user,
   %  we observe the execution trace as
   \\
   $rb(a^0) ((k, in, 2, \bullet))  = \vcounter(\trace_1) = 1$
   \\
   $\cdots$
  \\
   $rb(x^3) ((k, in, 2, \bullet))  = \vcounter(\trace_1) = 2$
    \\
    $\cdots$
    \\

    Then, for arbitrary initial trace $\trace_0 \in \mathcal{T}_0(\kw{twoRounds(k)})$,
    the \emph{may-dependency} quantity for every variable under $\trace_0$ is a function
    as follows,
    \\
    $rb(a^0) (\trace_0)  = 1$
    \\
    $\cdots$
    \\
    $rb(x^3) (\trace_0)  = \max\{0, \env(\trace_0) k \} $
\end{itemize}
    % \\
    % We modify the value assigned to $x$ when evaluating the command $ \clabel{\assign{x}{\query(\chi[j] \cdot \chi[k])} }^{3}$
    % in the first iteration.
    % By manipulating the event in the trace, 
    % the event $(x, 3, v_1, \chi[2]*\chi[2])$
    % is modified into $(x, 3, v_1', \chi[2]*\chi[2])$ where $v_1 \neq v_1'$.
    % Then, through executing the program from the execution point after executing line $3$, we observe another execution trace as follows,
    % \\
    % $
    % \left[\begin{array}{l}
    % % \trace_0 \tracecat
    %  (a, 0, 0, \bullet),
    % (j, 1, 2, \bullet),
    % (j>0, 2, \etrue, \bullet),
    % \highlight{(x, 3, v_1', \chi[2]*\chi[2])},
    % (j, 4, 1, \bullet),
    % (a, 5, v_1, \bullet),\\
    % (j>0, 2, \etrue, \bullet),
    % (x, 3, v_2, \chi[1]*\chi[2]),
    % (j, 4, 0, \bullet),
    % \highlight{(a, 5, v_1' + v_2, \bullet),}
    % (j>0, 4, \efalse, \bullet),\\
    % (l, 6, v_3, \chi[2]*( v_1' + v_2))
    % \end{array} \right]
    % $.   
    % \\
    % In this trace, the event $(a, 5, v_1' + v_2, \bullet),$  is different from $(a, 5, v_1 + v_2, \bullet),$ in the first 
    % trace.
    % \\
    % This change satisfies the Definition~\ref{def:event_dep}, so there exists the variable \emph{may-dependency} relation 
    % between variable $x^3$ and $a^5$.
\end{example}%
\subsubsection{Execution-Based Adaptivity Analysis}
\label{subsubsec:dynamic-adapt}

Based on the variable \emph{may-dependency} relation in Section~\ref{subsec:dynamic-datadep} and 
the dependency quantity analysis in Section~\ref{subsec:dynamic-reachability}.
% gives us the edges, 
I firstly define the execution-based dependency graph, then formalize the \emph{adaptivity} in this section.
% \wq{Just a few sentences here, some overview of this subsection. See 4.2 for instance.}
\paragraph{Execution Based Dependency Graph}
\label{para:execution-base-graph-def}
Based on the variable \emph{may-dependency} relation,
% gives us the edges, 
we define the execution-based dependency graph.
% \wq{Just a few sentences here, some overview of this subsection. See 4.2 for instance.}
\begin{defn}[Execution Based Dependency Graph]
\label{def:trace_graph}
Given a program ${c}$,
its \emph{execution-based dependency graph} 
$\traceG({c}) = (\traceV({c}), \traceE({c}), \traceW({c}), \traceF({c}))$ is defined as follows,
{
  \small
\[
\begin{array}{rlcl}
  \text{Vertices} &
  \traceV({c}) & := & \left\{ 
  x^l \in \mathcal{LV}
  ~ \middle\vert ~ x^l \in \lvar(c)
  \right\}
  \\
  \text{Directed Edges} &
  \traceE({c}) & := & 
  \left\{ 
  (x^i, y^j) 
%   \in \mathcal{LV} \times \mathcal{LV}
  ~ \middle\vert ~
  x^i, y^j \in \lvar(c) \land \vardep(x^i, y^j, c) 
  % \text{\mg{$\land$ instead of ,}}
  \right\}
  \\
  \text{Weights} &
  \traceW({c}) & := & 
%   \left
  \{ 
  (x^l, w) 
  % \in \mathcal{LV} \times \mathbb{N}
  ~ \vert ~ 
  w : \mathcal{T} \to \mathbb{N}
  \land
  x^l \in \lvar(c) 
  \\ & & &
  \land
  % n = \max \left\{ 
    % ~ \middle\vert~
  \forall \vtrace \in \mathcal{T}_0(c), \trace' \in \mathcal{T} \sthat \config{{c}, \trace} \to^{*} \config{\eskip, \trace\tracecat\vtrace'} 
  \implies w(\trace) = \vcounter(\vtrace', l) 
  %  \right\}
%   \right
\}
  \\
  % \text{Query Label} &
  \text{Query Annotation} &
  \traceF({c}) & := & 
\left\{(x^l, n)  
% \in  \mathcal{LV}\times \{0, 1\} 
~ \middle\vert ~
 x^l \in \lvar(c) \land
n = 1 \Leftrightarrow x^l \in \qvar(c) \land n = 0 \Leftrightarrow  x^l \notin \qvar(c)
\right\}
\end{array}.
\]
}
\end{defn}
%
There are four components of the execution-based dependency graph. 
The vertices $\traceV(c)$ is the set of program $c$'s labeled variables $\lvar(c)$,
which are statically collected.
The query annotation is 
a set of pairs $\traceF(c) \in \mathcal{P}(\mathcal{LV} \times \{0, 1\} )$ 
mapping each $x^l \in \traceV(c)$ to $0$ or $1$, 
indicating whether this labeled variable is in program $c$'s query variable set $\qvar(c)$.
{
The weights is a set of pairs, $(x^l, w) \in \mathcal{LV} \times (\mathcal{T} \to \mathbb{N})$,
with a labeled variable as first component and
its weight $w$ the second component.
Weight $w$ for
% a labeled variable 
$x^l$ is a function $w : \mathcal{T} \to \mathbb{N}$
mapping from a starting trace to a natural number.
When program executes under this starting trace $\trace$,
$\config{{c}, \trace} \to^{*} \config{\eskip, \trace\tracecat\vtrace'} $, it generates an execution trace $\trace'$.
This natural number is the evaluation times of the labeled command corresponding to the vertex, 
computed by the counter operator $w(\trace) = \vcounter(\vtrace', l)$.
We can see in the execution-based dependency graph of $\kw{twoRounds}$ in Figure~\ref{fig:overview-example}(b), the weight of vertices in the while loop is  $\env(\trace) k$, which depends on the value of the user input $k$ specified in the starting trace $\tau$.
The directed edges $\traceE({c})$ is also a set of pairs with two labeled variables $ (x^i, y^j) \in \mathcal{LV} \times \mathcal{LV}$, from $x^i$ pointing to $y^j$ in the graph.
The edges are constructed directly from our variable may-dependency relation. 
For any two vertices $x^{i}$ and $y^{j}$ in $\traceV(c)$, if they satisfy the variable may-dependency relation $\vardep(x^i, y^j, c)$, there is a direct edge between the two vertices in our execution-based dependency graph for program $c$.
} 
In most data analysis programs $c$ we are interested, there are usually some user input variables, such as $k$ in $\kw{twoRounds}$. 
We denote $\mathcal{T}_0(c)$ as the set of initial traces in which all the input variables in $c$ are initialized, it is also reflected in $\traceW({c})$.    

\paragraph{Trace-based Adaptivity}

% \wq{
% Given 
% a program $c$'s execution-based dependency graph 
% % $G_{trace}(c)(\trace) = (\vertxs, \edges, \weights, \qflag)$,
% $\traceG({c}) = (\traceV({c}), \traceE({c}), \traceW({c}), \traceF({c}))$
% we define adaptivity 
% with respect to $\trace$ by the finite walk in the graph, which has the most query requests along the walk.
% }

Given 
a program $c$'s execution-based dependency graph 
% $G_{trace}(c)(\trace) = (\vertxs, \edges, \weights, \qflag)$,
$\traceG({c})$,
we define adaptivity 
with respect to an initial trace $\trace_0 \in \mathcal{T}_0(c)$ by the finite walk in the graph, which has the most query requests along the walk.
We show the definition of a finite walk as follows.
%
% The query length of a walk $k$ is the number of vertices which correspond to query variables in the vertices sequence of this walk. 
% Instead of counting all 
% the vertices in $k$'s vertices sequence, i

\begin{defn}[Finite Walk (k)].
  \label{def:finitewalk}
  \\
%   Given a program $c$'s execution-based dependency graph $\traceG({c})(\trace)$, 
%   a \emph{finite walk} $fw$ in $\traceG({c})(\trace)$ is a sequence of edges $(e_1 \ldots e_{n - 1})$ 
%   for which there is a sequence of vertices $(v_1, \ldots, v_{n})$ such that:
%   \begin{itemize}
%       \item $e_i = (v_{i},v_{i + 1})$ for every $1 \leq i < n$.
%       \item every vertex $v \in \traceV({c}) $ appears in $(v_1, \ldots, v_{n})$ at most 
%       \wq{$\traceW({c})(\trace)$} times.  
%   \end{itemize}
%   %
%   The length of $fw$ is the number of vertices in its vertex sequence, i.e., $\len(k) = n$.
  Given the execution-based dependency graph $\traceG({c}) = (\traceV({c}), \traceE({c}), \traceW({c}), \traceF({c}))$ of a program $c$,
  a \emph{finite walk} $k$ in $\traceG({c})$ is a 
  function $k: \mathcal{T} \to $ sequence of edges.
  For a initial trace $\trace_0 \in \mathcal{T}_0(c)$, 
  $k(\trace_0)$ is a sequence of edges $(e_1 \ldots e_{n - 1})$ 
  for which there is a sequence of vertices 
  $(v_1, \ldots, v_{n})$ such that:
  \begin{itemize}
      \item $e_i = (v_{i},v_{i + 1}) \in \traceE(c)$ for every $1 \leq i < n$.
      \item every $v_i \in \traceV(c)$
      and $(v_i, w_i) \in \traceW(c)$, 
       $v_i$ appears in $(v_1, \ldots, v_{n})$ at most 
    %   \wq{$\traceW({c})(\trace)$} 
    $w(\trace_0)$
      times.  
  \end{itemize}
  %
  The length of $k(\trace_0)$ is the number of vertices in its vertices sequence, i.e., $\len(k)(\trace_0) = n$.
 \end{defn}

We use $\walks(\traceG(c))$ to denote 
% \mg{``the set'', not ``a set''}a set containing all finite walks $k$ in $G$;
the set containing all finite walks $k$ in $\traceG(c)$;
and $k_{v_1 \to v_2} \in \walks(\traceG(c))$ with $v_1, v_2 \in \traceV(c)$ denotes the walk from vertex $v_1$ to $v_2$ . 
\\
We are interested in queries, so we need to recover the 
variables corresponding to queries from the walk. We define the query length of a walk, 
instead of counting all 
the vertices in $k$'s vertices sequence, we just count the number of vertices which correspond to query variables in this sequence.
%
% \mg{I don't understand this definition. Is wrt a single query?if yes, who is chosing the query? Or is it any query?}
% \jl{It is for any query, as long as the vertex is a query variable, in another worlds, this length just counting the number of query variables in the walk, instead of counting all 
% the vertices.}
% \todo{Make the definition clear}
\begin{defn}[Query Length of the Finite Walk($\qlen$)].
\label{def:qlen}
\\
% Given 
% % labelled weighted graph $G = (\vertxs, \edges, \weights, \qflag)$, 
% a program $c$'s execution-based dependency graph $\traceG(c)(\trace)$
%  and a \emph{finite walk} $k$ in $\traceG(c)(\trace)$ with its vertex sequence $(v_1, \ldots, v_{n})$, 
% %  the length of $k$ w.r.t query is defined as:
% The query length of $k$ is the number of vertices which correspond to query variables in $(v_1, \ldots, v_{n})$ as follows, 
% \[
%   \qlen(k) = \len\big( v \mid v \in (v_1, \ldots, v_{n}) \land \qflag(v) = 1 \big)
% \]
% , where $\big(v \mid v \in (v_1, \ldots, v_{n}) \land \qflag(v) = 1 \big)$ is a subsequence of $(v_1, \ldots, v_{n})$.
Given 
% labelled weighted graph $G = (\vertxs, \edges, \weights, \qflag)$, 
the execution-based dependency graph $\traceG({c}) = (\traceV({c}), \traceE({c}), \traceW({c}), \traceF({c}))$ of a program $c$,
 and a \emph{finite walk} 
%  $k$ in $\traceG(c)(\trace)$
 $k \in \walks(\traceG(c))$. 
%  with its vertex sequence $(v_1, \ldots, v_{n})$, 
%  the length of $k$ w.r.t query is defined as:
The query length of $k$ is a function $\qlen(k): \mathcal{T} \to \mathbb{N}$, such that with an initial trace  $\trace_0 \in \mathcal{T}_0(c)$, $\qlen(k)(\trace_0)$ is
the number of vertices which correspond to query variables in the vertices sequence of the walk $k(\trace_0)$
$(v_1, \ldots, v_{n})$ as follows, 
\[
  \qlen(k)(\trace_0) = |\big( v \mid v \in (v_1, \ldots, v_{n}) \land \qflag(v) = 1 \big)|.
\]
\end{defn}
The definition of adaptivity is then presented in Def~\ref{def:trace_adapt} below.

\begin{defn}
  [Adaptivity of a Program].
  \label{def:trace_adapt}
  \\
  Given a program ${c}$, 
  its adaptivity $A(c)$ is function 
  $A(c) : \mathcal{T} \to \mathbb{N}$ such that for an
  % with respect to a starting trace $\trace$ 
  initial trace $\trace_0 \in \mathcal{T}_0(c)$, 
  % is defined as follows:
  %
 $$
  A(c)(\trace_0) = \max \big 
  \{ \qlen(k)(\trace_0) \mid k \in \walks(\traceG(c)) \big \} $$
  \end{defn}%
%
\subsection{Adaptivity through An Example}
\label{subsec:dynamic-examples}
\begin{example}[twoRounds]
    In this example program $\kw{towRounds(k)}$, the analyst asks in total $k+1$ queries to the mechanism in two phases.
    In the first phase, the analyst asks $k$ queries and stores the answers that are provided by the mechanism. 
    In the second phase, the analyst constructs a new query based on the results of the previous $k$ queries and sends this query to the mechanism. More specifically, we assume that, in this example, the domain $\dbdom$ 
    contains at least $k$ numeric attributes, which we index just by natural numbers. 
    The queries inside the while loop correspond to the first phase and compute an approximation of 
    the product of the empirical mean of the first $k$ attributes. 
    The query outside the loop corresponds to the second phase and computes an approximation of the empirical mean where each record is weighted by the sum of the empirical mean of the first $k$ attributes.
    %
    % Queries are of the form $q(e)$ where $e$ is an expression with a special variable $\chi$ representing a possible row. Mainly $e$ represents a function from $X$ to some domain $U$, for example $U$ could be $[-1,1]$ or $[0,1]$. This function characterizes the linear query we are interested in running. As an example, $x \leftarrow q(\chi[2])$ computes an approximation, according to the used mechanism, of the empirical mean of the second attribute, identified by $\chi[2]$. Notice that we don't materialize the mechanism but we assume that it is implicitly run when we execute the query. 
    % \jl{We use $\chi$ to abstract a possible row in the database and }
    % queries are of the form $\query(\qexpr)$, where $\qexpr$ is a special expression 
    %
    {Since statistical query computes the empirical mean of a function on rows, we use $\chi$ to abstract a possible row in the database and }
    queries are of the form $\query(\qexpr)$, where $\qexpr$ is a special expression 
    (as in our syntax in Section~\ref{sec:language})
    {
    % from $X$ to some domain $U$, 
    % for example $U$ 
      We use $U$ to denote the co-domain of queries, and it could be $[-1,1]$, $[0,1]$ or $[-R,+R]$, for some $R$ we consider.
      This function characterizes the linear query we are interested in running. 
      As an example, $x \leftarrow \query(\chi[j] \cdot \chi[k])$ computes an approximation, according to the used mechanism, of the empirical mean of the product of the $j^{th}$ attribute and $k^{th}$ attribute, identified by $\chi[j] \cdot \chi[k]$. Notice that we don't materialize the mechanism but we assume that it is implicitly run when we execute the query. } 

      The graph in Figure~\ref{fig:twoRounds_example}(b). This graph is built by considering all the possible execution traces of the program in   Figure~\ref{fig:twoRounds_example}(a).
      Each vertex in this graph has a superscript representing its weight, and a subscript $1$ or $0$ telling if the vertex corresponds to a query or not. We will call this subscript a query annotation. 
      For example the vertex $l^{6}:{}^{w_1}_1$, 
      % the superscript $1$ represents the weight $1$, and the subscript for the query annotation.
      has weight $w_1$, a constant function which returns $1$ for every starting state, since 
      this query at line $6$ is at most executed once regardless of the initial trace.
      The query annotation of this vertex is $1$, which  indicates that 
      $\clabel{\assign{l}{\query(\chi[k] * a)}}^6$ is a query request.
      % The assignment in the while loop, such as node $x^{3}$, 
      Another vertex, $x^{3}:{}^{w_k}_1$, appears in the while loop. 
      It has as weight a function $w_k$ that for every initial state returns the value that $k$ has in this state, since this is also the number the while loop will be iterated. 
      The node $j^{4}:{}^{w_k}_0$ has as a subscript $0$ representing a non-query assignment.
      
      
      Since the edges between two vertices represent the fact that one program variable may depend on the other,
      % the queries that are executed and the edges between two nodes represent the fact that one query may depend on the other. 
      we can define the program adaptivity with respect to a initial trace by means of a walk traversing the graph, visiting each vertex no more than its weight with respect to the initial trace, and visiting as many query nodes as possible.
      % In the walk that passes the most times of query nodes, the total visiting times of this walk on 
      % these query nodes is defined as adaptivity.
      %
      So, looking again at our example, we can see that
      % if the input variable $k$ is less than $1$ in an initial trace $\trace_1$, then it is easy to see the weight of vertex $x^3$, $w_3(\trace_1) < 1$ and we can only find a walk with one vertex $l^{6}$, according to  the definition of finite walk in Definition~\ref{def:finitewalk}. So the adaptivity for $\trace_1$, as the number of query vertices along the walk, is $1$. It is easy to understand because when $k <1$, the while loop will not be executed and only one query is asked in total. However, in reality, people want the adaptivity of this example when $k \geq 1$. With this initial trace, it is easy to see that 
      in the walk along the dotted arrows,  $l^{6} \to a^5 \to x^3 $, there are $2$ vertices with query annotation $1$ and that this number is maximal, i.e. we cannot find another walk having more than $2$ vertices with query annotation $1$, under the assumption that $k \geq 1$. So the adaptivity of the program in Figure~\ref{fig:twoRounds_example}(a)  is $2$,
      % longest walk in the graph in Figure~\ref{fig:twoRounds_example}(b), which we mark with a red dashed arrow, is $2$, 
      as expected.
{\small
\begin{figure}
\centering
\begin{subfigure}{.2\textwidth}
\begin{centering}
$
    \begin{array}{l}
    \kw{towRounds(k)} \triangleq \\
           \clabel{ \assign{a}{0}}^{0} ;
            \clabel{\assign{j}{k} }^{1} ; \\
            \ewhile ~ \clabel{j > 0}^{2} ~ \edo ~ \\
            \Big(
             \clabel{\assign{x}{\query(\chi[j] \cdot \chi[k])} }^{3}  ; \\
             \clabel{\assign{j}{j-1}}^{4} ;\\
            \clabel{\assign{a}{x + a}}^{5}       \Big);\\
            \clabel{\assign{l}{\query(\chi[k]*a)} }^{6}\\
        \end{array}
$
\caption{}
\end{centering}
\end{subfigure}
\begin{subfigure}{.75\textwidth}
%}
\qquad
\begin{centering}
 \begin{tikzpicture}[scale=\textwidth/18cm,samples=200]
\draw[] (0, 10) circle (0pt) node
{{ $a^0: {}^{w_1}_{0}$}};
\draw[] (0, 7) circle (0pt) node
{\textbf{$x^3: {}^{w_k}_{1}$}};
\draw[] (0, 4) circle (0pt) node
{{ $a^5: {}^{w_k}_{0}$}};
\draw[] (0, 1) circle (0pt) node
{{ $l^6: {}^{w_1}_{1}$}};
% Counter Variables
\draw[] (5, 9) circle (0pt) node {\textbf{$j^1: {}^{w_1}_{0}$}};
\draw[] (5, 6) circle (0pt) node {{ $j^4: {}^{w_k}_{0}$}};
%
% Value Dependency Edges:
\draw[ ultra thick, -latex, densely dotted,] (0, 1.5)  -- 
% The Weight for this edge
node [left] {\highlight{$\trace_0 \to 1 $}}(0, 3.5) ;
\draw[ ultra thick, -latex, densely dotted,] (0, 4.5)  -- 
node [left] {\highlight{$\trace_0 \to \env(\trace_0) k $}}(0, 6.5) ;
\draw[ thick, -latex] (0, 4.5)  to  [out=-230,in=230]  
node [left] {\highlight{$\trace_0 \to \env(\trace_0) k $}}(0, 9.5) ;
\draw[ thick, -Straight Barb] (1.5, 3.5) arc (120:-200:1);
    % The Weight for this edge
    \draw[](3, 3) node [] {\highlight{$\trace_0 \to \env(\trace_0) k  $}};
\draw[ thick, -Straight Barb] (6.5, 6.5) arc (150:-150:1);
    % The Weight for this edge
    \draw[](9, 6) node [] {\highlight{$\trace_0 \to \env(\trace_0) k  $}};
\draw[ thick, -latex] (5, 6.5)  -- 
% The Weight for this edge
node [right] {\highlight{$\trace_0 \to \env(\trace_0) k $}} (5, 8.5) ;
% Control Dependency
\draw[ thick,-latex] (1.5, 7)  -- (4, 9) ;
\draw[ thick,-latex] (1.5, 4)  -- 
% The Weight for this edge
node [] {\highlight{$\trace_0 \to \env(\trace_0) k $}} (4, 9) ;
\draw[ thick,-latex] (1.5, 7)  -- (4, 6) ;
\draw[ thick,-latex] (1.5, 4)  -- (4, 6) ;
\end{tikzpicture}
\caption{}
\end{centering}
\end{subfigure}
 \caption{(a) The program $\kw{towRounds(k)}$, an example 
%  of a program 
with two rounds of adaptivity (b) The corresponding execution-based dependency graph.}
\label{fig:twoRounds_example}
\end{figure}
}
\end{example}
%

% In terms of techniques, our work relies on ideas from both static analysis and dynamic analysis. 
We discuss closely related work in both areas.


%
% \subsection{Implementation}
% \label{subsec:dynamic-implementation}
%
% \cleardoublepage

% \chapter{The Program Static Analysis for Adaptivity}
\label{ch:adapt-algo}


\section{Introduction}
\label{sec:static-intro}

\section{Static Data Dependency Analysis}
\label{sec:static-datadep}

\section{Static Reachability Bound Analysis}
\label{sec:static-reachability}

\section{Static Adaptivity Analysis}
\label{sec:static-adapt}

\section{Examples}
\label{sec:static-examples}
%
\section{Implementation}
\label{sec:static-implementation}

\section{Related Work}
\label{sec:static-relatedwork}
In terms of techniques, our work relies on ideas from both static analysis and dynamic analysis. 
We discuss closely related work in both areas.


% \cleardoublepage

\section{Status and Plan}
The $\THESYSTEM$ presented in Section~\ref{sec:dynamic} and Section~\ref{sec:static} are submitted as a paper to POPL'2023 and implemented as well. Below is the timeline plan for the further features of this work.
\begin{itemize}
\item September 05, 2022: Finish the improved execution-based dependency depth analysis and implementation
\item September 20, 2022: Finish Path Sensitive Reachability Bound Algorithm Formalization and implementation
\item September 30, 2022: Improve the completeness of the static adaptivity computation algorithm and implementation
\item October 15, 2022:  Finish generalization on program resource cost analysis and implementation
\item November 05, 2022: Finish Path Sensitive Reachability Bound paper writing and Submit to PLDI 2023
\item November 25, 2022: Finish thesis
\item December 05, 2022: Defend
\end{itemize}


\bibliographystyle{plain}
\bibliography{main.bib}

\end{document}
