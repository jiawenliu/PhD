Data analyses are usually designed to identify some property of the population from which the data are drawn, 
generalizing beyond the specific data sample. For this reason, data analyses are often designed in a way that guarantees that they produce a low generalization error.
 That is, they are designed so that the result of a data analysis run on sample 
 data does not differ too much from the result one would achieve by running the analysis over the entire population. 
 
 An adaptive data analysis can be seen as a process composed of multiple queries interrogating some data, where the choice of which query to run next may rely on the results of previous queries. 
 The generalization error of individual query/analysis can be controlled by using an array of well-established statistical techniques.
 However, when queries are arbitrarily composed, the different errors can propagate through the chain of different queries and bring high generalization errors. 
 To address this issue, data analysts are designing several techniques that not only guarantee bounds on the generalization errors of single queries, but also guarantee bounds on the generalization error of the composed analyses. 
 The choice of which of these techniques to use, 
 often depends on the chain of queries that an adaptive data analysis can generate.
 Specifically, the total number of queries and the depth of the chain of queries is of great significance 
 to guarantee the generalization error, 
 when the composed data analyses are adaptive. 
 So in order to give a precise guarantee of generalization error
 for the program,
 I'm interested in analyzing the depth of the chain of queries in a program, i.e., the program's \emph{adaptivity} property.
 % Gap
 % Unfortunately, this depth which relies on the program(implementation) itself is costly in human efforts, and how to statically obtain this information is not well studied to support data analysts.

 In this proposal, I firstly focus on formalizing and analyzing the intuitive \emph{adaptivity} property for 
 the adaptive data analysis program
 and present 
 my full-spectrum \emph{adaptivity} analysis framework.
 Next, based on the implementation and experimental results of my \emph{adaptivity} analysis framework, 
 I propose three significant 
 further features can be improved in this framework.
 % and plan to finish the improvement 
 % before the final defense.
Then according to the connection between the \emph{adaptivity} and program's resource cost,
I propose 
 % I propose extensions of this analysis with improved techniques, 
 % and 
an accurate full-spectrum program resource cost analysis via
the generalization of my \emph{adaptivity} analysis framework.
% plan to finish the design and implementation in the thesis.
In the end, 
I propose an interesting further work on solving the 
CFL-Reachability problem by reducing it into my \emph{adaptivity} analysis framework, 
based on observing the similarities between them.
 % onto general program's resource cost analysis,
 % .