% Data analyses are usually designed to identify some property of the population from which the data are drawn, 
% generalizing beyond the specific data sample. For this reason, data analyses are often designed in a way that guarantees that they produce a low generalization error.
% That is, they are designed so that the result of a data analysis run on sample 
% data does not differ too much from the result one would achieve by running the analysis over the entire population. 
 
% An adaptive data analysis can be seen as a process composed of multiple queries interrogating some data, where the choice of which query to run next may rely on the results of previous queries. 
% The generalization error of individual query/analysis can be controlled by using an array of well-established statistical techniques.
% However, when queries are arbitrarily composed, the different errors can propagate through the chain of different queries and bring high generalization errors. 
% To address this issue, data analysts are designing several techniques that not only guarantee bounds on the generalization errors of single queries but also guarantee bounds on the generalization error of the composed analyses. 
% The choice of which of these techniques to use, 
% often depends on the chain of queries that an adaptive data analysis can generate.
% Specifically, the total number of queries and the depth of the chain of queries is of great significance 
% to guarantee the generalization error, 
% when the composed data analyses are adaptive. 
% So in order to give a precise guarantee of generalization error
% for the program,
% I am interested in analyzing the depth of the chain of queries in a program, i.e., the program's \emph{adaptivity} property.
 % Gap
 % Unfortunately, this depth which relies on the program(implementation) itself is costly in human efforts, and how to statically obtain this information is not well studied to support data analysts.
% \todo{Intro The Program Analysis On non-functional Property}
% \wqside{First of all, is "reachability property of program" something common?  }
% \highlight
% {
 Many program properties can be expressed in terms of reachability,
for example,
the dependency of a program's outputs and inputs can be expressed based on the reachability of the input data, 
the secret data's leakage in a program can be expressed through the reachability of the secret data, 
whether some pieces of code are executed can be described via the reachability of the code pieces, etc.
% }
% However, it is not useful enough to describe these properties only
However, it is not useful enough to describe these properties only
% in terms of reachability,
from the reachability perspective,
% Meanwhile, 
there are many other
% some other 
important information related to these properties from the quantitative aspect.
For example, the amount of leaked data, which is crucial in security;
the times that some pieces of code are executed, which is important in resource cost analysis; etc.
% Reachability can tell us the existence of these properties.
The properties that have both reachable and quantitative information are
% These kinds of properties are 
the program's quantitative reachability properties.
% which are program properties that can be expressed in terms of both reachability and quantity.
% \highlight{
 Providing both the reachability estimation and quantitative bounds on 
these program properties are more useful to improve software design
% for different potential applications 
 than only one.
For example, knowing both the reachability and reaching times of pieces of code
can better help to optimize the software in resource usage than only knowing the reachability of these code pieces;
knowing
% both the
% reachability and reaching times of 
how much secret data is leaked can
% better help
% to 
improve the software's security better than only knowing whether the date is leaked; etc.
% }
% Moreover, 
% by exploiting the program's quantitative reachability properties,
% other different program properties beyond them can also be 
% studied,
% based on it
% expressed
% in a 
% through quantitative reachability
% For example, in resource cost analysis,
Moreover, 
by exploiting the program's quantitative reachability properties,
other potential program properties beyond them can also be 
studied,
such as the number of function calls,
program execution steps,
lines of executed codes,
the power consumption, memory usage, etc. 
% \highlight{
 Controlling
these properties
% from the two aspects
% can improve program reliability, security, privacy, usability, etc.
% For instance, in security, whether and how the program's output is influenced by the user's input
% can be studied based on the 
% quantitative reachability
% of user-defined input w.r.t. the output of the program.
% the performance in terms of the execution time of two pieces of code for the same task can be used to decide which one to use. 
% and 
can further improve the software's reliability, security, privacy, usability, etc.
% }
To this end, I'm interested in
% designing the program analysis on 
analyzing the program's quantitative reachability properties.
% \highlight{
 Analyzing the quantitative reachability properties is challenging.
The techniques for analyzing the reachability can only help us to determine whether a program holds some of these properties.
However, they cannot tell us the quantitative information about these properties.
% such as whether the output relies on user input.
% for example whether the secret data is leaked in the security 
For instance, the reachability analysis tool
"Zoncolan" finds security issues in Meta's codebase by monitoring whether any sensitive user data is reached by the third-party code.
However, it can not tell us how many times this sensitive data is used.
% area.
% However, they cannot tell us
% how many times or how long these properties hold for the program.
% the quantitative information of these properties.
% For example, 
% it can not tell us how much data is leaked, or
The liveness analysis techniques can tell us whether some pieces of code will be executed, but not
% , which is crucial in the security analysis area.
% It can not tell us 
how many times some pieces of code are executed.
% which is important in resource cost analysis area.
% In data analysis area,
% reachability-based program analysis 
Dependency analysis can tell us whether an execution of a certain code piece is determined by
the execution of some other code pieces.
However, during multiple executions,
it cannot tell us how many times this dependency relation holds.
To have useful analysis results on these quantitative reachability properties,
we need to perform extra analyses
% of the program's quantitative property. 
to get the quantitative information
(in addition to the boolean information describing the reachability).
% in this proposal.
% the reachability and quantity aspects of the property.
In this sense, I'm interested in designing
new program analysis frameworks analyzing several quantitative reachability properties of programs,
% that can be used to improve the programs' performance, along different axes.
which can provide useful analysis results from both the reachability and quantity aspects.
% }

The first quantitative reachability property on which
 I focus is the adaptivity of the program in the adaptive data analysis.
%  \highlight{
    A data analysis is designed for identifying properties for unknown populations 
 through some data samples.
%  }
 % ?= is widely 
% used in research and industrial areas.
It can be modeled as a process composed of 
multiple queries interrogating the data.
% This is intuitively defined as the depth of the chain of dependent queries. 
 In an adaptive data analysis, the choice of some queries can depend on the results of their previous queries. 
% \highlight{
 Hence, these dependent queries form trees in which every query relies on its previous ones.
% the generalization error is propagated through the reliance between the choices of queries.
The \emph{adaptivity} is intuitively defined as the maximum depth of the trees that are formed by these dependent queries
in an adaptive data analysis.
% (i.e., how many queries are relied on others) in the analysis plays a key role in reducing the generalization error.
When generalizing the analysis result from the data samples to unknown populations, 
the generalization error is the key issue that needs to be resolved.
% }
% Many researchers have found that the generalization error can be controlled well
% % by choosing the appropriate statistic technique 
% according to the adaptivity. 
% Hence, reasoning about 
 Many researchers have found that the
 \emph{adaptivity} is useful for controlling the generalization error of an adaptive data analysis,
 which usually propagates sharply along with the dependent queries. 
 However, when the data analysts implement the adaptive
 data analysis as programs and execute them,
this intuitive \emph{adaptivity} is not always clear in these programs
when they want to use them to control the programs' generalization errors.
% \highlight{
    To assist them automatically and efficiently 
to control the generalization error of the program which implements an adaptive
data analysis,
% using the adaptivity information,
% when they need it. 
% To solve this dilemma, I
 I propose a program analysis framework in the \redd{PART I}.
 This analysis framework
% based on dependency graphs,
 provides the data analysts with upper bounds on their programs' \emph{adaptivity}.
% to assist data analysts
It first defines the \emph{adaptivity} for 
 the program that implements an adaptive data analysis
 and then
% designing
% a program analysis framework for 
 estimates its \emph{adaptivity} statically.
%  }
% This part is developed by 
% This 
% by first defining the \emph{adaptivity} for 
% the program that implements an adaptive data analysis
% and then designing
% a program analysis framework for estimating the program's \emph{adaptivity}.
% Next, based on the implementation and experimental results of my \emph{adaptivity} analysis framework, 
% I propose three significant 
% further features can be improved in this framework.
 % and plan to finish the improvement 
 % before the final defense.
% Then according to the connection between the \emph{adaptivity} and the program's resource cost,
% I propose 
% % I propose extensions of this analysis with improved techniques, 
% % and 
% an accurate full-spectrum program resource cost analysis via
% the generalization of my \emph{adaptivity} analysis framework.
% Then according to the similarities between the \emph{adaptivity} and the program's \emph{non-monotonic} resource cost,
% I present 
 % I propose extensions of this analysis with improved techniques, 
 % and 

% In Resource Cost Analysis Area, the Reachability-Bound is an execution property that has broad
% applications in bounding resources cost by a program or analyzing the program's other run-time behaviors.


% Another execution property which is also significant is the non-monotonic quantitative property.
The second quantitative reachability property which I'm interested in is 
% the program's reachability-bound.
% % This property is a worst-case bound on 
the number of times a given control location 
 inside a procedure is visited during the program execution.
% The number of times a given control location 
% inside a procedure is visited during the program execution
% is an importance quantitative reachability property for programs.
% Finding a tight bound on this quantitative reachability property is
% useful in different applications, 
% such as
% bounding 
% resources consumed by a program such as time, memory,
% network-traffic, power, 
% or helping to estimate other quantitative properties (as opposed to boolean properties)
% of data in programs, such as information leakage or uncertainty propagation.
% \highlight{
    The upper bound on this execution number for 
a given control location is referred to as the reachability-bound by \cite{GulwaniZ10} in the first place.
% Computing the
% the program's reachability-bound.
% This property is 
A tight reachability-bound
% on this property
% is referred to as the reachability-bound problem
% in \cite{GulwaniZ10}.
% This property 
can help to improve the analysis result on other program features.
% }
% has broad
% applications 
% in improving the program's performance from different aspects.
% in analyzing other program property beyond the quantitative reachability property.
For example, in the resource cost analysis, this quantitative reachability property
can help to provide a tighter
% improve the precision of the 
bound on the resources consumed by a program such as time, memory,
network traffic, power, etc.
In the data analysis area,
% In analyzing some quantitative properties
% % this problem is important in computing a precise analysis result.
% in the data analysis area, 
% such as the \emph{adaptivity}
% % For example, in the data analysis area 
% in PART I, 
the reachability-bound on each program control location
% helps
% in 
can help to improve the precision of the estimated bound on the program's \emph{adaptivity} as introduced in PART I.
In security, this reachability-bound helps to improve the accuracy
in estimating the program's information leakage or uncertainty propagation, etc.
% , it can also help to improve accuracy.
However, computing a tight reachability-bound for a given program control location is challenging.
Many works in the program analysis area develop algorithms estimating the program's loop bound or overall complexity.
% None of them solve the reachability-bound problem directly, or path-sensitively.
However, there isn't a general analysis algorithm that
computes the reachability-bound
% solves this problem 
directly or path-sensitively.
To leverage these limitations,
% and improve the analysis results on this problem,
I propose a path-sensitive reachability-bound algorithm
aiming to find a precise symbolic worst-case reachability-bound on the program's every control location
% than existing works, 
in the \redd{PART II}.
% through static analysis.

% in terms of the inputs to that procedure.
% aims to give the accurate reaching times bound
% for every program location.


% a static analysis framework for 
% \todo
% {
%  The third challenging quantitative reachability property I'm interested in is the program's {non-monotonic} quantitative property.
%  This kind of property includes memory usage in the presence of garbage collection,
% number of channel connections established that are later closed,
% or resources requested to a virtual host which is released after using them. 
% The non-monotonic quantitative property is significantly different from the traditional monotonic quantitative property,
% such as the program execution time, energy consumption,
% % etc. w.r.t. the physical resources,
% or the information leakage, etc. with different measurements in different areas.
% % ).
% These traditional quantitative properties only accumulate during the program execution. 
% However, the non-monotonic quantitative property could also decrease along the computation.
% To give a precise estimation on the {non-monotonic} quantity, I propose a program analysis framework in PART III.
% This new framework will give
% a symbolic bound on the program's non-monotonic quantitative property more accurately and efficiently
% than existing works.
% }
% analysis via
% the generalization of the \emph{adaptivity} analysis framework.
%
% plan to finish the design and implementation in the thesis.
% In the end, 
% I propose an interesting further work on solving the 
% CFL-Reachability problem by reducing it into my \emph{adaptivity} analysis framework, 
% based on observing the similarities between them.
 % onto general program's resource cost analysis,
 %.