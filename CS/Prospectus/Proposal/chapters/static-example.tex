My static analysis provides an upper bound on the adaptivity for this example as follows.
{\THESYSTEM} constructs a program-based dependency graph, for my example we show this graph in 
Figure~\ref{fig:twoRounds_example}(c).
The edges of this graph are built by considering both control flow and data flow between assigned variables (the algorithm is presented in Section~\ref{subsubsec:static-datadep}). 
The weight of every vertex is estimated by using a reachability-bound estimation algorithm 
(presented in Section~\ref{subsubsec:static-reachability}), which can be symbolic and provide a sound upper bound on the weight of the corresponding vertex in the execution-based dependency graph. 
For instance, the weight $k$ of the vertex $x^{3}$ in Figure~\ref{fig:twoRounds_example}(c) is a sound upper bound on the weight $w_k$ of vertex $x^{3}$ in Figure~\ref{fig:twoRounds_example}(b), with the same starting trace. 
The soundness of this step is proved in Theorem~\ref{thm:addweight_soundness}.

$\THESYSTEM$ search a walk on this graph which over-approximate the adaptivity of the program (this is done by an algorithm
$\pathsearch$ presented in  Section~\ref{subsubsec:static-adapt}). 
%  finds a finite walk on this graph.
% This finite walk traverses the maximum times of query variables, 
% and the visiting time of every vertex on this walk is restricted by its weight.
% The maximum number of vertices visited on this walk which correspond to query variables, is the final estimated adaptivity upper bound, for the program.
For instance, in Figure~\ref{fig:twoRounds_example}(c), $\pathsearch$ first finds a path $l^6:{}^1_1 \to a^5: {}^k_1 \to x^3: {}^k_1$, and then approximate a walk with this path.
Every vertex on this walk is visited once, and the number of vertices with query annotation $1$ traversed in this path is $2$, which is the upper bound we expect.
It is worth to note here that even though the node $x^3$ has weight depending on $k$, 
it is only visited once, similarly for $l^6$, hence the overall upper bound on the adaptivity is 2, as we expect.