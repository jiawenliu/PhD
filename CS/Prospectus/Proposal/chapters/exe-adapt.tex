Given 
a program $c$'s execution-based dependency graph 
% $G_{trace}(c)(\trace) = (\vertxs, \edges, \weights, \qflag)$,
$\traceG({c})$,
we define adaptivity 
with respect to an initial trace $\trace_0 \in \mathcal{T}_0(c)$ by the finite walk in the graph, which has the most query requests along the walk.
We show the definition of a finite walk as follows.
%
% The query length of a walk $k$ is the number of vertices which correspond to query variables in the vertices sequence of this walk. 
% Instead of counting all 
% the vertices in $k$'s vertices sequence, i

\begin{defn}[Finite Walk (k)].
  \label{def:finitewalk}
  \\
%   Given a program $c$'s execution-based dependency graph $\traceG({c})(\trace)$, 
%   a \emph{finite walk} $fw$ in $\traceG({c})(\trace)$ is a sequence of edges $(e_1 \ldots e_{n - 1})$ 
%   for which there is a sequence of vertices $(v_1, \ldots, v_{n})$ such that:
%   \begin{itemize}
%       \item $e_i = (v_{i},v_{i + 1})$ for every $1 \leq i < n$.
%       \item every vertex $v \in \traceV({c}) $ appears in $(v_1, \ldots, v_{n})$ at most 
%       \wq{$\traceW({c})(\trace)$} times.  
%   \end{itemize}
%   %
%   The length of $fw$ is the number of vertices in its vertex sequence, i.e., $\len(k) = n$.
  Given the execution-based dependency graph $\traceG({c}) = (\traceV({c}), \traceE({c}), \traceW({c}), \traceF({c}))$ of a program $c$,
  a \emph{finite walk} $k$ in $\traceG({c})$ is a 
  function $k: \mathcal{T} \to $ sequence of edges.
  For a initial trace $\trace_0 \in \mathcal{T}_0(c)$, 
  $k(\trace_0)$ is a sequence of edges $(e_1 \ldots e_{n - 1})$ 
  for which there is a sequence of vertices 
  $(v_1, \ldots, v_{n})$ such that:
  \begin{itemize}
      \item $e_i = (v_{i},v_{i + 1}) \in \traceE(c)$ for every $1 \leq i < n$.
      \item every $v_i \in \traceV(c)$
      and $(v_i, w_i) \in \traceW(c)$, 
       $v_i$ appears in $(v_1, \ldots, v_{n})$ at most 
    %   \wq{$\traceW({c})(\trace)$} 
    $w(\trace_0)$
      times.  
  \end{itemize}
  %
  The length of $k(\trace_0)$ is the number of vertices in its vertices sequence, i.e., $\len(k)(\trace_0) = n$.
 \end{defn}

We use $\walks(\traceG(c))$ to denote 
% \mg{``the set'', not ``a set''}a set containing all finite walks $k$ in $G$;
the set containing all finite walks $k$ in $\traceG(c)$;
and $k_{v_1 \to v_2} \in \walks(\traceG(c))$ with $v_1, v_2 \in \traceV(c)$ denotes the walk from vertex $v_1$ to $v_2$ . 
\\
We are interested in queries, so we need to recover the 
variables corresponding to queries from the walk. We define the query length of a walk, 
instead of counting all 
the vertices in $k$'s vertices sequence, we just count the number of vertices which correspond to query variables in this sequence.
%
% \mg{I don't understand this definition. Is wrt a single query?if yes, who is chosing the query? Or is it any query?}
% \jl{It is for any query, as long as the vertex is a query variable, in another worlds, this length just counting the number of query variables in the walk, instead of counting all 
% the vertices.}
% \todo{Make the definition clear}
\begin{defn}[Query Length of the Finite Walk($\qlen$)].
\label{def:qlen}
\\
% Given 
% % labelled weighted graph $G = (\vertxs, \edges, \weights, \qflag)$, 
% a program $c$'s execution-based dependency graph $\traceG(c)(\trace)$
%  and a \emph{finite walk} $k$ in $\traceG(c)(\trace)$ with its vertex sequence $(v_1, \ldots, v_{n})$, 
% %  the length of $k$ w.r.t query is defined as:
% The query length of $k$ is the number of vertices which correspond to query variables in $(v_1, \ldots, v_{n})$ as follows, 
% \[
%   \qlen(k) = \len\big( v \mid v \in (v_1, \ldots, v_{n}) \land \qflag(v) = 1 \big)
% \]
% , where $\big(v \mid v \in (v_1, \ldots, v_{n}) \land \qflag(v) = 1 \big)$ is a subsequence of $(v_1, \ldots, v_{n})$.
Given 
% labelled weighted graph $G = (\vertxs, \edges, \weights, \qflag)$, 
the execution-based dependency graph $\traceG({c}) = (\traceV({c}), \traceE({c}), \traceW({c}), \traceF({c}))$ of a program $c$,
 and a \emph{finite walk} 
%  $k$ in $\traceG(c)(\trace)$
 $k \in \walks(\traceG(c))$. 
%  with its vertex sequence $(v_1, \ldots, v_{n})$, 
%  the length of $k$ w.r.t query is defined as:
The query length of $k$ is a function $\qlen(k): \mathcal{T} \to \mathbb{N}$, such that with an initial trace  $\trace_0 \in \mathcal{T}_0(c)$, $\qlen(k)(\trace_0)$ is
the number of vertices which correspond to query variables in the vertices sequence of the walk $k(\trace_0)$
$(v_1, \ldots, v_{n})$ as follows, 
\[
  \qlen(k)(\trace_0) = |\big( v \mid v \in (v_1, \ldots, v_{n}) \land \qflag(v) = 1 \big)|.
\]
% , where $\trace_0 \in \mathcal{T}$ is the initial trace and $\big(v \mid v \in (v_1, \ldots, v_{n}) \land \qflag(v) = 1 \big)$ is a subsequence of $(v_1, \ldots, v_{n})$.
%  $k$'s vertex sequence.
% \mg{If I understand where you want to go, why don't you just use the cardinality of the set above, rather than taking the length of a subsequence?}
% \jl{because the same vertex could have multiple occurrence in the sequence, and we will count all the occurrence instead of just once.
% So the cardinality of set doesn't work.}
\end{defn}
%
%
% \subsection{SSA Transformation and Soundness of Transformation}
% in File {\tt ``ssa\_transform\_sound.tex''}
% \input{ssa_transform_sound}
% %
% %
% The following lemma describes a property of the trace-based dependency graph.
% For any program $c$ with a database $D$ and a initial trace $\trace$,
% the directed edges in its trace-based dependency graph can only be constructed from variable with  
% smaller labels variables of greater ones.
% There doesn't exist backward edges with direction from greater labelled variables to smaller ones.
% \begin{lem}
% \label{lem:edgeforwarding}
% [Edges are Forwarding Only].
% \\
% %
% %
% $$
% \forall \trace \in \mathcal{T}, D \in \dbdom \st G(c, D) =  (\vertxs, \edges, \weights, \qflag) 
% \implies
% \forall (\event', \event) \in \edges \st \event' \eventleq \event
% $$
% %
% \end{lem}
% %
% \begin{proof}
% Proof in File: {\tt ``edge\_forward.tex''}.
% % \begin{proof}
This lemma is proved by showing there is a contradiction. 

\jl{
Assume there exists an edge  $(\av', \av) \in \edges$ and $\av' \avgeq \av$, where $\av' = ({\qval}',l',w')$ and $\av = ({\qval},l,w)$.
%
According to the Definition~\ref{def:trace-based_graph}, we have:
%
$$
DEP(\av', \av, c, m, D) ~ (1)
$$
%
%
Unfolding the Definition \ref{def:query_dep} in $(1)$,
we know: there exists $t_1, t_3, m_1, m_3, c_2$ s.t.,
%
\[
\config{m, c, [], []} \rightarrow^{*} 
\config{m_1, [\assign{x}{\query({\qval'})}]^{l'} ; c_2,
  t_1, w'} 
\rightarrow^{\textbf{query-v}} 
\config{m_1[v_1/x], c_2,
(t_1 ++ [\av'], w_1} \rightarrow^{*} \config{m_3, \eskip, t_3,w_3} ~ (a)  
\]
%
and
%
\[
 \bigwedge
 \begin{array}{l}   
  % 
  \left( 
  \begin{array}{l}
  \av \avin (t_3 - (t_1 ++ [\av'])) 
  % 
  \\
  \implies 
  \exists v \in \qdom, v \neq v_1, m_3', t_3', w_3'. ~  
  \config{m_1[v/x], {c_2}, t_1 ++ [\av'], w_1} 
  \\ 
  \quad \quad 
  \rightarrow^{*}
  (\config{m_3', \eskip, t_3', w_3'} 
  \land 
  \av \not \avin (t_3'-(t_1 ++ [\av'])))
\end{array} \right ) ~(b)
\\
\left( 
  \begin{array}{l}
  \av \not\avin (t_3 - (t_1 ++ [\av']))
    % 
    \\
    \implies 
  \exists v \in \qdom, v \neq v_1, m_3', t_3', w_3'. 
  \config{m_1[v/x], {c_2}, t_1 ++ [\av'], w_1}
  \\ 
  \quad \quad 
  \rightarrow^{*} 
  (\config{m_3', \eskip, t_3', w_3'} 
  \land 
  \av  \avin (t_3' - (t_1 ++ [\av'])))
\end{array} \right ) ~ (c)
\end{array}
\]
%
%
According to the Theorem \ref{thm:os_wf_trace} and $(a)$, we know both $t_1$, $t_3$ are well-formed traces.
% }
% \\
% \jl{
\\
Consider 2 cases:
 \[\av \avin (t_3 - (t_1 ++[\av'])) ~(d) ~~ \lor ~~ \av \notin_{aq} (t_3 - (t_1 ++[\av'])) ~(e)\]
%
\begin{itemize}
%
  \caseL{ (d) \[\av \avin (t_3 - (t_1 ++[\av'])) ~ (d)\]}
  By unfolding the trace subtraction operations, we have;
  \[
    \exists t_2. ~ s.t., ~ t_1 ++[\av'] ++ t_2 = t_3 \land \av \avin t_2 ~ (3)
  \]
%
%
According to the Corollary~\ref{coro:aqintrace} and $\av \avin t_2$, we have:
%
\[
  \exists t_{21}, t_{22}, \av' ~ s.t.,~ (\av \aveq \av') \land t_{21} ++ [\av'] ++ t_{22} = t_2 ~ (4)
\]
%
By rewriting $(4)$ inside $(3)$, we have:
%
\[
  \exists t_{21}, t_{22}, \av'' ~ s.t.,~ (\av \aveq \av'') \land t_1 ++[\av'] ++ t_{21} ++ [\av''] ++ t_{22} = t_3 ~ \star
\]
%
By the \emph{ordering} property in definition \ref{def:wf_trace} and $(\star)$, we know
%
\[
  \av' <_{aq} \av
\]
%
This is contradict to our assumption, where $\av' \avgeq \av$. 
%
%
\caseL{(e), \[\av \notin_{aq} (t_3 - (t_1 ++[\av'])) ~(e)\]} 
%
According to the condition $(c)$, we know: $\exists v \in \qdom, v \neq v_1, m_3', t_3', w_3'.$
    % 
\[ 
  \config{m_1[v/x], {c_2}, t_1 ++ [\av'], w_1} 
  \rightarrow^{*} (\config{m_3', \eskip, t_3', w_3'} ~ (5)
  \land \av  \in_{q} (t_3' - (t_1 ++ [\av']))) ~ (6)
\] 
%
According to the Theorem \ref{thm:os_wf_trace}, Sub-Lemma-t and $(5)$,
we know $t_3'- (t_1 ++ [\av'])$ is a well-formed trace.
\\
From $(a)$, we know that the minimal line number of $c_2$ is greater than $l'$, so we know that : 
\[
  \forall \av'' \in t_3'- (t_1 ++ [\av']), \av''>_{aq} \av'
\]
%
Unfolding the trace subtraction operations in $(6)$, we have;
\[
  \exists t_2'. ~ s.t., ~ t_1 ++[\av'] ++ t_2' = t_3' \land \av \avin t_2' ~ (7)
\]
%
%
According to the sub-lemma and $(7)$, we have:
%
\[
  \exists t_{21}', t_{22}' ~ s.t.,~ t_{21}' ++ [\av] ++ t_{22}' = t_2' ~ (8)
\]
%
By rewriting $(8)$ inside $(7)$, we have:
%
\[
  t_1 ++[\av'] ++ t_{21}' ++ [\av] ++ t_{22}' = t_3' ~ \diamond
\]
%
By the \emph{ordering} property in definition \ref{def:wf_trace} and $(\diamond)$, we know
%
\[
  \av' <_{aq} \av
\]
%
This is contradict to the assumption,  $\av' \avgeq \av$.
%
\end{itemize}
%
From both cases, we derive $\av' \avgeq \av$, which is contradict to the hypothesis, i.e., $\av' \avgeq \av$.
Then, we can conclude that 
for any directed edge $(\av', \av) \in \edges$, 
this is not the case that:
%
$$\av' \avgeq \av$$
%
}
\end{proof}
%
%
%
% \end{proof}
%
%
% %
% \begin{lem}
% \label{lem:DAG}
% [Trace-based Dependency Graph is Directed Acyclic].
% \\
% %
% $\forall \trace \in \mathcal{T}, D \in \dbdom $, $G(c, D)$ is a directed acyclic graph.
% \end{lem}
%
