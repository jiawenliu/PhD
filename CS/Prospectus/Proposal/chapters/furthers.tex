\subsection{Accurate Execution-Based Dependency Depth Analysis}
\label{subsec:furthers-dep-depth}
%
The program's adaptivity in the formal model through the execution-based analysis,
% which we define over the program's execution-based dependency graph from the dynamic 
% analysis 
in Definition~\ref{def:trace_adapt}, isn't precise enough w.r.t. the intuitive adaptivity rounds.
It comes across an over-approximation 
% on the program's
%  intuitive adaptivity rounds.
% It is 
resulted from difference between its Dependency Depth analysis and the \emph{variable may-dependency} definition.
It occurs when the weight is computed over the traces different from the traces used in 
witness the \emph{variable may-dependency} relation.
As shown in the Example~\ref{ex:overdefined_adapt}.
\input{examples/multipleRoundsSingle}
%%
\subsubsection{Proposed Methodology}
\label{subsubsec:furthers-dep-depth}
% In terms of techniques, our work relies on ideas from both static analysis and dynamic analysis. 
We discuss closely related work in both areas.


1. In the first stage of the execution-based analysis, 
I will give an alternative variable \emph{may-dependency} definition 
by referring to the analysis methodology in \cite{Cousot19a}.
\\
Specifically, I will define the variables dependency relation over two witness traces and an initial trace. Comparing to 
the existing \emph{may-dependency} definition, which quantifies overall possible execution traces, the alternative
definition explicitly relies on two specific witness traces from execution.
This externalization helps in analyzing the dependency quantity through the same 
witness traces as the \emph{may-dependency} relation. In this way, the over-approximation as illustrated above
can be reduced.
\\
2. In the second stage of the execution-based analysis, 
based on the new \emph{may-dependency} definition,
I will compute the weight of every edge constructed from 
\emph{may-dependency} relation in the execution-based dependency graph, w.r.t. to the witness traces.
\\
3. Then in the third stage, I will formalize the \emph{adaptivity} still as the 
length of the longest finite walk. Differently from the previous one, I restrict 
the occurrence of every edge in a finite walk no more than its weight as well.
\\
Through the three steps above, I give a more accurate formalization of the intuitive \emph{adaptivity}.
%
\subsection{Static Path Sensitive Reachability Bound Analysis}
\label{subsec:furthers-reachability}
In static program analysis framework $\THESYSTEM$, specifically on the dependency quantity, 
I adopt the reachability bound analysis technique to estimate this dependency quantity.
% In existing static reachability bound analysis, 
However, it isn't precise enough w.r.t. the execution-based reachability bound on every program command.
It comes across an over-approximation on the estimation due to its path-insensitive nature. 
It occurs when the control flow can be decided in a particular way in front of conditional branches, 
while the static analysis fails to witness. 
As shown in Example~\ref{ex:overapproximate}.
\input{examples/multipleRoundsOdd}

% as follo
\subsubsection{Proposed Methodology}
\label{subsubsec:furthers-reachability}
Given the imprecision comes from the second stage of the static program analysis,
I will insist on the same $\THESYSTEM$ framework,
and design new algorithm for this stage.
In this stage, 
methodology on reachability bound analysis isn't path sensitive. 
I will design a path sensitive reachability bound analysis algorithm computing the 
reachability bounds for every labeled command taking the different paths inside while loop into consideration.
% Comparing to just compute the reachability bound for the while loop command, new methodology improves the accuracy of the 
% reachability bound for every labeled command.
\\
Then, I will use this improved analysis result to estimate the dependency quantity.
%
% \subsection{Static Adaptivity Computation towards Completeness}
% \label{subsec:furthers-adaptcomplete}
% The Algorithm is conditional completeness as proved in appendix, but Algorithm~\ref{alg:adaptscc} isn't.
% In the algorithm design at line: in Algorithm~\ref{alg:adaptscc}, an over-approximation happens here. 

% As  following motivating example shows.
% \subsubsection{Proposed Methodology}
% \label{subsubsec:furthers-adaptcomplete-methodology}
% %
% 1. looking into more over-approximated example and summarize the common properties of these examples.
% \\
% 2. Modify the Algorithm~\ref{alg:adaptscc}, targeting the line: 12 of the algorithm. 
% The goal is to reduce the over-approximation in computing adaptivity statically.
