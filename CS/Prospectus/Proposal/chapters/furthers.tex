\subsection{{\tt Query While} Language Extension}
\label{subsec:furthers-language}
The {\tt Query While} Language presented in Section~\ref*{sec:language} is limited to 
standard while language with only $\eif$ and $\ewhile$ commands and basic assignment commands.
I plan to extend it with inter-procedure call in following two steps,
\begin{itemize}
    \item I plan to firstly extend the syntax with  function definition and function call commands, sketching as follows,
\[
\highlight
{\mbox{Labeled Command} 
c ::= 
\clabel{\efun}^l: x(r, x_1, \ldots, x_n) := c
~|~ \clabel{\assign{x}{\ecall(x, e_1, \ldots, e_n)}}^l
}
\]
    \item Accordingly, I will extend the operational semantics with function definition and call rules.
\end{itemize}

\subsection{Execution-Based Adaptivity Analysis Extension}
\label{subsec:furthers-dep-depth}
%
The program's adaptivity in the formal model through the execution-based analysis in Section~\ref{sec:dynamic}
% which we define over the program's execution-based dependency graph from the dynamic 
% analysis 
(specifically in Definition~\ref{def:trace_adapt}), isn't precise enough w.r.t. the intuitive adaptivity rounds.
It comes across as an over-approximation 
% on the program's
% intuitive adaptivity rounds.
as shown in the Example~\ref{ex:multipleRoundsSingle_example}.
% It is 
In this example, the formalized adaptivity by execution-based analysis in Section~\ref{sec:dynamic} 
is the initial value of the input variable $k$.
However, the intuitive \emph{adaptivity} is only $2$ given whatever initial trace.
This is resulted from the
%  difference 
disconnection between the 
% dependency Depth 
dependency quantity analysis in Section~\ref{subsubsec:dynamic-reachability} and 
the data dependency analysis in \emph{variable may-dependency} definition in Section~\ref{subsubsec:dynamic-datadep}.
It occurs when the 
% weight 
dependency quantity is computed over the traces different from the traces used in 
% witness 
analyzing the \emph{variable may-dependency} relation.
% As shown in the Example~\ref{ex:multipleRoundsSingle_example}.
\input{examples/multipleRoundsSingle}
%%
\subsubsection{Proposed Methodology}
\label{subsubsec:furthers-dep-depth}
% In terms of techniques, our work relies on ideas from both static analysis and dynamic analysis. 
We discuss closely related work in both areas.


Based on the execution-based dependency analysis in Section~\ref{sec:dynamic}, the improvement on the accuracy 
of formalizing the
\emph{adaptivity} is planned to develop in following three-steps.
\begin{enumerate}
\item In the first stage of the execution-based analysis, 
I will give an alternative variable \emph{may-dependency} definition 
by referring to the analysis methodology in \cite{Cousot19a}.
%
Specifically, I will define the variables' dependency relation over two witness traces and an initial trace. Comparing to 
the existing \emph{may-dependency} definition, which quantifies overall possible execution traces, the alternative
the definition explicitly relies on two specific witness traces from execution.
This externalization helps in analyzing the dependency quantity through the same 
witness traces as the \emph{may-dependency} relation. In this way, the over-approximation as illustrated above
can be reduced.
%
\item In the second stage of the execution-based analysis, 
based on the new \emph{may-dependency} definition,
I will compute the weight of every edge constructed from 
\emph{may-dependency} relation in the execution-based dependency graph, w.r.t. to the witness traces.
%
\item Then in the third stage, I will formalize the \emph{adaptivity} still as the 
length of the longest finite walk. Differently from the previous one, I restrict 
the occurrence of every edge in a finite walk to no more than its weight as well.
\end{enumerate}
Through the three steps above, I give a more accurate formalization of the intuitive \emph{adaptivity}.
%
\subsection{Accurate Static Adaptivity Analysis}
\label{subsec:furthers-reachability}
In static program analysis framework $\THESYSTEM$, specifically on the dependency quantity, 
I adopt the reachability bound analysis technique to estimate this dependency quantity.
% In existing static reachability bound analysis, 
However, it isn't precise enough w.r.t. the execution-based reachability bound on every program command.
It comes across an over-approximation on estimation due to its path-insensitive nature,
as shown in the Example~\ref{ex:multipleRoundsOdd_example}.
% It is 
In this example, the estimated adaptivity by static analysis in Section~\ref{sec:static} 
is $1 + 2*k$ where $k$ is the initial value of the input variable.
However, the formal \emph{adaptivity} is only $1 + k$ from the execution-based analysis.

This is resulted from the  path-insensitive of the dependency quantity in
second stage of the static analysis in Section~\ref{subsubsec:static-reachability}
It occurs when the control flow can be decided in a particular way in front of conditional branches, 
while the static analysis fails to witness. 
% As shown in Example~\ref{ex:overapproximate}.
\input{examples/multipleRoundsOdd}

% as follo
\subsubsection{Proposed Methodology}
\label{subsubsec:furthers-reachability}
The accurate static adaptivity analysis is planned to be improved in following two steps.
\begin{enumerate}
    \item \textbf{Path Sensitive Reachability Bound Algorithm Design}
    
    The imprecision comes from the second stage of the static program analysis.
    The algorithm in this stage is to statically estimate the 
    reachability bound for every location of program.
    Specifically, this reachability bound analysis algorithm isn't path sensitive. 
    There are many works in the program complexity analysis area, estimating the path-sensitive loop bound 
    or reachability bound
    \cite{GustafssonEL05, HumenbergerJK18}, 
    \cite{BrockschmidtEFFG16,AlbertAGP08,AliasDFG10,Flores-MontoyaH14}, 
    \cite{GulwaniZ10, SinnZV17,GulwaniJK09, GulwaniMC09, abs-2203-04243}. 
    But they aren't analyzing the reachability
    bound in a path-sensitive way.
    \\
    Motivated by this observation, I will first design path-sensitive reachability bound analysis algorithm computing the 
    reachability bounds for every labeled command taking the different paths inside while loop into consideration.
% Compared to just computing the reachability bound for the while loop command, the new methodology improves the accuracy of the 
% reachability bound for every labeled command.
% \\
    \item \textbf{$\THESYSTEM$ via Path Sensitive Reachability Bound Algorithm Design}

    Then follow the same static program analysis based $\THESYSTEM$ framework in Section~\ref{sec:static},
% and design a new algorithm for this stage.
% Then, I will 
this improved analysis result will be applied to estimate the dependency quantity as follows,
\begin{itemize}
    \item $\THESYSTEM$ will construct different abstract control flow graph based on the requirement of the newly-designed
    path-sensitive reachability bound algorithm.

    \item The improved $\THESYSTEM$ will perform the same algorithm for analyzing the dependency relation in the first stage.
    \item But in the second stage, the improved $\THESYSTEM$ 
    apply the newly-designed  path-sensitive reachability bound algorithm for analyzing the dependency quantity.
    The result from the new algorithm 
    give a tighter approximation on the dependency quantity,
    specifically on the reachability times of the program's every location.
    \item Based on the two results analyzed from the steps above, 
    i.e., the estimated dependency relation and the tighter reachability bound, the improved $\THESYSTEM$ will construct
a more accurate program-based data dependency graph.
%  constructed in Section~\ref{subsubsec:static-adapt}
Then, by applying the same algorithm as in Section~\ref{subsubsec:static-adapt}, 
the improved $\THESYSTEM$ will compute a tighter \emph{adaptivity} for a program.%
\end{itemize}
\end{enumerate}
As shown in Figure~\ref{fig:multipleRoundsOdd_example}, 
through the newly-designed path-sensitive reachability bound analysis,
ideally the weight for vertex $p^6$ and $x^7$ will be $\frac{k}{2}$.
% \subsection{Static Adaptivity Computation towards Completeness}
% \label{subsec:furthers-adaptcomplete}