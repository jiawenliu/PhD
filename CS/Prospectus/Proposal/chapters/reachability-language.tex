The language is model is defined in the same way as Section~\ref{sec:adapt-language} by removing the query extension, summarized as follows.

\subsubsection{Syntax}
The selected syntax is shown as follows,
\[
\begin{array}{llll}
% \\ 
%
\mbox{Arithmetic Expression} 
& \aexpr & ::= & 
n ~|~ {x} ~|~ \aexpr \oplus_a \aexpr 
~|~ \elog \aexpr ~|~ \esign \aexpr ~|~ \max(a, a) ~|~ \min(a, a)
\\
%
\mbox{Boolean Expression} & \bexpr & ::= & 
%
\etrue ~|~ \efalse ~|~ \neg \bexpr
 ~|~ \bexpr \oplus_b \bexpr
%
~|~ \aexpr \sim \aexpr 
\\
%
\mbox{Expression} & \expr & ::= & v ~|~ \aexpr ~|~ \bexpr ~|~ [\expr, \dots, \expr]
\\ 
%
\mbox{Value} 
& v & ::= & { n ~|~ \etrue ~|~ \efalse ~|~ [] ~|~ [v, \dots, v]} 
\\
%
\mbox{Label} 
& l & \in & \mathbb{N} \cup \{\lin, \lex\} \\
\mbox{Labeled Command} 
& {c} & ::= & [\assign {{x}}{ {\expr}}]^{l}
~|~ {\ewhile [ \bexpr ]^{l} \edo {c} }
 \\
 &&&
~|~ {c};{c} 
~|~ \eif([\bexpr]{}^l , {c}, {c}) 
~|~ [\eskip]^l 
% \\
% %
\end{array}
\]
\subsubsection{Trace-based Operational Semantics}
\paragraph{Event}
%   Its first element is either 
% an assigned variable (from an assignment command) or a boolean expression (from the guard of if or while command), follows by 
%  the label associated with this event, the value evaluated either from the expression assigned to the variable,
% or the boolean expression in the guard.
%  The last element stores the query information, which is a query value whose default is $\bullet$. I declare event projection operator $\pi_i$ which projects the $i$th element from an event.
\[
\begin{array}{llll}
\mbox{Event} 
& \event & ::= & 
\mbox{Assignment Event}  ({x}, l, v) 
~|~ \mbox{Testing Event}(\bexpr, l, v) 
% \mbox{Trace} & \trace
% & ::= & [] ~|~ \trace :: \event
\end{array}
\]
\paragraph*{Trace}
The trace $\trace \in \mathcal{T} $ is a list of events same as Section~\ref{sec:adapt-language}.
\[
\begin{array}{llll}
\mbox{Trace} & \trace
& ::= & [] ~|~ \trace :: \event
\end{array}
\]
\paragraph*{Environment}
The function $\env : {\mathcal{T}} \to \mathcal{VAR} \to \mathcal{VAL} \cup \{\bot\}$, which maps a trace and a variable to the latest value assigned to this variable on the trace is defined as follows.
%
\paragraph{Operational Semantics}
The operational semantics rules are also defined in the same way as Section~\ref{sec:adapt-language} by removing the query rules.
A selection of rules of the trace-based operational semantics is presented in Figure~\ref{fig:reachability-os}. 
%
\begin{figure}
 \begin{mathpar}
 \boxed{
 \mbox{Command $\times$ Trace}
 \xrightarrow{}
 \mbox{Command $\times$ Trace}
 }
 \and
 \boxed{\config{{c, \trace}}
 \xrightarrow{} 
 \config{{c', \trace'}}
 }
 \\
 % \inferrule
 % {
 % \empty
 % }
 % {
 % \config{\clabel{\eskip}^l, \trace } 
 % \xrightarrow{} 
 % \config{\clabel{\eskip}^l, \trace}
 % }
 % ~\textbf{skip}
 %
 % \and
 %
 \inferrule
 {
 \config{\trace, \expr} \earrow v 
 \and
 \event = ({x}, l, v)
 }
 {
 \config{[\assign{{x}}{\expr}]^{l}, \trace } 
 \xrightarrow{} 
 \config{\clabel{\eskip}^l, \trace \traceadd \event}
 }
 ~\textbf{assn}
 %
 \and
 %
 \inferrule
 {
\config{ \trace, b} \barrow \etrue
 \and 
 \event = (b, l, \etrue)
 }
 {
 \config{{\ewhile [b]^{l} \edo c, \trace}}
 \xrightarrow{} 
 \config{{
 c; \ewhile [b]^{l} \edo c),
 \trace \traceadd \event}}
 }
 ~\textbf{while-t}
 %
 %
 \quad
 %
 \inferrule
 {
 \config{\trace, b} \barrow \efalse
 \and 
 \event = (b, l, \efalse)
 }
 {
 \config{{\ewhile [b]^{l}, \edo c, \trace}}
 \xrightarrow{} 
 \config{{
 \clabel{\eskip}^l,
 \trace \traceadd \event}}
 }
 ~\textbf{while-f}
 %
 %
 \and
 %
 %
 \inferrule
 {
 \config{{c_1, \trace}}
 \xrightarrow{}
 \config{{\clabel{\eskip}^l, \trace'}}
 \and 
 \config{{\clabel{\eskip}^l; c_2, \trace'}} \xrightarrow{} \config{{ \clabel{\eskip}^l, \trace''}}
 }
 {
 \config{{c_1; c_2, \trace}} 
 \xrightarrow{} 
 \config{{\clabel{\eskip}^l, \trace''}}
 }
 ~\textbf{seq}
 %
 % \and
 % %
 % \inferrule
 % {
 % \config{{c_2, \trace}}
 % \xrightarrow{}
 % \config{{c_2', \trace'}}
 % }
 % {
 % \config{{\clabel{\eskip}^l; c_2, \trace}} \xrightarrow{} \config{{ c_2', \trace'}}
 % }
 % ~\textbf{seq2}
 %
 \quad
 %
 %
 \inferrule
 {
 \trace, b \barrow \etrue
 \and 
 \event = (b, l, \etrue)
 }
 {
 \config{{
 \eif([b]^{l}, c_1, c_2), 
 \trace}}
 \xrightarrow{} 
 \config{{c_1, \trace \traceadd \event}}
 }
 ~\textbf{if-t}
 %
 \end{mathpar}
 % \end{subfigure}
 \vspace{-0.5cm}
 \caption{Trace-based Operational Semantics Rules for Reachability-Bound Analysis.}
 \label{fig:reachability-os}
 \end{figure}
 %