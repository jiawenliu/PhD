\subsection{Introduction and Related Work}
\label{subsec:cfl-backgroung}
Finally, based on the study on traditional way of performing data flow and control analysis,
   I identify the similarity between the traditional way of performing data flow and control analysis, and the 
   adaptivity analysis.  
   Specifically I identify the similarity between 
   solving the feasible path problem in the analysis by reducing to CFL-reachability problems,
   and the way of computing the adaptivity in my static analysis framework.
   Motivated by this observation, 
   % I'm insterested
   % the, There are similarity between
   % solving the data flow problem by reducing to CFL-reachability problem,
   % resource analysis through reducing to CFL-reachability problem, 
   I'm interested in showing that
   CFL-reachability problems can be solved by reducing it into my adaptivity analysis framework.

\subsection{Proposed Methodology}
\label{subsec:cfl-methodology}
Based on the paper\cite{Reps98} where Thomas shows 
four program analysis problems 
which can be solved by reducing to CFL-reachability problem, I will follow the same idea and show the reduction
in following steps.
\begin{enumerate}
   \item For the same four program analysis problems, each of them 
   can be solved through  my \emph{adaptivity} analysis framework with 
   a different generalization of $\THESYSTEM$ with higher accuracy.
   \item Based on summarizing the common property of the four program analysis problems,
   I will show a reduction from CFL-reachability problem 
   by showing that CFL-reachability problem is a special case of 
   computing the adaptivity. 
\end{enumerate}
%  system structure as $\THESYSTEM$,
% by modifying the restriction on finite walk, compute different resource cost for program.

