
Then based on the analysis results in Section~\ref{subsubsec:static-datadep} and 
Section~\ref{subsubsec:static-reachability},
specifically the estimated $\progE(c)$ and $\progW(c)$,
the program-based dependency graph is constructed as follows:
 %
{
   \[
     \progG(c) = (\progV(c), \progE(c), \progW(c), \progF(c))
     \]
 }
 with $\progV(c), \progE(c), \progW(c)$ and $ \progF(c)$ as computed in each steps above.
 % and $\progF(c) =\left\{(x^l, n)  \in  \mathcal{LV} \times \{0, 1\} 
 % ~ \middle\vert ~
 % x^l \in \lvar_{c},
 % n = 1 \iff x^l \in \qvar_{c} \land n = 0 \iff  x^l \in \qvar_{c} .
 % \right\} $,
 % This program-based graph program-based graph has a similar topology structure as 
 % % the one
 % % of 
 % the execution-based dependency graph. It has the same
 % vertices and query annotations, but approximated edges and weights.  
 % The algorithm computation is 
 Its formal definition can be found in the appendix.
 We compute the adaptivity upper bound for a program $c$ by the maximum query length over all finite walks in the program-based data dependency graph $\progG({c})$, defined  formally in Definition~\ref{def:prog_adapt}. 
 %
 We use $\walks(\progG(c))$ to represent the walks over the program-based dependency graph for $c$.
 Different from the walks on $\traceG(c)$, $k \in \walks(\progG(c))$ doesn't rely on the initial trace.
 The occurrence time of every $v_i $ in $k$'s vertices sequence is bound by 
 an arithmetic expression $w_i$ where $(v_i, w_i) \in \progW(c)$ is $v_i$'s estimated weight. Then we have $\qlen(k) \in \mathcal{A}_{in}$ as well. The full definition for $\walks(\progG(c))$ and $\qlen$ over $\walks(\progG(c))$ is in the appendix.
 %
 % The adaptivity upper bound 
 % is estimated as
 % Then the adaptivity bound based on program analysis for ${c}$ 
 % is the number of query vertices on a finite walk in $\progG({c})$. This finite walk satisfies:
 % \begin{itemize}
 % \item the number of query vertices on this walk is maximum
 % \item the visiting times of each vertex $v$ on this walk is bound by its reachability bound $\weights(v)$.
 % \end{itemize}
 
 % is computed as the maximum query length over all finite walks in $\walks(\progG({c}))$, and computed 
 
 %
 % It is formally defined in \ref{def:prog_adapt}.
 % defined formally as follows.
 %
 %
 \begin{defn}
 [{Program-Based Adaptivity}].
 \label{def:prog_adapt}
 \\
 {
 Given a program ${c}$ and its program-based graph 
 $\progG({c})$
 %  = (\vertxs, \edges, \weights, \qflag)$,
 %
 the program-based adaptivity for $c$ is 
 % a function $\progA({c}): \mathcal{T} \to\mathbb{N} $,
 % for an initial trace $\trace_0 \in \mathcal{T}$,
 defined as%
 \[
 \progA({c})
 \triangleq \max
 \left\{ \qlen(k) \ \mid \  k\in \walks(\progG({c}))\right \}.
 \]
 }
 \end{defn}
 Based on my soundness of the program-based adaptivity, my program-based adaptivity is a sound upper bound of its adaptivity in Definition~\ref{def:trace_adapt}. 
 \begin{thm}[Soundness of $\progA(c)$]
     \label{thm:sound_progadapt}
     For every program $c$, 
     % for any initial trace $\trace_0$, 
     its program-based adaptivity is a sound upper bound of its adaptivity.
      $$ \forall \trace_0 \in \mathcal{T}_{0}(c) \st 
 \config{\progA(c), \trace_0} \earrow n \implies n \geq A(c)(\trace_0) $$
 \end{thm}
 % By $\progA(c) \geq A(c)$, 
 % % between a function and an arithmetic expression,
 % we bound over all trace, $\forall \trace \in \mathcal{T} \st 
 % \config{\progA(c), \trace} \earrow n \implies n \geq A(c)(\trace) $.
 To estimate a sound and precise upper bound on adaptivity, we develop an adaptivity estimation algorithm called $\pathsearch$ (in the appendix Alg.~I), which uses both the deep first search and breath first search strategy to find the walk. We also show that the estimated adaptivity from my $\pathsearch$ is sound with respect to the program-based adaptivity. 
 \begin{thm}[Soundness of $\pathsearch$]
     \label{thm:sound_adaptalg}
     For every program $c$.
     % for any initial trace $\trace_0$,
      $$\pathsearch(\progG({c})) \geq \progA(c).$$
 \end{thm}
 The full proofs and details of the algorithm and the soundness can be found in the appendix.
 
% The key novelty of AdaptSearch is its walk finding part, we first discuss two challenges when we try to find the walks in the program-based dependency graph, and show that how we solve them using my algorithms.

% \textbf{Non-Termination Challenge:}
%  One naive walk finding method is to simply traverse on this program-based dependency graph by decreasing the weight of every node by one after every visit. However, this simple 
%  traversing strategy leads to a non-termination dilemma for most programs we are interested in because the weight can be a symbolic expression. 
%  Look at the simple example in Figure~\ref{fig:alg_adaptsearch_simplewhile}, where $k$ is the input variable from domain $\mathbb{N}$.
%  %  in Figure~\ref{} in Section~\ref{sec:overview},
%  \begin{figure}
%  \centering
%  {
%  \small
%  \begin{subfigure}{.4\textwidth}
%  \begin{centering}
%  $ 
%  \begin{array}{l}
%    \kw{whileSim(k)} \triangleq \\
%    \clabel{ \assign{j}{k} }^{0} ;
%    \clabel{ \assign{x}{\query(\chi[0])} }^{1} ; \\
%        \ewhile ~ \clabel{j > 0}^{2} ~ \edo ~ \\
%        \Big(
%         \clabel{\assign{x}{\query(\chi[x]) }}^{3}  ; 
%        \clabel{\assign{j}{j-1}}^{4}       \Big)
%    \end{array}
%  $
%  \caption{}
%  \end{centering}
%  \end{subfigure}
%  \quad
%    \begin{subfigure}{.45\textwidth}
%    \begin{centering}
%    \begin{tikzpicture}[scale=\textwidth/22cm,samples=200]
%  \draw[] (0, 7) circle (0pt) node
%  {\textbf{$x^1: {}^{1}_{1}$}};
%  \draw[] (0, 4) circle (0pt) node
%  {{ $x^3: {}^{k}_{1}$}};
%  % Counter Variables
%  \draw[] (10, 7) circle (0pt) node {{$j^2: {}^{1}_{0}$}};
%  \draw[] (10, 4) circle (0pt) node {{ $j^4: {}^{k}_{0}$}};
%  %
%  % Value Dependency Edges:
%  \draw[ ultra thick, -latex, densely dotted,] (0, 4.2)  -- (0, 6.5) ;
%  \draw[ ultra thick, -Straight Barb, densely dotted,] (2, 4.5) arc (150:-180:1);
%  \draw[ thick, -Straight Barb] (11.3, 4.7) arc (150:-150:1);
%  \draw[ thick, -latex] (10, 4.6)  -- (10, 6.5) ;
%  % Control Dependency
%  \draw[ thick,-latex] (1.5, 7)  -- (8, 7) ;
%  \draw[ thick,-latex] (1.5, 4)  -- (8, 7) ;
%  \draw[ thick,-latex] (1.5, 7)  -- (8, 4) ;
%  \draw[ thick,-latex] (1.5, 4)  -- (8, 4) ;
%  \end{tikzpicture}
%  \caption{}
%    \end{centering}
%    \end{subfigure}
%  }
%  % \end{wrapfigure}
%  % \end{equation*}
%  \vspace{-0.4cm}
%   \caption{(a) The simple while loop example (b) The program-based dependency graph generated from $\THESYSTEM$.}
%  \label{fig:alg_adaptsearch_simplewhile}
%  \vspace{-0.5cm}
%  \end{figure}
%  % Analysis Results: $ \progA(\kw{whileRec}(k)) = 1 + k$
%  %
%  If we traverse on the program-based dependency graph, and decrease the weight of $x^3$ (the weight $k$ is symbolic) by one after every visit,
%  % We can simply adopt either a deep first strategy to estimate the adaptivity as the length of the longest weight path, as 
%  % in Alg.~\ref{alg:overadp_alg}.
%  we will never terminate because we only know $k \in \mathbb{N}$.
 
%  To solve this non-termination challenge, we switch to another walk finding approach: we first find a longest path in the program-based dependency graph and then approximate the walk with the path.
%  Through a simple deep first search algorithm, we find the longest weighted path as the dotted arrow in Figure~\ref{fig:alg_adaptsearch_simplewhile},
%  $x^3: {}^k_1 \to x^1: {}^1_1 $.
%  Then, by summing up the weights on this path where the vertices have query annotation $1$, deep first search algorithm gives the adaptivity bound $1 + k$.
%  This is a tight bound for this program's adaptivity. This approach is the fundamental of $\pathsearch$.
%  \textbf{Approximation Challenge:}
%  % As in Definition~\ref{def:finitewalk}, w
%   We adopt a deep first strategy to search for the longest weighted path, and then use the path to approximate the adaptivity. We find that this gives us over-approximation to a large extend.
%   This over-approximation could result in an $\infty$ adaptivity upper bound on the program with actual adaptivity $2$.
%  Look at $\kw{twoRounds}$ in overview, 
%  we find the longest weighted path is  $x^3 : {}^{k}_{1} \to a^5 : {}^{k}_{0} \to l^6 : {}^{1}_{0}$ in Figure~\ref{fig:overview-example}(c)  with weighted query length $1 + k$.
%  If we use this path to approximate a finite walk, and weight of each vertex as
%  %  their visiting times, 
%  its visiting times,
%  we have the estimated walk $x^3 \to \cdots \to x^3 \to a^5 \to \cdots \to a^5 \to l^6$, in which $x^3$ appears $k$ times. 
%  Obviously,
%  % the with the weighted length $1 + k$. It is obviously
%  its weighted query length, $1 + k$, 
%  % which is 
%  over approximates 
%  % its 
%  the adaptivity of this example to a large extend, which is supposed to be $2$. 
%  % \wq{
%  % The reason of the over-approximation comes from the missing consideration of circle in the graph when we approximate the walk from a weighted path. 
%  % If there is a circle in the graph, we can estimate in this way, but if there is no circle in my two round example, it is an over-estimation.
%  % } 
%  The reason of the over-approximation comes from the missing consideration of the restriction on vertex's visiting times in a walk.
%  Moreover, the restriction plays different roles in cases
%  whether there is cycle in the graph.
%  This leads us to consider cycle in the program-based dependency graph in $\pathsearch$, and we develop an auxiliary algorithm to estimate the adaptivity for cycles in the graph, $\kw{\pathsearch_{scc}(G)}$ as in Alg.~\ref{alg:adaptscc}.
%  $\kw{\pathsearch_{scc}}$ takes an strong connected graph, which is an strong connected component (SCC) of $\progG(c)$
%  % representing a cycle 
%  and approximates the
%  path to a precise walk to compute the adaptivity of this subgraph. 
%  In general, $\pathsearch$ will now first find all the SCCs of $\progG(c)$ by using the Kosaraju’s algorithm\cite{sharirM81}.  
%  Then $\pathsearch$ simplifies  the input graph by replacing every strong connected components with, a vertex whose weight is the adaptivity of the SCC, calculated by $\kw{\pathsearch_{scc}(G)}$. Then $\pathsearch$ performs
%       the standard breath first search strategy to find the longest weighted path on this simplified graph and returns the estimation on adaptivity as we shown before.
%       The full details of $\pathsearch$ is in the appendix Alg.~2.
%  %
%  % \paragraph*{The Adaptivity Computation Algorithm ($\pathsearch$)}
%  % \begin{algorithm}
%  %     \caption{
%  %     {Adaptivity Computation Algorithm ($\pathsearch$)}
%  %     \label{alg:adpt_alg}
%  %     }
%  %     \begin{algorithmic}[1]
%  %     \REQUIRE $G = (\vertxs, \edges, \weights, \qflag)$ \#\{The program based dependency graph\}
%  %     % with a start vertex $s$ and destination vertex $t$ .
%  %     \STATE  {\bf {$\kw{\pathsearch(G)}$}:}  
%  %     \STATE {\bf init} 
%  %     % \\
%  %     % current node: $c$, 
%  %     \\
%  %     $q$: empty queue.
%  %     % \\
%  %     % $\kw{visited}$: List of length $|\vertxs|$, initialize with $\efalse$.
%  %     % \\
%  %     % $\kw{SSCvisited}$: List of length $|\vertxs|$, initialize with $\efalse$.
%  %     % \\ 
%  %     % $\kw{adapt_{scc}(SCC_i) = \pathsearch_{scc}(SCC_i)}$.
%  %     \\
%  %     $\kw{adapt}$ : the adaptivity of this graph initialize with $0$.
%  %     \\
%  %     \STATE Find all Strong Connected Components (SCC) in $G$: $\kw{SCC_1}, \cdots, \kw{SCC_n}, 0 \leq n \leq |\vertxs|$, 
%  %     % where $\kw{SCC_i} = (\vertxs_i, \edges_i, \weights_i, \qflag_i)$.
%  %     % and assign each vertex $x^i$ with an SCC number $\kw{SCC}(x^i)$
%  %     \STATE {\bf for} every SCC: $\kw{SCC_i}$, compute its Adaptivity $\kw{SCC_i}$:
%  %     \STATE \quad $\kw{adapt_{scc}[SCC_i] = \pathsearch_{scc}(SCC_i)}$;
%  %     \STATE {\bf for} every $\kw{SCC_i}$:
%  %     \STATE \qquad $q.append(\kw{SCC_i})$;
%  %     \STATE \qquad $\kw{adapt_{tmp}} = 0$;
%  %     \STATE \qquad {\bf while} $q$ isn't empty:
%  %     \STATE \qquad \qquad $\kw{s} = q.pop()$;  \#\{take the top SCC from head of queue\}
%  %     \STATE \qquad \qquad  $\kw{adapt_{tmp}}_0= \kw{adapt_{tmp}}$; \#\{record the adaptivity of last level\}
%  %     \STATE \qquad \qquad  $\kw{SCC_{max}}$;  \#\{record the SCC with longest walk in this level\}
%  %     % initialize cycle-adapt = 0.
%  %     \STATE \qquad \qquad {\bf for} every 
%  %     % SCC having a directed edge from $s$ of $s$: $\kw{SCC'}$:
%  %     % directed edge goes out of $\kw{s}$ and connects a 
%  %     different SCC, $\kw{s'}$ connected by $\kw{s}$ by a directed edge from $\kw{s}$:
%  %     % \STATE \qquad \qquad   cycle-adapt$ = \max($cycle-adapt, $\kw{dfs_{refine}(G, v, v)})$;
%  %     % \STATE \qquad \qquad \qquad \#\{compute the adaptivity of vertex $v$  on $\kw{SCC}(v)$, and update r[v] with the SCC-adapt\}
%  %     % \STATE \qquad \qquad \qquad $ r[v] = r[s] + \kw{dfs_{refine}(G, v, visited)})$; 
%  %     \STATE \qquad \qquad \qquad {\bf if} $(\kw{adapt_{tmp}} < \kw{adapt_{tmp}}_0 + \kw{adapt_{scc}[s']})$:
%  %     \STATE \qquad \qquad \qquad \qquad $\kw{adapt_{tmp}} = \kw{adapt_{tmp}}_0 + \kw{adapt_{scc}[s']}$; 
%  %     \STATE \qquad \qquad \qquad \qquad $\kw{SCC_{max} = s'} $; \#\{update the SCC with longest walk in this level\} 
%  %     % \STATE \qquad   $r[c] = r[c] + $cycle-adapt;
%  %     % \STATE \qquad for all unvisited vertex $v$ having directed edge from c and $! \kw{cycle}(c)$:
%  %     % \STATE \qquad \qquad $r[v] = r[c] + \flag(v)$; 
%  %     % \STATE \qquad \qquad \qquad  \#\{mark all the nodes with the same $\kw{SCC}$ number as visited\} 
%  %     % \STATE \qquad \qquad \qquad  \#\{append the unvisited vertex to the rear of the queue\}
%  %     % \STATE \qquad \qquad \qquad  \#\{mark all the nodes with the same $\kw{SCC}$ number as visited\} 
%  %     % \STATE \qquad \qquad for $v \in V$,   $\kw{visited}[s] = 1$;
%  %     \STATE \qquad \qquad \qquad $q.append(\kw{SCC_{max}})$;
%  %     \STATE \qquad $\kw{adapt} = \max(\kw{adapt}, \kw{adapt_{tmp}})$;    
%  %     \RETURN $\kw{adapt}$.
%  %     \end{algorithmic}
%  %     \end{algorithm}
%      %
%  %
%      % In Alg.~\ref{alg:adpt_alg}, 
%      % it 
%      % This algorithm first finds all the strong connected components (SCC) of $\progG(c)$ using the Kosaraju’s algorithm in line:3. 
%      % Every $\kw{SCC_1}, \cdots, \kw{SCC_n}$
%      % where $0 \leq n \leq |\vertxs|$ is a sub-graph of $\progG(c)$, where $\kw{SCC_i} = (\vertxs_i, \edges_i, \weights_i, \qflag_i)$.
%      % % where $\kw{SCC_i} = (\vertxs_i, \edges_i, \weights_i, \qflag_i)$.
%      % Then, 
%      % % we compute the adaptivity on every SCC, which is a subgraph of the $\progG(c)$, in line:4-5 by Alg.~\ref{alg:adaptscc}.
%      % it computes the adaptivity on every SCC
%      % % , which is a subgraph of the $\progG(c)$, 
%      % in line:4-5 by Alg.~\ref{alg:adaptscc}.
%      % % We guarantee the soundness of the adaptivity on SCC by Lemma~\ref{lem:sound_adaptalg_scc} with proof 
%      % in appendix.
%      % % ~\ref{apdx:adaptalg_soundness}.
%      % The $\progG(c)$ is then shrunk into an acyclic directed graph where 
%      % % vertices are all the SCCs and edges are between every SCCs with their adaptivities as weights.
%      % $\kw{SCC_1}, \cdots, \kw{SCC_n}$ are vertices with their adaptivities as weights.
%      % % , and directed edges are .
%      % For every $(v_i, v_j) \in \edges$ such that $v_1 \in \vertxs_i$, $v_j \in \vertxs_j$ and $i \neq j$,
%      % there is a edge $(s_i, s_j)$ in this shrank graph. \\ 
%      % Then, we use the standard breath first search strategy to find the longest weighted path
%      % %  w.r.t. all the SCCs and their adaptivities.
%      % on this shrank graph and return the length as adaptivity.
%      % \\
%      % We guarantee that 
%      % % this longest weighted path is a sound computation of the adaptivity on this,
%      % the length of this longest weighted path is a sound computation of the adaptivity for program $c$,
%      % % as well as 
%      % and this longest weighted path a sound computation of the finite walk having the longest query length 
%      % % on this graph, in Theorem~\ref{thm:sound_adaptalg}
%      % on $c$'s program based dependency graph, in Theorem~\ref{thm:sound_adaptalg}
%      % in appendix.
%      % ~\ref{apdx:adaptalg_soundness}.
%  %    
%  % \todo{add proof} 
%  % We also guarantee the conditional completeness of the adaptivity computation for graphs under the case that 
%  % $c$'s program-based dependency Graph $\progG(c)$ is acyclic directed
%  % % in Theorem~\ref{thm:adaptalg_pcomplete} 
%  % in appendix.
 
%  % ~\ref{apdx:adaptalg_completeness}.
%      % for every vertex which isn't on any SCC, it is easy to know that it will be visited 
%      % at most once given no edges going back to this vertex. We can know the adaptivity on the SCC 
%       %
%      % \begin{algorithm}
%      % \caption{
%      % {Longest Adaptivity Search Algorithm ($\pathsearch$)}
%      % \label{alg:adpt_alg}
%      % }
%      % \begin{algorithmic}
%      % \REQUIRE Weighted Directed Graph $G = (\vertxs, \edges, \weights, \flag)$ with a start vertex $s$ and destination vertex $t$ .
%      % \STATE  {\bf {bfs $(G)$}:}  
%      % \STATE {\bf init} 
%      % \\
%      % current node: $c$, 
%      % \\
%      % queue: $q$ : List, add into $a$ an arbitrary v from $\vertxs$. 
%      % \\
%      % visited: List of length $|\vertxs|$, initialize with $\efalse$.
%      % \\
%      % results: $r$ : List of length $|\vertxs|$, initialize with -1.
%      % \\
%      % curr$\kw{flowcapacity}$: INT, initialize MAXINT.
%      % \\
%      % querynum: INT, initialize 0. \#\{To count the query numbers when we are walking inside a cycle\}
%      % \\
%      % \STATE \qquad {\bf while} $q$ isn't empty:
%      % \STATE \qquad \qquad take the vertex from head $c= q.pop()$
%      % \STATE \qquad \qquad mark $c$ as visited, visited $[c] = 1$.
%      % \STATE \qquad \qquad {\bf if} $\kw{cycle}(c)$  \#\{we are inside a cycle\}
%      % \STATE \qquad \qquad \qquad curr$\kw{flowcapacity}$ = min($\weights$(c), curr$\kw{flowcapacity}$).
%      % \STATE \qquad \qquad \qquad querynum += $\flag(c)$.
%      % \STATE \qquad \qquad  \qquad for all unvisited vertex $v$ having directed edge from c:
%      % \STATE \qquad \qquad \qquad \qquad r[v] = r[c]; q.add(v)
%      % \STATE \qquad \qquad \qquad  {\bf if}  $v$ is visited, then the circle finished
%      % \STATE \qquad \qquad \qquad \qquad update the result $r[v] =  \max(r[v], r[c] + $curr$\kw{flowcapacity}$*querynum)
%      % \STATE \qquad \qquad \qquad \qquad curr$\kw{flowcapacity}$ = MAXINT
%      % \STATE \qquad \qquad \qquad \qquad querynum = 0.  
%      % \STATE \qquad \qquad {\bf else} 
%      % \STATE \qquad \qquad \qquad for all unvisited vertex $v$ having directed edge from c:
%      % \STATE \qquad \qquad \qquad  \qquad $r[v] = \max(r[v], r[c] + \flag(c))$; q.add(v)
%      % \RETURN max($r$)
%      % \end{algorithmic}
%      % \end{algorithm}
%      %
%      % \begin{algorithm}
%      %   \caption{
%      %   {Over-Approximated Adaptivity on SCC}
%      %   \label{alg:overadp_alg}
%      %   }
%      %   \begin{algorithmic}[1]
%      %   \REQUIRE $G = (\vertxs, \edges, \weights, \qflag)$ \#\{An Strong Connected Symbolic Weighted Directed Graph\}
%      %   % with a start vertex $s$ and destination vertex $t$ .
%      %   \STATE {\bf {$\kw{\pathsearch_{scc-naive}(G)}$}:}  
%      %   \STATE {\bf init} 
%      %   \\
%      %   $\kw{r_{scc}}$: the Adaptivity of this SCC
%      %   % \STATE  {\bf def} {$\kw{dfs_{naive}(G, c,visited)}$}: 
%      %   % % \STATE {\bf init} 
%      %   % % \\
%      %   % % current node: $c$, 
%      %   % % \\
%      %   % % visited: List of length $|\vertxs|$, initialize with $\efalse$.
%      %   % % \\
%      %   % % \STATE {\bf if} $c = s$:
%      %   % % \RETURN \qquad  $\weights(s)*\flag(s) $.
%      %   % \STATE \qquad $r[c] = \weights(c)*\qflag(c) $
%      %   % \STATE \qquad {\bf for}  all vertex $v$ having directed edge from $c$:
%      %   % \STATE \qquad \qquad {\bf if}  $v$ is unvisited:
%      %   % \STATE \qquad \qquad \qquad  \#\{mark $v$ as visited\} $\kw{visited}[v] = 1$;
%      %   % \STATE \qquad \qquad \qquad $r[c] += \kw{dfs_{naive}(G, v, visited)}$;
%      %   % \STATE \qquad {\bf else}: \#\{There is a cycle finished\}
%      %   % \RETURN \qquad \qquad $\weights(v)*\flag(v) $.
%      %   \STATE  {\bf for} every vertex $v$ in $\vertxs$:
%      %   % \STATE  \qquad initialize \kw{visited} with $\efalse$.
%      %   \STATE  \qquad $r_{scc} += \weights(v)*\qflag(v)$  
%      %   \RETURN $r[c]$
%      %   \end{algorithmic}
%      %   \end{algorithm}
%        %
%  %
%      % \begin{algorithm}
%      %     \caption{
%      %     {Over-Approximated Adaptivity on SCC}
%      %     \label{alg:overadp_alg}
%      %     }
%      %     \begin{algorithmic}
%      %     \REQUIRE Weighted Directed Graph $G = (\vertxs, \edges, \weights, \qflag)$ with a start vertex $s$ and destination vertex $t$ .
%      %     \STATE  {\bf {$\kw{dfs_{naive}(G, c,visited)}$}:}  
%      %     % \STATE {\bf init} 
%      %     % \\
%      %     % current node: $c$, 
%      %     % \\
%      %     % visited: List of length $|\vertxs|$, initialize with $\efalse$.
%      %     % \\
%      %     % \STATE {\bf if} $c = s$:
%      %     % \RETURN \qquad  $\weights(s)*\flag(s) $.
%      %     \STATE $r[c] = \weights(c)*\qflag(c) $
%      %     \STATE {\bf for}  all vertex $v$ having directed edge from $c$:
%      %     \STATE \qquad {\bf if}  $v$ is unvisited:
%      %     \STATE \qquad \qquad  \#\{mark $v$ as visited\} $\kw{visited}[v] = 1$;
%      %     \STATE \qquad \qquad $r[c] += \kw{dfs_{naive}(G, v, visited)}$;
%      %     % \STATE \qquad {\bf else}: \#\{There is a cycle finished\}
%      %     % \RETURN \qquad \qquad $\weights(v)*\flag(v) $.
%      %     \RETURN $r[c]$
%      %     \end{algorithmic}
%      %     \end{algorithm}%
%          %
%    \paragraph*{Adaptivity Computation Algorithm on SCC Graph ($\kw{\pathsearch_{scc}(G)}$)}
%      \begin{algorithm}
%              \caption{
%              {Adaptivity Computation Algorithm on SCC Graph  {\bf {$\kw{\pathsearch_{scc}(G)}$}:} }
%              \label{alg:adaptscc}
%              }
%              \begin{algorithmic}[1]
%                \REQUIRE $G = (\vertxs, \edges, \weights, \qflag)$ \#\{An Strong Connected program based dependency Graph\}
%              % \STATE  {\bf {$\kw{\pathsearch_{scc}(G)}$}:}  
%              \STATE {\bf init} 
%              % \\
%              $\kw{r_{scc}}$: $\aexpr$, arithmetic expression, initialized $0$, the Adaptivity of this SCC
%              % \STATE \qquad {\bf init} 
%              % \STATE \qquad current node: $c$, 
%              % \\
%              % visited: List of length $|\vertxs|$, initialize with $\efalse$.
%              % \\ \qquad  $\kw{r_{scc}}$ : initialize $0$, the adaptivity of this graph
%              \\ \qquad  $\kw{visited}$ : $\{0, 1\}$ List,  \#\{length $|\vertxs|$, initialize with $0$ for every vertex\}
%              % \\ \qquad  \#\{length $|\vertxs|$, initialize with $0$ for every vertex, recording whether a vertex is visted.\}
%              \\ \qquad  $\kw{r}$ : $\aexpr$, arithmetic expression List, 
%              \#\{length $|\vertxs|$, initialize with $\qflag(v)$ for every vertex \}
%              % \\ \qquad  \#\{length $|\vertxs|$, initialize with $\qflag(v)$ for every vertex, recording the adaptivity reaching each vertex.\}
%              \\ \qquad  $\kw{flowcapacity}$: $\aexpr$ List, 
%              \#\{length $|\vertxs|$, initialize with $\infty$ for every vertex\}
%              % INT List of length $|\vertxs|$, initialize MAXINT. 
%              % \\ \qquad  \#\{length $|\vertxs|$, initialize with $\infty$ for every vertex,
%              % % \#\{For every vertex, 
%              % recording the minimum weight when the walk reaching 
%              % that vertex, inside a cycle\}
%              \\ \qquad  $\kw{querynum}$: INT List,  \#\{length $|\vertxs|$, initialize with $\qflag(v)$ for every vertex.
%              % % \#\{For every vertex, 
%              % recording the query numbers when the path reaching 
%              % that vertex, inside a cycle
%              \}
%              % %  of length $|\vertxs|$, initialize with $\qflag(v)$ for every vertex. 
%              % \\ \qquad  \#\{length $|\vertxs|$, initialize with $\qflag(v)$ for every vertex, 
%              % \#\{For every vertex, 
%              % recording the query numbers when the path reaching 
%              % that vertex, inside a cycle\}
%              \STATE {\bf if} $|\vertxs| = 1$ and $|\edges| = 0$: {\bf return}  $\qflag(v)$
%              % \STATE \qquad {\bf return}  $\qflag(v)$
%              \STATE  {\bf def} {$\kw{dfs(G, c,visited)}$}:
%              % \STATE \qquad update the length of the longest path reaching this vertex
%              % $r[s] =  r[s] + $$\kw{flowcapacity}$[s] * querynum[s].
%              % \RETURN  \qquad $r[s]$.      
%              \STATE \qquad {\bf for} every vertex $v$ 
%              % having directed edge from $c$:
%              connected by a directed edge from $c$:
%              \STATE \qquad \qquad {\bf if} $\kw{visited}[v] = \efalse$:
%              \STATE \qquad \qquad \quad $\kw{flowcapacity[v] = \min(\weights(v), {flowcapacity}[c])}$;
%              \STATE \qquad \qquad \quad $\kw{querynum[v] = querynum[c] + \qflag(v)}$;
%              % \STATE \qquad \qquad \qquad \#\{do not update the length of the longest walk reaching $v$ until the cycle is finished\}
%              % \STATE \qquad \qquad \qquad $\kw{r[v] =  r[c] + flowcapacity[v] \times querynum[v]} $; \#\{do not update the length of the longest walk reaching $v$ until the cycle is finished\}
%              \STATE \qquad \qquad \quad $\kw{r[v] =  \max(r[v], flowcapacity[v] \times querynum[v]}) $; 
%              % \#\{do not update the length of the longest walk reaching $v$ until the cycle is finished\}
%              \STATE \qquad \qquad \quad  $\kw{visited}[v] = 1$;
%              $\kw{dfs(G, v, visited)}$; %\#\{mark $v$ as visited\}
%              % \STATE \qquad \qquad \qquad $\kw{dfs(G, v, visited)}$;
%              \STATE \qquad \quad {\bf else}: \#\{There is a cycle finished, update the adaptivity\}
%              % \STATE \qquad \qquad \qquad \#\{update the length of the longest path reaching this vertex\}
%              \STATE \qquad \qquad \quad 
%              $\kw{r[v] =  \max(r[v], r[c] +  \min(\weights(v), {flowcapacity}[c]) * (querynum[c] + \qflag(v)))}$;
%              % \#\{update the length of the longest walk reaching this vertex on this cycle\}
%              %  $\kw{r[v] =  \max(r[v], r[c] + flowcapacity[v] * querynum[v])}$; \#\{update the length of the longest walk reaching this vertex on this cycle\}
%              %  \STATE \qquad \qquad \qquad \#\{Recover the $\kw{flowcapacity}$ and querynumber to previous state, for different loops\}
%              % \STATE \qquad \qquad \qquad $\kw{flowcapacity[v] = flowcapacity[c]}$; \#\{Recover the $\kw{flowcapacity}$\}
%              % \STATE \qquad \qquad \qquad $\kw{querynum[v] = querynum[c]}$;\#\{Recover the $\kw{querynum}$\}
%              \STATE \qquad {\bf return}  $\kw{r[c]}$
%              \STATE  {\bf for} every vertex $v$ in $\vertxs$:
%              initialize the $\kw{visited, r, flowcapacity, querynum}$;
%              % \STATE  \qquad initialize the $\kw{visited, r, flowcapacity, querynum}$;
%              \STATE  \qquad $\kw{r_{scc} = \max(r_{scc}, dfs(G, v, \kw{visited} ))}$ ; 
%              \RETURN  $\kw{r_{scc}}$
%              \end{algorithmic}
%              \end{algorithm}
%              % \\ 
 
%  This algorithm in Alg.~\ref{alg:adaptscc} takes a subgraph of the program-based dependency graph(SCC), and the output is the adaptivity of the input. 
%  For an SCC containing only one vertex, it returns the query annotation of this vertex as adaptivity.
%  For SCC containing at least one edge, 
%  There are three steps: 1. find out all the paths in the input 2. calculate the adaptivity of every path using my designed adaptivity counting method. 3. return the maximal adaptivity among all the paths. 
%  Because my input graph is SCC, when we start traversing from a vertex, we will finally go back to this vertex. The paths we find in step 1 are all those with the same starting and ending vertex. The most interesting part is step 2. 
%  For the SCC containing at least one edge, 
%  we compute the adaptivity for each path 
%  on the fly of searching for the paths 
%  in the
%  recursion algorithm $\kw{dfs}$. It is designed based on 
%  a deep first search strategy 
%  % of this algorithm is described as follows,
%  from line: 6-16.
%  In order to guarantee the visiting times of each vertex by its weight
%  and compute the adaptivity accurately, 
%  we use a special parameter $\kw{flowcapacity}$  to track the minimum weight
%  along the path during the 
%  searching procedure, 
%  and a parameter $\kw{querynum}$
%  % to compute the query length of this walk
%  to track the total number of vertices with query annotation $1$
%  % which are query vertices 
%  along the path.
%  $\kw{flowcapacity}$ is a list of arithmeitc expression $\mathcal{A}_{in}$ 
%  % for every vertex,
%  recording the minimum weight when the path reaches that vertex, which is initialized by $\infty$.
%  $\kw{querynum}$ is a list of integer
%  initialized by query annotation $\qflag(v)$ for every vertex. 
%  The updating operations
%  during the traverse 
%  (line: 8) and 
%  at the end of the traverse (line: 11) are interesting,
%  specifically $\kw{flowcapacity[v] \times querynum[v]}$ 
%  computes the query length for this path. 
%  it guarantees 
%  the visiting times of each vertex on the path reaching a vertex $v$ is no more than 
%  the maximum visiting times it can be on a qualified walk, through $\kw{flowcapacity[v]}$,
%  and in the same time  compute the query length instead of weighted length accurately through 
%  $\kw{ querynum[v]}$.
%  In this way, we resolve the \textbf{Approximation Challenge} without losing the soundness, formally in the appendix.