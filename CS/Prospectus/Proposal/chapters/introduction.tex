% %%%% Benefit of reasoning about 
% \subsection{Motivation of Reasoning about Adaptivity}
% % 
In Section~\ref{sec:intro-background},
I introduce the background and limitation of the 
adaptive data analysis, 
and the motivation of reasoning about the \emph{adaptivity} quantity property 
for adaptive data analysis.
%  in Section~\ref{sec:intro-motivation}.
% analyzing 
In order to analyze this quantity property for the adaptive data analysis, there are 3 challenges
% problems encountered.
introduced.
% I introduce these three problems
% and the full-spectrum analysis methodologies developed according to these problems 
Targeting to the three challenges, I introduce the methodologies 
of the program analysis framework for the adaptive data analysis's adaptivity property
accordingly, in Section~\ref{sec:intro-adapt}.
Concretely, 
% the full-spectrum 
this analysis framework is developed through the language formalization,
the execution-based analysis, and the static-based program analysis.
%
Based on the implementation and experimental results on this analysis framework, 
I propose three significant 
further features can be improved 
% in the analysis methodologies 
in Section~\ref{sec:intro-improve}, 
and plan to finish these improvements 
before the final defense.
%
 % Next, based on the implementation and experimental results, I proposed two significant 
 % further features can be improved for my analysis framework, and plan to finish the improvement 
 % before the final defense.
 %
 Then, in Section~\ref{sec:intro-cost}, through two observations, 
 I introduce the motivations and methodologies
 for 
%  the accurate 
analyzing program's \emph{non-monotonic} quantitative property accurately.
%   with implicit cost decreases.
% \\
% 1. traditional program's resource cost analysis failed to consider the case where the program's cost could decrease 
% implicitly, 
% \\
% 2. and 
% % when there isn't a dependency relation between variables.
% the resource consumption during the program 
% execution increases and particularly decreases implicitly in the same way as the program's adaptivity, 
% % Specifically, in line 5 
% % where the list is re-written and the heap consumption is decreased implicitly. 
% % This implicit decrease 
% % of the cost works the same as the program's adaptivity decreases.
% I am interested in improving the accuracy of the program's general resource cost analysis
% by 
% % onto the program's resource cost analysis. 
% % Use this framework,
% Through the generalized \emph{adaptivity} analysis framework.
% I will give
% a more accurate resource cost estimation by taking the program's implicit resource cost into consideration, comparing 
% to the worst case cost analysis in a traditional way.

\subsection{Background and Motivation\todo{Rewrite into Program Analysis of the Quantitative Property}}
 \label{sec:intro-background}
 Skeleton:
Usefulness of the Program Analysis on Quantitative Property in different areas.
\\
In Machine Learning Area, the Adaptivity Quantity is significant
==> Major Work I
\\
In Resource Cost Analysis Area, the Reachability-Bound is significant.
==> Major Work II
Technique Limitation:
\\
Path-Sensitive Reachability-Bound 
\\
In Resource Cost Analysis Area, the Reachability-Bound is significant.


\subsection{Proposal Outline}
\label{sec:intro-outline}
\paragraph{Automated Program Analysis Framework for Adaptive Data Analysis (In Improvements)}
\label{sec:intro-adapt}
Based on the implementation and experimental results of the program analysis framework,
%  on the $\THESYSTEM$,
I plan to focus on the following three further features which can be extended and improved.
%  in my full-spectrum analysis.
\begin{enumerate}
    \item In the adaptive data analysis formalization, I plan to extend the {\tt Query While} Language with inter-procedure call.
    \item In the execution-based \emph{adaptivity},
    I plan to improve the precision of the \emph{adaptivity} formalization w.r.t. the program's intuitive adaptivity rounds,
%  in the formal  model 
and extend this analysis with inter-procedure call.
\item In static \emph{adaptivity} analysis, I plan to give a tighter estimated upper bound on \emph{adaptivity}.
%  give a tighter estimated upper bound 
Specifically, I will focus on improving the accuracy of the static \emph{dependency quantity} analysis in the second step through 
developing a new path sensitive reachability bound analysis technique. 
% \item In the third step of static program analysis, I will improve the accuracy of the adaptivity computation algorithm,
% compute a tighter adaptivity upper bound as well.
\end{enumerate}
\paragraph{Path-Sensitive Reachability-Bound Analysis (In Preparation)}
\label{sec:intro-reachability}

\paragraph{Towards Accurate Program Non-Monotonic Quantitative Property Analysis (In Preparation)}
\label{sec:intro-cost}
Moving towards the area of program's quantitative property analysis,
% Then, motivated by the two following aspects, 
there are two interesting observations as follows.
% I am interested 
These two observations motivated me in 
% improving the accuracy of the program's general resource cost analysis
improving the accuracy of the program's general resource cost analysis
by generalizing this \emph{adaptivity} analysis framework.
\begin{itemize}
 \item 
%  In the traditional program's resource cost and quantitative property analysis,
 There are two research areas in the traditional program's resource cost and quantitative property analysis.
%  of program cost analysis, 
One area is type-system based and the other is data-flow/control-flow analysis based. 
In the type-system design-based areas (\cite{GustafssonEL05}, \cite{hoffmann_jost_2022}), 
the analysis technique requires explicit abstraction or data structure de-allocation in order to save or reduce the cost.
 The
 works in both of these two areas fail to recognize the case where program resource consumption or quantitative properties 
 are decreased implicitly or increased \emph{non-monotonically}.
 \item This kind of resource consumption or quantitative properties during the program 
 execution increase and particularly decrease implicitly in the same way as the program's adaptivity. 
 This is explained in detail through an example in Section~\ref*{sec:generalization}.
\end{itemize}
 Based on the observations above, 
 I plan to develop
 an accurate program \emph{non-monotonic} quantitative property analysis framework through generalizing 
 my \emph{adaptivity} analysis framework.
 This framework can give more accurate cost bound than traditional worst-case resource cost estimation methods,
 by taking the program's implicit resource cost decreasing into consideration.
%  compared 
%  to the worst-case cost analysis in the traditional way.

\paragraph*{Proposal Structure}
To sum up, this proposal covers the following topic in each following section.
\begin{enumerate}
\item A while-like language extended with query request feature, named {\tt Query While} Language, 
used to implement 
the adaptive data analysis in Section~\ref{sec:language}.
\item A formal adaptivity model through execution-based adaptivity analysis in Section~\ref{sec:dynamic}.
\item A static program analysis algorithm named {\THESYSTEM} in Section~\ref{sec:static}.
\item Three proposed further features to be improved for the adaptivity analysis framework,
% based on the 
% % formal adaptivity model and {\THESYSTEM}, 
%  full-spectrum 
in Section~\ref{sec:furthers}.
%  presented in Section~\ref{sec:language},~\ref{sec:dynamic} and~\ref{sec:static},
 This proposed work is planned to be done before the final defense.
\item A proposed accurate program resource cost analysis framework generalized from {\THESYSTEM} in Section~\ref{sec:generalization}. 
The analysis framework design is expected to be done with the implementation start off before the final defense.
% \item A proposed plan for solving the CFL-reachability problem via reduction into the {\THESYSTEM} framework in Section~\ref{sec:cfl_reduction},
% expected to start before final defense and developing further after.
\end{enumerate}