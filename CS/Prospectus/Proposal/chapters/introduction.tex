% %%%% Benifit of reasoning about 
% \subsection{Motivation of Reasoning about Adaptivity}
% % 
Consider a dataset $X$ consisting of $n$ independent samples from some unknown population $\dist$.  How can I ensure that the conclusions drawn from $X$ \emph{generalize} to the population $\dist$?  Despite decades of research in statistics and machine learning on methods for ensuring generalization, there is an increased recognition that many scientific findings generalize poorly (e.g. 
\cite{Ioannidis05,GelmanL13}
).  While there are many reasons a conclusion might fail to generalize, one that is receiving increasing attention is \emph{adaptivity}, which occurs when the choice of method for analyzing the dataset depends on previous interactions with the same dataset~\cite{GelmanL13}.

 Adaptivity can arise from many common practices, such as exploratory data analysis, using the same data set for feature selection and regression, and the re-use of datasets across research projects.  Unfortunately, adaptivity invalidates traditional methods for ensuring generalization and statistical validity, which assume that the method is selected independently of the data. The misinterpretation of adaptively selected results has even been blamed for a ``statistical crisis'' in empirical science~\cite{GelmanL13}.
%  ~\cite{GelmanL13}.

\begin{figure}
    \centering
    \includegraphics[width=0.7\columnwidth]{figures/data_analysis_model.png}
    \caption{Overview of our Adaptive Data Analysis model.}
    \label{fig:adaptivity-model-overview}
\vspace{-0.5cm}
\end{figure}

A line of work initiated by \cite{DworkFHPRR15}, \cite{HardtU14} posed the question: Can I design \emph{general-purpose} methods that ensure generalization in the presence of adaptivity, together with guarantees on their accuracy?  
The idea that has emerged in these works is to use randomization to help ensure generalization. 
Specifically, these works have proposed to mediate the access of an adaptive data analysis to the data by means of queries from some pre-determined family (I will consider here a specific family of queries often called "statistical" or "linear" queries) that are sent to a 
\emph{mechanism} which uses some randomized process to guarantee that the result of the query does not depend too much on the specific
sampled dataset. 
This guarantees that the result of the queries generalizes well. 
This approach is described in Figure~\ref{fig:adaptivity-model-overview}, where
I have a population that I'm interested in studying, and a dataset containing individual samples from this population. The adaptive data analysis I'm interested in running has access to the dataset through queries of some pre-determined family (e.g., statistical or linear queries) mediated by a mechanism. 
This mechanism uses randomization to reduce the generalization error of the queries issued to the data.
This line of work has identified many new algorithmic techniques for ensuring generalization in adaptive data analysis, leading to algorithms with greater statistical power than all previous approaches. 
Common methods proposed by these works include, the addition of noise to the result of a query, data splitting, etc. 
Moreover, these works have also identified problematic strategies for adaptive analysis, showing limitations on the statistical power one can hope to achieve. 
Subsequent works have then further extended the methods and techniques in this approach and further extended the theoretical underpinning of this approach, 
e.g.~\cite{dwork2015reusable,dwork2015generalization,BassilyNSSSU16,UllmanSNSS18,FeldmanS17,jung2019new,SteinkeZ20,RogersRSSTW20}.
%

A key development in this line of work is that the best method for ensuring generalization in an adaptive data analysis depends to a large extent on the number of \emph{rounds of adaptivity}, the depth of the chain of queries. 
As an informal example, the program $x \leftarrow q_1(D);y \leftarrow q_2(D,x);z \leftarrow q_3(D,y)$ has three rounds of adaptivity, since $q_2$  depends on $D$ not only directly because it is one of its input but also via the result of $q_1$, 
which is also run on $D$, and similarly,  $q_3$ depends on $D$ directly but also via the result of $q_2$, which in turn depends on the result of $q_1$. 
The works I discussed above showed that, not only does the analysis of the generalization error depend on the number of rounds, but knowing the number of rounds actually allows one to choose methods that lead to the smallest possible generalization error - I will discuss this further in Section~\ref{sec:overview}. 

% \mg{Check the following - also the plots need to be on the same scale!}
For example, these works showed that when an adaptive data analysis uses a large number of rounds of adaptivity then a low generalization error can be achieved by mechanism of 
adding to the result of each query Gaussian noise scaled to the number of rounds. When instead  an adaptive data analysis uses a small number of rounds of adaptivity then a low generalization error can be achieved by using more specialized methods, such as data splitting mechanism or the reusable holdout technique from~\cite{DworkFHPRR15}.
To better understand this idea, I show in Figure~\ref{fig:generalization_errors} two experiments showcasing these situations. 
More precisely, in Figure~\ref{fig:generalization_errors}(a) I show the results of a specific analysis\footnote{I will use formally a program implementing this analysis (Figure~\ref{fig:overview-example}) as a running example in the rest of the paper.} with two rounds of adaptivity. 
This analysis can be seen as a classifier which first runs 500 non-adaptive queries on the first 500 attributes of the data, looking for correlations between the attributes and a label, and then runs one last query which depends on all these correlations. 
Without any mechanism the generalization error is pretty large, and the lower generalization error is achieved when the data-splitting method is used. 
In Figure~\ref{fig:generalization_errors}(b), I show the results of a specific analysis\footnote{I will present this analysis formally in Section~\ref{sec:examples}.} with four hundreds rounds of adaptivity. 
This analysis can be seen as a classifier which at each step runs an adaptive query based on the result of the previous ones. 
Again, without any mechanism the generalization error is pretty large, and the lower generalization error is achieved when the Gaussian noise is used. 
{\small
\begin{figure}
\centering
\begin{subfigure}{.48\textwidth}
\begin{centering}
\includegraphics[width=0.9\textwidth]{figures/tworound.png}
\caption{}
\end{centering}
\end{subfigure}
%}
\quad
\begin{subfigure}{.48\textwidth}
\begin{centering}
\includegraphics[width=0.9\textwidth]{figures/multipleround.png}
\caption{}
\end{centering}
\end{subfigure}
\vspace{-0.4cm}
 \caption{
 The generalization errors of two adaptive data analysis examples, under different choices of mechanisms.
 (a) Data analysis with adaptivity 2, 
 (b) Data analysis with adaptivity 400. 
}
\label{fig:generalization_errors}
\vspace{-0.5cm}
\end{figure}
}


%gap
This scenario motivates us to explore the design of program analysis techniques that can be used to estimate the number of \emph{rounds of adaptivity} that a program implementing a data analysis can perform. These techniques could be used to help a data analyst in the choice of the mechanism to use,
and they
could be ultimately be integrated into a tool for adaptive data analysis such as the \emph{Guess and Check} framework by~\cite{RogersRSSTW20}. 
%
The first problem is \emph{how to define formally} a model for adaptive data analysis which is general enough to support the methods I discussed above and would permit to formulate the notion of adaptivity these methods use. 
I take the approach of designing a programming framework for submitting queries to some \emph{mechanism} giving access to the data mediated by one of the techniques I mentioned before, e.g., adding Gaussian noise, randomly selecting a subset of the data, using the reusable holdout technique, etc. 
In this approach, a program models an \emph{analyst} asking a sequence of queries to the mechanism. The mechanism runs the queries on the data applying one of the methods discussed above and returns the result to the program. The program can then use this result to decide which query to run next. 
Overall, I'm interested in controlling the generalization of the results of the queries which are returned by the mechanism, by means of the adaptivity. 

The second problem is \emph{how to define the adaptivity of a given program}.
Intuitively, a query $Q$ may depend on another query $P$, if there are two values that $P$ can return which affect in different ways the execution of $Q$. 
For example, as shown in \cite{dwork2015reusable}, and as I did in our example in Figure~\ref{fig:generalization_errors}(a), one can design a machine learning algorithm for constructing a classifier which first computes each feature's correlations with the label via a sequence of queries, and then constructs the classifier based on the correlation values. 
If one feature's correlation changes, the classifier depending on features is also affected.  
This notion of dependency builds on the execution trace as a \emph{causal history}. 
In particular, I'm interested in the history or provenance of a query up until this is executed, I'm not then concerned about how the result is used --- except for tracking whether the result of the query may further cause some other query. 
This is because I focus on the generalization error of queries and not their post-processing. % 
To formalize this intuition as a quantitative program property,
% I first consider all the possible evaluations of a programs  --- I do this by 
I use a trace semantics recording the execution history of programs on some given input --- and I create a dependency graph, where the dependency between different variables (query is also assigned to variable) is explicit and track which variable is associated with a query request. 
I then enrich this graph with weights describing the maximal number of times each variable is evaluated in a program evaluation starting with an initial state. The adaptivity is then defined as the length of the walk visiting most query-related variables on this graph. 

The third problem is \emph{how to estimate the adaptivity of a given program}. 
The adaptive data analysis model I consider and our definition of adaptivity suggest that for this task I can use a  program analysis that is based on some form of dependency analysis. This analysis needs to take into consideration:
1) the fact that, in general, a query $Q$ is not a monolithic block but rather it may depend, through the use of variables and values, on other parts of the program. 
Hence, it needs to consider some form of data flow analysis. 
2) the fact that, in general, the decision on whether to run a query or not may depend on some other value. Hence, 
 it needs to consider some form of control flow analysis.
3) the fact that. in general, I'm not only interested in whether there is a dependency or not, but in the length of the chain of dependencies. 
Hence, it needs to consider some quantitative information about the program dependencies. % {A quick example is that : I store the result of query $Q_1$ in variable $x$ and use variable $y$ to record the result of query $Q_2$. I want to construct the third query $Q_3$ which relies on the value stored in $x$, let us say, $Q_3$ will ask for the sum of the first column of a table if $x$ is positive and the sum of the second column otherwise. In this situation, I need data flow analysis. On the other hand, if I need the value of $y$ to help us decide whether I should ask $Q_3$, for example, I ask the third query if $y$ is odd, and do not ask if $y$ is even. Naturally, to be able to handle this case, control flow analysis comes into play. Formally speaking,  }
To address these considerations and be able to estimate a sound upper bound on the adaptivity of a program, 
I will develop a program analysis algorithm, named {\THESYSTEM}, which combines data flow and control flow analysis with reachability bound analysis~\cite{GulwaniZ10}. 
This new program analysis gives tighter bounds on the adaptivity of a program than the ones one would achieve by directly using the data and control flow analyses or the ones that one would achieve by directly using reachability bound analysis techniques alone.
%%%%% To reason about

Then, through observation in following example, the heap resource consumption during the program 
execution is accumulating in the same way as the program's adaptivity. Specifically, in line 5 
where the list is re-wrote and the heap consumption is decreased implicitly. 
This implicit decrease 
of the cost works exactly the same as program's adaptivity decrease.
\\
This motivates the generalization of the analysis framework onto the program's resource cost analysis. Use this framework,
I will give
a more accurate resource cost estimation by taking the program's implicit resource cost into consideration, comparing 
to the worst case cost analysis in traditional way.
\\
There are two categories of the program cost analysis, Type-System Based and Static Program Analysis Based. But both of the
works in these two areas fails to recognize the case where program resource consumption is decreased implicitly.


To sum up, this proposal covers the following topic.
\begin{enumerate}
\item A while-like language extended with query request feature, {\tt Query While} Language used to implement 
the adaptive data analysis.
\item A formal adaptivity model through execution-based adaptivity analysis.
\item A static program analysis algorithm, named {\THESYSTEM}.
\item Extended features on the formal adaptivity model and {\THESYSTEM}.
\item Generalization of {\THESYSTEM} on program resource cost analysis.
\end{enumerate}