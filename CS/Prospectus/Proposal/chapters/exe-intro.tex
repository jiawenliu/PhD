
% In order to formalize a quantitative property w.r.t. the dependency relation in the program, I
% use a three-step analysis methodology developed as follows,
% \\
%  a. The dependency relation between every query, through the methodology of semantic data dependency analysis.
% \\
%  b. The dependency quantity analysis, through the methodology of execution-based data reachability bound analysis. Then 
% \\
%  c. The adaptivity analysis, based on the two analysis results above, 
%  I construct an execution-based dependency graph combining the dependency relation and the dependency quantity
%  and give the formal \emph{adaptivity} definition 
%  for program.
% \\
% The construction of this graph requires me to think about the dependency relation between two queries using what we have at hand - 
% the trace generated in Section~\ref{sec:language}. 
 \paragraph*{Background and Related Work}
 {
My framework constructs an execution-based dependency graph based on the execution traces of a program. I define semantic dependence on this graph by considering (intraprocedural) data and control dependency~\cite{bilardi1996framework,cytron1991efficiently,pollock1989incremental}. 
One related work 
\cite{austin1992dynamic} presents a methodology to construct a dynamic dependency graph (DDG) based on the dynamic execution of a program in an imperative language, where edges represent dependency between instructions. Data dependency, control dependency, storage dependency, and resource dependency between instructions are all considered. My execution-based dependency graph only needs data dependency and control dependency between variable assignment results. 
% Critical path length analysis on DDGs is useful for understanding the scope for parallelization, while we use the length of the longest path to define adaptivity. 
%
DDGs have been used in many other domains. \cite{nagar2018automated} use DDGs to find serializability violations. \cite{hammer2006dynamic} use similar \emph{program dependency graphs} \cite{ferrante1987program} for dynamic program slicing.
\cite{mastroeni2008data} propose ways of constructing different kinds of program slices, by choosing different program dependencies. 
% For example, in either syntactic or semantics sense.
% This abstract dependency is based on properties rather than exact data.
% Aims to give finer and smaller program slice. 
They use a combination of 
static and dynamic dependency graphs but in a manner that is different from how we use the two. Their slicing uses both static and dynamic dependency graphs, while we use the dynamic dependency graph as the basis of a definition, which is then soundly approximated by an analysis based on the static dependency graph.}

{My execution-based data dependency relation definition over variables 
is inspired by the method in \cite{Cousot19a}, where the dependency relation is also identified by looking into the differences on two execution traces. 
However, Cousot excludes timing channels~\cite{SabelfeldM03} and empty observation, which are also not considered as a form of dependency in traditional dependency analysis \cite{DenningD77}.
% In the cases of empty observation and timing channels, the second query is executed 
% in one trace and isn't in another trace by modifying the value of the first query. 
% Then, the second query is indeed dependent on the first query and there exists an
% adaptivity round between the two queries. 
My definition includes timing channels and empty observation by observing both the disappearance and value variation.
}
\paragraph*{Execution-Based Adaptivity Analysis Overview}
To formalize this intuition as a quantitative program property, I develop an execution-based analysis
% I first consider all the possible evaluations of a program --- I do this by 
% I use a trace semantics recording the execution history of programs on some given input --- and I create a dependency graph, where the dependency between different variables (query is also assigned to a variable) is explicit and track which variable is associated with a query request. 
% I then enrich this graph with weights describing the maximal number of times each variable is evaluated in a program evaluation starting with an initial state. The adaptivity is then defined as the length of the walk visiting most query-related variables on this graph. 
% Through two aspects: the execution-based analysis and static-based program analysis.
% In the execution-based analysis, I will formalize the intuitive notion of \emph{adaptivity} as a quantitative 
% property of programs. This analysis is developed 
 in three steps as follows,
 \begin{enumerate}
 \item The first step on \emph{dependency relation} analysis is presented in Section~\ref{subsubsec:dynamic-datadep}.
 In this step, I define the variable \emph{may-dependency} relation based on the trace semantics in Section~\ref{sec:language-os}.
%   to analyze the \emph{dependency relation} between every query, 
%  through the methodology of semantic data dependency analysis.
%  %
%  Specifically through a trace semantics recording the execution history of programs on given input,
%  % --- and I create a dependency graph, 
%  the dependency between different variables (query is also assigned to a variable) is explicitly tracked and 
%  analyzed.
%   and 
%   which variable is associated with a query request. 
% I then enrich this graph with weights describing the maximal number of times each variable is evaluated in a program evaluation starting with an initial state. The adaptivity is then defined as the length of the walk visiting most query-related variables on this graph. 
% In the execution-based analysis, I will formalize the intuitive notion of \emph{adaptivity} as a quantitative 
% property of programs. This analysis is developed 
% \\
 \item In the second step in Section~\ref{subsubsec:dynamic-reachability}, I analyze the \emph{dependency quantity} through the methodology of execution-based reachability bound analysis.
%  As 
% %  analysis, 
% based on the \emph{dependency relation} above.
% This analysis is developed through the methodology of execution-based reachability bound analysis.
% \\
 \item The last step is the intuitive \emph{adaptivity} quantity analysis presented in Section~\ref{subsubsec:dynamic-adapt}.
 According to the two analysis results above, specifically \emph{dependency relation} and \emph{dependency quantity},
 I define the formal \emph{adaptivity} model in definition~\ref{def:trace_adapt} through 
 constructing a dependency graph.
%  This analysis is developed through the formal \emph{adaptivity} definition. \\
%  Specifically, I create a dependency graph, where the dependency between different variables (query is also assigned to a variable) is explicit and track which variable is associated with a query request. 
%  I then enrich this graph with weights describing the maximal number of times each variable is evaluated in a program evaluation starting with an initial state. 
%  The adaptivity is then defined as the length of the walk visiting most query-related variables on this graph. 
 \end{enumerate}


