
% In order to estimate weight for every vertex in the program-based dependency graph,
In order to estimate the \emph{Dependency-Quantity} from the execution-based analysis,
I perform the symbolic reachability bound analysis based on constructing the abstract control flow graph $\absG(c)$ for the program $c$.
\highlight{
 The previous static analysis doesn't have an analysis on the quantitative property.
 Compared to it, the new technique of estimating the quantity property helps to improve the adaptivity estimation.
}
% and add weight on that graph as shown in Figure~\ref{fig:abscfg_tworound}(c). 
% This because
% the vertices in the two graphs share the same unique label, the line number.
% We would like to generate the closure of every edge, which is an equal relation between variables. Solving this closure gives us the reachability bound for this edge. With all the bound for all the edges in the abstract control flow graph, we can calculate the weight for every vertex in this graph. For example, we show the closure generated for the edge $(4, j < j - 1, 5)$, 
% $\absclr(4, 5) = \varinvar(j)$. The invariant for variable $j$, $\varinvar(j)$ used here is 
% $\varinvar(j) = k * \absclr(1, 2)$, which is generated by all the difference constraints involving $j$ in the graph. Notice the $k$ in $\varinvar(j)$ comes from considering both difference constraint $j<=k$ from edge (1,2) and $j<=j-1$ from (4,5), which intuitively reflects the while loop whose counter is set to $k$ at the beginning and decreases by 1 at each iteration.
% We first generate the invariant for every variable showing up in the difference constraint and not user-defined. For example, in the Figure~\ref{def:abscfg_tworound}(b), there are three variables $a$, $x$, $j$ that we will generate invariant for. We show the invariant for the counter $k$, which is $j = k$. 
% $\absclr(1, 2) = 1$
% \\
% => 
% $\absclr(4, 5) = k$
% using the difference constraint on the edges for all 
% through the edges in $\absG(c)$, which correspond to $c$'s abstract transition between labels.
% We infer the invariant for every variable and compute the transition closure for every abstract transition. By solving the closure
% with the invariants of variables involved in this closure for every transition, we compute
% the symbolic reachability bound of every command corresponding to this transition. Specifically, this analysis can be performed in four steps:
% Variable Modification Tracking, Local Bounds Computation,
% Invariant Inference and Closure Generation, and Reachability Bound Computation,
% 
% We present the details of invariant, closure generation, and reachability bound computation as follows.
% with details as follows.
% \\
% %
% 1. Variable Modification Tracking: collecting the dc for where each variable is increased, decreased, and reset: 
% \\
% 2. Local Bounds Computation: 
% \\
% 3. Invariant Inference and Closure Generation
% \\
% 4. Reachability Bound Computation,
% %
% \paragraph*{Variable Modification Tracking}
% Identify the abstract events where each variable is increased, decreased, and reset:
% \\
% $\inc: \mathcal{VAR} \to \mathcal{P}(\absevent) $
% the set of the abstract events where the variable increase.
% \\
% $\inc(x) = \{(\absevent, c) | \absevent = (l, l', x' \leq x + v)\}$
% \\
% $\reset: \mathcal{VAR} \to \mathcal{P}(\absevent) $
% The set of the abstract events where the variable is reset.
% \\
% $\dec: \mathcal{VAR} \to \mathcal{P}(\absevent) $
% The set of abstract events where the variable decrease.
% % \\
% % $\dec(x) = \{(\absevent, c) | \absevent = (l, l', x' \leq x - v)\}$
% \\
% $Incr(v) \triangleq \sum\limits_{(\absevent, c) \in \inc(v)}\{\absclr(\absevent) \times v\}$
% %
% \paragraph*{Local Bounds}
% Given a program $c$ with its abstract control flow graph 
% $\absG(c) = (\absV, \absE)$
% \\
% Local Bounds Computation:
% $\locbound: \absevent \to \mathcal{VAR} \cup \constdom$.
% %
% \[ 
% \begin{array}{ll}
% \locbound(\absevent) \triangleq 1 
% & \absevent \notin SCC(\absG(c))
% \\
% \locbound(\absevent) \triangleq (x, v) 
% & \absevent \in SCC(\absG(c)) \land \absevent \in \dec(x) \land \absevent = (\_, \_ , x' \leq x - v) \\
% \locbound(\absevent) \triangleq (x, \max(\dec(x))) 
% & \absevent \in SCC(\absG(c)) \land 
% \absevent \notin \bigcup_{x \in \mathcal{VAR}} \dec(x)
% \land \absevent \notin SCC(\absG(c) \setminus \dec(x)) 
% \end{array}
% \]
% The first case is straightforward. Since variable's visiting time outside of any while loop is at most 1, we do not need to analyze the visiting times of every node in the graph from phase 1.
% The second and third step is guaranteed by the \emph{Discussion on Soundness} in Section 4 of \cite{sinn2017complexity}.
% Then soundness proof is in Lemma~\ref{lem:local_bound_sound} in appendix~\ref{apdx:reachability_soundness}.
% %
% \paragraph*{Invariant Inference and Closure Generation }
% Then, computing the bound invariants for variables and the transition closures for abstract events:
% \\ 
% $ \varinvar: \mathcal{VAR} \cup \constdom \to EXPR(\constdom)$
% \\
% $\absclr: \absevent \to EXPR(\constdom)$
% \\
% Then, the symbolic invariant for each variable 
% as well as the symbolic transition closure for each transition is calculated as follows:
% \[ 
% \begin{array}{lll}
% \varinvar(x) & \triangleq c & c \in \constdom \\
% \varinvar(x) & \triangleq Incr(v) + \max(\{\varinvar(a) + c | (t, a, c) \in \reset(x)\}) & c \notin \constdom
% \end{array}
% \]
% %
% \begin{defn}
% \label{def:transition_closure_base}
% \[ 
% \begin{array}{lll}
% \absclr(\absevent) 
% & \triangleq x / v & \\ 
% & \locbound(\absevent) = (x, v) \in \constdom \times \mathbb{N} & \\
% \absclr(\absevent) 
% & \triangleq (Incr(x) + 
% \sum\limits_{(\absevent', y, v') \in \reset(x)}
% \absclr(\absevent') \times \max(\varinvar(y) + v', 0) ) / v & \\
% & \locbound(\absevent) = (x, v) \land x \notin \constdom & 
% \end{array}
% \]
% \end{defn}
% %
% \paragraph*{Improved Variable Modification Tracking}
% Instead of just identifying the abstract events where each variable is reset,
% this improvement identifies the chain of the events where a given variable is reset by the 
% variables of the abstract events through the chain.
% \\
% $\resetchain: \mathcal{VAR} \to \mathcal{P}(\mathcal{P}(\absevent)) $
% The set of the chain of abstract events where the variable is reset through the chain.
% % \\
% % $Incr(v) \triangleq \sum\limits_{(\absevent, c) \in \inc(v)}\{\absclr(\absevent) \times v\}$
% %
% \paragraph*{Improved Invariant Inference and Closure Generation}
% Then, computing the bound invariants for variables and the transition closures for abstract events:
% \\ 
% $ \varinvar: \mathcal{VAR} \cup \constdom \to EXPR(\constdom)$
% \\
% $\absclr: \absevent \to EXPR(\constdom)$
% \\
% Then, the symbolic invariant for each variable 
% as well as the symbolic transition closure for each transition is calculated as follows:
% \[ 
% \begin{array}{lll}
% \varinvar(x) & \triangleq c & c \in \constdom \\
% \varinvar(x) & \triangleq Incr(v) + \max(\{\varinvar(a) + c | (t, a, c) \in \reset(x)\}) & c \notin \constdom
% \end{array}
% \]
% %
% \begin{defn}
% \label{def:transition_closure}
% \[ 
% \begin{array}{lll}
% \absclr(\absevent) 
% & \triangleq x / v & \\ 
% & \locbound(\absevent) = (x, v) \in \constdom \times \mathbb{N} & \\
% \absclr(\absevent) 
% & \triangleq \Big(
% \sum\limits_{y \in \{ y ~|~ 
% ch \in \resetchain(x), (l_1, x, y, v, l_2) \in ch \} } Incr(x) & \\
% & \quad + 
% \sum\limits_{ch \in \resetchain(x)}
% \big( \min\limits_{\absevent' \in ch}({\absclr(\absevent')}) \times 
% \max(\varinvar(y) + \sum\limits_{(l_1, x, y, v, l_2) \in ch } v, 0)\big) \Big) / v & \\
% & \locbound(\absevent) = (x, v) \land x \notin \constdom & 
% \end{array}
% \]
% \end{defn}
 %
% \paragraph*{Adding the Reachability Bounds for Every Vertex in the Data-Control Flow Graph}
% Updating the weight of every vertex in the $\progG(c) = (\progV, \progE, \progW, \progF)$ for program $c$ generated from phase 1. 
% For every $x^l \in \progV$, find the abstract event $\absevent \in \absflow(c)$ of the form $(l, \_, \_)$, updating the $\progW(x^l) $ by the transition closure of this event.
% \\
% \paragraph*{Reachability Bound Computation}
% With all the closures for all the edges of the abstract control flow graph, we can solve them to obtains the reachability bound of every edges. We decide the weight for every vertex in the abstract control flow graph by using the bound of the edges which head out from this vertex, by taking the max of the bound from these involving edges. For instance, 
% By the constraint on the edge $(4, j \leq j - 1, 5)$, we get bound $k$ for this edge.
% Then, we assign vertex $4$ by reachability bound $k$, as in Figure~\ref{fig:abscfg_tworound}(c). 
% Another interesting vertex is $2$, which has more than one edge heading out from it, $(2, \top, 3)$ and $(2,\top, 6)$. For the weight for vertex $2$, we choose the max between the bound $k$ from $(2,\top, 3)$ and $1$ from $(2,\top, 6)$.
% we first infer its local bound as variable $j$.
% Then by solving the invariant for variable $j$,
% we infer the value bound for $j$, which is $k$.
% Then the reachability bound for this abstract transition, (i.e., edge $4, j \leq j - 1, 5$) 
% is computed as $k$ as well through Definition~\ref{def:abs_trace}.
% \\
% $
% \progW(x^l) 
% \triangleq \absclr(\absevent)
% $
% Through the transition closure computed above, 
% The weight of every label in 
% % Then we update 
% the program $c$'s abstract control flow graph,
% $\absG(c) =(\absV, \absE, \absW)$
% is 
% computed as the maximum over all the abstract events $\absevent \in \absE$ heading out from this vertex, formally as follows.
% % by annotating each vertex with a symbolic weight. 
% % This weight corresponds to 
% %reachability bounds of
% \\
% $\absW 
% \triangleq \left\{ (l, w) \in \mathbb{N} \times EXPR(\constdom) | w = \max\limits_{\absevent = (l, \_, \_)} \{ \absclr(\absevent)\} \right\}$.
% % \\
% \paragraph*{Example}
% Still looking at the two-round example as in Figure~\ref{fig:abscfg_tworound}(b) where 
% each label $l$ is added with a weight by $absW$.
% This weight represents the maximum reaching times of this location $l$, in the other word,
% the estimated maximum visiting times of the command labeled with $l$.
% For example, looking at the vertex $1$,
% by analysis steps, since it isn't in any SCC, it's estimated reachability bound is computed as $1$.
% However, for the vertex $4$ which is involved in an SCC, the reachability bound is inferred in another way.
% By the constraint on the edge $4, j \leq j - 1, 5$,
% we first infer its local bound as variable $j$.
% Then by solving the invariant for variable $j$,
% we infer the value bound for $j$, which is $k$.
% Then the reachability bound for this abstract transition, (i.e., edge $4, j \leq j - 1, 5$) 
% is computed as $k$ as well through Definition~\ref{def:abs_trace}.
% In this abstract control flow graph, every vertex is a label,
% corresponding to a label command in the program.
% Each directed 
% edge represents an abstract transition 
% between two control locations, 
% i.e., the labels of two commands (we call the labels also control location and they refer to the same thing), 
% where the second labeled command will be executed after execution of the command with first label.
% For example, the edge $0, a \leq 0, 1$ on the top, represents,
% from location $0$, the command 
% $\clabel{\assign{a}{0}}^0$ is executed with next continuation location $1$,
% where the 
% command $\clabel{\assign{j}{k}}^1$ will be executed next.
% The constraint $a \leq 0$ is generated by abstracting from the assignment command $\assign{a}{0}$,
% representing that value of $a$ is less than or equals to $0$ after 
% location $0$ before executing command at line $1$.
%
% The same way for the rest weights' computation.
% We use for 
\paragraph*{Abstract Control Flow Graph}
The abstract control flow graph is a control flow graph, with annotations on every edge. 
It is constructed through program abstraction method,
the detail of the abstract control flow graph construction can be found in the appendix.
The formal abstract control flow graph definition is presented in
% Through a program $c$'s abstract execution trace, its abstract control flow graph is computed 
% defined in 
Definition~\ref{def:abs_cfg}.
% Given program $c$ with its abstract control flow $\absflow(c)$, the Abstract Control Flow Graph:
% \\
\begin{defn}[Abstract Control Flow Graph]
\label{def:abs_cfg}
Given a program $c$, 
with its abstract control flow $\absflow(c)$
its abstract control flow graph $\absG(c) =(\absV(c), \absE(c), \absW(c))$ is defined as follows,
\\
% \highlight{
% :
%
% \\
$\absE(c) = \{(l_1, dc, l_2) | (l_1, dc, l_2) \in \absflow(c)\}$,
\\
$\absV(c) = labels(c)\cup\{l_{ex}\}$
\\
 $\absW(c) 
\triangleq \left\{ (l, w) \in \mathbb{L} \times EXPR(\constdom) \right\}$.
% }
% \\
% , where the weight of every label to be computed in the next step.
\end{defn}
%
The vertices of this graph are the set of $c$'s labels with an exit label $l_{ex}$.
The edge $(l_1, dc, l_2)$ in the abstract control flow graph, represents an abstract transition 
from $l_1$ to $l_2$, with a set of difference constraints $dc$. The $\absflow(c)$ is generated by the
program abstraction method with full details in the appendix.
The weight $\absW(c)$ is the set of pairs $(l,w)$ where $w$ is the weight for label $l$. 
$w$ is an arithmetic expression over $\mathbb{N}$ and input variables.
It is the estimated upper bound on the number of visiting times for every control location
through the reachability bound analysis in the next step.

\paragraph{Reachability Bound Analysis}
The new analysis then performs the reachability-bound analysis algorithm on the abstract control flow graph $\absG(c)$ for the program
$c$ on every label.
The computed weights $\absW(c)$ is
a set of pairs 
$(l, w)$ where 
$w$ is the weight 
for label $l$ from the abstract control flow graph of $c$.
% The weight $w$ for each label $l$ 
$w$ is an arithmetic expression over $\mathbb{N}$ and
input variables, denoted by $\mathcal{A}_{in}$.
This analysis is inspired by the program complexity analysis method in \cite{sinn2017complexity}.
The detail of our symbolic reachability bound analysis which uses the difference constraint of the abstract control flow graph can be found in the appendix.

\paragraph{Dependency Quantity Estimation}
%
%
% The weight for each vertex in $\progV(c)$ is computed as follows,
With the computed weights $\absW(c)$ for program $c$,
we compute the reachability bound for each labeled variable $x^l$ as the estimation of the 
execution-based \emph{Dependency-Quantity}. 
% vertex in $\progV(c)$,
% as a set of pairs $(x^l, w) \in \ldom \times \mathcal{A}_{in}$
% $\progW(c) \in \mathcal{P}(\mathcal{LV} \times \mathcal{LV} \times EXPR(\constdom))$ 
% is the set of pairs 
% The weight for each vertex in $\progV(c)$ is computed 
The estimated reachability bound mapping each $x^l \in \lvar(c)$ to an arithmetic expression over $\mathbb{N}$ and
input variables. 
\[ 
 rb_{\kw{prog}}(x^l) = w \st (l, w) \in \absW(c)
 \]
% Because
% the vertices in the two graphs share the same unique label, the line number of the same command,
% we define 
%
We prove that this 
% symbolic expression is the upper bound for $x^l$'s 
arithmetic expression for $x^l \in \progV(c)$ is a sound upper bound of 
the maximum visiting times of $x^l$ over all execution traces of $c$, with the full proof in the appendix.
 \begin{thm}[Soundness of the Reachability Bounds Estimation]
 \label{thm:addweight_soundness}
 Given a program ${c}$ with its estimated weight $\progW(c)$
% program-based dependency graph 
% $\progG = (\progV, \progE, \progW, \progF)$,
 % $\traceG = (\traceV, \traceE, \traceW, \traceF)$, 
 we have:
 %
 \[
 \forall x^l \in \lvar(c), \vtrace_0 \in \mathcal{T}_0(c), \trace \in \mathcal{T},
 v \in \mathbb{N}
 \st 
 % \max \left\{ 
 % \vcounter(\vtrace') l ~ \middle\vert~
 % \forall \vtrace, \trace' \in \mathcal{T} \st 
 % \config{{c}, \trace_0} \to^{*} \config{\eskip, \trace_0 \tracecat\vtrace} 
 % \land 
 \config{\vtrace_0, w} \earrow v
 \implies
 % \right\} 
 rb(x^l) \trace_0 \leq v
 \]
 \end{thm}
% appendix~\ref{apdx:reachability_soundness}. 
%
% Based on $\rb_{\kw{prog}}$, we construct a set of weights as follows. 
% This set of weights is used to in estimating the dependency graph in step3.
% $\progW(c)$ is defined
% % denoted by $\mathcal{A}_{in}$
% as follows,
% % :
% % \\
% \[\progW(c) \triangleq
% \left
% \{ (x^l, rb_{\kw{prog}}(x^l)) \mid x^l \in \lvar(c) 
% \right \}.
% \]
% \paragraph*{Example} 
% As in
% Figure~\ref{fig:twoRounds_example}(c),
% the weight for $a^5$ is $k$. which is a sound estimated weight.
% % in program-based dependency Graph as $k$ as well.
% For any initial $\trace_0 \in \mathcal{T}_0(c)$, we know $\config{\trace_0, k} \earrow \env(\trace_0) k$ and
% the weight $w_k$ for vertex $a^5$ from Figrue~\ref{fig:twoRounds_example}(b)
% $w_k(\trace_0) = \env(\trace_0) k$. 
% %
% In the same way, the weights for all the other vertices are sound.
% % for the rest weights' computation.