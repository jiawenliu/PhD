\chapter{Appendix for {\BIAREL}}
\label{AppB}

We use some abbreviations throughout in the appendix, "TS" for "To show", RTS stands
for ``remains to show", STS stands for ``suffices to
show".  "IH" for induction hypothesis, "IH2" for induction hypothesis on the second goal, which is usually when simultaneously proving multiple goals in a lemma, similar for "IH3", "IH4", and so on.

\section{\relstlc}
\label{appendixb:relstlc}
 \subsection{Syntax of \relstlc}
 \label{appendixb:relstlc-syntax}
  \input{chapters/birelcost/RSTLC/relstlc_figures}
  \clearpage
  \subsection{Metatheory of \relstlc}
  \label{appendixb:relstlc_lemma}
 \input{chapters/birelcost/RSTLC/relstlc_lemmas} 
  \clearpage

\section{\relref}
  \subsection{Syntax of \relref}
  \label{appendixb:relref_syntax}
  \input{chapters/birelcost/RSTLC/relref_figures}
  \clearpage
  \subsection{Metatheory of \relref}
   \label{appendixb:relref_lemma}
  \input{chapters/birelcost/RSTLC/relref_lemmas}
  \clearpage
  
  \section{\relinf}
\subsection{Syntax of \relinf}
 \label{appendixb:relinf_syntax}
   \input{chapters/birelcost/RSTLC/relinf_figures}
  \clearpage
  \subsection{Metatheory of \relinf}
   \label{appendixb:relinf_lemma}
   \input{chapters/birelcost/RSTLC/relinf_lemmas}
   \clearpage
   
\section{\relinfref}
 \subsection{Syntax of \relinfref}
  \label{appendixb:relinfref_syntax}
 \input{chapters/birelcost/RSTLC/relinfref_figures}
 \clearpage
 \subsection{Metatheory of \relinfref}
  \label{appendixb:relinfref_lemma}
  We also use the name RelInfRef for {\relinfref}.
 \input{chapters/birelcost/RSTLC/relinfref_lemmas}   
  \clearpage
  
\section{\Relcost}
\wq{The content of this section is from the Ph.D. thesis of Ezgi Cicek~\citet{cicek2018:thesis}.}
% \begin{itemize}
% \item The constrained type $\tcimpl{C}{\tau}$, which is eliminated with
%   the ``$\ecelim e$'' construct.
% \item The type $\achange\,\grt$ is generalized to
%   $\tch{\grt_1}{\grt_2}$ (like in {\Relcost}'s appendix), allowing us to
%   relate two expressions of two different unary types $\grt_1$ and
%   $\grt_2$, respectively. As a result of this change, \textbf{switch}
%   rule, \textbf{$\to \wexec$} subtyping rule, and some of the
%   asynchronous rules are also generalized.%  The advantage of this
%   % generalization is useful for the $\kw{2Dcount}$ example, explained
%   % in depth in Section \ref{sec:examples}.

% \end{itemize}

It first presents {\Relcost}'s syntax, typing and subtyping rules. Then,
it introduces {\relcostmin} and the embedding of {\Relcost} into
{\relcostmin}. Finally, the bidirectional system {\birelcost} is
introduced. 

\clearpage

 
The syntax is shown as follows,
\[
\begin{array}{llll}
\mbox{Arithmetic Operators} 
& \oplus_a & ::= & + ~|~ - ~|~ \times 
%
~|~ \div ~|~ \max ~|~ \min\\  
\mbox{Boolean Operators} 
& \oplus_b & ::= & \lor ~|~ \land
\\
%
\mbox{Relational Operators} 
& \sim & ::= & < ~|~ \leq ~|~ == 
\\  
%
\mbox{Label} 
& l & \in & \mathbb{N} \cup \{\lin, \lex\} 
\\ 
%
\mbox{Arithmetic Expression} 
& \aexpr & ::= & 
n \in \mathbb{N}^{\infty} ~|~ {x} ~|~ \aexpr \oplus_a \aexpr 
 ~|~ \elog \aexpr  ~|~ \esign \aexpr
\\
%
\mbox{Boolean Expression} & \bexpr & ::= & 
%
\etrue ~|~ \efalse  ~|~ \neg \bexpr
 ~|~ \bexpr \oplus_b \bexpr
%
~|~ \aexpr \sim \aexpr 
\\
%
\mbox{Expression} & \expr & ::= & v ~|~ \aexpr ~|~ \bexpr ~|~ [\expr, \dots, \expr]
\\  
%
\mbox{Value} 
& v & ::= & { n ~|~ \etrue ~|~ \efalse ~|~ [] ~|~ [v, \dots, v]}  
\\
%
\mbox{Query Expression} 
& {\qexpr} & ::= 
& { \qval ~|~ \aexpr ~|~ \qexpr \oplus_a \qexpr ~|~ \chi[\aexpr]} 
\\
%
\mbox{Query Value} & \qval & ::= 
& {n ~|~ \chi[n] ~|~ \qval \oplus_a  \qval ~|~ n \oplus_a  \chi[n]
    ~|~ \chi[n] \oplus_a  n}
    \\
\mbox{Labeled Command} 
& {c} & ::= &   [\assign {{x}}{ {\expr}}]^{l} ~|~  [\assign {{x} } {{\query(\qexpr)}}]^{l}
~|~ {\ewhile [ \bexpr ]^{l} \edo {c} }
\\
&&&
~|~ {c};{c}  
~|~ \eif([\bexpr]{}^l , {c}, {c}) 
~|~ [\eskip]^l\\ 
\mbox{Event} 
& \event & ::= & 
    ({x}, l, v, \bullet) ~|~ ({x}, l, v, \qval)  ~~~~~~~~~~~ \mbox{Assignment Event} \\
&&& ~|~(\bexpr, l, v, \bullet)   ~~~~~~~~~~~~~~~~~~~~~~~~~~~~~~~~~~ \mbox{Testing Event}
\\
\end{array}
\]
We use following notations to represent the set of corresponding terms:
\[
\begin{array}{lll}
\mathcal{V} & : & \mbox{Set of Variables}  
\\ 
%
\mathcal{VAL} & : & \mbox{Set of Values} 
\\ 
%
\mathcal{QVAL} & : & \mbox{Set of Query Values} 
\\ 
%
\cdom & : & \mbox{Set of Commands} 
\\ 
%
\eventset  & : & \mbox{Set of Events}  
\\
%
\eventset^{\asn}  & : & \mbox{Set of Assignment Events}  
\\
%
\eventset^{\test}  & : & \mbox{Set of Testing Events}  
\\
%
\ldom  & : & \mbox{Set of Labels}  
\\
%%
\mathcal{VAL}  & : & \mbox{Set of Labeled Variables}  
\\
%%
\dbdom  & : & \mbox{{Set of Databases}} 
\\
%
{\mathcal{T}} & : & \mbox{Set of Traces}
\\
%
\qdom & : & \mbox{{Domain of Query Results}}\\
\end{array}
\]
%
Expressions include
standard arithmetic (with value $n \in \mathbb{N}^{\infty}$) and boolean expression, ($\aexpr$ and $\bexpr$) and extended query expressions $\qexpr$.
A query expression $\qexpr$ can be either a simple arithmetic expression $a$, an expression of the form $\chi[\aexpr]$ where $\chi$ represents a row of the database  and  $\aexpr$ represents an index used to identify a specific attribute of the row $\chi$, a combination of two query expressions, $\qexpr \oplus_{a} \qexpr$, or a normal form $\qval$.
For example, the query expression $\chi[3] + 5$  denotes the computation that
obtains
the value in the $3$rd column of $\chi$ in one row and then add $5$ to it.
\\
%
\paragraph{Standard Expression}
The expressions are either the standard one or the extended one.
A standard expression is
% can be 
either a standard arithmetic expression or a boolean expression, or a list of expressions.
An arithmetic expression can be a constant $n$ denoting integer, a variable $x$ from some countable set $\mathcal{VAR}$, binary operation $\oplus_a$ such as addition, product, subtraction, etc, over arithmetic expressions, and also log and sign operation. 
%
A boolean expression can be either {\tt true} or {\tt false}, basic boolean connectives such as logical negation, logical and and or denoted by $\oplus_b$, and basic comparison $sym$ between arithmetic expressions, e.g., $\leq,=,<,$ etc.
Additionally, I also introduce list in expression.
Our language supports primitives for queries, 
where a specific query is specified by a query expression $\qexpr$. 
A query expression contains the necessary information for a query request, for example, 
$\chi[\aexpr]$ represents the values at a certain index $\aexpr$ in a row $\chi$ of the database. 
Query expressions combine access to the database with other expressions, 
for example, $\chi[3] + 5$ represents a query which asks the value from the column 3 of each database raw $\chi$, adds 5 to each of these values, 
and then computes the average of these values.
\paragraph{Query Expression}
The key extension is
%  language supports 
the primitive for queries, where a specific query is specified by a query expression $\qexpr$. 
A query expression contains the necessary information for a query request, 
for example, $\chi[\qexpr]$ represents the values at a certain index $\qexpr$ in a row $\chi$ of the database. 
When this expression is encapsulated by the symbol $\query$,
 $ \query(\chi[\qexpr]) $ represents a query request computation, this computation will compute the average value at certain index over each row of the database as follows,
 \[
  \query(\chi[\qexpr]) = \frac{1}{n}\sum\limits_{i = 0}^{n}\chi_i[\qexpr]
  \]
For example, the query expression $\chi[3]$  denotes the computation that
obtains
the average value in the $3$rd column of $\chi$.

Query expressions combine access to the database with other expressions, 
for example, 
$\chi[3] + 5$ represents a query that asks the value from column 3 of each database raw $\chi$, 
adds 5 to each of these values, and then computes the average of these values as follows, where $n$ is 
database $\chi$'s number of raw.
%
\[
  \query(\chi[3] + 5) = \frac{1}{n}\sum\limits_{i = 0}^{n}\chi_i[3] + 5
  \]


% %
\paragraph{Labeled Command}
Commands are the typical ones from while languages with an additional command $\assign{x}{\query(\qexpr)}$ 
for query requests which can be used to interrogate the database and compute the linear query corresponding to $\qexpr$.
In this the command, the query expression $\qexpr$ is sent to the database server as a request.
Then the server will compute the average value of $\qexpr$ over each row of the hidden database $\chi$ and return us the result.
For instance, when we execute the command $\assign{x}{\query(\chi[3] + 5)}$,
the server will receive query request in form of $\chi[3] + 5$,
then compute the average value of $\chi[3] + 5$ over each raw of $\chi$, and return the result to us. 
The server is used as external API for computing the results over the hidden database $\chi$.
Each command is annotated with a label $l$, we will use natural numbers as labels and we will use them to record
the location of each command, so that we can uniquely identify them.


 
\paragraph{Labeled Variables}
The labeled variables and assigned variables are set of variables annotated by a label. 
We use  
$\mathcal{LV}$ represents the universe of all the labeled variables and 
$\lvar(c) \in \mathcal{P}(\mathcal{V} \times \mathcal{L}) \subseteq \mathcal{LV}$,
represents the set of labeled variables in program $c$,
defined in Definition~\ref{def:lvar}.
Labeled variables in $c$ is the set of assigned variables.
%
%
\begin{defn}[labelled Variables $\lvar$]
  \label{def:lvar}
  {
  $$
    \lvar(c) \triangleq
    \left\{
    \begin{array}{ll}
        \{{x}^l\}           
        & {c} = [{\assign x e}]^{l} 
        \\
        \{{x}^l\}            
        & {c} = [{\assign x \query(\qexpr)}]^{l} 
        \\
        \lvar(c_1) \cup \lvar(c_2) 
        & {c} = {c_1};{c_2}
        \\
        \lvar(c) \cup \lvar(c_2)
        & {c} =\eif([\bexpr]^{l}, c_1, c_2) 
        \\
        \lvar(c')
        & {c}   = \ewhile ([\bexpr]^{l}, {c}')
  \end{array}
  \right.
  $$
  }
  \end{defn}
  $FV: \expr \to \mathcal{P}(\mathcal{V})$, computes the set of free variables in an expression. To be precise,
  $FV(\aexpr)$, $FV(\bexpr)$ and $FV(\qexpr)$ represent the set of free variables in arithmetic
  expression $\aexpr$, boolean expression $\bexpr$ and query expression $\qexpr$ respectively.
  The free variables
  showing up in $c$, which aren't defined before using, are actually the input variables of this program.
  
% \begin{defn}[Assigned Variables ($\avar : \cdom \to \mathcal{P}(\mathcal{V} \times \mathbb{N})$)]
% \label{def:avar}
% {\footnotesize
% $$ \avar_{c} \triangleq
%   \left\{
%   \begin{array}{ll}
%       \{{x}^l\}                   
%       & {c} = [{\assign x e}]^{l} 
%       \\
%       \{{x}^l\}                   
%       & {c} = [{\assign x \query(\qexpr)}]^{l} 
%       \\
%       \avar_{{c_1}} \cup \avar_{{c_2}}  
%       & {c} = {c_1};{c_2}
%       \\
%       \avar_{{c}} \cup \avar_{{c_2}} 
%       & {c} =\eif([\bexpr]^{l}, c_1, c_2) 
%       \\
%       \avar_{{c}'}
%       & {c}   = \ewhile ([\bexpr]^{l}, {c}')
% \end{array}
% \right.
% $$
% }
% \end{defn}
%

% \begin{defn}[labelled Variables $\lvar$]
% \label{def:lvar}
% {\footnotesize
% $$
%   \lvar_{c} \triangleq
%   \left\{
%   \begin{array}{ll}
%       \{{x}^l\} \cup FV(\expr)^{\lin}                  
%       & {c} = [{\assign x e}]^{l} 
%       \\
%       \{{x}^l\}   \cup FV(\qexpr)^{\lin}                
%       & {c} = [{\assign x \query(\qexpr)}]^{l} 
%       \\
%       \lvar_{{c_1}} \cup \lvar_{{c_2}}  
%       & {c} = {c_1};{c_2}
%       \\
%       \lvar_{{c}} \cup \lvar_{{c_2}} \cup FV(\bexpr)^{\lin}
%       & {c} =\eif([\bexpr]^{l}, c_1, c_2) 
%       \\
%       \lvar_{{c}'} \cup FV(\bexpr)^{\lin}
%       & {c}   = \ewhile ([\bexpr]^{l}, {c}')
% \end{array}
% \right.
% $$
% }
% \end{defn}
%
We also defined the set of query variables for a program $c$,
it is the set of variables set to the result of a query in the program formally in Definition~\ref{def:qvar}.
\begin{defn}[Query Variables ($\qvar: \cdom \to \mathcal{P}(\mathcal{LV})$)] 
  \label{def:qvar}
Given a program $c$, its query variables 
$\qvar(c)$ is the set of variables set to the result of a query in the program.
It is defined as follows:
{\footnotesize
$$
  \qvar(c) \triangleq
  \left\{
  \begin{array}{ll}
      \{\}                  
      & {c} = [{\assign x \expr}]^{l} 
      \\
      \{{x}^l\}                  
      & {c} = [{\assign x \query(\qexpr)}]^{l} 
      \\
      \qvar(c_1) \cup \qvar(c_2)  
      & {c} = {c_1};{c_2}
      \\
      \qvar(c_1) \cup \qvar(c_2) 
      & {c} =\eif([\bexpr]^{l}, c_1, c_2) 
      \\
      \qvar(c')
      & {c}   = \ewhile ([\bexpr]^{l}, {c}')
\end{array}
\right.
$$
}
\end{defn}
%
It is easy to see that a program $c$'s query variables is a subset of 
its labeled variables, $\qvar(c) \subseteq \lvar(c)$.
%
%
Every labeled variable in a program is unique, formally as follows with proof in Appendix~\ref{apdx:lvar_unique}.
\begin{lem}[Uniqueness of the Labeled Variables]
  \label{lem:lvar_unique}
  Given any program $c \in \cdom$ with its label variable set $\lvar(c)$,
  every two labeled variables in this set are different.
  \[
    \forall c \in \cdom, x^i, y^j \in \lvar(c) \sthat i \neq j.
    \]
\end{lem}



% \begin{figure}
%   \centering
%   \[ \begin{array}{@{}l@{~~}l@{~}l@{~}l@{}l}
%        \text{meta~variables} & M & \bnfdef & i,n,m, \cdots \\
%        \text{index~terms} & I,k,t & \bnfdef & \cdots~|~ M \\
%        \text{meta~contexts} & \psi &  \bnfdef & \emptyset~ |~ \psi, M : \sort\\
%        \text{meta~substitutions} & \theta &  \bnfdef & []~ |~ \theta[M \mapsto I]\\
%        \text{constraints} & C &  \bnfdef & \cdots ~ |~ \cforallS{i}{S}{C} ~|~ \cexistsK{i}{S}{C} \\
%        \text{expressions} & e &  \bnfdef & \cdots ~ |~ \eanno{e}{\grt}{k}{t} ~|~ \eannobi{e}{\tau}{t} \\
%      \end{array}\]
%    \caption{Extended {\tnamemin} syntax for {\bitname}}
%  \end{figure}


\input{chapters/birelcost/appendix/types-wf}
\input{chapters/birelcost/appendix/constraint-gen}

%\input{subtyping-rules-defs}
\input{chapters/birelcost/appendix/subtyping-rules}


%\input{typing-rules-defs}
\input{chapters/birelcost/appendix/typing-rules}

% Elaboration
%\input{elaboration-rules-defs} 
\input{chapters/birelcost/appendix/elaboration-rules} 

% RelCost Core System
%\input{type-eq-rules-defs}
\input{chapters/birelcost/appendix/type-eq-rules}
%\input{typing-rules-min-defs}
\input{chapters/birelcost/appendix/typing-rules-min}

\clearpage
% Algorithmic System
%\input{alg-typing-rules-defs}
\input{chapters/birelcost/appendix/alg-typing-rules}
%\input{alg-subtyping-rules}
%\input{alg-type-eq-rules-defs}
\input{chapters/birelcost/appendix/alg-type-eq-rules} 
\input{chapters/birelcost/appendix/ty-anno-erasure}
\clearpage

\subsection{Metatheory}
% \input{subtyping-proj}
% \input{substitution-lemma}
% \input{subtyping-refl}
% \input{alg-trans}
% \input{transitivity}


\input{chapters/birelcost/appendix/subtyping-elab}

\input{chapters/birelcost/appendix/alg-binary-type-eq-reflexivity}
\input{chapters/birelcost/appendix/alg-unary-subtyping-reflexivity}
\input{chapters/birelcost/appendix/alg-unary-subtyping-transitivity}


\input{chapters/birelcost/appendix/alg-sub-soundness}
\input{chapters/birelcost/appendix/alg-sub-completeness}
 
\input{chapters/birelcost/appendix/alg-ty-eq-soundness}
\input{chapters/birelcost/appendix/alg-ty-eq-completeness}

\input{chapters/birelcost/appendix/soundness-relcostM}
\input{chapters/birelcost/appendix/completeness-relcostM}

\input{chapters/birelcost/appendix/invariant-alg-typing}
\input{chapters/birelcost/appendix/alg-soundness}
\input{chapters/birelcost/appendix/alg-completeness} 

\clearpage
%  \input{chapters/birelcost/appendix/examples} 
 
\section{\BIAREL} 
\input{chapters/birelcost/appendix/biarel}
\clearpage

% \section{Appendix Tables}
% If you plan on using tables and/or figures in an Appendix, make sure that you use \verb+\chapter{}+ instead of \verb+\chapter*{}+. This is to ensure that your figures and tables that are included in the Appendix have a leading letter for enumeration. 

% \begin{figure}[H] %"H" says to put the figure in that exact location
% \centering
% \includegraphics[width=0.7\textwidth]{./figures/proflike_example}
% \caption[External figure]{Caption line below the figure.} % Caption line pasted below puts \includegraphics[]{} the caption below the figure. 
% \label{fig:proflike}
% \end{figure}