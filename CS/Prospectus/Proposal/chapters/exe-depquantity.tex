For a program $c$, the reachability bound for every labeled variable overall $c$'s execution traces,
w.r.t. an initial trace is computed as follows,
\[
  rb(x^l) \triangleq \forall \vtrace \in \mathcal{T}_0(c), \trace' \in \mathcal{T} \st \config{{c}, \trace} \to^{*} \config{\eskip, \trace\tracecat\vtrace'} 
  \implies rb(x^l)(\trace) = \vcounter(\vtrace', l) 
  \]
When program executes under this starting trace $\trace$,
$\config{{c}, \trace} \to^{*} \config{\eskip, \trace\tracecat\vtrace'} $, it generates an execution trace $\trace'$.
This natural number is the evaluation times of the labeled command corresponding to the vertex, 
computed by the counter operator $w(\trace) = \vcounter(\vtrace', l)$.


In most data analysis programs $c$ we are interested, there are usually some user input variables, such as $k$ in $\kw{twoRounds}$. 
We denote $\mathcal{T}_0(c)$ as the set of initial traces in which all the input variables in $c$ are initialized, it is also reflected in $\traceW({c})$.    
