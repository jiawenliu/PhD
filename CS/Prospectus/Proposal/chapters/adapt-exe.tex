According to the second challenge introduced in Section~\ref{sec:adapt-challenge},
i.e., formalizing the intuitive \emph{adaptivity} quantity,
and the limitation in previous \emph{adaptivity} formalization analysis in Section~\ref{sec:prework-formalization},
I develop a new execution-based program analysis in formalizing this intuitive \emph{adaptivity} quantity
precisely and efficiently.
%

In summary, this new execution-based program analysis has three steps as follows,
\begin{enumerate}
\item The first step on \emph{dependency relation} analysis is presented in Section~\ref{sec:dynamic-datadep}.
In this step, I define the variable \emph{may-dependency} relation based on the trace semantics in Section~\ref{sec:language-os}.
\item In the second step in Section~\ref{sec:dynamic-reachability}, I analyze the \emph{dependency quantity} through the methodology of execution-based reachability bound analysis.
%  As 
% %  analysis, 
% based on the \emph{dependency relation} above.
% This analysis is developed through the methodology of execution-based reachability bound analysis.
% \\
\item The last step is the intuitive \emph{adaptivity} quantity analysis presented in Section~\ref{sec:dynamic-adapt}.
According to the two analysis results above, specifically \emph{dependency relation} and \emph{dependency quantity},
I define the formal \emph{adaptivity} model in definition~\ref{def:trace_adapt} through 
constructing a dependency graph.
\end{enumerate}

\subsubsection{Data Dependency Analysis}
\label{sec:dynamic-datadep}
\paragraph*{Challenge}
In the data analysis model our programming framework supports, 
%  an \emph{analyst} asks a sequence of queries to the mechanism, and receives the answers to these queries from the mechanism. In this model, the adaptivity we are interested in is the length of the longest sequence of such adaptively chosen queries, among all the queries the data analyst asks. 
  we define that a query is adaptively chosen when it is affected by answers of previous queries. The next thing is to decide how do we define whether one query is "affected" by previous answers, with the limited information we have? As a reminder, 
 when the analyst asks a query, the only known information will be the answers to previous queries and the current execution trace of the program.


There are two possible situations that a query will be "affected",  
either when the query expression directly uses the results of previous queries (data dependency), or when the control flow of the program with respect to a query (whether to ask this query or not) depends on the results of previous queries (control flow dependency).
% As a first step, we give a definition of when one query may depend on a previous query, which is supposed to consider both control dependency and data dependency. We first look at two possible candidates:
% \begin{enumerate}
%     \item One query may depend on a previous query if and only if a change of the answer to the previous query may also change the result of the query.
%     \item One query may depend on a previous query if and only if a change of the answer to the previous query may also change the appearance of the query.
% \end{enumerate}


Since the the results of previous queries can be stored or used in variables
which aren't associated to the query request,
it is necessary to track the dependency between queries, through all the program's variables,  
and then we can distinguish variables which are assigned with query requests.
 We give a definition of when one variable \emph{may-depend} on a previous variable with two candidates.
{
\begin{enumerate}
    \item One variable may depend on a previous variable if and only if a change of the value assigned to the previous variable may also change the value assigned to the variable.
    \item One variable may depend on a previous variable if and only if a change of the value assigned to the previous variable may also change the appearance of the assignment command to this variable 
    % in\wq{during?} 
    during execution.
\end{enumerate}
}
%   The first candidate works well by witnessing the result of one query according to the change of the answer of another query. We can easily find that the two queries have nothing to do with each other in a simple example   

{   
% The first situations works well by witnessing the result assigned to variable 
% according to the change of the value assigned to another query. 
% We can easily find that the two queries have nothing to do with each other in a simple example 
% In the first one, by defining the dependency as
The first definition is defined as
% witnessing 
% the query expressions equivalence (or the value equality for non-query assignment )
the witness of a variation on the value assigned to the same variable through two executions,
% assigned to the same variable through two executions, 
according to the change of the value assigned to another variable in pre-trace.
% the situation of data-dependency works well. \wq{long sentence, make it short?}
In particular for query requests, the variation we observe is on the query value instead of on the query requesting results.
% We can find that two queries 
% % have nothing to do with each other in this simple example 
% % depends on each other\wq{not each other, one direction.} 
% satisfy this definition
In 
%this 
the simple program $c_1 =\assign{x}{\query(\chi[2])} ;\assign{y}{\query(\chi[3] + x)}$.
 %
 From our perspective, $\query(\chi[1])$ is different from $\query(\chi[2]))$. Informally, we think $\query(\chi[3] + x)$ may depend on the query $\query(\chi[2]))$, because equipped function of the former $\chi[3] + x$ may depend on the data stored in x assigned with the result of $\query(\chi[2]))$, according to this definition. }
%
% in this example: $c_1 = \assign{x}{\query(0)}; \assign{z}{\query(\chi[x])}$.
% This candidate definition works well 
Nevertheless, the first definition fails to catch control dependency because it just monitors the changes to a query, but misses the appearance of the query when the answers of its previous queries change. 
For instance, it fails to handle $
      c_2 = \assign{x}{\query(\chi[1])} ; \eif( x > 2 , \assign{y}{\query(\chi[2])}, \eskip )
   $, but the second definition can. However, it only considers the control dependency and misses the data dependency. This reminds us to define a \emph{may-dependency} relation between labeled variables by combining the two definitions to capture the two situations.
%
%
%
%
\paragraph{Dependency}
 To define the may dependency relation on two labeled variables, we rely on the limited information at hand - the trace generated by the operational semantics. In this end, we first define the \emph{may-dependency} between events, and use it as a foundation of the variable may-dependency relation.
\begin{defn}[Events Different up to Value ($\diff$)]
  Two events $\event_1, \event_2 \in \eventset$ are  \emph{Different up to Value}, 
  denoted as $\diff(\event_1, \event_2)$ if and only if:
  \[
    \begin{array}{l}
  \pi_1(\event_1) = \pi_1(\event_2) 
  \land  
  \pi_2(\event_1) = \pi_2(\event_2) \\
  \land  
  \big(
    (\pi_3(\event_1) \neq \pi_3(\event_2)
  \land 
  \pi_{4}(\event_1) = \pi_{4}(\event_2) = \bullet )
  % \qquad \qquad 
  \lor 
  (\pi_4(\event_1) \neq \bullet
  \land 
  \pi_4(\event_2) \neq \bullet
  \land 
  \pi_{4}(\event_1) \neq_q \pi_{4}(\event_2)) 
  \big)
  \end{array}
  \]
  \end{defn}
 %
 We compare two events by defining $\diff(\event_1, \event_2)$. We use $\qexpr_1 =_{q} \qexpr_2$ and $\qexpr_1 \neq_{q} \qexpr_2$ to notate query expression equivalence and in-equivalence, distinct from standard equality. A program $c$'s
%  , its 
 labeled variables 
%  and assigned variables are subsets of 
is a subset of
the labeled variables $\mathcal{LV}$, denoted by $\lvar(c) \in \mathcal{P}(\mathcal{VAR} \times \mathcal{L}) \subseteq \mathcal{LV}$.
% annotated by a label. 
% We use  
%$\mathcal{LVAR} = \mathcal{VAR} \times \mathcal{L} $ 
% $\mathcal{LV}$ represents the universe of all the labeled variables and 
% $\avar(c) \in \mathcal{P}(\mathcal{VAR} \times \mathbb{N}) \subset \mathcal{LV}$ and 
% $\lvar(c) \in \mathcal{P}(\mathcal{VAR} \times \mathcal{L}) \subseteq \mathcal{LV}$ for them. 
We also define the set of query variables for a program $c$, $\qvar: \cdom \to 
\mathcal{P}(\mathcal{LV})$.

A program $c$'s query variables is a subset of 
its labeled variables, $\qvar(c) \subseteq \lvar(c)$. We have the operator $\tlabel : \mathcal{T} \to \ldom$, which gives the set of labels in every event belonging to the trace.
Then we introduce a counting operator $\vcounter : \mathcal{T} \to \mathbb{N} \to \mathbb{N}$, 
% \wq{which counts the occurrence of of a variable in the trace,} 
which counts the occurrence of a labeled variable in the trace,
whose behavior is defined as follows,
\[
\begin{array}{ll}
\vcounter(\trace :: (\_, l, \_, \_), l ) \triangleq \vcounter(\trace, l) + 1
&
\vcounter(\trace  ::(b, l, v, \bullet), l) \triangleq \vcounter(\trace, l) + 1
\\
\vcounter(\trace  :: (x, l, v, \qval), l) \triangleq \vcounter(\trace, l) + 1
&
% \vcounter(\trace :: (\_, l', \_, \_), l ) \triangleq \vcounter(\trace, l), l' \neq l 
% &
\vcounter(\trace  :: (x, l', v, \bullet), l) \triangleq \vcounter(\trace, l), l' \neq l
\\
\vcounter(\trace  :: (b, l', v, \bullet), l) \triangleq \vcounter(\trace, l), l' \neq l
&
\vcounter(\trace  :: (x, l', v, \qval), l) \triangleq \vcounter(\trace, l), l' \neq l
\\
\vcounter({[]}, l) \triangleq 0
\end{array}
\]
The full definitions of these above operators can be found in the appendix.
\begin{defn}[Event May-Dependency].
\label{def:event_dep}
\\ 
  An event $\event_2$ is in the \emph{event may-dependency} relation with an assignment
  event $\event_1 \in \eventset^{\asn}$ in a program ${c}$
  with a hidden database $D$ and a trace $\trace \in \mathcal{T}$ denoted as 
  %
  $\eventdep(\event_1, \event_2, [\event_1 ] \tracecat \trace \tracecat [\event_2], c, D)$, iff
  %
  \[
    \begin{array}{l}
  \exists \vtrace_0,
  \vtrace_1, \vtrace' \in \mathcal{T},\event_1' \in \eventset^{\asn}, {c}_1, {c}_2  \in \cdom  \sthat
  \diff(\event_1, \event_1') \land 
      \\ \quad
      (
        \exists  \event_2' \in \eventset \sthat 
    \left(
    \begin{array}{ll}   
   & \config{{c}, \vtrace_0} \rightarrow^{*} 
  \config{{c}_1, \vtrace_1 \tracecat [\event_1]}  \rightarrow^{*} 
    \config{{c}_2,  \vtrace_1 \tracecat [\event_1] \tracecat \vtrace \tracecat [\event_2] } 
    % 
   \\ 
   \bigwedge &
    \config{{c}_1, \vtrace_1 \tracecat [\event_1']}  \rightarrow^{*} 
    \config{{c}_2,  \vtrace_1 \tracecat[ \event_1'] \tracecat \vtrace' \tracecat [\event_2'] } 
  \\
  \bigwedge & 
  \diff(\event_2,\event_2' ) \land 
  \vcounter(\vtrace, \pi_2(\event_2))
  = 
  \vcounter(\vtrace', \pi_2(\event_2'))\\
  \end{array}
  \right)
  \\ \quad
  \lor 
  \exists \vtrace_3, \vtrace_3'  \in \mathcal{T}, \event_b \in \eventset^{\test} \sthat 
  \\ \quad
  \left(
  \begin{array}{ll}   
    & \config{{c}, \vtrace_0} \rightarrow^{*} 
      \config{{c}_1, \vtrace_1 \tracecat [\event_1]}  \rightarrow^{*} 
      \config{c_2,  \vtrace_1 \tracecat [\event_1] \tracecat \trace \tracecat [\event_b] \tracecat  \trace_3} 
    \\ 
    \bigwedge &
    \config{{c}_1, \vtrace_1 \tracecat [\event_1']}  \rightarrow^{*} 
    \config{c_2,  \vtrace_1 \tracecat [\event_1'] \tracecat \trace' \tracecat [(\neg \event_b)] \tracecat \trace_3'} 
    \\
    \bigwedge &  \tlabel_{\trace_3} \cap \tlabel_{\trace_3'} = \emptyset
     \land \vcounter(\trace', \pi_2(\event_b)) = \vcounter(\trace, \pi_2(\event_b)) 
    %   \land \event_2 \eventin \trace_3
    % \land \event_2 \not\eventin \trace_3'
    \land \event_2 \in \trace_3
    \land \event_2 \not\in \trace_3'
  \end{array}
  \right)
  )
\end{array}
   \]
% , where ${\tt label}(\event_2) = \pi_2(\event_2)$.
  %  
%
\end{defn}
% \todo{add explnanation}
% \jl{
Our event \emph{may-dependency} relation of 
two events $\event_1 \in \eventset^{\asn}$ and $\event_2 \in \eventset$, 
for a program $c$ and hidden database $D$ is w.r.t to
a trace $[\event_1 ] \tracecat \trace \tracecat [\event_2]$.
The $\event_1 \in \eventset^{\asn}$ is an assignment event because only a change on an assignment event will affect the execution trace, according to our operational semantics.
In order to observe the changes of $\event_2$ under the modification of $\event_1$, this trace 
$[\event_1 ] \tracecat \trace \tracecat [\event_2]$
starts with $\event_1$ and ends with $\event_2$.
% }
{The \emph{may-dependency} relation considers both the value dependency and value control dependency as discussed in Section~\ref{sec:design_choice}. The relation can be divided into two parts naturally in Definition~\ref{def:event_dep} (line $2-4$, $5-8$ respectively, starting from line $1$). The idea of the event $\event_1$ may depend on $\event_2$ can be briefly described:
we have one execution of the program as reference (See line $2$ and $6$, for the two kinds of dependency). 
When the value assigned to the 
% first variable 
first variable in $\event_1$ is modified, the reference trace $\trace_1 \tracecat [\event_1]$ is modified correspondingly to $\trace_1 \tracecat [\event_1']$.
We use $\diff(\event_1, \event_1')$ at line $1$ to express this modification, which guarantees that $\event_1$ and $\event_1'$ only differ in their assigned values and are equal on variable name and label. We perform a second run of the program by continuing the execution of the same program from the same execution point, 
but with the modified trace $\trace_1 \tracecat [\event_1']$ (See line $3$, $7$). 
The expected may dependency will be caught by observing two different possible changes (See line $4, 8$ respectively) when comparing the second execution with the reference one (similar definitions as in \cite{Cousot19a}). 

% \wq{
% In the first situation, we are witnessing 
In the first part (line $2-4$ of Definition~\ref{def:event_dep}), we witness
% that the value assigned to the second variable in $\event_2$
the appearance of $\event_2'$ in the second execution, and
% a variation in $\event_2$, which changes into $\event_2'$.
a variation between $\event_2$ and $\event_2'$ on their values.
% changes in $\event_2'$.
% \jl{
We have special requirement $\diff(\event_2, \event_2')$, which guarantees that they
have the same variable name and label but only differ 
% % in their assigned value. 
in their evaluated values.
% assigned to the same variable. 
In particularly for queries, if $\event_2$ and $\event_2'$ are 
% query assignment events, then 
generated from query requesting, then $\diff(\event_2, \event_2')$ guarantees that
they differ in their query values rather than the 
% query requesting value. 
query requesting results. 
Additionally, in order to handle multiple occurrences of the same event through iterations of the while loop,
 where  $\event_2$ and $\event_2'$ could be 
in different while loops,
we restrict the same occurrence of $\event_2$'s label in $\trace$ from the first execution with  the occurrence of $\event_2'$'s label in $\trace'$ from the second execution,
through $\vcounter(\vtrace, \pi_2(\event_2))
= 
\vcounter(\vtrace', \pi_2(\event_2'))$ at line $4$.
% }
% }

% \wq{
In the second part (line $5-8$ of Definition~\ref{def:event_dep}), we 
% are witnessing 
witness
the disappearance of $\event_2$ through observing the change of a testing event $\event_b$.
% In order to change the appearance of 
% % and event, the command that generating $\event_2$ must not be executed in 
% 5yhan event, 
To witness
the disappearance, the command that generates $\event_2$ must not be executed in 
the second execution. 
The only way to control whether a command will be executed, is through the change of a guard's 
evaluation result in an if or while command, which generates a testing event $\event_b$ in the first place.
So we observe when
$\event_b$ changes into $\neg \event_b$ in the second execution firstly, 
whether it follows with the disappearance of $\event_2$ in the second trace. We restrict the occurrence of $\event_b$'s label in the two traces being the same
}
% s to the occurrence times of $\event_2'$'s label in the second trace,
through $\vcounter(\trace', \pi_2(\event_b)) = \vcounter(\trace, \pi_2(\event_b))$ to handle the while loop.
% changes in $\event_2'$, have the same variable and label and only differ in their assigned value. 
Again, for queries, we observe the disappearance based on the query value equivalence.
% if $\event_2$ and $\event_2'$ are query assignment events, then 
% they differ in their query value rather than the assigned value. 
% }
%
% \mg{I don't understand this explanation. What are the ``assignment commands associated to the two labelled variables''}
% \jl{revised but need more think}
% Explanation: 

{Considering 
% a program's all possible executions 
all events generated during a program's executions
under an initial trace,
% among all events generated during these executions
% and the variables and labels of these events are 
% corresponding to the two labeled variables,
% evaluations of the assignment commands associated to the two labelled variables respectively, 
as long as there is one pair of events satisfying the \emph{event may-dependency} relation in Definition~\ref{def:event_dep}, 
 we say the two 
related
variables satisfy the \emph{variable may-dependency} relation, in Definition~\ref{def:var_dep}.
}

\begin{defn}[Variable May-Dependency].
  \label{def:var_dep}
  \\
  A variable ${x}_2^{l_2} \in \lvar(c)$ is in the \emph{variable may-dependency} relation with another
  variable ${x}_1^{l_1} \in \lvar(c)$ in a program ${c}$, denoted as 
  %
  $\vardep({x}_1^{l_1}, {x}_2^{l_2}, {c})$, if and only if.
\[
  \begin{array}{l}
\exists \event_1, \event_2 \in \eventset^{\asn}, \trace \in \mathcal{T} , D \in \dbdom \sthat
% (\pi_{1}{(\event_1)}, \pi_{2}{(\event_1)}) = ({x}_1, l_1)
% \land
% (\pi_{1}{(\event_2)}, \pi_{2}{(\event_2)}) = ({x}_2, l_2)
\pi_{1}{(\event_1)}^{\pi_{2}{(\event_1)}} = {x}_1^{l_1}
\land
\pi_{1}{(\event_2)}^{\pi_{2}{(\event_2)}} = {x}_2^{l_2}% \\ \quad 
\land 
\eventdep(\event_1, \event_2, \trace, c, D) 
  \end{array}
\]  %
\end{defn}

\subsubsection{Data Dependency Quantity Analysis}
\label{sec:dynamic-reachability}
For a program $c$, there are two data \emph{dependency quantities} we are considering.
The first quantity is the reachability times of each labeled variable during the program execution.
The second quantity is the reachability time for every pair of labeled variables with variable \emph{may-dependency} relation.
% \paragraph*{Variable Reachability}
\paragraph{The Dependency Quantity for Labeled Variables}
The reachability time of a labeled variable indicates the evaluation times of the assignment command assigning a value to this variable.  
\begin{defn}[Reachability Time of Labeled Variable]
  \label{def:adapt-var_reachability}
The reachability for every labeled variable overall $c$'s execution traces,
w.r.t. an initial trace $\vtrace \in \mathcal{T}_0(c)$ is defined as follows,
\[
  rb(x^l) \triangleq \forall \vtrace \in \mathcal{T}_0(c), \trace' \in \mathcal{T} \sthat \config{{c}, \trace} \to^{*} \config{\eskip, \trace\tracecat\vtrace'} 
  \implies w(\trace) = \vcounter(\vtrace', l) 
  \]
\end{defn}
%
$(x^l, w) \in \mathcal{LV} \times (\mathcal{T} \to \mathbb{N})$,
with a labeled variable as first component and
its weight $w$ the second component.
Weight $w$ for
% a labeled variable 
$x^l$ is a function $w : \mathcal{T} \to \mathbb{N}$
mapping from a starting trace to a natural number.
When program executes under this starting trace $\trace$,
$\config{{c}, \trace} \to^{*} \config{\eskip, \trace\tracecat\vtrace'} $, it generates an execution trace $\trace'$.
This natural number is the evaluation times of the labeled command corresponding to the vertex, 
computed by the counter operator $w(\trace) = \vcounter(\vtrace', l)$.


In most data analysis programs $c$ we are interested, there are usually some user input variables, such as $k$ in $\kw{twoRounds}$. 
We denote $\mathcal{T}_0(c)$ as the set of initial traces in which all the input variables in $c$ are initialized, it is also reflected in $\traceW({c})$.    
%
\paragraph{Dependency Quantity for the Pair of Labeled Variables}
% \paragraph*{Dependent Variables Reachability}
%
% For a program $c$ I compute the reachability bound for every labeled variable overall $c$'s execution traces,
% w.r.t. an initial trace as follows,
\begin{defn}[Reachability Time of Dependent Variables]
  \label{def:adapt-depvar_reachability}
  The execution-based reachability time for every pair of 
  labeled in the
  \emph{may-dependency} relation w.r.t. an initial trace. Formally as follows,
    \[
    \begin{array}{l}
        rb(x^i, y^j) \triangleq 
%   x^i, y^j \in \lvar(c)
%   \land w \in \mathcal{P}( \mathcal{T}_0(c) \to \mathbb{N})
%   \land 
%   \exists \trace \in \mathcal{T}_0(c), 
%   \trace_1, \trace_2 \in \mathcal{T} \sthat \dep(x^i, y^j,\trace_1, \trace_2, \trace_0, c)
%   \\
%   \land 
\forall \trace_0 \in \mathcal{T}_0(c) \sthat
  w (\trace_0) = \max \left\{ | \sdiff(\trace_1, \trace_2, y)|
  ~\middle\vert~
  \forall \trace_1, \trace_2 \in \mathcal{T} \sthat \dep(x^i, y^j,\trace_1, \trace_2, \trace_0, c) \right\}
\end{array}
\]
\end{defn}
%
For any pair of labeled variable $(x^i, y^j) \in \ldom$, 
$ rb(x^i, y^j)$ is a function $w: \mathcal{T}_0(c) \to \mathbb{N}$,
    where given an initial trace $\trace_0$,
    it is the maximum length of the difference sequence between all pairs of the witness traces $\trace_1, \trace_2$ 
    satisfying the dependency relation.

    \highlight{\paragraph*{Improvements Analysis}
    Previous works do not have any quantity analysis on the dependency relation.
    Comparing to them, this part is stronger in following senses.
    % It is more scalable to general program, and it provides the program with preciser formal definition for \emph{Adaptivity} than previous definition,
    % specifically as follows.
    % language and operational semantics design improves the expressiveness, efficiency, and the accuracy to a large extend.
    \todo{Add details}
    \begin{itemize}
      \item \textbf{Improvements on Efficiency}
      \\
      It is also efficient.
      \item \textbf{Improvements on Accuracy}
      This quantity analysis can help to improve the precision of the adaptivity formalization.
      \end{itemize}
      }

\paragraph*{The Dependency Quantity through The Two Rounds Example}
\begin{example}[Variable \emph{May-Dependency} Quantity in The Two Rounds Data Analysis Example Program]
    In the same $\kw{towRounds(k)}$ example Program,    the analyst asks in total $k+1$ queries to the mechanism in two phases.
    %
    \[         \begin{array}{l}
      \kw{towRounds(k)} \triangleq \\
             \clabel{ \assign{a}{0}}^{0} ;
              \clabel{\assign{j}{k} }^{1} ;\\
              \ewhile ~ \clabel{j > 0}^{2} ~ \edo ~
              \Big(
               \clabel{\assign{x}{\query(\chi[j] \cdot \chi[k])} }^{3}  ;
               \clabel{\assign{j}{j-1}}^{4} ;
              \clabel{\assign{a}{x + a}}^{5}       \Big);\\
              \clabel{\assign{l}{\query(\chi[k]*a)} }^{6}
          \end{array}
          \]    %
    % Queries are of the form $q(e)$ where $e$ is an expression with a special variable $\chi$ representing a possible row. Mainly $e$ represents a function from $X$ to some domain $U$, for example $U$ could be $[-1,1]$ or $[0,1]$. This function characterizes the linear query I are interested in running. As an example, $x \leftarrow q(\chi[2])$ computes an approximation, according to the used mechanism, of the empirical mean of the second attribute, identified by $\chi[2]$. Notice that I don't materialize the mechanism but I assume that it is implicitly run when I execute the query. 
    % \jl{We use $\chi$ to abstract a possible row in the database and }
    % queries are of the form $\query(\qexpr)$, where $\qexpr$ is a special expression 
    With the initial trace
    $[(k, in, 2, \bullet)]$ and following execution trace, 
    \\
    $
    \trace_1 \triangleq 
    \left[\begin{array}{l}
    % \trace_0 \tracecat
     (a, 0, 0, \bullet),
    (j, 1, 2, \bullet),
    (j>0, 2, \etrue, \bullet),
    (x, 3, v_1, \chi[2]*\chi[2]),
    (j, 4, 1, \bullet),
    (a, 5, v_1, \bullet),\\
    (j>0, 2, \etrue, \bullet),
    (x, 3, v_2, \chi[1]*\chi[2]),
    (j, 4, 0, \bullet),
    (a, 5, v_1 + v_2, \bullet),
    (j>0, 4, \efalse, \bullet),\\
    (l, 6, v_3, \chi[2]*( v_1 + v_2))
    \end{array} \right]
    $.
    Based on these observations, we analyze the \emph{may-dependency} quantity for every labeled variable,
    and pairs of dependent variables as follows.
\begin{itemize}
    \item \textbf{The Dependency Quantity for Labeled Variables}
    \\
    For the specific two execution traces above,
    the \emph{may-dependency} quantity for every variable
    is computed as follows,
   %   where $k$ is the 
   %  initial value of input variable $k$ given by user,
   %  we observe the execution trace as
   \\
   $rb(a^0) ((k, in, 2, \bullet))  = \vcounter(\trace_1) = 1$
   \\
   $\cdots$
  \\
   $rb(x^3) ((k, in, 2, \bullet))  = \vcounter(\trace_1) = 2$
    \\
    $\cdots$
    \\

    Then, for arbitrary initial trace $\trace_0 \in \mathcal{T}_0(\kw{twoRounds(k)})$,
    the \emph{may-dependency} quantity for every variable under $\trace_0$ is a function
    as follows,
    \\
    $rb(a^0) (\trace_0)  = 1$
    \\
    $\cdots$
    \\
    $rb(x^3) (\trace_0)  = \max\{0, \env(\trace_0) k \} $
    \item \textbf{Dependency Quantity for the Pair of Labeled Variables}
    For the specific two execution traces above,
    the \emph{may-dependency} quantity for every variable
    is computed as follows,
   %   where $k$ is the 
   %  initial value of input variable $k$ given by user,
   %  we observe the execution trace as
   \\
   $rb(a^0) ((k, in, 2, \bullet))  = \vcounter(\trace_1) = 1$
   \\
   $\cdots$
  \\
   $rb(x^3) ((k, in, 2, \bullet))  = \vcounter(\trace_1) = 2$
    \\
    $\cdots$
    \\

    Then, for arbitrary initial trace $\trace_0 \in \mathcal{T}_0(\kw{twoRounds(k)})$,
    the \emph{may-dependency} quantity for every variable under $\trace_0$ is a function
    as follows,
    \\
    $rb(a^0) (\trace_0)  = 1$
    \\
    $\cdots$
    \\
    $rb(x^3) (\trace_0)  = \max\{0, \env(\trace_0) k \} $
\end{itemize}
    % \\
    % We modify the value assigned to $x$ when evaluating the command $ \clabel{\assign{x}{\query(\chi[j] \cdot \chi[k])} }^{3}$
    % in the first iteration.
    % By manipulating the event in the trace, 
    % the event $(x, 3, v_1, \chi[2]*\chi[2])$
    % is modified into $(x, 3, v_1', \chi[2]*\chi[2])$ where $v_1 \neq v_1'$.
    % Then, through executing the program from the execution point after executing line $3$, we observe another execution trace as follows,
    % \\
    % $
    % \left[\begin{array}{l}
    % % \trace_0 \tracecat
    %  (a, 0, 0, \bullet),
    % (j, 1, 2, \bullet),
    % (j>0, 2, \etrue, \bullet),
    % \highlight{(x, 3, v_1', \chi[2]*\chi[2])},
    % (j, 4, 1, \bullet),
    % (a, 5, v_1, \bullet),\\
    % (j>0, 2, \etrue, \bullet),
    % (x, 3, v_2, \chi[1]*\chi[2]),
    % (j, 4, 0, \bullet),
    % \highlight{(a, 5, v_1' + v_2, \bullet),}
    % (j>0, 4, \efalse, \bullet),\\
    % (l, 6, v_3, \chi[2]*( v_1' + v_2))
    % \end{array} \right]
    % $.   
    % \\
    % In this trace, the event $(a, 5, v_1' + v_2, \bullet),$  is different from $(a, 5, v_1 + v_2, \bullet),$ in the first 
    % trace.
    % \\
    % This change satisfies the Definition~\ref{def:event_dep}, so there exists the variable \emph{may-dependency} relation 
    % between variable $x^3$ and $a^5$.
\end{example}%
\subsubsection{Adaptivity Formalization}
\label{sec:dynamic-adapt}

Based on the variable \emph{may-dependency} relation in Section~\ref{subsec:dynamic-datadep} and 
the dependency quantity analysis in Section~\ref{subsec:dynamic-reachability}.
% gives us the edges, 
I firstly define the execution-based dependency graph, then formalize the \emph{adaptivity} in this section.
% \wq{Just a few sentences here, some overview of this subsection. See 4.2 for instance.}
\paragraph{Execution Based Dependency Graph}
\label{para:execution-base-graph-def}
Based on the variable \emph{may-dependency} relation,
% gives us the edges, 
we define the execution-based dependency graph.
% \wq{Just a few sentences here, some overview of this subsection. See 4.2 for instance.}
\begin{defn}[Execution Based Dependency Graph]
\label{def:trace_graph}
Given a program ${c}$,
its \emph{execution-based dependency graph} 
$\traceG({c}) = (\traceV({c}), \traceE({c}), \traceW({c}), \traceF({c}))$ is defined as follows,
{
  \small
\[
\begin{array}{rlcl}
  \text{Vertices} &
  \traceV({c}) & := & \left\{ 
  x^l \in \mathcal{LV}
  ~ \middle\vert ~ x^l \in \lvar(c)
  \right\}
  \\
  \text{Directed Edges} &
  \traceE({c}) & := & 
  \left\{ 
  (x^i, y^j) 
%   \in \mathcal{LV} \times \mathcal{LV}
  ~ \middle\vert ~
  x^i, y^j \in \lvar(c) \land \vardep(x^i, y^j, c) 
  % \text{\mg{$\land$ instead of ,}}
  \right\}
  \\
  \text{Weights} &
  \traceW({c}) & := & 
%   \left
  \{ 
  (x^l, w) 
  % \in \mathcal{LV} \times \mathbb{N}
  ~ \vert ~ 
  w : \mathcal{T} \to \mathbb{N}
  \land
  x^l \in \lvar(c) 
  \\ & & &
  \land
  % n = \max \left\{ 
    % ~ \middle\vert~
  \forall \vtrace \in \mathcal{T}_0(c), \trace' \in \mathcal{T} \sthat \config{{c}, \trace} \to^{*} \config{\eskip, \trace\tracecat\vtrace'} 
  \implies w(\trace) = \vcounter(\vtrace', l) 
  %  \right\}
%   \right
\}
  \\
  % \text{Query Label} &
  \text{Query Annotation} &
  \traceF({c}) & := & 
\left\{(x^l, n)  
% \in  \mathcal{LV}\times \{0, 1\} 
~ \middle\vert ~
 x^l \in \lvar(c) \land
n = 1 \Leftrightarrow x^l \in \qvar(c) \land n = 0 \Leftrightarrow  x^l \notin \qvar(c)
\right\}
\end{array}.
\]
}
\end{defn}
%
There are four components of the execution-based dependency graph. 
The vertices $\traceV(c)$ is the set of program $c$'s labeled variables $\lvar(c)$,
which are statically collected.
The query annotation is 
a set of pairs $\traceF(c) \in \mathcal{P}(\mathcal{LV} \times \{0, 1\} )$ 
mapping each $x^l \in \traceV(c)$ to $0$ or $1$, 
indicating whether this labeled variable is in program $c$'s query variable set $\qvar(c)$.
{
The weights is a set of pairs, $(x^l, w) \in \mathcal{LV} \times (\mathcal{T} \to \mathbb{N})$,
with a labeled variable as first component and
its weight $w$ the second component.
Weight $w$ for
% a labeled variable 
$x^l$ is a function $w : \mathcal{T} \to \mathbb{N}$
mapping from a starting trace to a natural number.
When program executes under this starting trace $\trace$,
$\config{{c}, \trace} \to^{*} \config{\eskip, \trace\tracecat\vtrace'} $, it generates an execution trace $\trace'$.
This natural number is the evaluation times of the labeled command corresponding to the vertex, 
computed by the counter operator $w(\trace) = \vcounter(\vtrace', l)$.
We can see in the execution-based dependency graph of $\kw{twoRounds}$ in Figure~\ref{fig:overview-example}(b), the weight of vertices in the while loop is  $\env(\trace) k$, which depends on the value of the user input $k$ specified in the starting trace $\tau$.
The directed edges $\traceE({c})$ is also a set of pairs with two labeled variables $ (x^i, y^j) \in \mathcal{LV} \times \mathcal{LV}$, from $x^i$ pointing to $y^j$ in the graph.
The edges are constructed directly from our variable may-dependency relation. 
For any two vertices $x^{i}$ and $y^{j}$ in $\traceV(c)$, if they satisfy the variable may-dependency relation $\vardep(x^i, y^j, c)$, there is a direct edge between the two vertices in our execution-based dependency graph for program $c$.
} 
In most data analysis programs $c$ we are interested, there are usually some user input variables, such as $k$ in $\kw{twoRounds}$. 
We denote $\mathcal{T}_0(c)$ as the set of initial traces in which all the input variables in $c$ are initialized, it is also reflected in $\traceW({c})$.    

\paragraph{Trace-based Adaptivity}

% \wq{
% Given 
% a program $c$'s execution-based dependency graph 
% % $G_{trace}(c)(\trace) = (\vertxs, \edges, \weights, \qflag)$,
% $\traceG({c}) = (\traceV({c}), \traceE({c}), \traceW({c}), \traceF({c}))$
% we define adaptivity 
% with respect to $\trace$ by the finite walk in the graph, which has the most query requests along the walk.
% }

Given 
a program $c$'s execution-based dependency graph 
% $G_{trace}(c)(\trace) = (\vertxs, \edges, \weights, \qflag)$,
$\traceG({c})$,
we define adaptivity 
with respect to an initial trace $\trace_0 \in \mathcal{T}_0(c)$ by the finite walk in the graph, which has the most query requests along the walk.
We show the definition of a finite walk as follows.
%
% The query length of a walk $k$ is the number of vertices which correspond to query variables in the vertices sequence of this walk. 
% Instead of counting all 
% the vertices in $k$'s vertices sequence, i

\begin{defn}[Finite Walk (k)].
  \label{def:finitewalk}
  \\
%   Given a program $c$'s execution-based dependency graph $\traceG({c})(\trace)$, 
%   a \emph{finite walk} $fw$ in $\traceG({c})(\trace)$ is a sequence of edges $(e_1 \ldots e_{n - 1})$ 
%   for which there is a sequence of vertices $(v_1, \ldots, v_{n})$ such that:
%   \begin{itemize}
%       \item $e_i = (v_{i},v_{i + 1})$ for every $1 \leq i < n$.
%       \item every vertex $v \in \traceV({c}) $ appears in $(v_1, \ldots, v_{n})$ at most 
%       \wq{$\traceW({c})(\trace)$} times.  
%   \end{itemize}
%   %
%   The length of $fw$ is the number of vertices in its vertex sequence, i.e., $\len(k) = n$.
  Given the execution-based dependency graph $\traceG({c}) = (\traceV({c}), \traceE({c}), \traceW({c}), \traceF({c}))$ of a program $c$,
  a \emph{finite walk} $k$ in $\traceG({c})$ is a 
  function $k: \mathcal{T} \to $ sequence of edges.
  For a initial trace $\trace_0 \in \mathcal{T}_0(c)$, 
  $k(\trace_0)$ is a sequence of edges $(e_1 \ldots e_{n - 1})$ 
  for which there is a sequence of vertices 
  $(v_1, \ldots, v_{n})$ such that:
  \begin{itemize}
      \item $e_i = (v_{i},v_{i + 1}) \in \traceE(c)$ for every $1 \leq i < n$.
      \item every $v_i \in \traceV(c)$
      and $(v_i, w_i) \in \traceW(c)$, 
       $v_i$ appears in $(v_1, \ldots, v_{n})$ at most 
    %   \wq{$\traceW({c})(\trace)$} 
    $w(\trace_0)$
      times.  
  \end{itemize}
  %
  The length of $k(\trace_0)$ is the number of vertices in its vertices sequence, i.e., $\len(k)(\trace_0) = n$.
 \end{defn}

We use $\walks(\traceG(c))$ to denote 
% \mg{``the set'', not ``a set''}a set containing all finite walks $k$ in $G$;
the set containing all finite walks $k$ in $\traceG(c)$;
and $k_{v_1 \to v_2} \in \walks(\traceG(c))$ with $v_1, v_2 \in \traceV(c)$ denotes the walk from vertex $v_1$ to $v_2$ . 
\\
We are interested in queries, so we need to recover the 
variables corresponding to queries from the walk. We define the query length of a walk, 
instead of counting all 
the vertices in $k$'s vertices sequence, we just count the number of vertices which correspond to query variables in this sequence.
%
% \mg{I don't understand this definition. Is wrt a single query?if yes, who is chosing the query? Or is it any query?}
% \jl{It is for any query, as long as the vertex is a query variable, in another worlds, this length just counting the number of query variables in the walk, instead of counting all 
% the vertices.}
% \todo{Make the definition clear}
\begin{defn}[Query Length of the Finite Walk($\qlen$)].
\label{def:qlen}
\\
% Given 
% % labelled weighted graph $G = (\vertxs, \edges, \weights, \qflag)$, 
% a program $c$'s execution-based dependency graph $\traceG(c)(\trace)$
%  and a \emph{finite walk} $k$ in $\traceG(c)(\trace)$ with its vertex sequence $(v_1, \ldots, v_{n})$, 
% %  the length of $k$ w.r.t query is defined as:
% The query length of $k$ is the number of vertices which correspond to query variables in $(v_1, \ldots, v_{n})$ as follows, 
% \[
%   \qlen(k) = \len\big( v \mid v \in (v_1, \ldots, v_{n}) \land \qflag(v) = 1 \big)
% \]
% , where $\big(v \mid v \in (v_1, \ldots, v_{n}) \land \qflag(v) = 1 \big)$ is a subsequence of $(v_1, \ldots, v_{n})$.
Given 
% labelled weighted graph $G = (\vertxs, \edges, \weights, \qflag)$, 
the execution-based dependency graph $\traceG({c}) = (\traceV({c}), \traceE({c}), \traceW({c}), \traceF({c}))$ of a program $c$,
 and a \emph{finite walk} 
%  $k$ in $\traceG(c)(\trace)$
 $k \in \walks(\traceG(c))$. 
%  with its vertex sequence $(v_1, \ldots, v_{n})$, 
%  the length of $k$ w.r.t query is defined as:
The query length of $k$ is a function $\qlen(k): \mathcal{T} \to \mathbb{N}$, such that with an initial trace  $\trace_0 \in \mathcal{T}_0(c)$, $\qlen(k)(\trace_0)$ is
the number of vertices which correspond to query variables in the vertices sequence of the walk $k(\trace_0)$
$(v_1, \ldots, v_{n})$ as follows, 
\[
  \qlen(k)(\trace_0) = |\big( v \mid v \in (v_1, \ldots, v_{n}) \land \qflag(v) = 1 \big)|.
\]
\end{defn}
The definition of adaptivity is then presented in Def~\ref{def:trace_adapt} below.

\begin{defn}
  [Adaptivity of a Program].
  \label{def:trace_adapt}
  \\
  Given a program ${c}$, 
  its adaptivity $A(c)$ is function 
  $A(c) : \mathcal{T} \to \mathbb{N}$ such that for an
  % with respect to a starting trace $\trace$ 
  initial trace $\trace_0 \in \mathcal{T}_0(c)$, 
  % is defined as follows:
  %
 $$
  A(c)(\trace_0) = \max \big 
  \{ \qlen(k)(\trace_0) \mid k \in \walks(\traceG(c)) \big \} $$
  \end{defn}%
%
% \subsection{Adaptivity through An Example}
% \label{subsec:dynamic-examples}
% \begin{example}[twoRounds]
    In this example program $\kw{towRounds(k)}$, the analyst asks in total $k+1$ queries to the mechanism in two phases.
    In the first phase, the analyst asks $k$ queries and stores the answers that are provided by the mechanism. 
    In the second phase, the analyst constructs a new query based on the results of the previous $k$ queries and sends this query to the mechanism. More specifically, we assume that, in this example, the domain $\dbdom$ 
    contains at least $k$ numeric attributes, which we index just by natural numbers. 
    The queries inside the while loop correspond to the first phase and compute an approximation of 
    the product of the empirical mean of the first $k$ attributes. 
    The query outside the loop corresponds to the second phase and computes an approximation of the empirical mean where each record is weighted by the sum of the empirical mean of the first $k$ attributes.
    %
    % Queries are of the form $q(e)$ where $e$ is an expression with a special variable $\chi$ representing a possible row. Mainly $e$ represents a function from $X$ to some domain $U$, for example $U$ could be $[-1,1]$ or $[0,1]$. This function characterizes the linear query we are interested in running. As an example, $x \leftarrow q(\chi[2])$ computes an approximation, according to the used mechanism, of the empirical mean of the second attribute, identified by $\chi[2]$. Notice that we don't materialize the mechanism but we assume that it is implicitly run when we execute the query. 
    % \jl{We use $\chi$ to abstract a possible row in the database and }
    % queries are of the form $\query(\qexpr)$, where $\qexpr$ is a special expression 
    %
    {Since statistical query computes the empirical mean of a function on rows, we use $\chi$ to abstract a possible row in the database and }
    queries are of the form $\query(\qexpr)$, where $\qexpr$ is a special expression 
    (as in our syntax in Section~\ref{sec:language})
    {
    % from $X$ to some domain $U$, 
    % for example $U$ 
      We use $U$ to denote the co-domain of queries, and it could be $[-1,1]$, $[0,1]$ or $[-R,+R]$, for some $R$ we consider.
      This function characterizes the linear query we are interested in running. 
      As an example, $x \leftarrow \query(\chi[j] \cdot \chi[k])$ computes an approximation, according to the used mechanism, of the empirical mean of the product of the $j^{th}$ attribute and $k^{th}$ attribute, identified by $\chi[j] \cdot \chi[k]$. Notice that we don't materialize the mechanism but we assume that it is implicitly run when we execute the query. } 

      The graph in Figure~\ref{fig:twoRounds_example}(b). This graph is built by considering all the possible execution traces of the program in   Figure~\ref{fig:twoRounds_example}(a).
      Each vertex in this graph has a superscript representing its weight, and a subscript $1$ or $0$ telling if the vertex corresponds to a query or not. We will call this subscript a query annotation. 
      For example the vertex $l^{6}:{}^{w_1}_1$, 
      % the superscript $1$ represents the weight $1$, and the subscript for the query annotation.
      has weight $w_1$, a constant function which returns $1$ for every starting state, since 
      this query at line $6$ is at most executed once regardless of the initial trace.
      The query annotation of this vertex is $1$, which  indicates that 
      $\clabel{\assign{l}{\query(\chi[k] * a)}}^6$ is a query request.
      % The assignment in the while loop, such as node $x^{3}$, 
      Another vertex, $x^{3}:{}^{w_k}_1$, appears in the while loop. 
      It has as weight a function $w_k$ that for every initial state returns the value that $k$ has in this state, since this is also the number the while loop will be iterated. 
      The node $j^{4}:{}^{w_k}_0$ has as a subscript $0$ representing a non-query assignment.
      
      
      Since the edges between two vertices represent the fact that one program variable may depend on the other,
      % the queries that are executed and the edges between two nodes represent the fact that one query may depend on the other. 
      we can define the program adaptivity with respect to a initial trace by means of a walk traversing the graph, visiting each vertex no more than its weight with respect to the initial trace, and visiting as many query nodes as possible.
      % In the walk that passes the most times of query nodes, the total visiting times of this walk on 
      % these query nodes is defined as adaptivity.
      %
      So, looking again at our example, we can see that
      % if the input variable $k$ is less than $1$ in an initial trace $\trace_1$, then it is easy to see the weight of vertex $x^3$, $w_3(\trace_1) < 1$ and we can only find a walk with one vertex $l^{6}$, according to  the definition of finite walk in Definition~\ref{def:finitewalk}. So the adaptivity for $\trace_1$, as the number of query vertices along the walk, is $1$. It is easy to understand because when $k <1$, the while loop will not be executed and only one query is asked in total. However, in reality, people want the adaptivity of this example when $k \geq 1$. With this initial trace, it is easy to see that 
      in the walk along the dotted arrows,  $l^{6} \to a^5 \to x^3 $, there are $2$ vertices with query annotation $1$ and that this number is maximal, i.e. we cannot find another walk having more than $2$ vertices with query annotation $1$, under the assumption that $k \geq 1$. So the adaptivity of the program in Figure~\ref{fig:twoRounds_example}(a)  is $2$,
      % longest walk in the graph in Figure~\ref{fig:twoRounds_example}(b), which we mark with a red dashed arrow, is $2$, 
      as expected.
{\small
\begin{figure}
\centering
\begin{subfigure}{.2\textwidth}
\begin{centering}
$
    \begin{array}{l}
    \kw{towRounds(k)} \triangleq \\
           \clabel{ \assign{a}{0}}^{0} ;
            \clabel{\assign{j}{k} }^{1} ; \\
            \ewhile ~ \clabel{j > 0}^{2} ~ \edo ~ \\
            \Big(
             \clabel{\assign{x}{\query(\chi[j] \cdot \chi[k])} }^{3}  ; \\
             \clabel{\assign{j}{j-1}}^{4} ;\\
            \clabel{\assign{a}{x + a}}^{5}       \Big);\\
            \clabel{\assign{l}{\query(\chi[k]*a)} }^{6}\\
        \end{array}
$
\caption{}
\end{centering}
\end{subfigure}
\begin{subfigure}{.75\textwidth}
%}
\qquad
\begin{centering}
 \begin{tikzpicture}[scale=\textwidth/18cm,samples=200]
\draw[] (0, 10) circle (0pt) node
{{ $a^0: {}^{w_1}_{0}$}};
\draw[] (0, 7) circle (0pt) node
{\textbf{$x^3: {}^{w_k}_{1}$}};
\draw[] (0, 4) circle (0pt) node
{{ $a^5: {}^{w_k}_{0}$}};
\draw[] (0, 1) circle (0pt) node
{{ $l^6: {}^{w_1}_{1}$}};
% Counter Variables
\draw[] (5, 9) circle (0pt) node {\textbf{$j^1: {}^{w_1}_{0}$}};
\draw[] (5, 6) circle (0pt) node {{ $j^4: {}^{w_k}_{0}$}};
%
% Value Dependency Edges:
\draw[ ultra thick, -latex, densely dotted,] (0, 1.5)  -- 
% The Weight for this edge
node [left] {\highlight{$\trace_0 \to 1 $}}(0, 3.5) ;
\draw[ ultra thick, -latex, densely dotted,] (0, 4.5)  -- 
node [left] {\highlight{$\trace_0 \to \env(\trace_0) k $}}(0, 6.5) ;
\draw[ thick, -latex] (0, 4.5)  to  [out=-230,in=230]  
node [left] {\highlight{$\trace_0 \to \env(\trace_0) k $}}(0, 9.5) ;
\draw[ thick, -Straight Barb] (1.5, 3.5) arc (120:-200:1);
    % The Weight for this edge
    \draw[](3, 3) node [] {\highlight{$\trace_0 \to \env(\trace_0) k  $}};
\draw[ thick, -Straight Barb] (6.5, 6.5) arc (150:-150:1);
    % The Weight for this edge
    \draw[](9, 6) node [] {\highlight{$\trace_0 \to \env(\trace_0) k  $}};
\draw[ thick, -latex] (5, 6.5)  -- 
% The Weight for this edge
node [right] {\highlight{$\trace_0 \to \env(\trace_0) k $}} (5, 8.5) ;
% Control Dependency
\draw[ thick,-latex] (1.5, 7)  -- (4, 9) ;
\draw[ thick,-latex] (1.5, 4)  -- 
% The Weight for this edge
node [] {\highlight{$\trace_0 \to \env(\trace_0) k $}} (4, 9) ;
\draw[ thick,-latex] (1.5, 7)  -- (4, 6) ;
\draw[ thick,-latex] (1.5, 4)  -- (4, 6) ;
\end{tikzpicture}
\caption{}
\end{centering}
\end{subfigure}
 \caption{(a) The program $\kw{towRounds(k)}$, an example 
%  of a program 
with two rounds of adaptivity (b) The corresponding execution-based dependency graph.}
\label{fig:twoRounds_example}
\end{figure}
}
\end{example}
%

% In terms of techniques, our work relies on ideas from both static analysis and dynamic analysis. 
We discuss closely related work in both areas.


%
% \subsection{Implementation}
% \label{subsec:dynamic-implementation}
%