\highlight{Improved from previous analysis,
this new analysis algorithm combines the reaching definition analysis into the standard data flow analysis.
It designs an improved version of data flow analysis named feasible 
data flow, estimating a tighter bound on the data dependency relation. 
}
\paragraph{Standard Reaching Definition Analysis}
This analysis first performs the standard reaching definition algorithm and computes the result as $\live(l, c)$ for every label $l$ in a program $c$.
The $\live(l, c)$ represents the analysis result, which is the set of 
reachable labeled variables in program $c$ at the location of label $l$.
For every labeled variable $x^l$ in this set, 
the value assigned to that variable
in the assignment command associated with that label is reachable at the entry point of executing the command of label $l$.

\paragraph{Feasible Data-Flow Analysis}
Then this analysis performs the feasible data-flow analysis by using both control and data flow in the following $\flowsto$ relation.
% which uses the results of reaching definition analysis, as $\live(l, c)$ for every label $l$ in a program $c$. 
% $\mathsf{FV}$ computes the set of free variables in an expression. 
% $\mathsf{blk}$ returns a list of labeled computation blocks of a labeled command $c$.
%
% \[
% \begin{array}{ll}
% dcdg([x := e]^{l}) & = \{ (y^i, x^l) | y \in VAR(e) \land (y,i) \in \live_{in}(l) \} \\
% dcdg([x := q(e)]^{l}) & = \{ (y^i, x^l) | y \in VAR(e) \land (y,i) \in \live_{in}(l) \} \\
% dcdg([skip]^{l}) & = \emptyset \\
% dcdg([if [b]^l then C_1 else C_2) & = dcdg(c_1) \cup dcdg(c_2)\\ & \cup \{(x^i,y^j) | x \in VAR(b) \land (x,i) \in \live_{in}(l) \land ([y = \_]^j) \in blocks(c_1) \} \\
% &\cup \{(x^i,y^j) | x \in VAR(b) \land (x,i) \in \live_{in}(l) \land ([y = \_]^j) \in blocks(c_2) \} \\
% dcdg([while [b]^l do c) & = dcdg(c) \cup \{(x^i,y^j) | x \in VAR(b) \land (x,i) \in \live_{in}(l) \land ([y = \_]^j) \in blocks(C) \} \\
% dcdg(c_1 ;c_2) & = dcdg(c_1) \cup dcdg(c_2) \\
% \end{array}
% \]
%
\begin{defn}[Feasible Data-Flow]
 \label{def:feasible_flowsto}
 Given a program $c$ and two labeled variables $x^i, y^j$ in this program, 
 $\flowsto(x^i, y^j, c)$ is 
 {\small
 \[
 \begin{array}{ll}
 \flowsto(x^i, y^j, \clabel{\assign{y}{\expr}}{}^j) 
 & \triangleq (x^i, y^j) \in \left\{ (x^i, y^j) | x \in \mathsf{FV}(\expr) 
 % \land (y,i) \in \live(l, \clabel{\assign{x}{\expr}}^l) \} \\
 \land x^i \in \live(l, \clabel{\assign{y}{\expr}}^j) \right\} \\
 \flowsto(x^i, y^j, \clabel{\assign{y}{\query(\qexpr)}}{}^j) 
 & \triangleq (x^i, y^j) \in \left\{ (x^i, y^j) | x \in \mathsf{FV}(\qexpr) 
 % \land (y,i) \in \live(l,\clabel{\assign{x}{\query(\qexpr)}}^l) \} \\
 \land x^i \in \live(l,\clabel{\assign{y}{\query(\qexpr)}}^j) \right\} \\
 \flowsto(x^i, y^j, [\eskip]^{l}) & = \emptyset \\
 \flowsto(x^i, y^j, \eif ([b]^l, c_1, c_2)) & \triangleq \flowsto(x^i, y^j, c_1) \lor \flowsto(x^i, y^j, c_2) \\ 
 & \lor (x^i, y^j) \in
 \left\{(x^i,y^j) | x \in \mathsf{FV}(b) \land 
 % (x,i) 
 x^i \in \live(l, \eif ([b]^l, c_1, c_2)) \land y^j \in \lvar(c_1) \right\} \\
 % ([y = \_]^j) \in \mathsf{blk}(c_1) \} \\
 &\lor (x^i, y^j) \in \left\{(x^i,y^j) | x \in \mathsf{FV}(b) \land 
 % (x,i) 
 x^i\in \live(l, \eif ([b]^l, c_1, c_2)) \land y^j \in \lvar(c_2) \right\} \\
 % \land ([y = \_]^j) \in \mathsf{blk}(c_2) \} \\
 \flowsto(x^i, y^j, \ewhile [b]^l \edo c_w) & \triangleq \flowsto(x^i, y^j, c_w) \lor
 \\ & 
 (x^i, y^j) \in \left\{(x^i,y^j) | x \in \mathsf{FV}(b) \land 
 % (x,i) 
 x^i \in \live(l, \ewhile [b]^l \edo c_w) \land y^j \in \lvar(c_w) \right\} \\
 % ([y = \_]^j) \in \mathsf{blk}(c_w) \} \\
 \flowsto(x^i, y^j, c_1 ;c_2) & \triangleq \flowsto(x^i, y^j, c_1) \lor \flowsto(x^i, y^j, c_2) \\
 \end{array}
 \]
 }
 \end{defn}
%
This \emph{Feasible Data-Flow} relation is a sound approximation 
of the \emph{variable may-Dependency} relation over labeled variables for every program.
The soundness is proved in the full paper.
% in the appendix.
% ~\ref{apdx:flowsto_soundness}.
% %
% \paragraph*{Edge Estimation for Step 3}
% Based on the \emph{Feasible Data-Flow} relation,
% we estimate a set of directed edges, which is 
% used to construct the estimated program-based dependency graph for step 3.
% % for each vertex in $\progV(c)$,
% The set of directed edges between labeled variables is defined
% as a set of pairs 
% % $\progW(c) \in \mathcal{P}(\mathcal{LV} \times \mathcal{LV} \times EXPR(\constdom))$ 
% $\progE(c) \in \mathcal{P}( \mathcal{LV} \times \mathcal{LV})$.
% % is the set of pairs 
% % The weight for each vertex in $\progV(c)$ is computed 
% Each pair of vertices indicates a directed edge from the first vertex to the second one
% % in each pair
% formally as follows.
% {
% \[
% \begin{array}{ll}
% \progE(c) \triangleq &
% % \left
% \{ 
% (y^j, x^i) 
% % \in \mathcal{LV} \times \mathcal{LV}
% ~ \vert ~
% % \begin{array}{l}
% y^j, x^i \in \progV(c)
% \land
% % \\
% \exists n,
% % \in \mathbb{N}, 
% z_1^{r_1}, \ldots, z_n^{r_n} \in \lvar(c) \st 
% % 
% \\ 
% & \qquad \qquad
% n \geq 0 \land
% % \\
% \flowsto(x^i, z_1^{r_1}, c) 
% \land \cdots \land \flowsto(z_n^{r_n}, y^j, c) 
% % \end{array}
% % \right
% \}
% \end{array}
% \]
% }
% This estimated directed edge set $\progE(c)$ is a sound approximation of the 
% edge set in $c$'s execution-based dependency graph, which is proved in the full paper.

