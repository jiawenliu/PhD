I develop a variant of data flow analysis, called Feasible Data-Flow Generation, which produces a sound approximation of the edges in the execution based dependency graph.
%  $\live_{in}(l,c)$ and $\live_{out}(l, c)$ is computed by the Standard worklist algorithm. (For simplicity, we use $\live(l,c)$ to represent $\live_{in}(l,c)$ in the other part of the paper.
%%
\paragraph{Feasible Data-Flow Generation}
We generate edges by using both control and data flow in the following $\flowsto$ relation, which uses the results of reaching definition analysis, as $\live(l, c)$ for every label $l$ in a program $c$. $\mathsf{FV}$ computes the set of free variables in an expression. 
% $\mathsf{blk}$ returns a list of labeled computation blocks of a labeled command $c$.
%
%   \[
%  \begin{array}{ll}
%     dcdg([x := e]^{l})  & = \{ (y^i, x^l) | y \in VAR(e) \land (y,i) \in \live_{in}(l) \}  \\
%      dcdg([x := q(e)]^{l})  & = \{ (y^i, x^l) | y \in VAR(e) \land (y,i) \in \live_{in}(l) \}  \\
%      dcdg([skip]^{l})  & = \emptyset \\
%      dcdg([if [b]^l then C_1 else C_2)  & =  dcdg(c_1) \cup dcdg(c_2)\\ & \cup \{(x^i,y^j) | x \in VAR(b) \land (x,i) \in \live_{in}(l) \land ([y = \_]^j) \in blocks(c_1) \} \\
%      &\cup \{(x^i,y^j) | x \in VAR(b) \land (x,i) \in \live_{in}(l) \land ([y = \_]^j) \in blocks(c_2) \} \\
%      dcdg([while [b]^l do c)  & =  dcdg(c) \cup \{(x^i,y^j) | x \in VAR(b) \land (x,i) \in \live_{in}(l) \land ([y = \_]^j) \in blocks(C) \} \\
%      dcdg(c_1 ;c_2)  & = dcdg(c_1) \cup  dcdg(c_2) \\
%  \end{array}
%  \]
%
\begin{defn}[Feasible Data-Flow]
  \label{def:feasible_flowsto}
  Given a program $c$ and two labeled variables $x^i, y^j$  in this program, 
  $\flowsto(x^i, y^j, c)$ is 
    {\footnotesize
    \[
   \begin{array}{ll}
    \flowsto(x^i, y^j, \clabel{\assign{x}{\expr}}{}^l)  & \triangleq (x^i, y^j) \in \{ (y^i, x^l) | y \in \mathsf{FV}(\expr) 
    % \land (y,i) \in \live(l, \clabel{\assign{x}{\expr}}^l) \}  \\
    \land y^i \in \live(l, \clabel{\assign{x}{\expr}}^l) \}  \\
    \flowsto(x^i, y^j, \clabel{\assign{x}{\query(\qexpr)}}{}^l)  & \triangleq (x^i, y^j) \in \{ (y^i, x^l) | y \in \mathsf{FV}(\qexpr) 
    % \land (y,i) \in \live(l,\clabel{\assign{x}{\query(\qexpr)}}^l) \}  \\
    \land y^i \in \live(l,\clabel{\assign{x}{\query(\qexpr)}}^l) \}  \\
    \flowsto(x^i, y^j, [\eskip]^{l})  & = \emptyset \\
    \flowsto(x^i, y^j, \eif ([b]^l, c_1, c_2))  & \triangleq \flowsto(x^i, y^j, c_1) \lor \flowsto(x^i, y^j, c_2) \\ 
        & \lor (x^i, y^j) \in
       \{(x^i,y^j) | x \in \mathsf{FV}(b) \land 
      %  (x,i) 
      x^i \in \live(l, \eif ([b]^l, c_1, c_2)) \land  y^j \in \lvar(c_1) \\
    %   ([y = \_]^j) \in \mathsf{blk}(c_1) \} \\
       &\lor (x^i, y^j) \in \{(x^i,y^j) | x \in \mathsf{FV}(b) \land 
      %  (x,i) 
      x^i\in \live(l, \eif ([b]^l, c_1, c_2))  \land  y^j \in \lvar(c_2) \\
    %   \land ([y = \_]^j) \in \mathsf{blk}(c_2) \} \\
       \flowsto(x^i, y^j, \ewhile [b]^l \edo c_w)  & \triangleq  \flowsto(x^i, y^j, c_w)  \lor
       \\ & 
       (x^i, y^j) \in  \{(x^i,y^j) | x \in \mathsf{FV}(b) \land 
      %  (x,i) 
      x^i \in \live(l,   \ewhile [b]^l \edo c_w) \land  y^j \in \lvar(c_w) \\
    %   ([y = \_]^j) \in \mathsf{blk}(c_w) \} \\
       \flowsto(x^i, y^j, c_1 ;c_2)  & \triangleq \flowsto(x^i, y^j, c_1) \lor \flowsto(x^i, y^j, c_2) \\
   \end{array}
   \]
   }
   \end{defn}
%
We prove that this \emph{Feasible Data-Flow} relation is a sound approximation 
of the \emph{variable may-Dependency} relation over labeled variables for every program,
in the appendix.
% ~\ref{apdx:flowsto_soundness}.
%
\paragraph*{Edge Estimation}
Then we define the estimated directed edges
% for each vertex in $\progV(c)$,
between vertices in $\progV(c)$,
as a set of pairs 
% $\progW(c) \in \mathcal{P}(\mathcal{LV} \times \mathcal{LV} \times EXPR(\constdom))$ 
$\progE(c) \in \mathcal{P}( \mathcal{LV} \times \mathcal{LV})$
% is the set of pairs 
% The weight for each vertex in $\progV(c)$ is computed 
indicating a directed edge from the first vertex to the second one in each pair
as follows,
{
  \[
    \begin{array}{ll}
    \progE(c) \triangleq &
    % \left
    \{ 
    (y^j, x^i) 
    % \in \mathcal{LV} \times \mathcal{LV}
    ~ \vert ~
    % \begin{array}{l}
      y^j, x^i \in \progV(c)
    \land
      % \\
      \exists n,
    %   \in \mathbb{N}, 
      z_1^{r_1}, \ldots, z_n^{r_n} \in \lvar(c) \st 
    %   
    \\ 
    & \qquad \qquad
      n \geq 0 \land
    %   \\
      \flowsto(x^i,  z_1^{r_1}, c) 
      \land \cdots \land \flowsto(z_n^{r_n}, y^j, c) 
    % \end{array}
    % \right
    \}
    \end{array}
    \]
}
We prove that this estimated directed edge set $\progE(c)$ is a sound approximation of the 
edge set in $c$'s execution-based dependency graph 
in the appendix.
% ~\ref{apdx:adapt_soundness}.
%  \begin{defn}[Feasible Data-Flow]
%   \label{def:feasible_flowsto}
%     {\footnotesize
%     \[
%    \begin{array}{ll}
%       dcdg(\clabel{\assign{x}{\expr}}{}^l)  & = \{ (y^i, x^l) | y \in FV(e) \land (y,i) \in \live_{in}(l, \clabel{\assign{x}{\expr}}^l) \}  \\
%        dcdg(\clabel{\assign{x}{\query(\qexpr)}}{}^l)  & = \{ (y^i, x^l) | y \in FV(e) \land (y,i) \in \live_{in}(l,\clabel{\assign{x}{\query(\qexpr)}}^l) \}  \\
%        dcdg([\eskip]^{l})  & = \emptyset \\
%        dcdg([\eif [b]^l \ethen c_1 \eelse c_2)  & =  dcdg(c_1) \cup dcdg(c_2)\\ & \cup 
%        \{(x^i,y^j) | x \in FV(b) \land (x,i) \in \live_{in}(l) \land ([y = \_]^j) \in blocks(c_1) \} \\
%        &\cup \{(x^i,y^j) | x \in FV(b) \land (x,i) \in \live_{in}(l) \land ([y = \_]^j) \in blocks(c_2) \} \\
%        dcdg([\ewhile [b]^l \edo c)  & =  dcdg(c) \cup \{(x^i,y^j) | x \in FV(b) \land (x,i) \in \live_{in}(l) \land ([y = \_]^j) \in blocks(C) \} \\
%        dcdg(c_1 ;c_2)  & = dcdg(c_1) \cup  dcdg(c_2) \\
%    \end{array}
%    \]
%    }
%    \end{defn}
%    For any two labeled variables $x^i, y^j$ in a program $c$, 
%   %  it is easy to see that there is a one-on-one correspondence between 
%   %  $\flowsto$ relation of the two variables, and the $dcdg$ analysis result on $c$.
%   we use $\flowsto()$ denote if they have a feasible data-flow relation in Definition~\ref{def:flowsto}.
%    \begin{defn}[Feasible Data-Flow ($\flowsto$)]
%    \label{def:flowsto}
%    \[
%    \forall c \in \cdom, x^i, y^j \in \lvar_c \st 
%    \flowsto(x^i, y^j, c) \iff (x^i, y^j) \in dcdg(c)
%    \]
%    \end{defn}
  %  This soundness is proved in Proof~\ref{pf:rd_soundness} in appendix~\ref{apdx:rd_soundness}.
  %  For any two labeled variables in a program $c$, it is easy to see that there is a one-on-one correspondence between 
  %  $\flowsto$ relation of the two variables, and the $dcdg$ analysis result on $c$.
  %  \begin{thm}[Soundness of the Feasible Data-Flow Analysis]
  %  \label{thm:rd_soundness}
  %  \[
  %  \forall c \in \cdom, x^i, y^j \in \lvar_c \st 
  %  \flowsto(x^i, y^j, c) \iff (x^i, y^j) \in dcdg(c)
  %  \]
  %  \end{thm}
  %  This soundness is proved in Proof~\ref{pf:rd_soundness} in appendix~\ref{apdx:rd_soundness}.
  \paragraph*{Example}
 Look at Figure~\ref{fig:overview-example}(c),
and take the edge $(l^6, a^5)$ for example.
By $\flowsto(l^6, a^5, c)$, we can see $a$ is used directly in the query expression $\chi[k]*a$,
in the assignment command $\clabel{\assign{l}{\query(\chi[k]*a)}}^l$,
i.e., $a \in FV(\chi[k]*a)$.
Also, from the reaching definition analysis, we know $a^5 \in \live(6, \kw{twoRounds(k)})$.
Then we have $\flowsto(l^6, a^5, c)$ and construct the edge $(l^6, a^5)$.
And the same way for constructing the rest edges. Also, the edge $(x^3,j^5)$ in the same graph represents the control flow, caught by our $\flowsto$ relation.
%
%   \subsection{Program-Based Data Dependency Graph Generation}
%   %  Weighted Data Dependency Graph Generation}
%   \label{sec:alg_graphgen}
   %
%    Each directed edge represents an abstract transition 
%    between two control locations, i.e., the labels of two commands (we call the labels also control location and they refer to the same thing in the follows), 
%    where the second labeled command will be executed after execution of the command with first label.
%    The abstract transition contains a set of difference constraints for variables, generated by abstracting the command of the first label.
%   \item Computing 
%   % we get the reachability bound for each command.
%   the symbolic reachability bound for each command,
%   % the value bound invariant for each variable in the event and 
%   by inferring the value bound invariant for each variable 
%   % the event transition closure over the abstract control flow graph,
%   and the transition closure for every abstract transition through the constraints over the abstract control flow graph.
%   % \\
%   % Through this graph and constraint for every transition, we infer the  invariant for every variable,
%   % and compute the transition closure for every abstract transition.
%   % By solving the closure with the invariants of variables involved in this closure for every transition, 
%   % we compute the symbolic reachability bound of every commands corresponding to 
% %     % this transition.
% %     \item Performing a feasible data-flow analysis from the reachable definition algorithm. 
% % %  By generating set of all the reachable variables at location of label $l$ in the program $c$.
% % and generating the set of all the reachable variables for every program location.
% % For every labelled variable $x^l$ in this set, 
% % the value assigned to that variable
% % in the assignment command associated to that label is reachable at the entry point of  executing the command of label $l$.
% % \item Refining the abstract control flow graph into a weighted-data dependency graph, 
% % by annotating each vertex with reachability bounds and 
% % removing unfeasible edges and redundant edges and vertices.
% % adding edges between
% %     variables having data-flow relations, and
% % removing the edges between locations where the variables associated to that labeled command isn't reachable from the second location.
% % \\
% % first annotate each vertex of label $l$ with the variable 
% % assigned in that labeled command, and remove the rest doesn't correspond to an assignment command.
% % Then 
% % add direct edge between two labeled variables,
% % where the first variable 
% % is directly used in the assignment expression to the second variable, by restricting 
% % the first labeled variable is reachable at the the second label.
% %
% \item Computing the adaptivity through this weighted data dependency graph,
%   by finding a finite walk on this weighted graph, 
% traversing the maximum times of query variables, by restricting the visiting time of every vertex on this walk to its weight.
% The maximum number of vertices corresponding to a query variables visited on this walk is the estimated upper bound, for program's adaptivity.

%    In this step, $\THESYSTEM$ refines the abstract control flow graph into the program-based weighted-data dependency graph, 
% by annotating each vertex with reachability bounds and 
% removing unfeasible edges and redundant edges and vertices,
% % This graph is used 
% for approximating the trace-based weight-data dependency graph.
% \\
% Specifically, we first annotate each vertex of label $l$ with the variable 
% assigned in that labeled command, and remove the rest doesn't correspond to an assignment command.
% Then 
% add direct edge between two labeled variables,
% where the first variable 
% is directly used in the assignment expression to the second variable, by restricting 
% the first labeled variable is reachable at the second label.
% % \\
% The formal definition is as follows.
% 