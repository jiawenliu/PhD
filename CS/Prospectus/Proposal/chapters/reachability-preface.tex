\highlight{The number of times a given control location 
inside a procedure is visited during the program execution
is one of the importance quantitative reachability properties for programs.
Finding a tight bound on this quantitative reachability property is
useful in different applications, such as
bounding 
resources consumed by a program such as time, memory,
network-traffic, power, 
or helping to estimate other quantitative properties (as opposed to boolean properties)
of data in programs, such as information leakage or uncertainty propagation.
In this part, I focus on analyzing this quantitative reachability property and propose a new program analysis algorithm.
This algorithm can estimate a tighter upper bound on the number of execution
times on a given control location in a program than previous works more efficiently.
}