The previous works on formalizing the adaptivity is through a query-based dependency graph.
Below is a summary of their query-based dependency graph construction and adaptivity formal definition, followed
by the weaknesses of this formalization method.
\subsection*{Query-Based Dependency Graph}
They build their dependency graph over annotated queries, in order to construct the edge between queries for this graph, they first
define a query \emph{May-Dependency} relation. Below is a summary of this definition from
% Below is the formal definition of query may dependency based on their trace-based operational semantics 
from Thesis~\cite{weihao22}.
The notations $\in_q$ and $\not\in_q$ a query belongs to the trace or not respectively, with the full details in Thesis~\cite{weihao22}.
\begin{defn}[Query may dependency]
One query $q({v_q}_1)$ may depend on another query $q({v_q}_2)$ in a program $c$, with a starting loop maps $w$, a starting memory $m$, hidden database $D$, denoted as \\
$\mathsf{DEP}(q({v_q}_1)^{(l_1, w_1)}, q({v_q}_2)^{(l_2, w_2)}, c,w, m, D)$ is defined below. 
\[
  {\footnotesize
\begin{array}{l}
\forall  t. \exists m_1,m_3,t_1,t_3,c_2.\\
  \left (\begin{array}{l}   
\config{m, c,  t,w} \rightarrow^{*} \config{m_1, [\assign{x}{q({v_q}_1)}]^{l_1} ; c_2,
  t_1,w_1} \rightarrow \\ \config{m_1[q({v_q}_1)(D)/x], c_2,
  t_1++[q({v_q}_1)^{(l_1, w_1)}], w_1} \rightarrow^{*} \config{m_3, \eskip,
  t_3,w_3} \\  
  \land \\
\Big( q({v_q}_1)^{(l_1,w_1)} \in_{q} (t_3-t) \land q({v_q}_2)^{(l_2,w_2)} \in_{q} (t_3-t_1) \\ \implies  \exists v \in \codom(q({v_q}_1)), m_3', t_3', w_3'.  \\
 \config{m_1[v/x], {c_2}, t_1++[q({v_q}_1)^{(l_1,w_1)}], w_1} \rightarrow^{*} \config{m_3', \eskip, t_3', w_3'} \\ \land (q({v_q}_2)^{(l_2,w_2)}) \not \in_{q} (t_3'-t_1)
\Big)\\
\land \\
\Big(q({v_q}_1)^{(l_1,w_1)} \in_{q} (t_3-t) \land q({v_q}_2)^{(l_2,w_2)} \not\in_{q} (t_3-t_1) \\ \implies  \exists v \in \codom(q({v_q}_1)),  m_3', t_3', w_3'. \\
 \config{m_1[v/x], {c_2}, t_1++[q({v_q}_1)^{(l_1,w_1)}], w_1} \rightarrow^{*} \config{m_3', \eskip, t_3', w_3'} \\ \land (q({v_q}_2)^{(l_2,w_2)})  \in_{q} (t_3'-t_1)
\Big)
\end{array} \right )
\end{array}
}
\]
\end{defn}
%
Based on the \emph{may-dependency} between the two queries above,
below is their formal definition of query-based dependency graph. 
For every execution of a program $c$ staring with a certain memory and while map
%  configurations, 
they construct a corresponding dependency graph.  
% graph definition
\begin{defn}[Query-based Dependency Graph]
Given a program $c$, a database $D$, a starting memory $m$, an initial loop map $w$, the query-based dependency graph $G(c,D,m,w) = (V, E)$ is defined as: \\
$V =\{q({v_q})^{l,w} \in \mathcal{AQ} \mid \forall t. \exists m',  w', t'.  \config{m ,c, t, w}  \to^{*}  \config{m' , \eskip, t', w' }  \land q({v_q})^{l,w} \in {(t'-t)}  \}$.
\\
$E = \left\{(q({v_q})^{(l,w)},q({v_q}')^{(l',w')}) \in \mathcal{AQ} \times \mathcal{AQ} 
~ \left \vert ~ \mathsf{DEP}(q({v_q}')^{(l',w')},q({v_q})^{(l,w)}, c,w,m,D)
 \right.\right\}$.
\end{defn}
%
% The function $\mathsf{To}(q(v')^{(l',w')}, q(v)^{(l,w)}$ tells that the query request $q(v')^{(l',w')}$ appears after the query request $q(v)^{(l,w)}$ in the trace, by comparing the annotation $(l',w')$ and $(l,w)$. It helps to decide on the direction of one edge.
The edge is directed, when an annotated query $q({v_q})^{(l,w)}$ may depend on its previous query $q({v_q}')^{(l',w')}$, we have the directed
edge $(q({v_q})^{(l,w)}, q({v_q}')^{(l'.w')})$, from $q({v_q})^{(l,w)} $ to $q({v_q}')^{(l'.w')}$.

The query-based dependency graph only considers the newly generated annotated queries during the execution of the program $c$,
so they see the nodes coming from the trace $t'-t$.
The previous trace before the execution of $c$ is excluded when constructing the graph.
% To summarize, for every execution of a program $c$ staring with different configurations, 
% they can construct a corresponding dependency graph. 

\subsection*{Adaptivity Formalization}
Below is the definition of adaptivity from Thesis~\cite{weihao22}, by means of the query-based dependency graph. 
\begin{defn}[Adaptivity in {loop} language]
Given a program $c$, and a memory $m$, a database $D$, a starting loop map $w$, the adaptivity of the dependency graph $G(c, D,m,w) = (V, E)$ is the length of the longest path in this graph. We denote the path from $q({v_q})^{(l,w)}$ to $q({v_q}')^{(l',w')}$ as $p(q({v_q})^{(l,w)}, q({v_q}')^{(l',w')} )$. The adaptivity denoted as $A(c, D, m, w)$.
%
$$A(c, D, m, w) = \max\limits_{q({v_q})^{(l,w)},q({v_q}')^{(l',w')} \in V } |p(q({v_q})^{(l,w)}, q({v_q}')^{(l',w')} )| $$
\end{defn}
% \subsection*{Limitations}
% Limited by their language design and the dependency graph construction,
% this adaptivity definition is weak in the following aspects.
% \highlight{
%  \begin{enumerate}
%  \item \textbf{Expressiveness Limitation}
%  \\
%  The definition isn't general enough to give the formal adaptivity for the adaptive data analysis programs with non-constant
%  while loops.
%  However, programs containing while loops of non-deterministic iteration times are very common in the adaptive data analysis area.
%  This is caused by the expressiveness limitation in their language design and operational semantics design as analyzed
%  in Section~\ref{sec:prework-language}.
%  \\
%  In the following example program similar to the one in Section~\ref{sec:prework-language} with two extra query request commands,
%  \[
%  {\assign{x}{0}};
%  \assign{y}{5};
%  \assign{z}{q(x + y)};
%  \ewhile (x < y) \edo 
%  \{
%  \assign{x}{x + 1};
%  \assign{y}{y - 1};
%  \assign{z}{q(x+y+z)};
%  \}\}
%  \] 
%  the number of iterations cannot be evaluated to a nature number in advance of entering this loop for the same reason. 
%  The iteration number is only
%  able to be decided during executing this while loop body.
%  % This program represents a class of data analysis programs with non-constant loop iterations, which is very common in data analysis. However, it isn't supported by their design.
%  \\
%  Limited by this, they cannot give the adaptivity for this program even though the adaptivity in this program is 4, which is straightforward to observe.
%  % a trace generated for this program
%  \item \textbf{Accuracy Limitation}
%  \\
%  The in-accuracy of this \emph{adaptivity} definition is caused by both their language design and their graph construction.
%  \\
%  As analyzed in Section~\ref{sec:prework-language}, the information in non-query requesting variables are lost.
%  The dependency that passes through these variables is lost, and the adaptivity is lost as well.
%  \\
%  The other reason for this in-accuracy comes from their graph definition.
%  This dependency graph definition relies on a specific memory, while-map, and a specific execution trace.
%  It limits the adaptivity defined for the adaptive data analysis program to be w.r.t. one specific
%  execution.
%  In the other words, this adaptivity definition doesn't correspond to the \emph{adaptivity} for this program,
%  but for the program in a certain execution.
%  % to The trace only tracks the query requests. This lost the information of the variables
%  % which are assigned by query values even if they are not assigned by query requests. 
%  \item \textbf{Efficiency Limitation}
%  \\
%  This definition is in-efficient in the sense that it requires the full unfolding of every iteration for the while loop.
%  There are two reasons for this unfolding operation.
%  The first one comes from the low efficiency of their trace-based operational semantics,
%  which requires the trace to track the loop iteration number in the annotated query.
%  The other comes from their dependency graph generation, which requires every query request evaluated during the program execution
%  as the graph nodes.
%  Both of the two processes generate unnecessary and duplicate nodes during the while iterations.
%  % annotatation of the the query request executed in the program
%  % with integer indicating the while loops. 
% \end{enumerate}
% }
% This operational semantics is limited in the following aspects:
% \begin{itemize}
%  \item \textbf{Expressiveness Limitation}
%  \\
%  The operational-semantics causes a similar limitation as their syntax.
%  The loop iteration number can only be a constant number or an arithmetic expression evaluated to a nature number.
%  This is caused by their operational semantics rule design, trace design, and their annotated query design.
%  In the rule \rname{l-loop}, a nature number $v_N$ is required on the premise of tracking the iteration times.
%  This requires that the loop iteration number has to be a nature number or evaluated to a nature number in advance to execute the loop.
%  Then in their trace and annotated query design,
%  % generated through the operational semantics rules.
%  the trace tracks the annotated query, which requires an integer annotation explicitly indicating the iteration number of the loop.
%  % annotatation of the the query request executed in the program
%  % with integer indicating the while loops. 
%  \\
%  For example, in the following program with a simple while loop,
%  \[
%  {\assign{x}{20}};
%  \assign{y}{100};
%  \ewhile (x < y) \edo 
%  \{
%  \assign{x}{x + 1};
%  \assign{y}{y - 2};
%  \}\}
%  \] 
%  the number of iterations cannot be evaluated to a nature number in advance of entering this loop. 
%  The iteration number is only
%  able to be decided during executing this while loop body.
%  This program represents a class of data analysis programs with non-constant loop iterations
%  which is very common in data analysis. However, it isn't supported by their design.
%  \item \textbf{Accuracy Limitation}
%  \\
%  This operational semantics design causes in-precision in formalizing the \emph{adaptivity}.
%  Because there is information loss in their trace generated through the operational semantic rules.
%  The trace only tracks the query requests. This lost the information of the variables
%  which are assigned by query values even if they are not assigned by query requests. However, these variables
%  are critical in analyzing the \emph{adaptivity}.
%  This will be analyzed in detail in the limitation in Section~\ref{sec:prework-formalization}.
%  \item \textbf{Efficiency Limitation}
%  \\
%  There are four components in their configuration in order to evaluate the program. 
%  The update operations for this quadruple configuration are low-efficient, especially the update operation of the while map.
% \end{itemize}