This section analyzes the limitations of previous works and motivation for developing a new
program analysis for the adaptivity of the adaptive data analysis.
% Limited by their language design, the adaptivity definition and the techniques in
% their adaptivity estimation, t
The weaknesses mainly lay in the three aspects, the expressiveness, accuracy and efficiency
throughout their analysis framework.
% \highlight{
%     \paragraph{Limitations of The Language Syntax}
% \begin{enumerate}
% % \item  It supports limited query expressions.
% % %
% \item  It doesn't support the program which contains while loop with non-deterministic iterations.
% In the other words, it only supports the simple loop programs with
% constant (or arithmetic expression which is evaluated to constant) number of iterations.
% However, the adaptive data analysis programs with non-constant (or non-deterministic) iteration numbers are very common.
% %
% \item  It doesn't support the program which contains the user inputs.
% This kind of program is also common in the data analysis area.
% \end{enumerate}
% }
\highlight{
% \paragraph{Limitations in The Language Model}
\begin{itemize}
 \item \textbf{Expressiveness Limitation}
%  \\
 \begin{enumerate}
    % \item  It supports limited query expressions.
    % %
    \item  This language syntax doesn't support programs with standard while loop.
    %  the program which contains while loop with non-deterministic iterations.
    In the other words, it only supports the programs with the loops of constant iteration numbers,
    % constant (or 
    arithmetic expression which is evaluated to constant
    % ) number of iterations.
    However, the adaptive data analysis programs with non-constant (or non-deterministic) iteration numbers are very common.
    %
    \item  The language syntax doesn't support the programs with user inputs, which are also common in the data analysis area.
    \item 
    % The operational semantics design
    % %  causes a similar limitation as their syntax.
    % % The 
    % restrict the loop iteration number only be a constant number or an arithmetic expression evaluated to a nature number too.
    % This is caused by the operational semantics rule design, trace definition, and the annotated query definition.
    % In the rule \rname{l-loop}, a nature number $v_N$ is required on the premise of tracking the iteration times.
    % This requires that the loop iteration number has to be a nature number or evaluated to a nature number in advance to execute the loop.
    % Then in their trace and annotated query design,
    % % generated through the operational semantics rules.
    % the trace tracks the annotated query, which requires an integer annotation explicitly indicating the iteration number of the loop.
    The operational semantics design further
    %  causes a similar limitation as their syntax.
    % The 
    restricts the expressiveness of the loop command.
    In the rule \rname{{l-loop-a}}, the iterator expression in the guard has to be evaluated a constant $V_N$ first,
    then the loop iteration starts the $v_N$ times iteration. 
    As a result, they cannot execute the following example program with a simple while loop, which is common in data analysis.
    % For example, in the following program with a simple while loop,
    \[
    {\assign{x}{20}};
    \assign{y}{100};
    \ewhile (x < y) \edo 
    \{
    \assign{x}{x + 1};
    \assign{y}{y - 2};
    \}
    \] 
    Because the number of iterations cannot be evaluated to a nature number in advance of entering this loop, 
    % The iteration number 
    is only
    able to be decided during executing this while loop body.
    % The iteration number only be a constant number or an arithmetic expression evaluated to a nature number too.
    % This is caused by the operational semantics rule design, trace definition, and the annotated query definition.
    % In the rule \rname{l-loop}, a nature number $v_N$ is required on the premise in order to track the current iteration number.
    % This requires that the loop iteration number has to be a nature number or evaluated to a nature number in advance to execute the loop.
    % Then in the trace and annotated query definition,
    % % generated through the operational semantics rules.
    % the trace tracks the annotated query, which requires an integer annotation explicitly indicating the iteration number of the loop.
    %    % % annotatation of the the query request executed in the program
    % % with integer indicating the while loops. 
    % % \\
    % For example, in the following program with a simple while loop,
    % \[
    % {\assign{x}{20}};
    % \assign{y}{100};
    % \ewhile (x < y) \edo 
    % \{
    % \assign{x}{x + 1};
    % \assign{y}{y - 2};
    % \}
    % \] 
    % the number of iterations cannot be evaluated to a nature number in advance of entering this loop. 
    % The iteration number is only
    % able to be decided during executing this while loop body.
    % This program represents a class of data analysis programs with non-constant loop iterations
    % which is very common in data analysis. However, it isn't supported by their design.
    \item Limited by the language model, the adaptivity definition is unable to give
    the formal adaptivity for the adaptive data analysis programs
    with non-constant
    while loops either.
    \item The static analysis algorithms \textbf{variable estimation} and the algorithm \textbf{graph generation}
    for estimating the adaptivity further limits the expressiveness.
%     Limited by both of their \textbf{variable estimation} algorithm and the \textbf{graph generation} algorithm,
% their static analysis algorithm is unable to analyze the programs with
%  non-constant (or non-deterministic) loop iteration numbers.
 This is caused by their \textbf{ag-loop} rule in Figure~\ref{fig:prework-static_alg1}.
 This rule unfolds every iteration of the while loop, and creates new annotated variables for every iteration.
 In order to guarantee the termination of the analysis, they have to limit the loop with the constant number of loop iterations.
 Specifically in rule \textbf{ag-loop} and \textbf{ad-loop},
 % Their variable estimation for the while loop is low-efficient. As shown in rule \textbf{ag-loop},
 % they unfold every iteration of the while loop and create new annotated variables for every iteration.
 % This rule causes a major efficiency limitation of their static analysis.
 % It also causes a critical expressiveness limitation. 
 the premise in these two rules requires
 the arithmetic expression
 % is required 
 to be a natural number. 
 This rule limits the program cannot even
 have a loop with an arithmetic expression like $10 + 4$ in the guard.
%  Concretely, a simple example program with a loop iterating $5$ times as follows isn't allowed in their work.
%  \[
%  {\assign{x}{5}};
%  \assign{z}{q(x)};
%  \eloop (x ) \edo 
%  \{
%  \assign{z}{q(x + z)};
%  \}
%  \] 
    \end{enumerate}
 \item \textbf{Accuracy Limitation}
 \\
 The adaptivity definition and the estimated adaptivity are both inaccurate.
 \\
 There are two reasons for the inaccuracy of adaptivity definition.
 \\
 The first cause is the operational semantics design.
%   in-precision in formalizing the \emph{adaptivity}.
 Because important information is lost in their trace.
%  generation.
%   generated through the operational semantic rules.
 The trace only tracks the query requests, which losses the information of the variables
 that use the query results but not assigned by query requests.
%   even if they are not assigned by query requests. 
These variables
 are critical in analyzing the \emph{adaptivity}.
 \\
 The second cause of the adaptivity definition is the dependency graph definition.
 It relies on a specific memory, while-map, and a specific execution trace.
 It limits the adaptivity defined for the adaptive data analysis program to be w.r.t. one specific
 execution.
 In the other words, this adaptivity definition doesn't correspond to the \emph{adaptivity} for this program,
 but for the program in a certain execution.
\\
The estimated adaptivity from this framework is loose.
It naively composed three standard static estimation algorithms without improvement.
%   in the sense that all three steps are standard.
%  And the framework is simply a straightforward composition of the three steps.
The naive \textbf{graph generation} algorithm over-approximates data dependency relation in a large scale.
%  This will be analyzed in detail in the limitation in Section~\ref{sec:prework-formalization}.
 \item \textbf{Efficiency Limitation}
 \begin{enumerate}
    \item In the operational semantics design, there are four components in their configuration in order to evaluate the program. 
    The update operations for this quadruple configuration are low-efficient, especially the update operation of the while map.   
    \item
    In order to address the issue of re-assignment of different queries requesting results to the same variable, they re-write the program from the loop language into SSA form.
    However, this rewriting is low-efficient and unnecessary.
    The re-assignment problem can be resolved efficiently and accurately through many state-of-art static program analysis techniques, such as the
    variable reachable analysis, etc..
    \item Their \textbf{variable estimation} and \textbf{graph generation}  algorithm is low-efficient by the \textbf{ag-loop} rule in Figure~\ref{fig:prework-static_alg1},
    and rule \textbf{ad-loop} in Figure~\ref{fig:prework-static_alg2}.
    These two rules unfold every iteration of the while loop and create new annotated variables for every iteration.
    This operation increased the complexity of the program analysis exponentially. 
   %  \item For the same reason as above, their \textbf{Graph Generation} algorithm is low-efficient as well.
    \end{enumerate}
\end{itemize}
}  

% \highlight{
% \paragraph{Limitations in The Adaptivity Definition}
% Their adaptivity definition is also limited by the three aspects, the expressiveness, accuracy and efficiency.
%  \begin{enumerate}
%  \item \textbf{Expressiveness Limitation}
%  \\
%  The definition isn't general enough to give the formal adaptivity for the adaptive data analysis programs
%  with non-constant
%  while loops.
% %  However, programs containing while loops of non-deterministic iteration times are very common in the adaptive data analysis area.
%  This is caused by the expressiveness limitation in their language model.
% %   design and operational semantics design as analyzed
% %  in Section~\ref{sec:prework-language}.
% %  \\
% The following example program
% %   similar to the one in Section~\ref{sec:prework-language} 
% has two query request commands.
%  \[
%  {\assign{x}{0}};
%  \assign{y}{5};
%  \assign{z}{q(x + y)};
%  \ewhile (x < y) \edo 
%  \{
%  \assign{x}{x + 1};
%  \assign{y}{y - 1};
%  \assign{z}{q(x+y+z)};
%  \}
%  \] 
%  Because the number of iterations cannot be evaluated to a nature number
%  before executing the loop body, the trace cannot be generated for this program.
% %   in advance of entering this loop for the same reason. 
% %  The iteration number is only
% %  able to be decided during executing this while loop body.
%  % This program represents a class of data analysis programs with non-constant loop iterations, which is very common in data analysis. However, it isn't supported by their design.
% %  \\
%  Limited by this,
%  they cannot give the adaptivity for this program even though the adaptivity in this program is 4,
%  which is straightforward to observe.
%  % a trace generated for this program
%  \item \textbf{Accuracy Limitation}
%  \\
%  The in-accuracy of this \emph{adaptivity} definition 
%   is caused by
% %    both their language design and their 
% two limitations in the data dependency graph design.
% %  \\
% %  As analyzed in Section~\ref{sec:prework-language}, 
% % Limited by their operational semantics design, the information in non-query requesting variables are lost.
% Firstly, the data dependency graph construction relies on the execution trace.
% But information of non-query requesting variables are lost in the trace.
% In this sense,
%  the data dependency relation that passes through these variables is lost,
% %   and the adaptivity is lost as well.
% as well as the adaptivity.
% %  \\
% %  The other cause of this in-accuracy is the design of their data dependency graph.
%  Secondly, the dependency graph definition relies on a specific memory, while-map, and a specific execution trace.
%  It limits the adaptivity defined for the adaptive data analysis program to be w.r.t. one specific
%  execution.
%  In the other words, this adaptivity definition doesn't correspond to the \emph{adaptivity} for this program,
%  but for the program in a certain execution.
%  % to The trace only tracks the query requests. This lost the information of the variables
%  % which are assigned by query values even if they are not assigned by query requests. 
%  \item \textbf{Efficiency Limitation}
%  \\
%  This definition is in-efficient in the sense that it requires the full unfolding of every iteration for the while loop.
%  There are two reasons
%  behind
% %   for 
% this unfolding operation.
%  One is the low-efficiency of the
% %  comes from the low efficiency of their 
%  trace-based operational semantics,
%  which requires the trace to track the loop iteration number in the annotated query.
%  The other comes from their dependency graph generation, which requires every query request evaluated during the program execution
%  as the graph nodes.
%  Both of the two processes generate unnecessary and duplicate nodes during the while iterations.
%  % annotatation of the the query request executed in the program
%  % with integer indicating the while loops. 
% \end{enumerate}
% }
% \highlight{
%     \paragraph{Limitations in The Adaptivity Estimation}
%     % The adaptivity estimation algorithm is mainly limited by its efficiency and the accuracy with weakness
%     % % It is also weak 
%     % in terms of expressiveness as well.
% \begin{enumerate}
%     \item \textbf{Expressiveness Limitations}
% %  \begin{enumerate}
% %  \item 
% \\
% Limited by both of their \textbf{variable estimation} algorithm and the \textbf{graph generation} algorithm,
% their static analysis algorithm is unable to analyze the programs with
%  non-constant (or non-deterministic) loop iteration numbers.
%  This is caused by their \textbf{ag-loop} rule in Figure~\ref{fig:prework-static_alg1}.
%  This rule unfolds every iteration of the while loop, and creates new annotated variables for every iteration.
%  In order to guarantee the termination of the analysis, they have to limit the loop with the constant number of loop iterations.
%  Specifically in rule \textbf{ag-loop} and \textbf{ad-loop},
%  % Their variable estimation for the while loop is low-efficient. As shown in rule \textbf{ag-loop},
%  % they unfold every iteration of the while loop and create new annotated variables for every iteration.
%  % This rule causes a major efficiency limitation of their static analysis.
%  % It also causes a critical expressiveness limitation. 
%  the premise in these two rules requires
%  the arithmetic expression
%  % is required 
%  to be a natural number. 
%  This rule limits the program cannot even
%  have a loop with an arithmetic expression like $10 + 4$ in the guard.
%  Concretely, a simple example program with a loop iterating $5$ times as follows isn't allowed in their work.
%  \[
%  {\assign{x}{5}};
%  \assign{z}{q(x)};
%  \eloop (x ) \edo 
%  \{
%  \assign{z}{q(x + z)};
%  \}
%  \] 
% %  \item For the same reason as above, their \textbf{Graph Generation} algorithm is limited as well.
% %  \end{enumerate}
%  \item \textbf{Efficiency Limitations}
%  \begin{enumerate}
%  \item
%  In order to address the issue of re-assignment of different queries requesting results to the same variable, they re-write the program from the loop language into SSA form.
%  However, this rewriting is low-efficient and unnecessary.
%  The re-assignment problem can be resolved efficiently and accurately through many state-of-art static program analysis techniques, such as the
%  variable reachable analysis, etc..
%  \item Their \textbf{variable estimation} and \textbf{graph generation}  algorithm is low-efficient by the \textbf{ag-loop} rule in Figure~\ref{fig:prework-static_alg1},
%  and rule \textbf{ad-loop} in Figure~\ref{fig:prework-static_alg2}.
%  These two rules unfold every iteration of the while loop and create new annotated variables for every iteration.
%  This operation increased the complexity of the program analysis exponentially. 
% %  \item For the same reason as above, their \textbf{Graph Generation} algorithm is low-efficient as well.
%  \end{enumerate}
%  %
%  \item \textbf{Accuracy Limitations}
%  The estimated adaptivity from this framework is loose.
%  It naively composed three standard static estimation algorithms without improvement.
% %   in the sense that all three steps are standard.
% %  And the framework is simply a straightforward composition of the three steps.
%  The naive \textbf{graph generation} algorithm over-approximates data dependency relation in a large scale.
% %  \begin{enumerate}
% %  \item The estimated adaptivity from this framework is loose,
% %  because the naive \textbf{graph generation} algorithm over-approximates data dependency relation.
% % %   in the cases where there isn't semantic dependency between variables even though the variables
% % %  are explicitly used in the query request.
% %  \item This program analysis framework is naive in the sense that all three steps are standard.
% %  And the framework is simply a straightforward composition of the three steps.
% %  \end{enumerate}
% \end{enumerate}
% }